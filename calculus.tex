%%%%%%%%%%%%%%%%%%%%%%%%%%%%%%%%%%%%%%%%%%%%%%%%%%%%%%%%%%%%%%%%%%%
%%%%%%%%%%%%%JIBLM Formatting Package%%%%%%%%%%%%%%%%%%%%%%%%%%%%%%
%%%%%%%%%%%%%Version 1.2: August, 2008%%%%%%%%%%%%%%%%%%%%%%%%%%%
%%%%%%%%%%%%%Author: Paul J. Kapitza, Berry College%%%%%%%%%%%%%%%%
%%%%%%%%%%%%%%%%%%%%%%%%%%%%%%%%%%%%%%%%%%%%%%%%%%%%%%%%%%%%%%%%%%%

\documentclass[oneside]{book}
%%%%%%%%%%%%%journal additions%%%%%%%%%%%%%%%%%%%%%%%%%%%%%%%%%%%%%
\usepackage{hyperref}
\usepackage{time}%make system time available
\usepackage{enumerate}%extended enumeration package
%%%%%%%%%%%%%Symbol libraries%%%%%%%%%%%%%%%%%%%%%%%%%%%%%%%%%%%%%
\usepackage{amssymb}
\usepackage{amsmath}
\usepackage{latexsym}
\usepackage{amsthm}%extended ams-theorem environment

\usepackage{lettrine}%Drop-caps for Masthead
\usepackage{mathptmx}%Times Roman type package for both math and text


\usepackage{endnotes}%Footnotes to the instructor.

%%%%%%%%%%%%%Header Customization%%%%%%%%%%%%%%%%%%%%%%%%%%%%%%%%%
\usepackage{fancyhdr}%Header customization
\pagestyle{fancy}
%%%%%%%%%%%%%Chapter headings%%%%%%%%%%%%%%%%%%%%%%%%%%%%%%%%%
\renewcommand{\chaptermark}[1] {\markboth{#1}{}}%

%%%%%%%%%%%%%Page Formatting%%%%%%%%%%%%%%%%%%%%%%%%%%%%%%%%%%
\setlength{\oddsidemargin}{63pt}%%%%%One-sided printing values for 10pt. text-Remove for two sided print
\setlength{\evensidemargin}{63pt}%%%%%One-sided printing values for 10pt. text-Remove for two sided print

\setlength{\parskip}{1mm}
\setlength{\textwidth}{5.0in}
\setlength{\textheight}{8.0in}

%%%%%%%%%%%%%%%%%%%%%%%%%%%%AUTHOR MASTHEAD%%%%%%%%%%%%%%%%%%%%%%%%%%%%%
\newcommand{\authormasthead}{
\begin{flushleft}
\hspace{4.4mm}
\rule{0.3\linewidth}{0.3mm}
\lettrine[lines=2]{J}{ournal of Inquiry-Based Learning in Mathematics}\\
\rule{0.3\linewidth}{0.3mm}
%\hspace{1mm} Issue~\textbf{#1}, Volume #2        Issue 1 (August, 2007)
\vspace{0.2in}
\end{flushleft}
}
%%%%%%%%%%%%%%%%%%%%%%%%%%%%AUTHOR MASTHEAD%%%%%%%%%%%%%%%%%%%%%%%%%%%%%

%%%%%%%%%%%%%%%%%%%%%%%%%%%%TIMESTAMP%%%%%%%%%%%%%%%%%%%%%%%%%%%%%
%%Uses the ``time" package to stamp the time-Editing Feature
\newcommand{\timestamp}{{Edited: \texttt{\now , \today}}}
%%%%%%%%%%%%%%%%%%%%%%%%%%%%TIMESTAMP%%%%%%%%%%%%%%%%%%%%%%%%%%%%%


\let\affiliation\date


%%%%%%%%%%%%%%%%%%%%%%%%%%%% TITLEPAGE%%%%%%%%%%%%%%%%%%%%%%%%%%%%%
%
\makeatletter
\def\maketitle{%
  \null
  \thispagestyle{empty}%
  \timestamp
  \authormasthead
  %\vfill
  \normalfont
  \vspace{2in}
\begin{center}\leavevmode
{\Huge \@title\par}%
\vspace{20mm}
{\Large \@author\par}%
\vspace{5mm}
{\Large \@date\par}% pass affiliation
{\Large \ }
\end{center}
  \vfill
  \null
  \cleardoublepage
 \let\newauthor\@author%transfer to footer line
 }%
\makeatother
%%%%%%%%%%%%%%%%%%%%%%%%%%%% END OF TITLEPAGE%%%%%%%%%%%%%%%%%%%%%%%%%%%%%

%Customized headers and footers- replace authorname with register
\lhead{ \leftmark} \chead{} \rhead{\thepage}
\lfoot{\newauthor} \cfoot{} \rfoot{\emph{math.lamar.edu/faculty/mahavier/}}
\renewcommand{\headrulewidth}{0.4pt}
\renewcommand{\footrulewidth}{0.4pt}
%
%%%%%%%%%%%%%%%%%%%%%%%%%%%% Annotation Environment %%%%%%%%%%%%%%%%%%%%%%%%%%%%%
\usepackage{comment}
\newcommand{\InstructorVersion}{\includecomment{annotation}}
\newcommand{\StudentVersion}{\excludecomment{annotation}}
%%%%%%%%%%%%%%%%%%%%%%%%%%%% END OF Annotation Environment%%%%%%%%%%%%%%%%%%%%%%%%%%%%%



%%%%%%%%%%%%%%%%%%%%%%%%%%%% Begin--Sectioning Redefines%%%%%%%%%%%%%%%%%%%%%%%%%%%%%
%
\makeatletter
\renewcommand{\@makechapterhead}[1]{%
\vspace*{50\p@}%
  {\parindent \z@ \raggedright \normalfont
    \ifnum \c@secnumdepth >\m@ne
      \if@mainmatter
        \huge \@chapapp\space \thechapter
        \par\nobreak
        \vskip 20\p@
      \fi
    \fi
    \interlinepenalty\@M
    \LARGE\bfseries  #1\par\nobreak
    \vskip 40\p@
  }}


\renewcommand{\@makeschapterhead}[1]{%
  \vspace*{50\p@}%
  {\parindent \z@ \raggedright
    \normalfont
    \interlinepenalty\@M
    \LARGE\bfseries  #1\par\nobreak
    \vskip 40\p@
  }}

\makeatother
%%%%%%%%%%%%%%%%%%%%%%%%%%%% End--Sectioning Redefines%%%%%%%%%%%%%%%%%%%%%%%%%%%%%

%%%%%%%%%%Theorem Environments%%%%%%%%%%%%%%%%%%%%%%%%
\newtheorem{theorem}{Theorem}
\newtheorem{acknowledgment}[theorem]{Acknowledgment}
\newtheorem{algorithm}[theorem]{Algorithm}
\newtheorem{axiom}[theorem]{Axiom}
\newtheorem{case}[theorem]{Case}
\newtheorem{claim}[theorem]{Claim}
\newtheorem{conclusion}[theorem]{Conclusion}
\newtheorem{condition}[theorem]{Condition}
\newtheorem{conjecture}[theorem]{Conjecture}
\newtheorem{corollary}[theorem]{Corollary}
\newtheorem{criterion}[theorem]{Criterion}
\newtheorem{definition}[theorem]{Definition}
\newtheorem{example}[theorem]{Example}
\newtheorem{exercise}[theorem]{Exercise}
\newtheorem{lemma}[theorem]{Lemma}
\newtheorem{notation}[theorem]{Notation}
\newtheorem{problem}[theorem]{Problem}
\newtheorem{proposition}[theorem]{Proposition}
\newtheorem{remark}[theorem]{Remark}
\newtheorem{solution}[theorem]{Solution}
\newtheorem{summary}[theorem]{Summary}
%%%%%%%%%%Theorem Environments%%%%%%%%%%%%%%%%%%%%%%%% 

%%%%%%%%%%%%%%%%%%%%% Annotation Environment Switch%%%%%%%%%%%%
%\StudentVersion
\InstructorVersion
%%%%%%%%%%%%%%%%%%%%% Annotation Environment Switch%%%%%%%%%%%%

%
%As of 2015, these have been used at Lamar, Texas State (Terry McCabe), St. Benedict St. Johns (Brett Benesh, Robert Campbell), Kim Brown (Tarrant County College), Susan Zielinski (highschool?), Universidade Federal Do Ceara (Don Girao, Brazil), Tarleton (Jeremiah Bass), Harvey Johnson (St. Andrews - Delaware), Potsdam (Victoria Klawitter), Michael Gagliardo (Cal Lutheran), somebody in colorado,

% Resource List
% goegebra, kahn academy, wolfram alpha

% changes
% 1.15.16 -- replaced all trig^{-1} with invtrig
% 1.15.16 -- rewrote limits at infinity to use DNE(\infty) instead of limit = infinity
% 1.27.16 -- pulled many problems from hyperbolic section and moved them to the drill section
% 1.28.16 -- added a problem at end of every chapter saying to do the practice
% 1.29.16 -- added some problems on work back to applications section
% 1.29.16 -- added application practice on WORK to appropriate practice section (spring, pump, cable pull)
% 1.29.16 -- renamed all practice problem sections to ''Chapter X Practice''
% 1.29.16 -- eliminated the use of the word ''Hint'' from the notes
% 2.1.16 --- searched web for calculus applets -- found geogebra,

% TO DO LIST

% consider reindexing series to a_k so that we don't have little and big ''n'' all over the place (Jason)

% the work section around line 2940 should be rewritten and re-incorporated

% consider killing method of slicing and putting disk and shell in


% the practice problems need to be addressed, ordered from simplest to hardest, arranged to exactly match sections/chapters?

% consider the format of the free linear aglebra text  (xml)

% consider incorporating sage experiments to lead them to certain conclusions

% consider incorporating mini-lectures from the web

% review moments application section and practice -- too long?  (you added apps to cII, must take something out)

% determine if world is moving away from early transendentals


% consider recommending external resources (Geogebra, Kahn Academy, Wolfram Alpha, Google for graphing)

% finish solutions to integration section or cut some of them out

% 2-3 practice problems with the divergence theorem

% need surface integrals and stokes theorem

% need to add practice on these

% need a practice section called Line Integrals on Vector Fields

% separate 2-D (divg/greens)  from 3-D (divg/stokes)

\setlength{\oddsidemargin}{0in}
\setlength{\leftmargin}{1in}
\setlength{\textwidth} {6.5in}
\setlength{\headwidth}{\textwidth}
%\setlength{\parindent} {0in}
%\setlength{\parskip} {7mm}
%\addtolength{\textheight} {1.5 in}
%\setlength{\headheight}{0in}
%\setlength{\topmargin}{-.7in}
%\setlength{\textheight} {10in}

\setlength{\headheight}{0in}
\setlength{\topmargin}{-.715in}
\setlength{\textheight} {9.5in}

\newcommand\re{\mathbb R}
\newcommand\nat{\mathbb N}
\newcommand\sech{\operatorname{sech}}
\newcommand\csch{\operatorname{csch}}
\newcommand\cotanh{\operatorname{cotanh}}
\newcommand\invsin{\operatorname{invsin}}
\newcommand\invcos{\operatorname{invcos}}
\newcommand\invtan{\operatorname{invtan}}
\newcommand\invsec{\operatorname{invsec}}
\newcommand\invcsc{\operatorname{invcsc}}
\newcommand\invcot{\operatorname{invcot}}
\newcommand\invsinh{\operatorname{invsinh}}
\newcommand\invcosh{\operatorname{invcosh}}
\newcommand\invtanh{\operatorname{invtanh}}
\newcommand\invcoth{\operatorname{invcoth}}
\newcommand\invsech{\operatorname{invsech}}
\newcommand\invcsch{\operatorname{invcsch}}


\def\oa{\overrightarrow}
\def\bm{\boldmath}
\def\ubm{\unboldmath}

\newtheorem{thm}{Theorem}
\newtheorem{qsn}{Question}
\newtheorem{dfn}{Definition}
\newtheorem{prb}{Problem}
\newtheorem{axm}{Axiom}
\newtheorem{lem}{Lemma}
\newtheorem{prp}{Proposition}
\newtheorem{con}{Conjecture}
\newtheorem{que}{Reflective Question}
\newtheorem{expl}{Example}
\newtheorem{sol}{Solution}
\newtheorem{ted}{TED COMMENT}
\newtheorem{epl}{Example}
\newtheorem{prf}{Proof}

\def\dsp{\displaystyle}

\begin{document}
\large
\frontmatter
\title{Calculus I, II, and III\\A Problem-Based Approach\\with Early Transcendentals}
\author{Mahavier $<$ Allen, Browning, Daniel, and So}
\affiliation{Lamar University}
\maketitle
\tableofcontents


\begin{annotation}
\chapter{To the Instructor}

``A good lecture is usually systematic, complete, precise -- and dull; it is a bad teaching instrument.'' - Paul Halmos

\section*{About these Notes}
\addcontentsline{toc}{section}{About these Notes}

These notes constitute a self-contained sequence for teaching, using an inquiry-based pedagogy, most of the traditional topics of Calculus I, II, and III in classes of up to 40 students.  The goal of the course is to have students constantly working on the problems and presenting their attempts in class every day, supplemented by appropriate guidance and input from the instructor.  For lack of a better word, I'll call such input \emph{mini-lectures}.  When the presentations in class and the questions that arise become the focus, a team-effort to understand and then explain becomes the soul of the course.  The instructor will likely need to provide input that fills in gaps in pre-calculus knowledge, ties central concepts together, foreshadows skills that will be needed in forthcoming material, and discusses some real-life applications of the techniques that are learned.  The key to success is that mini-lectures are given when the need arises and after students have worked hard.  At these times, students' minds are primed to take in the new knowledge just when they need it.   Sometimes, when the problems are hard and the students are struggling, the instructor may need to serve as a motivational speaker, reminding students that solving and explaining problems constitute the two skills most requested by industry.  Businesses want graduates who can understand a problem, solve it, and communicate the solution to their peers.  That is what we do, in a friendly environment, every day in class.  At the same time, we get to discover the amazing mathematics that is calculus in much the same way as you discover new mathematics -- by trying new things and seeing if they work!

Even though the vast majority of students come to like the method over time, I spend a non-trivial amount of time in the beginning explaining my motivation in teaching this way and encouraging students both in my office and in the classroom.  I explain that teaching this way is much harder for me, but that the benefit I see in students' future course performances justifies my extra workload.  And I draw parallels, asking a student what she is good at.  Perhaps she says basketball.  Since I don't play basketball, I'll ask her, ``If I come to your dorm or house this evening and you lecture to me about basketball and I watch you play basketball, how many hours do you think will be required before I can play well?''  She understands, but I still reiterate, that I can't learn to play basketball simply by watching it.  I can learn a few rules and I can talk about it, but I can't \emph{do} it.  To learn to play basketball, I'll need to pick up the ball and shoot and miss and shoot again.  And I'll need to do a lot of that to improve.  Mathematics is no different.   Sometimes a student will say, ``But I learn better when I see examples.''  I'll ask if that is what happened in his pre-calculus class and he'll say ``Yes.''  Then I'll point out that the first problems in this course, the very ones he is struggling with, are material from pre-calculus and the fact that he is struggling shows clearly that he did not learn.   His pre-calculus teacher did not do him any favors.  As soon as a problem is just barely outside of the rote training he saw in class and regurgitated on tests, he is unable to work the problems.  This brings home the fact that rote training has not been kind to him and that, as hard as it is on me, I am going to do what I think is best for him.  This blunt, but kind honesty with my students earns their trust and sets the stage for a successful course.

I've taught calculus almost every long semester since 1988.  When I first taught calculus, I lectured. A very few years later when I taught calculus, I sent my students to the board one day each week.  Using each of these methods, I saw students who demonstrated a real talent for mathematics by asking good questions and striving to understand concepts deeply.  Some of these students did poorly on rote tests.  Others did well.  But I was only able to entice a small number of either group to consider a major in mathematics.  Today when I teach Calculus, students present every day and mathematical inquiry is a major part of what occurs in the classroom.  And the minors and majors flow forward from the courses because they see the excitement that we as mathematicians know so well from actually doing mathematics.  Many are engineers who choose to earn a double major or a second degree.  Others simply change their major to mathematics.  The key to this sea change is that the course I now teach is taught just as I teach sailing.  I simply let them sail and offer words of encouragement along with tidbits of wisdom, tricks of the trade and the skills that come from a half-life of doing mathematics.  And  ``just sailing'' is fun, but a lecture on how to sail is not, at least not until you have done enough of it to absorb information from the lecture.

When teaching calculus, no two teachers will share exactly the same goals.  More than anything else, I want to excite them about mathematics by enabling them to succeed at doing mathematics and explaining to others what they have done.  In addition to this primary goal, I want to train the students to read mathematics carefully and critically.   I want to teach them to work on problems on their own rather than seeking out other materials and I emphasize this, even as I allow them to go to other sources because of the size of the classes.  I want them to see that there are really only a few basic concepts in calculus, the limiting processes that recur in continuity, differentiability, integration, arc-length, and Gauss' theorem.  I want them to note that calculus in multiple dimensions really is a parallel to calculus in one dimension.  For these reasons, I try to write tersely without trying to give lengthy ``intuitive'' introductions, and I write without graphics.  I believe it is better to define annulus carefully and make the students interpret the English than to draw a picture and say ``This is an annulus.''  The latter is more efficient but does not train critical reading skills.  I want to see them communicate clearly to one another in front of the class, both in their questions and in their responses.  I want them to write correct mathematics on the board.  The amount of pedagogical trickery that I use to support these goals is too long to list here, but it all rests firmly on always emphasizing that they will succeed and responding to everything, even negative comments, in a positive light.  If a student raises a concern about the teaching method, I explain how much more work it is to me and how much easier it would be to lecture and give monkey-see, monkey-do homework and that I sacrifice my own time and energy for their benefit.  If a student makes mistakes at the board, then as I correct it, I always say something like, ``These corrections don't count against a presentation grade because this was a good effort, yet we want to make sure that we see what types of mistakes I might count off for on an exam.''

\section*{Grading}
\addcontentsline{toc}{section}{Grading}

Over the years I have used many grading systems for the course.  The simplest was:
\begin{enumerate}
\item 50\% -- average of three tests and a comprehensive final
\item 50\% -- presentation grade
\end{enumerate}
The most recent was:
\begin{enumerate}
\item 40\% -- average of three tests and a comprehensive final
\item 50\% -- presentation grade
\item 10\% -- weekly graded homework
\end{enumerate}

The weekly graded homework was to be one problem of their choice to be written up perfectly with every step explained.   Early in the semester, they could use problems that had been presented.  Later I required that the turned-in material be problems that had not been presented.\\ \\
Another was:
\begin{enumerate}
\item 40\% -- average of 2 tests and a comprehensive final
\item 45\% -- presentation grade
\item 15\% -- weekly quizzes
\end{enumerate}

Regardless of what grading scheme I used, I always emphasized after each test that I reserved the right to give a better grade than the average based on consistently improving performance, an impressive comprehensive final, or significant contributions to the class.  Contributions to the class could entail consistently good questions about problems at the board, problems worked in unusual or original ways, or organizing help sessions for students.  Once a student filmed my lectures along with student presentations, and placed them on our Facebook group to help the class.

\section*{Some Guidance}
\addcontentsline{toc}{section}{Some Guidance}

Years ago I read a paper, which I can no longer locate, claiming that most students judge the teacher within the first fifteen minutes of the first day of class.  This supports my long-held belief that the first day may well be the most important of the semester, especially in a student-centered course.  The main goal is to place the class at ease and send a few students to the board in a relaxed environment.  For that reason, we don't start with my syllabus, grading, or other mundane details.  After all, this class isn't about a syllabus or grades, it's about mathematics.  Thus, after assuring that students are in the right room and telling the students where they can download the course notes and the syllabus, I learn a few names and ask a few casual questions about how many have had calculus and how many are first-semester students. Typically I give a twelve-minute, elementary quiz over college algebra (no trigonometry, no pre-calculus) that I tell them won't count for or against them.   This is graded and returned the next day.  I tell them that 100\% of the people who earned less than 50\% on this quiz last semester and stayed in the course failed the course.   This seems harsh, but it enables me to help put students who can't graph a line or factor a quadratic into the right course before it is too late.  While they are taking the quiz, I sketch the graph of a few functions, the graph of a non-function, and then write the equation for a few functions on the board.  Sometimes I put questions beneath the graphs or instructions such as ``Is this a function?'' or ``Graph this.''  All of these are very elementary and intended to get the students talking.  After the quiz is collected, I'll ask a student if she solved one of the problems on the quiz or on the board.  If she says yes, I ask her to write the solution on the board.  I ask someone else if he can graph one of the equations.  And in this relaxed way, perhaps four to eight students present at the board on the first day.  Then we discuss their solutions.  When reviewing these solutions, I compliment every problem (even if only the penmanship) and simultaneously feed them definitions and examples for function, relation, open interval and closed interval.  At the end of class, I pass out the first few pages of the notes along with a link to where they may find the syllabus and notes.   And I tell them their homework is to read the introduction and try to work as many of Problems 1-10 as they can, as I will ask for volunteers at the beginning of the next class meeting.

The next day, I ask for volunteers to present and by end of the first week, someone will usually ask if we just present problems every day and discuss them.  I'll ask if they like it and if they have learned anything from it so far.  When they say yes, I'll say, ``Well, then that seems a good enough justification to continue in this manner.''  I don't know what I'll say if they ever say no.  At some point I'll talk about grading and the syllabus, but hopefully even this will be in response to a question as I really don't like to talk about anything they are not interested in.   I really want the majority of the first few days concentrating on three things: putting my students at ease, creating an enjoyable environment, and focusing on mathematics. For the next week or two, I'll likely never lecture except in response to questions or problems that were presented.

You are reading the \emph{Instructor's Version} of these notes, so you will find endnotes \endnote{This is an example of an endnote.} where I have recorded information about the problems, guidance for you, and discussions about the mini-lectures that I gave during the most recent iteration.  Such discussions vary based on student questions, but these are representative of what I cover and include both the discussion and the examples that I use to introduce topics.   There is no way to include all the discussions that result from student questions, as this is where the majority of class interaction occurs.  What I've offered here are only the major presentations that I gave during the past iteration.  Additionally, I have been unable to refrain from shedding light on my own motivation and perspective in creating the course in this fashion, so you'll see discussions on how I attempt to attract majors, to motivate students to take other courses, to connect the material to other courses, and even to tie the course to my own experiences in industry.

While the mini-lectures, endnotes and examples appear in the notes at the approximate places where I presented them this semester, that may not be the optimal timing for presentation of this material in a different semester. Such lectures are presented in response to questions from students, at a time when students are stalled, or just-in-time to prepare students for success on upcoming problems.  The number and length of these lectures depend on the strength and progress of the class.  If no student has a problem to present or very few have a problem, then I will present.  If we are about to embark on another section that I feel needs an introduction, then I will present.  I have taught such strong students that I almost never introduced upcoming material and only offered additional information in response to students' questions. I have also taught classes where I carefully timed when to introduce new material in order to assure sufficient progress.

The reader will note that the lectures are shorter and less frequent at the beginning of the course than at the end.  There are two reasons for this.  First, I want the students to buy in to taking responsibility for presenting the material at the beginning.   If I present very much, they will seek to maximize my lectures in order to minimize their work and optimize their grades. Second, the material steadily progresses in level of difficulty and I doubt anyone's ability to create notes that enable students to ``discover'' the concepts of Lagrange Multipliers, Green's Theorem, Gauss' Theorem, and Stoke's Theorem in the time allotted.  By the third semester, you'll note that there are no sub-sections and the material for each chapter is co-mingled in one large problem set. This is intentional.   The level of difficulty of the material and the burden on the student to learn independently increases through each of the three semesters.

In the first semester, I will typically cover all of Chapters 1 - 5.1.  In the second semester, I'll repeat Section 5.1 and cover Chapters 5 - 7.3, although I often assign 5.6 as homework to be done out of class and not for presentation.  The third semester will consist of all of Chapters 8 - 13.  If $N$ is an integer satisfying $1 \leq N \leq 13$, then Chapter \ref{practiceproblems} Section $N$ contains problems with solutions for the students to work as practice for Chapter $N$.  I usually tell them when I think they are ready to work on these practice problems, and I'll work any of them at the beginning of class when a student asks.  The answers are given, so if they don't get the same answer, they are expected to ask.

The vast majority of class time goes to students putting up problems.  As discussed, on the first day I always pose problems at the board and students present them.  From that day forward, I begin each class by first asking if there are questions on anything that has already been presented or any of the practice problems.  It not, I call out problem numbers and choose from the class the students with the least number of presentations.  I break ties by test grades or by my estimate of the tied students' abilities, allowing apparently weaker students preference.  For the first few class periods, I give students who have not taken a course from me preference over those who have.  I allow multiple problems to go on the board at once, between five and fifteen on any given day by fully utilizing large white boards at the front and rear of the classroom.  As problems are being written on the board, I circulate and answer questions at the desks.  Classes average around fifteen problems per week, but that is a bit misleading as many problems have multiple parts and others lead to discussions and additional examples.  Once all students have completed writing their solutions, I encourage them to explain their solutions to the class.  If they are uncomfortable doing so, I will go over the problem, asking them questions about their work. Most problems in the notes are there to illustrate an important point, so after each problem, I may spend a few minutes discussing the purpose of this particular problem and foreshadowing important concepts with additional examples, pictures, or ideas.  If time is short after the problems are all written on the board (perhaps I talked a bit too much or perhaps I underestimated how long it would take to put these problems up), I may go over the problems myself.  This is a very difficult judgement to make because there are times when I struggle to make sense of the student's work and s/he will sit me down and explain his/her own solutions more eloquently than I! On the other hand, there are also times when a student's explanation is so lacking (or approach so obscure) that I feel the need to work a clarifying example in order to bring the class along and ensure future success.  There is no guarantee that I will make the right call!

Another difficult judgement call that I make regularly is how much to foreshadow via mini-lectures and when to give them.  If no student has a presentation, then clearly it is time for me to present. It is also common that only the best students may have something to present.  If after a quick survey, only a few top students have problems, then I may choose to lecture rather than let the class turn into a lecture class where the lectures are given by only a few students.  This is a delicate decision.  On the one hand, other students with less presentation will have these problems the next day, so these talented students have been ``cheated'' out of a presentation.  On the other hand, I record in my grade book that they had these problems and am typically already recruiting such students as majors so they know I think they are doing good work in the class and that they have not lost any points.  On occasion, I tell the class that I'm not sure if they need a lecture or if student presentations are the best choice and let them vote. Surprisingly, the vote may go either way, but is almost always near unanimous for one of the two choices: mini-lecture or presentations.  The students seem to know exactly what they need on a given day, even when I do not!

A slightly challenging symptom of having some students who have had calculus and of allowing students to look at other resources is that sometimes students will use tools that we have not yet developed.  For example they might apply the power rule or chain rule before we have derived them.  When this happens, I first thank the student for showing us a quick and correct solution using a formula and note that this receives full credit.  Then I use this as an opportunity to reinforce the axiomatic nature of mathematics and point out that we have not yet developed the tools that were used.   I might provide guidance and ask the student to solve it using the techniques we have, for example the definition of the derivative vs. the power rule.  Or I might give an on-the-spot mini-lecture where I rework the problem from first principles and state that I'm glad we now know these formulas, but we must derive them before we can trust them. I'll tell them that, were we simply engineers, we would accept anything that was fed to us, but as mathematicians we must validate such formulae before we use them. I do this somewhat tongue-in-cheek, but they  understand that I am coming at this with the perspective of a mathematician who  wants to deeply understand all that I use. I'll also tell them which problems are aimed at deriving the theorems and give guidance on how we might do so.  In an optimal  situation, the student who looked up the formula will dig in and derive the formula.  If not, another student will. I'll always praise both students because  now we \emph{have} the formula and we \emph{know} that it is valid.   Turning such potentially negative situations into positives is one of the keys to creating a class that the students respect and enjoy.  They must \emph{know} that whatever they do at the board, I will respond to it positively in an effort to teach them more and will give them some credit every time they present.

\section*{History of the Course}
\addcontentsline{toc}{section}{History of the Course}

The problems from these notes come from several sources.  Professors Carol Browning of Drury University, Charles Allen of Drury University, Dale Daniel of Lamar University and Shing So of the University of Central Missouri all contributed material in Chapters 1 - 7.   Material also comes from the multitude of calculus books I have taught from over the years, a list too extensive and outdated to include here.  All of these materials have been massaged and modified extensively as I taught this course over the past fifteen years.  They still won't meet your needs exactly, so I encourage you to take these notes and modify them to meet the needs of your students.  Then these notes will be titled \emph{Your Name $<$ My Name $<$ Other Names}.  Hopefully the authorship will grow as others adapt these notes to their students' needs.

Each of \emph{Calculus I, II,} and \emph{III} has been taught at Lamar University six or seven times over a period of twelve or thirteen years with an average class size of perhaps thirty-five students who meet for four fifty-minute periods each week for fifteen weeks, although recently the department has moved to five fifty-five minute periods over fourteen weeks for each of \emph{Calculus I} and \emph{II}. The large class size and full syllabus keep me from running a pure Moore Method approach where students present problems one at a time and work through a fully self-contained set of notes.  Therefore, to maximize the odds for success and to significantly increase the rate of coverage, I make three concessions that cause me to label this course a problem-based course or active-learning course rather than a Moore Method course.
\begin{enumerate}
\item I allow students to use other resources (web, books, tutoring lab) to help them understand the concepts.
\item I provide practice problems with answers in Chapter \ref{practiceproblems} and students are expected to work these after we have covered a given topic.
\item If needed to assure coverage, I supplement the notes with mini-lectures, on average once a week, typically about twenty-five minutes in length.
\end{enumerate}
I have conducted the class without allowing students to look at any other sources. While the top performing
students do no worse, too many students struggled.  I believe that this could easily be overcome if the class size were smaller, the syllabus not quite so full, or with an extra meeting per week to help weaker students.  When classes met four days a week, I would often offer a problem session on the fifth day and we would address problems from the practice problems.

\section*{Student Feedback}
\addcontentsline{toc}{section}{Student Feedback}

A recent survey found that 90\% of drivers believe that they are better-than-average drivers.  While this is not mathematically impossible, it seems improbable.  My guess is that 90\% of faculty believe they are above average teachers as well.  I am not a great teacher.  I am a teacher who has found a way to create an environment in which students become great students and great teachers of one another.  Over the years I have been exposed to countless teachers who simply amazed me with their dedication to their students, their knowledge of their subject and their ability to connect with students.  On the other hand, even though it is the students who carry my classes, they seem to offer me some of the credit and I'll share with you every student comment from the last semester I taught \emph{Calculus I} so that you might think about whether you would like your students to say such things about your course.  To assure you that I did not hand pick a semester (or make all these up!), I offer to have our departmental administrative assistant, who types all these up from the handwritten forms, send you all comments from any semester of any of my \emph{Calculus I, II} or \emph{III} courses.  I've given these evaluations, and modified my courses based on them, for more than twelve years.  I take no credit for these comments -- all credit is due to my students, my father, my advisor and the great teachers I had over the years from whom I have borrowed or taken ideas.  Here is what the students said, exactly as they wrote it, and exactly as my secretary typed them.

\begin{enumerate}
\item I believe the most effective part of this course was:
\begin{enumerate}
\item learning my mistakes at the board.  However, I would have had less mistakes if the professor spent more time explaining problems before I had to do them.
\item the presentation and student driven part of the course.  It has really helped.
\item The use of boardwork and homework in learning
\item everything!  He was a very fair and realistic professor.
\item You did a brilliant job explaining limits.  After 4 calculus teachers, you are the only one to adequately explain them,
\item putting homework problems on the board
\item interaction with the students, professors and graduate students.  You felt the bond and understanding when you had a problem.
\item the highly interactive socratic method used in the presentations and lectures
\item the ability to be able to go to the board and get hands on learning
\item I don't know
\item that he had everyone go up to the board to demonstrate whether or not they understood what they did
\item Proving almost every formula before actually applying them.
\item Doing problems on the board.
\item the most effective part was derivatives
\item the presentations
\item having to figure everything out on our own but still being able to ask questions and have others work them out if needed.
\item the days you lectured on a difficult concept
\item the combination of lecture and practice of the concepts in class
\item learning how to solve problems different ways
\item presentations on the board.  Really helped me to understand the problems
\item having to figure out the homework for ourselves
\item lecture days
\item giving the freedom he does to students.  I've never been in a class like this, and even though I have a very hard time presentation in front of people, let alone presenting on a subject I feel I am not good at, I feel it was incredibly effective to be allowed to teach to the class.
\item the way the homework is set up so that the class makes the pace
\end{enumerate}
\item To improve this course in future semesters, the professor should:
\begin{enumerate}
\item spend more time explaining concepts, and especially creating formulas for problems like related rates.
\item maybe lecture a little more.  I actually liked his industry based examples.
\item not change a thing
\item I wouldn't change a thing
\item Keep the students informed of what problems need to be completed.  Also, if students are getting to study, make an appearance to help answer questions
\item keep it the same
\item make lectures before going into a new subject.  It's a little frightening jumping into a material that you had no idea of.
\item keep improving in the direction it seems he's going in his teaching method
\item try to focus a couple more times a week doing a lecture and introducing new material
\item have the students explain their problems more
\item randomly call students to go up to the board that way everyone remains on edge
\item lecture more
\item do more practice problems on the board.  Have a day that you ask on how to do homework problems
\item I couldn't even think of a way because he's already perfect
\item maybe lecture a little more
\item maybe spend a little more time lecturing when people don't bring in problems for the board
\item possibly designate a day (like Mondays, after a weekend full of forgetting) to lecture on what's going on, if only for the start of class.
\item try to better balance the lecture and presentation
\item it's fine just like it is
\item not very much.  Maybe more quizzes?
\item make more detailed notes
\item spend more days lecturing, but I like the board problems too
\item do nothing.  Ted was/is (also Wes and Jeff) awesome.  One of my favorite professors.
\item stay the same
\end{enumerate}
\item Talking to another student thinking about taking this instructor's class, I would say:
\begin{enumerate}
\item take it if you want a challenge, or if you want to really learn the material
\item I would recommend it because I like the teaching method.  It is an effective way to learn
\item Definitely take it if at all possible
\item I recommend Mr. Mahavier for Calculus!  Specifically because I've taken calculus the previous semester and I did not understand a thing.  In Mr. Mahavier's class I've gained my love back for math, whether I am right or wrong.
\item This class is wonderful if you want to learn the material well!
\item He has good teaching skills
\item you would have to stay on top of your game.  It's all about teaching yourself as Ted is just there to guide you through or when you're confused.  Good luck!
\item Take it if you believe an interactive learning environment would be conducive to your education.  Take if you have interest in a degree in mathematics.
\item that having a self paced class is the most beneficial thing in a math course
\item that the teaching style is rather different and difficult to get used to, but effective.  I would tell students that learn by listening not to take this class, but students who learn by doing to definitely take it.
\item It is a very fun class that helps you understand just how much you think you know.
\item stay on top of your work.  Don't let yourself lag too far behind if possible, always try to stay ahead
\item Yes you should take Ted, he is a great teacher and you learn a lot by doing board work
\item He's the most cool teacher
\item Most definitely, take this course
\item definitely take him, but be prepared to work.
\item take this class to learn calculus, but only if you're willing to work
\item that you will learn the concepts but you have to practice after class
\item this class if very interesting and if you give your absolutely best you should be fine
\item take it!
\item the class was eye opening
\item do it!  I have told people to take Ted
\item very exposing in a good way, humbling
\item take it
\end{enumerate}
\item I would like the professor to know that:
\begin{enumerate}
\item He is awesome
\item before this class I have always struggled with math.  I feel like I might be more interested in taking higher math now instead of being afraid of it
\item He's the best professor I've had
\item You are a wonderful professor
\item I love that most of the grade is in class work.  Bad test grades freak me out when I know I've studied in class
\item I would hope to have him in the future
\item He is a great teacher, easy to communicate with and very friendly.  I'll be glad to have him again if these math courses were in my major!
\item His method has worked well for me
\item I definitely learned calculus this semester
\item He's great
\item It was definitely an interesting course
\item I will be taking him for Cal 2 and likely Cal 3
\item I am very thankful that he was my calculus teacher.  I had never had a math teacher that made me understand the material that well.  Thank you.
\item He really did help me and worked with me during the semester and I appreciate it.
\item He is an amazing professor
\item I actually enjoyed this class
\item this class has uncovered an unknown skill and passion of mine, mathematics, that I want to pursue as a definite career
\item I enjoyed the class
\item That I've tried and gave my all and if I fall short, I still appreciate all the methods and problems I did learn
\item I really enjoyed the class, the method of teaching works extremely well for me.
\item yes
\item I'll see him next semester
\item He is phenomenal.  He's mathematical!
\item He was very inspirational and kept me from giving up.
\end{enumerate}
\end{enumerate}
\end{annotation}

\chapter{To the Student}

``The men who try to do something and fail are infinitely better than those who try to do nothing and succeed.''
  -  Martin Lloyd Jones

\section*{How this Class Works}
\addcontentsline{toc}{section}{How this Class Works}

This class will be taught in a way that is likely to be quite different from mathematics classes you have encountered in the past.  Much of the class will be devoted to students working problems at the board and much of your grade will be determined by the amount of mathematics that you produce in this class.  I use the word produce because it is my belief that the best way to learn mathematics is by doing mathematics.

Therefore, just as I learned to ride a bike by getting on and falling off, I expect that you will learn mathematics by attempting it and occasionally falling off!  You will have a set of notes, provided by me, that you will turn into a book by working through the problems.  If you are interested in watching someone else put mathematics at the board, working ten problems like it for homework, and then regurgitating this material on tests, then you are not in the correct class.  Still, I urge you to seriously consider the value of becoming an independent thinker who tackles doing mathematics, and everything else in life, on your own rather than waiting for someone else to show you how to do things.

\section*{A Common Pitfall}
\addcontentsline{toc}{section}{A Common Pitfall}

There are two ways in which students often approach my classes.  The first is to say, ``I'll wait and see how this works and then see if I like it and put some problems up later in the semester after I catch on.''   Think of the course as a forty-yard dash.  Do you really want to wait and see how fast the other runners are? If you try every night to do the problems then either you will get a problem (Yay!) and be able to put it on the board with pride and satisfaction or you will struggle with the problem, learn a lot in your struggle, and then watch someone else put it on the board. When this person puts it up you will be able to ask questions and help yourself and others understand it, as you say to yourself, ``Ahhhh, now I see where I went wrong and now I can do this one and a few more for next class.''  If you do not try problems each night, then you will watch the student put the problem on the board, but perhaps will not quite catch all the details and then when you study for the tests or try the next problems you will have only a loose idea of how to tackle such problems.  Basically, you have seen it only once in this case.  The first student saw it once when s/he tackled it on her/his own, again when either s/he put it on the board or another student presented it, and then a third time when s/he studies for the next test or quiz. Hence the difference between these two approaches is the difference between participating and watching a movie. Movies are filled with successfully married couples and yet something like 60\% of marriages end in divorce.  Watching successful marriages doesn't teach one to be successfully married any more than watching mathematics go by will teach you to do mathematics.   I hope that each of you will tackle this course with an attitude that you will learn this material and thus will both enjoy and benefit from the class.

\section*{Board Work}
\addcontentsline{toc}{section}{Boardwork}

Because the board work constitutes a reasonable amount of your grade, let's put your mind at ease regarding this part of the class.  First, by coming to class everyday you have a 60\%  on board work.  Every problem you present pushes that grade a little higher.  You may come see me any time for an indication of what I think your current level of participation will earn you by the end of the semester for this portion of the grade.  Here are some rules and  guidelines associated with the board work.  I will call for volunteers every day and will pick the person with the least presentations to present a given problem.  You may inform me that you have a problem in advance (which I appreciate), but the problem still goes to the person with the least presentations on the day I call for a solution.  Ties are broken randomly at the beginning of the course.  Once the first test has been returned, ties are broken by giving precedence to the student with the lower test average.  A student who has not gone to the board on a given day will be given precedence over a student who has gone to the board that day.  To ``present'' a problem at the board means to have written the problem statement up, to have written a correct solution using complete mathematical sentences, and to have answered all students' questions regarding the problem.  Since you will be communicating with other students on a regular basis, here are several guidelines that will help you.

Most importantly, remember that the whole class is on your side and wants to see you succeed, so questions are intended to help everyone, not to criticize you.

When you speak, don't use the words ``obvious,'' ``stupid,'' or ``trivial.''  Don't attack anyone personally or try to intimidate anyone.  Don't get mad or upset at anyone.  If you do, try to get over it quickly.  Don't be upset when you make a mistake -- brush it off and learn from it.  Don't let anything go on the board that you don't fully understand. Don't say to yourself, ``I'll figure this out at home.''  Don't use concepts we have not defined. Don't use or get examples or solutions from other books.  Don't work together without acknowledging it at the board.  Don't try to put up a problem you have not written up.

Do prepare arguments in advance.  Do be polite and respectful.  Do learn from your mistakes. Do ask questions such as, ``Can you tell me how you got the third line?''  Do let people answer when they are asked a question.  Do refer to earlier results and definitions by number when possible.

\section*{How to Study each Day}
\addcontentsline{toc}{section}{How to Study each Day}

\begin{enumerate}
\item Read over your notes from class that day.
\item Make a list of questions to ask me at the beginning of the next class. (I love these!)
\item Review the recent problems.
\item Work on \emph{several} new problems.
\item Write up as many solutions as you can so that you can copy these onto the board the next day.
\end{enumerate}

\section*{What is Calculus?}
\addcontentsline{toc}{section}{What is Calculus?}

The first semester of Calculus consists of four main concepts: {\it limits, continuity, differentiation} and {\it integration}. Limits are required for defining each of continuity, differentiation, and integration.
All four concepts are central to an understanding of applications in fields including biology, business, chemistry, economics, engineering, finance, and physics. The second semester extends your study of \emph{integration techniques} and adds \emph{sequences} and \emph{series}. The third semester of calculus is a repeat of these concepts in higher dimensions with a few other topics tossed in.   All of these topics depend on a deep understanding of what might be called \emph{limiting} processes, which is why limits are one of the topics that we will spend a lot of time with in the beginning and will recur throughout the entire course.

In addition to mastering these concepts, I hope to impart in you the essence of the way a mathematician thinks of the world, an axiomatic way of viewing the world. And I hope to help you master the important skill of solving some difficult problems on your own, communicating these solutions to your peers, posing questions respectfully, and responding to questions your peers have.   These skills transcend calculus and will help you in all aspects of your life.


\mainmatter

\chapter{Functions, Limits, Velocity, and Tangents}

``The harder I work, the luckier I get.'' - Samuel Goldwyn

\section{Average Speed and Velocity}

The \textbf{average speed} of an object that has traveled a distance \textbf{d} in time \textbf{t} is \textbf{v = d/t}.  The terms velocity and speed are interchangeable except that velocity may be negative to indicate direction. For example, if a rock is thrown in the air, we would typically assign a positive velocity on the way up and a negative velocity on the way down.

\begin{prb}
\label{l2} A car goes from Houston, Texas to Saint Louis, Missouri (a distance of 600 miles). The trip takes 9 hours and 28 minutes. What was the average speed of the trip?  What is one true statement that can you make about the time(s) (if any) at which the car moved at this speed?
\end{prb}

\begin{prb}
\label{l3} Robbie the robot is walking down a road in a straight line, away from a pothole. Table \ref{t1} gives the distance between Robbie and the pothole at different times, where the time is measured in seconds and distance is measured in feet. Write a formula that expresses Robbie's distance from the pothole as a function of time.  Use $R$ to represent distance and $t$ to represent time in your formula.  This function will be called Robbie's position function.
\end{prb}

\begin{table}[h!]
\caption{Position Table} \label{t1}
\begin{center}
\begin{tabular}{ |c|c| }
\hline
 \multicolumn{1}{|c|}{Time (sec)} &\multicolumn{1}{c|}{Position (feet)}\\
\hline \hline
$0$ & $2$  \\
\hline
$1$ & $3$  \\
\hline
$2$ & $10$  \\
\hline
$3$ & $29$  \\
\hline
$4$ & $66$  \\
\hline
$5$ & $127$  \\
\hline
\end{tabular}
\end{center}
\end{table}

\begin{dfn}
A \textbf{relation} is a collection of points in the plane.
\end{dfn}

\begin{dfn}
A \textbf{function} is a collection of points in the plane with the property that no two points have the same first coordinates. The set of all first coordinates of these points is referred to as the \textbf{domain} of the function and the set of all second coordinates of these points is referred to as the \textbf{range} of the function.
\begin{annotation}
\endnote{Almost always during the first ten problems, there will be questions or problems that provide natural opportunities to discuss just-in-time basic concepts including (1) the definition of a function including domain and range (2) the distinction between a function (a collection of points), the graph of the function, and an equation that defines a function (3) useful graphing techniques such as translations, reflections, and symmetry.}
\end{annotation}
\end{dfn}


\begin{expl} Functions may be expressed in many different forms. \end{expl}
The function

$$f = \{ (x,y) : y = 6x + 2 \} =
\{ \dots , (-1,-4), (0, 2), (1,8), (\pi, 6\pi+2), \dots  \}$$

might be expressed via the short-hand,

$$ y = 6x + 2  \; \; \mbox{  or  }  \; \; f(x) = 6x +2 \; \;  \mbox{   or   }  \; \;  f(z) = 6z + 2.$$

The function is not the written equation, rather it is the set of points in the plane that satisfy the equation. While every function is a relation, every relation is not necessarily a function.  Examples of relations that are not functions would be the vertical line,
$$ \{ (x,y) : x = 6 \}\; \; \mbox{written in short-hand as} \; \; x=6,$$
and the circle,
$$ \{ (x,y) : x^2 + y^2 = 9 \} \; \; \mbox{written in short-hand as} \; \;x^2 + y^2 =9.$$


\begin{prb}
Express Robbie's position function in the three different forms just discussed. Graph Robbie's position function, labeling the axes and several points.
\end{prb}

\textbf{Interval and Set Notation.}  We use $\re$  to denote the set of all real numbers and $ x \in S$ to mean that $x$ is a member of the set $S.$  If $a \in \re$ and $b \in \re$ and $a < b$ then the \textbf{open interval} $(a,b)$ denotes the set of all numbers between $a$ and $b.$
For example,
\begin{enumerate}
\item $(a,b) = \{ x \in \re : a < x < b \}$ -- an open interval
\item $[a,b] = \{ x \in \re : a \leq x \leq b \}$ -- a closed interval
\item $(a,b] = \{ x \in \re : a < x \leq b \}$ -- a half-open interval
\end{enumerate}

When convenient, we'll use the notation $(-\infty, b)$ to mean all numbers less than $b,$ even though $-\infty$ is not a real number.  We won't define $\infty$ or $-\infty$ in this class, although you can read a bit about them in Appendix \ref{appreal}.

\begin{prb}
Sketch the graph of each of the following and state the domain and range of each.
\begin{enumerate}
\item $\dsp f(x) = \frac{1}{x-1}$
\item $g(x) = \sqrt{4-x}$
\item $x = y^2 + 1$
\item $i(x) = -(x-2)^3 +4$
\end{enumerate}
\end{prb}

The next several problems deal with Robbie's motion where $R(t)$ represents Robbie's distance from the pothole in feet at time $t$ in seconds.

\begin{prb}
Assume c is a positive number and compute $R(3)$ and $R(c).$ Write a sentence that says what $R(3)$ means.  What does $R(c)$ mean?
\end{prb}

\begin{prb}
\label{l9} Compute the value of $R(5) - R(3).$ What is the physical (real world, useful) meaning of this number? Assume u and v are positive numbers and compute $R(u) - R(v).$ What is the physical meaning of
$R(u)-R(v)?$
\end{prb}

\begin{prb}
Evaluate each of the following ratios and explain in detail their physical meaning.  Assume that $u$ and $v$ are positive numbers and $u > v.$
\begin{enumerate}
\item $\displaystyle{ \frac{R(5)-R(3)}{5-3}}$
\item $\displaystyle{ \frac{R(u)-R(v)}{u-v}}$
\end{enumerate}
\end{prb}

\begin{prb}
Assuming that the function you found is valid for all times $t > 4,$ where is Robbie when the time is $t=5$ seconds? $t=6$ seconds? $t=5.5$ seconds? When is Robbie $106$ feet from the pothole?
\end{prb}

\begin{prb}
Recall from Problem \ref{l2} that the distance from Houston, Texas to Saint Louis, Missouri is 600 miles.  A car goes from Houston to Saint Louis traveling at 65 mph for the first 22 minutes, accelerates to 95 mph, and remains at that speed for the rest of the trip. Estimate the average speed of the trip.  Explain whether your answer is exact or an approximation. \begin{annotation}
\endnote{This problem is intentionally ill-posed and provides a good opportunity for me to explain to you how I ``lecture'' reactively, in a way that may seem arbitrary to the students but is actually carefully orchestrated to foreshadow concepts and material.  Students typically work this problem under the assumption that the car accelerated from 65 to 95 in 0 seconds.  After presentation and questions, if this has not been noted by the presenter or the class I'll ask, ``At what time did the car travel at that average speed?''  This foreshadows the Mean Value Theorem and allows me to use Socratic-style questioning to encourage discovery of the ill-posed nature of the problem.  After a string of questions, I'll praise the students who realize the impossibility of such a scenario.  To further illustrate the concept I will graph the velocity function and ask if we could add assumptions to make this a more realistic problem.  I'll leave the class with this open-ended challenge, labeling it as a starred problem and saying I'd be happy to see any progress.  One semester a student worked on this problem for months, first assuming constant acceleration over a one-second period and then realizing that this makes the acceleration discontinuous.  He then created a realistic acceleration curve and presented his work at the Sectional MAA conference.  Such open-ended problems are great for the inquiring minds and model the problems of industry where we take a rough cut at the problem first and then try to model a more realistic problem.}
\end{annotation}
\end{prb}


\section{Limits and Continuity}

\begin{dfn}
If f is a function and each of $a$ and $L$ are numbers, then we say that the \textbf{limit of $f$ as $x$ approaches $a$ is $L$} if the values for $f(x)$ get close to the number $L$ as the values for $x$ get close to the number $a.$ We write this as
$$\lim_{x \rightarrow a} f(x) = L \; \; \; \mbox{ or } \; \; \; f(x) \rightarrow L \mbox{ as } x \rightarrow a.$$
\end{dfn}

\begin{expl}
What number is the function $\dsp f(x) = \frac{x^5-32}{x-2},$ approaching as $x$ approaches $2$?
\end{expl}

\begin{table}[h!]
\caption{First Limit Table} \label{t2}
\begin{center}
\begin{tabular}{ |c|c| }
\hline
 \multicolumn{1}{|c|}{x} &\multicolumn{1}{c|}{f(x)}\\
\hline \hline
$1.9$ & $72.39$  \\
\hline
$1.99$ & $79.20$  \\
\hline
$1.999$ & $79.92$  \\
\hline
$1.9999$ & $79.992$  \\
\hline \hline
$2.1$ & $88.41$  \\
\hline
$2.01$ & $80.80$  \\
\hline
$2.001$ & $80.08$  \\
\hline
$2.0001$ & $80.008$  \\
\hline
\end{tabular}
\end{center}
\end{table}

Table \ref{t2} shows that as $x$ approaches $2$ from either the left or the right, $f(x)$ approaches $80.$  From this table, we assume that the limit as $x$ approaches $2$ of the function $\displaystyle{f(x) = \frac{x^5-32}{x-2}}$ is $80.$  Our notation for this will be: $$\lim_{x\rightarrow 2} \frac{x^5-32}{x-2} = 80.$$

This is an intuitive approach to the notion of a limit because we did not give specific meaning to the phrases ``x approaches 2'' or ``f(x) approaches 80.'' To give a mathematically accurate definition for the word ``limit'' took mathematicians almost 200 years, yet our intuitive approach will serve us well for this course. If you are interested in a more precise definition of the limit, read Appendix \ref{applim} or take \emph{Real Analysis}.

\begin{prb}
Compute $\dsp \lim_{x\rightarrow 2} \frac{x^3-8}{x-2}$ by constructing a limit table like Table \ref{t2}.  What is $\dsp \lim_{x\rightarrow -2} \frac{x^3-8}{x-2}$?
\end{prb}


\begin{prb}
Arguments for trigonometric functions in calculus are in radians unless stated otherwise, so be sure your calculator is in radian mode. Compute $\dsp \lim_{x\rightarrow 0} \frac{\sin(x)}{2x}$ by constructing a limit table.
\end{prb}

\begin{expl}
Computing limits using algebra.
\end{expl}

A reasonable attempt to find the limit of the function $f$ as $x$ approaches $a$ would be to simply plug $a$ in for $x$ in the function. As illustrated in the last two problems, \emph{this does not always work.} There is an important distinction between the value $f(a)$ and the limit of $f$ as $x$ approaches $a.$  In addition to tables and graphing, we may also use algebra to evaluate some limits.

$$\lim_{x \rightarrow 3} \frac{x^2 - 9}{x - 3} = \lim_{x \rightarrow 3} \frac{(x-3)(x+3)}{x - 3} = \lim_{x \rightarrow 3} \left( x+3 \right) = 6$$

\begin{prb}
Graph the two functions $f(x) = x+3$ and $\dsp g(x) =\frac{x^2-9}{x-3},$ and list their domains. Are $f$ and $g$ the same function?
\end{prb}

\begin{prb}
Compute the following limit: $\dsp \lim_{x \rightarrow 2} \frac{x^2 -4}{x - 2}.$
\end{prb}

\begin{prb}
Compute these limits and make a note that you will need them in a later problem.
\begin{enumerate}
\item $\dsp \lim_{t\rightarrow 0}\frac{\sin(t)}{t}.$
\item $\dsp \lim_{t \rightarrow 0} \frac{1 - \cos(t)}{t}$
\end{enumerate}
\end{prb}

\begin{prb}
Let $x$ be a real number and compute  $\dsp \lim_{h \rightarrow 0} \frac{hx^2 + h}{h}$. Your answer will be a number written in terms of the number $x.$
\begin{annotation}
\endnote{This problem demonstrates how I gently try to introduce them to more abstract concepts, even ones as simple as ``$h$ represents some number, but we don't know which number $h$ is.''  When a student reading the problem indicates a lack of understanding, I simply say, ``Pick a value for $h$ and work it with that value; then pick another.'' If the problem is presented by someone who uses a specific value, I'll choose another value and then show the general case.}
\end{annotation}
\end{prb}

\begin{prb}
Compute the following limits.
\begin{enumerate}
\item $\dsp \lim_{x \rightarrow 3} \frac{x^3-27}{x-3}$
\item $\dsp \lim_{x \rightarrow 1} \frac{\frac{1}{x} - 1}{x^2-1}$
\item $\dsp \lim_{x \rightarrow -2} \frac{x^3+3x^2+3x+2}{x^2+x-2}$
\end{enumerate}
\end{prb}

There are rules (theorems) that will facilitate the computation of limits. The next problem will help you find these basic rules.


\begin{prb}
Read Problem \ref{l30} and then compute the following limits.
\begin{enumerate}
\item $\dsp \lim_{x \rightarrow 3} 5$

\item $\dsp \lim_{x \rightarrow 3} x$

\item $\dsp \lim_{x \rightarrow \frac{\pi}{2}} 6\sin(x)$ and $\dsp 6\lim_{x \rightarrow \frac{\pi}{2}} \sin(x)$

\item $\dsp (\lim_{x \rightarrow 4} x^2 + \lim_{x \rightarrow 4} 4x^5)$ and $\dsp \lim_{x \rightarrow 4} (x^2 + 4x^5)$

\item $\dsp \left( \lim_{x \rightarrow -3} x^2 \right) \cdot \left( \lim_{x \rightarrow -3} \frac{1}{x} \right)$ and $\dsp \lim_{x \rightarrow -3} \left( x^2  \cdot \frac{1}{x} \right)$

\item $\dsp\lim_{x \rightarrow 5} \frac{x^2}{x-4}$ and $\dsp \frac{ \lim_{x\rightarrow 5} x^2 }{ \lim_{x \rightarrow 5} x-4 }$

\item $\dsp\lim_{x \rightarrow 2} \sqrt{x^{2}+x+3}$ and $\dsp \sqrt{\lim_{x \rightarrow 2} (x^{2}+x+3)}$

\item $\dsp\lim_{x \rightarrow -1} (4x+1)^{3}$ and $\dsp [\lim_{x \rightarrow -1} (4x+1)]^{3}$
\end{enumerate}
\end{prb}

\begin{prb}
\label{l30}
Write five theorems (facts) about limits from your observations in the previous problems.
\end{prb}

\begin{prb}
Graph and state the domain for the following functions.
\begin{enumerate}
\item $\dsp p(x) = 3x^2 - 6x$ \item $\dsp f(x) =  \left\{
         \begin{array}{ll}
         -2x-4 & \mbox{if} \; \; \; \; x < 1 \\
         x+1 & \mbox{if} \; \; \; \; x > 1
    \end{array}
    \right.$
\item $\dsp g(x) =  \left\{
         \begin{array}{ll}
         x^2+2 & \mbox{if} \; \; \; \; x < 1 \\
         0 & \mbox{if} \; \; \; \; x = 1 \\
         2x+1 & \mbox{if} \; \; \; \; x > 1
    \end{array}
    \right.$
\end{enumerate}
\end{prb}

Sometimes we are interested in the limit of a function $f$ as $x$ approaches $a$ from the left or right side. To make clear which side we are discussing, we write $\dsp \lim_{x \rightarrow a^-} f(x)$ (\emph{left-hand limit}) or $\dsp \lim_{x \rightarrow a^+} f(x)$ (\emph{right-hand limit}). If both the left- and right-hand limits exist and are equal, then the limit exists.  Otherwise, the limit does not exist.  This can happen when either the left-hand limit or the right-limit does not exist, or when both exist but are not equal.
\begin{annotation}
\endnote{At some point during our march (trudge) through limits, I always explain that, while limits may seem abstract and we may not yet see the need for them, they are the foundations of all the future topics we will study in the course -- continuity, differentiation, integration and many applications.  I also encourage them by saying that limits may be the most challenging portion of the course, so if they will just hang for a bit longer, it will all fit together! }
\end{annotation}
\begin{thm} \textbf{Limit Existence Theorem.}
If f is a function, then the $\lim_{x \rightarrow a} f(x)$ exists if and only if $\lim_{x \rightarrow a-} f(x)$  and $\lim_{x \rightarrow a+} f(x)$ exist and are equal.
\end{thm}

\begin{prb}
\label{l1} Using the functions from the previous problem, compute the following limits if they exist. If they do not exist, state why.
\begin{enumerate}
\item $\lim_{x \rightarrow -3^+} p(x),
\lim_{x \rightarrow -3^-} p(x),  \mbox{ and }
\lim_{x \rightarrow -3} p(x)$

\item $\lim_{x \rightarrow 1^-} f(x),
\lim_{x \rightarrow 1^+} f(x),  \mbox{ and }
\lim_{x \rightarrow 1} f(x)$

\item $\lim_{x \rightarrow 1^-} g(x),
\lim_{x \rightarrow 1^+} g(x),  \mbox{ and }
\lim_{x \rightarrow 1} g(x)$
\end{enumerate}
\end{prb}

\begin{dfn}
If $f$ is a function and $a$ is in the domain of $f$ then we say that \textbf{f is continuous at a} if $\lim_{x \rightarrow a} f(x) = f(a)$.
\end{dfn}

There are three subtle points in this definition.  The definition requires that for $f$ to be continuous at $a,$ each of the following conditions must be met.
\begin{enumerate}
\item $a$ must be in the domain of $f$ (otherwise $f(a)$ doesn't exist),
\item $\dsp \lim_{x \rightarrow a} f(x)$ must exist, and
\item these two numbers must be equal.
\end{enumerate}

The \emph{absolute value} function is an example of what we call a \emph{piecewise defined} function.

\begin{dfn}
The \textbf{absolute value function} is defined by:
$$|x| =  \left\{
         \begin{array}{ll}
         +x & \mbox{if} \;\;\;\; x \geq 0 \\
         -x & \mbox{if}  \;\;\;\;x < 0
    \end{array}
    \right.$$
\end{dfn}


\begin{prb}
Determine whether the following functions are continuous at the indicated points.
\begin{enumerate}
\item $\dsp f(x) = \frac{1}{x+2} \;\;\;\;\; \mbox{at} \;\;\; x=-2.$
\item $\dsp g(x) =
\left\{
         \begin{array}{ll}
         -x-1 & \mbox{if} \;\;\;\; x < 1 \\
         x-3 & \mbox{if} \;\;\;\; x \geq 1
    \end{array}
    \right. \;\;\;\; \mbox{at} \;\;\;\; x=1.$
\item $\dsp k(x) =  \left\{
         \begin{array}{ll}
         \frac{x^2-4}{x-2} & \mbox{if}  \;\;\;\;x \neq 2 \\ \\
         5 & \mbox{if} \;\;\;\; x = 2 \\
    \end{array}
    \right. \;\;\;\;\;\; \mbox{at} \;\;\;\;\; x=1.$
\item $f(x)=4x^{3}+x^{2}-6x-7$ \;\;\;\; at  \;\;\;\; $x=-2$
\item $\dsp C(x)=\cos(x)$ \;\;\;\; at \;\;\;\; $x=\frac{\pi }{3}$
\item $\dsp a(x) = \frac{|x-2|}{x-2} \;\;\;\;\; \mbox{at} \;\;\;\; x=2.$
\end{enumerate}
\end{prb}

\begin{dfn}
We say that a function is \textbf{continuous on a set} (such as an open interval) if it is continuous at every point in the set.
\end{dfn}

\begin{prb}
For the following, list the open intervals on which the function is continuous.
\begin{enumerate}
\item $f(x)=x^{2}+5x+3$
\item $\dsp g(x)=\frac{3x-1}{2x+5}$
\item $H(x)=\sin(x)+5x+\sqrt{1-x}$
\item $T(z)=2+\tan(z)$
\end{enumerate}
\end{prb}

\begin{thm} \label{continuous}
\textbf{Continuous Functions Theorem.}
\begin{enumerate}
\item Every polynomial is continuous on $(-\infty ,\infty )$.
\item Every rational function is continuous on its domain.
\item Every trigonometric function is continuous on its domain.
\item The sum, product, and difference of continuous functions is continuous.
\item The quotient $\dsp \frac{f}{g}$ of continuous functions $f$ and $g$ is
    continuous wherever $g$ is non-zero.
\item If $g$ is continuous at $a$ and $f$ is continuous at $g(a)$ then $f \circ g$ is continuous at $a$.
\begin{annotation}
\endnote{Once this theorem is stated, I must be careful on tests when I ask if a function is continuous and make clear if I want them to use the definition of continuity or if they may use the theorems.}
\end{annotation}
\end{enumerate}
\end{thm}

\begin{prb}
Compute the following limits:
\begin{enumerate}
\item $\dsp \lim_{h \rightarrow 0} \frac{9 h^2 - 3h}{h}$
\item $\dsp \lim_{h \rightarrow 0} \frac{x^2h^2 - xh}{h},$ where $x$ is some real number.
\item $\dsp \lim_{x \rightarrow 2} \left( 2xt + 3x^2t^2 \right),$ where t is some real number.
\end{enumerate}
\end{prb}

\begin{prb}
Let $f(x) = x^2 + x + 1.$
\begin{enumerate}
\item Compute and simplify $\dsp \lim_{h \rightarrow 0} \frac{f(x+h) - f(x)}{h}.$
\item Compute and simplify $\dsp \lim_{t \rightarrow x} \frac{f(t) - f(x)}{t-x}.$
\end{enumerate}
\end{prb}

\begin{expl}
Graph $f(x) = x^2, g(x) = -x^2, $ and $\dsp b(x) = x^2\sin \left( \frac{1}{x} \right).$ Compute $\dsp \lim_{x \rightarrow 0} f(x)$ and $\dsp \lim_{x \rightarrow 0} g(x).$ What can you conclude about $\dsp \lim_{x \rightarrow 0} b(x)?$
\end{expl}

\begin{expl}
Show that $\dsp \lim_{h \rightarrow 0} \frac{\sin(h)}{h} =1$ using the unit circle and starting with the fact that $\sin(h) \leq h \leq \tan(h)$.
\end{expl}

The previous two examples illustrate the \textbf{Squeeze Theorem}.  The importance of the second example will  become clear later in the course when we take the derivative of the trigonometric function, sine. The notation $f \in C_{[a,b]}$  translates as  ``$f$ is an element of the set of all continuous function on the interval $[a,b],$'' which  means that $f$ is continuous at every number $x$ in the interval $[a,b].$  When we write $g$ is \emph{between} $f$ and $h,$ on $[a,b]$ we mean that for every number $x \in [a,b]$ we have either $f(x) \leq g(x) \leq h(x)$ or $h(x) \leq g(x) \leq f(x).$

\begin{thm} \label{squeeze}
\textbf{The Squeeze Theorem.} If $f,g, h \in C_{[a,b]}$ and $c \in [a,b]$ and $g$ is between $f$ and $h$ on $[a,b],$ and $\lim_{x \rightarrow c} f(x) = \lim_{x \rightarrow c} h(x) = L$ for some number L, then $\lim_{x \rightarrow c} g(x) = L.$
\end{thm}

\begin{prb}
Use the \textbf{Squeeze Theorem} to solve each of these problems.
\begin{enumerate}
\item Compute $\dsp \lim_{x\rightarrow 0} \left( x^2\cos(x) \right).$

\item Compute $\dsp \lim_{x \rightarrow 0} x\sin \left(\frac{1}{x} \right)$
using $\dsp f(x) = -x$, $\dsp g(x) = x\sin \left(\frac{1}{x} \right)$, and $\dsp h(x) = x.$

\item Compute $\dsp \lim_{x \rightarrow \infty} \frac{x}{x^2+x+1}$
using $\dsp f(x) = \frac{1}{x}$, $\dsp g(x) = \frac{x}{x^2+x+1}$, and $\dsp h(x) = 0.$

\end{enumerate}
\end{prb}


\section{Instantaneous Velocity and Growth Rate}

If you have been struggling, relax.  This is where it all comes together.
\begin{annotation}
\endnote{Having encouraged them to be patient and that the big picture will emerge soon, this section is where the idea of instantaneous rates of change begins to gel as they see the commonality between the problems associated with computing velocity, plant growth rate, population growth rate and later slope.}
\end{annotation}

\begin{prb}
\label{Robbie}
Recall from Problem \ref{l3}, that Robbie's position function is $R(t) = t^3 + 2.$
\begin{enumerate}
\item What is Robbie's average velocity from time $t=3$ till time $t=4$?
\item What is Robbie's average velocity from time $t=3$ till time $t=3.5$?
\item What is Robbie's average velocity from time $t=3$ till time $t=3.1$?
\item What is Robbie's \emph{instantaneous} velocity at time t=3?
\end{enumerate}
\end{prb}

\begin{prb}
\label{l5} Fill in Table \ref{t7} of average velocities for Robbie. Write down an equation for a function with input time and output the average velocity of Robbie over the time interval [3,t].
\end{prb}


\begin{table}[h!]
\caption{Robbie's Average Velocity Table} \label{t7}
\begin{center}
\begin{tabular}{ |c|c|c| }
\hline
\multicolumn{1}{|c}{Time} & \multicolumn{1}{|c|}{Time Interval} &\multicolumn{1}{c|}{Average Velocity}\\
\hline \hline
$4$ & $[3,4]$ &   \\
\hline
$3.5$ & $[3,3.5]$ &   \\
\hline
$3.1$ & $[3,3.1]$ &   \\
\hline
$3.01$ & $[3,3.01]$ &   \\
\hline
\end{tabular}
\end{center}
\end{table}


\begin{prb}
\label{l4} A population P of robots (relatives of Robbie) is growing according to Table \ref{t4}, where time is measured in days and population is measured in number of robots. Determine a population function P that represents the number of robots as a function of time and graph this function.
\end{prb}

\begin{table}[h!]
\caption{Population Table} \label{t4}
\begin{center}
\begin{tabular}{ |c|c| }
\hline
 \multicolumn{1}{|c|}{Time} &\multicolumn{1}{c|}{Population}\\
\hline \hline
$0$ & $2$  \\
\hline
$1$ & $4$  \\
\hline
$2$ & $12$  \\
\hline
$3$ & $32$  \\
\hline
$4$ & $70$  \\
\hline
$5$ & $132$  \\
\hline
\end{tabular}
\end{center}
\end{table}



\begin{prb}
Using the function P from the previous problem:
\begin{enumerate}
\item Assume that c is a positive number and compute $P(3)$ and $P(c)$. What do these numbers represent about the  robots?
\item Determine the value of $P(5) - P(3)$. What does this number say about the robot population?
\item Assume that $u$ and $v$ are positive numbers with $u > v$. Compute the value of $P(u) - P(v)$ and explain what this expression says about the robot population.
\end{enumerate}
\end{prb}

\begin{prb}
Using the function, P, from the previous problem:
\begin{enumerate}
\item Compute the value of $\dsp \frac{P(5)-P(3)}{5-3}$ and explain the physical meaning of this number.
\item Assume that $u$ and $v$ represent times. Simplify and explain the meaning of $\dsp \frac{P(u)-P(v)}{u-v}$.
\item What is the population of robots when time is equal to 5?  At time equal to 5.5?
\item At what time is there a population of 106 robots?
\end{enumerate}
\end{prb}

\begin{prb}
A Maple leaf grows from a length of 5.62 cm to a length of 8.12 cm. in 1 year and 3 months. How fast did the leaf grow?  Does your answer represent an instantaneous rate of growth or an average rate of growth?
\end{prb}

\begin{prb}
Experimental data indicates that on January 1, the leaf is 8.37 centimeters long and it grows at the rate of 0.18 centimeters per month for 3.5 months. From that time on, the leaf grows at a rate of 0.12 centimeters per month until it reaches a length of 9.6 centimeters. Determine the average growth rate for the whole time period.  At what time did the leaf grow at this rate?
\end{prb}

\begin{prb} \label{l6}
Use the robot population function you found in Problem \ref{l4} to fill in Table \ref{t5} and determine how fast the robot population is growing at time t=4. This is called the instantaneous growth rate and is measured in robots per day.
\end{prb}

\begin{table}[h!]
\caption{Population Growth Rate Table} \label{t5}
\begin{center}
\begin{tabular}{ |c|c|c| }
\hline \multicolumn{1}{|c}{Time} & \multicolumn{1}{|c|}{Time
Interval} &
\multicolumn{1}{c|}{Average Growth Rate}\\
\hline \hline
$5$ & $[4,5]$ &   \\
\hline
$4.5$ & $[4,4.5]$ &   \\
\hline
$4.1$ & $[4,4.1]$ &   \\
\hline
$4.01$ & $[4,4.01]$ &   \\
\hline
\end{tabular}
\end{center}
\end{table}

\begin{prb}
Write a function with input time and output the average growth rate of the population over the time interval [4,t].
\end{prb}


\begin{prb}
\label{l8}
State the limit that would produce the limit table from Problem \ref{l5} and hence establish the instantaneous velocity of Robbie at time t=3.
\end{prb}

\begin{prb}
State the limit that would produce the limit table from Problem \ref{l6}  and hence establish the instantaneous growth rate of the population at time t=4.
\end{prb}

At this point, you know how to compute average velocity, average growth rate, instantaneous velocity, and instantaneous growth rate.  You don't want to create a limit table every time you want the instantaneous velocity, so we want a function for Robbie's velocity. That is, we want a function that takes {\it time} as input and produces Robbie's {\it velocity} as output. Similarly, we want a function that takes \emph{time} as input and produces the \emph{robot population growth rate} as output.

\begin{prb}
\label{l10}
Write a function for the instantaneous velocity of Robbie.
\end{prb}

\begin{prb}
Write a function for the instantaneous growth rate of the robot population.
\end{prb}

\textbf{Roberta the Robot} (Robbie's potential girlfriend) is walking along the same road as Robbie. Roberta's
distance from the same pothole after $t$ seconds is given by $f(t)= 2t^2 - 16$ feet. Recall from Problem \ref{l10}, we know that if we have a position function then we can find the corresponding velocity function by taking a certain limit.

\begin{prb}
Find the velocity function for Roberta.
\end{prb}

\begin{prb}
\label{l11} Determine Roberta's velocity at 10 seconds. At what time(s) is Roberta $2$ feet from the pothole?
\end{prb}

\begin{prb}
At what time(s) is Roberta stopped? At what time is Roberta
traveling at a velocity of $-4$ feet per second?  Do Robbie and Roberta
ever meet?  If so, when and where?
\end{prb}

\textbf{A colony of androids} is threatening our robot population.
On day $t$, there are $N(t) = 2t^2  +4t + 6$ androids.

\begin{prb}
Find the growth rate function for the android population.  What are
the units? At what time(s) is the android population equal to 200?
\end{prb}

\begin{prb}
When is the android population increasing at a rate of 10 androids per day?
How fast is the androids population growing on day 10?
\end{prb}

\section{Tangent Lines}

We are going to define \textbf{tangent line} from a physical point-of-view rather than
give a rigorous mathematical definition.  Sketch any function, $f$, and pick a point on it.  Label
your point $(a,f(a)).$  Place your pencil on the graph at a point to the left of $(a,f(a)).$ Imagine that
your pencil is a car, the graph is a road, and you are driving toward the point $(a,f(a))$ from the left.
Suppose that there is a perfectly frictionless patch of ice at the point $(a,f(a))$ and keep
drawing in the direction the car would travel as it slides off the road. Now do the same from the right.
If the two lines you drew form a line, then you have drawn the
\textbf{tangent line to $f$ at the point $(a,f(a))$}.  If they don't form a line,
then $f$ does not have a tangent line at this point.

\begin{dfn}
A \textbf{secant line} for the graph $f$ is  a line passing through two points on the graph of $f.$
\end{dfn}

\begin{prb}
Let $f(x) = x^2+1$ and graph $f.$  Graph and compute the slope of the secant lines passing through:
\begin{enumerate}
\item $(2,f(2))$ and $(4,f(4))$
\item $(2,f(2))$ and $(3,f(3))$
\item $(2,f(2))$ and $(2.5,f(2.5))$
\item $(2,f(2))$ and $(2.1,f(2.1))$
\end{enumerate}
What is the slope of the line tangent to $f$ at the point $(2,5)$?
\end{prb}

\begin{prb}
Write down a limit that represents the work you did in the last problem.
\end{prb}

\begin{prb}
\label{l13}
Find the slope of the tangent line to the curve $f(x)= x^3 + 2$ at the point $(3, 29)$.
\end{prb}

\begin{prb}
Find the equation of the tangent line in Problem \ref{l13}.
\end{prb}

\begin{prb}
\label{l14} Assume $a \in \re.$ Find the equation of the line
tangent to the graph of $f(x) = x^3 + 2$ at the point $(a,f(a)).$
\end{prb}

\section{Practice} \label{chap1probs}

We will not present the problems from this section, although you are welcome to ask about them in class.

\vskip .1in
\noindent
\textbf{Graphing}
\begin{enumerate}
\item Graph $f(x) = x^2 - 4$, listing $x-$ and $y-$intercepts.
\item Graph $\dsp g(x) = \frac{x-1}{x+2}$, listing both vertical and horizontal asymptotes.
\item Graph $h(x) = \sin(x)$, listing the domain and range.
\item Graph $r(x) = \sqrt{3x-2}$, listing the domain and $x-$intercept.
\item Graph $\dsp f(x) =  \left\{
         \begin{array}{ll}
          x-5 & \mbox{if} \; \; \; \; x>1 \\
          x^{2}+1 & \mbox{if} \; \; \; \; x\leq 1
    \end{array}
    \right.$
\end{enumerate}

\noindent
\textbf{Evaluate the following limits or state why they do not exist.}

\begin{enumerate}

\item  $\lim_{x\to 2} (x^{3}+2x^{2}-5x+7)$
\item  $\dsp \lim_{x\to \frac{\pi}{2}}\cos(2x)$
\item $\dsp \lim_{x\to -2}\frac{x+2}{x^{2}-4}  \;\;\; \mbox{and} \;\;\; \lim_{x\to -2}\frac{x^{2}-4}{x+2}$
\item $\lim_{x\to 1} f(x)$, where $\dsp f(x) =  \left\{
         \begin{array}{ll}
          x-5 & \mbox{if} \; \; \; \; x>1 \\
          x^{2}+1 & \mbox{if} \; \; \; \; x\leq 1
    \end{array}
    \right.$
\item $\lim_{x\to 3}\sqrt{x-3}$
\item $\dsp \lim_{h\to 0}\frac{3xh+h}{h}$
\item  $\dsp \lim_{x\to -3} \frac{x^{2}+5x+6}{x^{2}-x-12}$
\item  $\dsp \lim_{t\to 1} \frac{t^{3}-1}{t^{2}-1}$
\item  $\dsp \lim_{x\to -1} \frac{1+2x}{1+x}$
\item  $\dsp \lim_{x\to 0} \frac{\sqrt{3-x}-\sqrt{3}}{x}$
\item  $\lim_{x\to \frac{\pi}{8}^{+}}\cos(2x)$, $\lim_{x\to \frac{\pi}{8}^{-}}\cos(2x)$, and $\lim_{x\to \frac{\pi}{8}}\cos(2x)$
\item $lim_{x\to 1^{-}}f(x)$, $lim_{x\to 1^{+}}f(x)$, and $lim_{x\to 1}f(x)$  \\
         where $\dsp f(x) =  \left\{
         \begin{array}{ll}
         x+3 & \mbox{if} \; \; \; \; x>1 \\
         9x^{2}-5 & \mbox{if} \; \; \; \; x\leq 1
    \end{array}
    \right.$
\item  Let $f(x)=1-2x^2$. Compute $\dsp \lim_{t \to x} \frac{f(t)-f(x)}{t-x}$.
\item Use the Squeeze Theorem to compute $\dsp \lim_{x \to \infty} \frac{\sin(x) + 2}{x}$.
\end{enumerate}

\noindent
\textbf{List the interval(s) on which each of the following is continuous.}

\begin{enumerate}
\item  $\dsp f(x)=\frac{3x+2}{6x^{2}-x-1}$
\item  $G(x)=\sqrt{x+5}$
\item $f(x)=x^{4}+2x^{3}-6x+1$
\item $h(x)=\sqrt{2-4x}$
\item $\dsp g(t)=\frac{t}{t-2}$
\item $f(\beta)=3\tan (\beta)$
\item $\dsp f(x) =  \left\{
        \begin{array}{ll}
         x^{2}+2 & \mbox{if} \; \; \; \; x \geq 1 \\
         5x-2 & \mbox{if} \; \; \; \; x < 1
    \end{array}
\right. $
\item  $ f(x) =  \left\{
         \begin{array}{ll}
         3-x^{2} & \mbox{if}  \;\;\;\;x\geq \pi \\
         \cos(2x) & \mbox{if} \;\;\;\; x<\pi \\
    \end{array}
    \right. $
\end{enumerate}

\noindent
\textbf{Velocity}

\begin{enumerate}

\item Given position function $f(t)=3t^{2}+t-1$, where $t$ is in seconds and $f(t)$ is in feet,
find the average velocity between $1$ and $4$ seconds. Find the average velocity between
$p$ and $q$ seconds.

\item Suppose that $y(t)=-5t^{2}+100t+5$ gives the height $t$
(in hundreds of feet) of a rocket fired straight up as a function
of time $t$ (in seconds).  Determine the instantaneous velocity at times $t=2$ and at $t=4$.


\end{enumerate}

\noindent
\textbf{Secant and Tangent Lines}

\begin{enumerate}

\item Find the equation of the secant line of $g(x)=1+\sin(x)$ containing the points $(0,1)$ and $(\frac{\pi }{2},2)$.
\item  Find the slope of the tangent line to $f(t)=5t-2t^{2}$ at $t=-1$.
\item Find the equation of the tangent line to $\dsp F(x) = \frac{x}{x+1}$ at $x=-3.$ Graph both $F$ and the tangent line.
\item  Find the equation of the tangent line to $H(x)=\sqrt{1+x}$ at $x=8$ and sketch the graph of both $H$ and the tangent line.

\end{enumerate}

\vskip .5in
\noindent
\textbf{Chapter 1 Solutions}\\ \\

\noindent
\textbf{Graphing}
\begin{enumerate}
\item parabola with vertex at $(0,-4)$, y-intercept $(0,-4)$, x-intercepts at $(\pm 2,0)$
\item rational function with horizontal asymptote $y=1$ and vertical asymptote $x=-2$
\item be sure you can graph all your trig functions and state their domains and ranges
\item root functions look like half a parabola sideways, x-intercept $(2/3,0)$, domain $[2/3, \infty)$
\item half parabola and half straight line, domain $(-\infty,\infty)$
\end{enumerate}

\noindent
\textbf{Evaluate the following limits or state why they do not exist.}

\begin{enumerate}
\item $13$
\item  $-1$
\item $-\frac{1}{4} \;\; \mbox{and} \;\; -4$
\item DNE since LHL = 2 doesn't equal RHL = -4
\item DNE since LHL DNE (note that RHL = 0)
\item $3x+1$
\item $\frac{1}{7}$
\item $\frac{3}{2}$
\item limit does not exist; $x=-1$ is a horizontal asymptote, so neither left nor right limits exist
\item $\frac{-1}{2\sqrt{3}}$
\item $\sqrt{2}/2$ for all three answers
\item $4$ for all three answers
\item  $-4x$
\item $0$
\end{enumerate}

\noindent
\textbf{List the interval(s) on which each of the following is continuous.}

\begin{enumerate}
\item  $(-\infty,-\frac{1}{3}), (-\frac{1}{3},\frac{1}{2}), (\frac{1}{2},\infty)$
\item  $[-5,\infty)$
\item  $(-\infty,\infty)$
\item $(-\infty, \frac{1}{2}]$
\item $(-\infty,2), (2,\infty)$
\item continuous everywhere except $\beta = \frac{(2k-1)\pi}{2}$ where k is an integer
\item $(-\infty,\infty)$
\item  all $x$ except $x=\pi$
\end{enumerate}

\noindent
\textbf{Velocity}

\begin{enumerate}
\item 16 ft/sec, $3(p+q)+1$ ft/sec
\item 8000 and 6000 ft/sec
\end{enumerate}

\noindent
\textbf{Secant and Tangent Lines}

\begin{enumerate}
\item $y = \frac{2}{\pi}x+1$
\item  $y=9x+2$
\item  $y= \frac{1}{4} x + \frac{9}{4}$
\item  $y = \frac{1}{6}x + \frac{5}{3}$
\end{enumerate}

\chapter{Derivatives}

``...the secret of education lies in respecting the pupil...'' -  Ralph Waldo Emerson\\ \\

Up to this point, we have solved several different problems and discovered that the same concept has generated a solution each time.  Limits were used to find:
\begin{itemize}
\item Robbie's velocity function given his position function,
\item the instantaneous growth function for the robot population given the population function, and
\item the function that tells us the slope of the line tangent to a function at any point.
\end{itemize}

The following definition is the result of all of our work to this point.
\begin{annotation}
\endnote{By this point, a majority of the class sees the connection between average rates of change and instantaneous rates of change.   Whenever a key problem was worked, I emphasized the commonality between the three problems (slope, velocity, plant growth rate, population growth rate).   Now, I'll likely summarize this one final time to show that Definition \ref{derivdfn} encompasses all the work we have done on each of these problems by giving us a formulaic way to compute instantaneous rates of change. On a recent evaluation form, ``You did a brilliant job explaining limits.  After 4 calculus teachers, you are the only one to adequately explain them.'' An interesting comment given that my explanations are primarily in response to student questions or work that students have done!}
\end{annotation}

\begin{dfn}
\label{derivdfn}
Given a function $f$, the \textbf{derivative} of $f$, denoted by $f'$, is the function so that for every point $x$ in the domain of $f$ where $f$ has a tangent line, $f'(x)$ equals the slope of the tangent line to $f$ at $(x,f(x)).$  Either
$$f'(x) = \lim_{b \rightarrow x} \frac{f(b)-f(x)}{b-x} \; \; \; \;    \mbox{      or     } \; \; \; \;    f'(x) = \lim_{h \rightarrow 0} \frac{f(x+h)-f(x)}{h}$$ may be used to compute the derivative.
\end{dfn}

While both limits above will always yield the same result, there are times when one may be preferable to the other in terms of the algebraic computations required to obtain the result. Given a function $f$, the derivative of $f$ is usually denoted by $\dsp f', \;\;  \frac{df}{dx} \;\; \mbox{or} \;\; \dot{f}.$  Regardless of how it is written, the derivative of $f$ at $x$ represents the slope of the line tangent to $f$ at the point $(x,f(x))$ or the rate at which the quantity $f$ is changing at $x$. The origins of calculus are generally credited to Leibnitz and Newton. The first notation, credited to Lagrange, is referred to as {\it prime} or {\it functional} notation. The second notation is due to Leibnitz.  The third is due to Newton.  Engineers sometimes use the dot to represent derivatives that are taken with respect to time.

If $f(x) = x^2,$ then each of $$f'(x) \mbox{ and } ( x^2 )' \mbox{ and } \frac{d}{dx} (x^2)$$ all mean ``the derivative of $f$ at $x$.''\\

\section{Derivatives of Polynomial, Root, and Rational Functions}

\begin{prb}
Compute the derivative of each function using Definition \ref{derivdfn}.
\begin{enumerate}
\item $f(x) = x^2 - 3x$
\item $g(x) = 3x-7$
\item $h(t) = 4t^3$
\item $\dsp u(t) = \frac{1}{t}$
\item $\dsp v(x) = \frac{x-5}{x+2}$
\end{enumerate}
\end{prb}

\begin{prb}
Compute the following using the derivatives you computed in the previous problem. Graph both the original function and the tangent line at the point of interest.
\begin{enumerate}
\item $f'(x)$ at $x=5$
\item $h'(t)$ at $t=e$ (Recall that Euler's constant, $e$, is approximately $2.71828...$)
\item $v'(x)$ at $x=-2$
\end{enumerate}
\end{prb}

\textbf{Algebra Reminder.} An expression such as  $\dsp \frac{\sqrt{a}-\sqrt{b}}{a-b},$ can be simplified by multiplying the numerator and the denominator of the fraction by $\dsp \sqrt{a}+\sqrt{b}.$  This is called \emph{rationalizing the numerator} and $\dsp \sqrt{a}+\sqrt{b}$ is called the \emph{conjugate} of $\dsp \sqrt{a}-\sqrt{b}.$

\begin{prb}
\label{l15}
If $f(x) = \sqrt{x+1}$, compute $f'$.  Graph $f$ and the tangent line to $f$ at $(1,f(1))$ and state the slope of the tangent line.
\end{prb}

\begin{prb}
\label{l16}
If $f(x) = 2x^3 + 3x^2 - 7$, compute $f'$.   Graph $f$ and compute the equation of the tangent line to $f$ at $(1,f(1)).$
\end{prb}

\begin{prb}
\label{l17}
If $\dsp f(x) = \frac{1}{x+1}$, compute $f'.$
\end{prb}

\begin{prb}
If $f(x) = \sqrt{x+1}$, find the equation of the tangent line at $x = a.$
\end{prb}

\begin{prb}
For each of the following, check the left- and right-hand limits to determine if the limit exists:
\begin{enumerate}
\item $\dsp \lim_{x \rightarrow 0}\frac{0}{x}$
\item $\dsp \lim_{x \rightarrow 0}\frac{|x|}{x}$
\end{enumerate}
\end{prb}

\begin{dfn}
We say that a function, f, is \textbf{differentiable at a} if $f'(a)$ exists.
\end{dfn}

\begin{dfn}
We say that a function, f, is \textbf{differentiable} if $f$ is differentiable at every point in its domain.
\end{dfn}

\begin{dfn}
We say that a function, f, is \textbf{left differentiable at a} if $\dsp \lim_{t \rightarrow a^-} \frac{f(t) - f(a)}{t-a}$ exists. If
$f$ is left differentiable at $a$, we denote the \textbf{left derivative} of $f$ at $a$ by $f'(a-).$  Right differentiability is defined similarly.
\end{dfn}

\begin{dfn}
We say that $f$ is \textbf{differentiable on (a,b)} if $f$ is differentiable at $t$ for all $t$ in $(a,b)$.
\end{dfn}

\begin{dfn}
We say that f is \textbf{differentiable on [a,b]} if f is differentiable on (a,b) and $f'(a+)$ and $f'(b-)$ both exist.
\end{dfn}

\begin{prb}
\label{absolutederiv}
Let $f(x) = |x - 2|.$
\begin{enumerate}
\item Graph $f.$
\item Determine the equation of the line tangent to $f$ at $(-1,3).$
\item Determine the equation of the line tangent to $f$ at $(5,3).$
\item Compute $f'(2+)$ and $f'(2-).$
\item Does $f'(2)$ exist? Why?
\item What is the derivative of $f?$
\end{enumerate}
\end{prb}

\begin{expl}
Illustrate ways in which a function might fail to be differentiable at a point $c$ such as:
\begin{itemize}
\item the point $c$ might not be in the domain of the function,
\item the function might have a {\it vertical tangent line} at $c$, or
\item the left and right-hand derivatives might not agree at $c$ (i.e. $f'(c+) \neq f'(c-)$).
\end{itemize}
\end{expl}

\begin{prb}
For each of the following, use a sketch of the graph to determine the set of those points at which the function is not differentiable.
\begin{enumerate}
\item $\dsp f(x) = \frac{3}{1-x^{2}}$
\item $g(x) =\sqrt{5+x}$
\item $i(x) = \sec(x)$
\end{enumerate}
\end{prb}

\begin{thm}
\textbf{Differentiable Functions Theorem.}
\begin{annotation}
\endnote{After this theorem is stated, I must be careful on a test to indicate clearly what I want when I ask whether a function is differentiable.   And I simply warn them in advance that I will be clear and ask something like, ``Compute the left- and right-hand derivatives to determine if $f$ is differentiable at a point'' or, ``Just list the points where $f$ is differentiable.''}
\end{annotation}
\begin{enumerate}
    \item Every polynomial is differentiable on $(-\infty ,\infty )$.
    \item Every rational function is differentiable on its domain.
    \item Every trigonometric function is differentiable on its domain.
    \item The sum, product, and difference of differentiable functions is differentiable.
    \item The quotient $\frac{f}{g}$ of differentiable functions $f$ and $g$ is differentiable wherever $g$ is not zero.
    \item If $g$ is differentiable at $x=a$ and $f$ is differentiable at $x=g(a)$ then $f \circ g$ is differentiable at $x=a$.
    \end{enumerate}
\end{thm}
We know how to compute derivatives of functions using limits.  By proving theorems
we will increase the ease with which we compute derivatives of more challenging functions.
\begin{prb}
\label{l18} Compute $f'$ for each function listed, using the
definition of the derivative.
\begin{enumerate}
\item $f(x) = a,$ where $a$ is a constant
\item $f(x) = x$
\item $f(x) = x^2$
\item $f(x) = x^3$
\item $f(x) = x^4$
\end{enumerate}
\end{prb}
At this point we have used the definition of the derivative (limits) to compute the derivatives
of many different functions, including: $a$ (constant function), $x$ (the identity function),  $x^2$,  $x^3$,  $4x^3$, $x^4$, $\dsp \sqrt{x+1}$, $2x^3 + 3x^2 - 7$, $\dsp \frac{1}{x}$, $\dsp \frac{1}{x+1}$, $\dsp \frac{x-5}{x+2}$, $|x-2|$, and  $x^3 + 2$. Based on the derivatives we have computed thus far, you can likely guess the answer to the next question.\\ \\
\noindent
\emph{Question.} What is the derivative of $f(x) = x^{200}$?  \hfill \emph{Guess.} $f'(x)=200x^{199}$.\\ \\
How do we {\it know} if our guess is correct?  We don't.  We must show that it is true using the
definition of the derivative.  Rather than work this one problem and still not know what the derivative
of $g(x) = x^{201}$ is, we'll let $n$ represent some unknown natural number and compute the derivative of
$f(x) = x^n.$  Because we will not choose a specific value for $n,$ we will now \emph{know} the derivative
of an entire class of functions.


\begin{prb}
\label{powerrule}
\textbf{Power Rule.}
\begin{annotation}
\endnote{When I first taught this way, I thought that no one would solve this problem because, like so many students, I had been shown a proof using the binomial theorem.  But when one sets it up with Problem \ref{l18}, several students will invariably see a pattern depending on which definition they are using.  Having introduced both definitions doubles the odds that a student has success, since if they don't see how to simplify one expression, they might still see how to simplify the other.  I have seen students discover the pattern for the long division $\dsp \frac{x^n - t^n}{x-t}$ and I have seen students recognize that even though they couldn't figure out the coefficients for what you and I know as the binomial theorem, they did not need the coefficients because the terms for which they did not know the coefficients tended to zero anyway! In either case, the students really learned a tremendous amount working on the problem and I shared the binomial theorem with the class after the presentation.}
\end{annotation}
Let $n$ be a natural number and use Definition \ref{derivdfn} to show that the derivative of $f(x) = x^n$ is $f'(x)=nx^{n-1}$.
\end{prb}

\begin{prb}
\label{constantrule}
\textbf{Constant Rule.} Show that if $k$ is a constant (i.e. any real number) and $h(x) = kf(x)$, then $h'(x) = kf'(x)$.
\end{prb}

\textbf{Note.}  The previous problem is not stated accurately.  Here is a precise statement. Show that if $k$ is a constant (i.e. any real number) and $f$ is differentiable on $(a,b)$ and $h(x) = kf(x)$ for all $x$ in the interval $(a,b)$, then $h'(x) = kf'(x)$ for all $x$ in $(a,b)$.

\begin{prb}
\label{sumrule}
\textbf{Sum Rule.} Show that if each of $f$ and $g$ is a function and $S(x) = f(x) + g(x)$, then $S'(x) = f'(x) + g'(x)$.
\end{prb}

\begin{prb}
\label{differencerule}
\textbf{Difference Rule.}
\begin{annotation}
\endnote{If they do Problem \ref{differencerule} using the definition, then I show them that this can be done using the sum and constant rules.  If they use the rules, then I show that it can be done with the definition.}
\end{annotation}
Show that if each of $f$ and $g$ is a function and $D(x) = f(x) - g(x)$, then $D'(x) = f'(x) - g'(x)$.
\end{prb}

\begin{prb}
Use Problems \ref{powerrule}, \ref{constantrule}, \ref{sumrule}, and \ref{differencerule} to compute $f'$ when $f(x) = 6x^5 - 4x^4 + 7x^3$. Apply one rule per equal sign and be sure to explain which rules you are using.
\end{prb}

The next theorem allows us to avoid using limits to determine if a function is continuous, since it states that all differentiable functions are also continuous functions.

\begin{thm} \label{diffcont}
\textbf{A Continuity Theorem.} If $x$ is a number and $f$ is differentiable at $x$, then $f$ is continuous at $x.$
\end{thm}

\begin{prb}
\label{productrule}
\textbf{Product Rule.} Show that if $P(x) = f(x)g(x)$, then $P'(x) = f'(x)g(x) + f(x)g'(x)$.
\end{prb}

\begin{prb}
\label{quotientrule}
\textbf{Quotient Rule.}
\begin{annotation}
\endnote{At some point, I always discuss what is missing from this statement -- the full hypothesis regarding that each of $f$ and $g$ are differentiable at the point and that $g$ is non-zero.}
\end{annotation}
Show that if $\dsp Q(x) = \frac{f(x)}{g(x)}$, then $\dsp Q'(x) = \frac{f'(x)g(x) - f(x)g'(x)}{(g(x))^2}.$
\end{prb}

Problem \ref{powerrule} only proved the power rule for exponents that were positive integers. With the new tools, the next problem extends the power rule to include negative and zero exponents. A natural question would then be, ``Is this true for exponents that are real numbers but not integers?'' The answer is ``yes'' but we'll simply assume this and not prove it in this course.

\begin{prb}
\label{powerrule2}
\textbf{Power Rule Revisited.} Show that if $n$ is any integer and $f(x) =x^n$, then $f'(x) = nx^{n-1}$.  Before you start, ask yourself how this is different from Problem \ref{powerrule}.
\end{prb}

\begin{prb}
Suppose $a,b,c,$ and $d$ are constants such that $d \neq 0$, and find $f'$ where $\dsp f(x) = \frac{ax^2 + bx + c}{d}.$
\end{prb}

\begin{prb}
Find $f'$ when $f$ is the function defined by $\displaystyle{f(x) = (x - \frac{1}{x})(2x^2+3x+4)}.$
\end{prb}

\begin{prb}
\label{l20} Find $f'$ when $f$ is the function defined by $\displaystyle{f(x) = \frac{x-1}{x+1} \cdot \frac{x-2}{x+2}}.$
\end{prb}

\begin{prb}
\label{l21} Find $f'$ when $f$ is the function defined by $\displaystyle{f(r) = \frac{r^2s^2-r^4}{s-m}}$ and $s$ and $m$ are real numbers and $s \neq m$.
\end{prb}
Since the derivative of a function is itself a function, there is no reason not to take the derivative of the derivative. When this is done, the new function is called the \textbf{second derivative} and is denoted by $f''$.  Similarly the \textbf{third derivative} is denoted by $f'''$.

\begin{prb}
Find $f''$ when $\displaystyle{f(x) = \frac{2}{x} - \frac{x^4}{2}}$.
\end{prb}

You have solved {\it algebraic} equations such as $$5x + 2 = 3x + 4 \;\;\; \mbox{or} \;\;\; 3x^2 - 2x + 1 = 0$$ where the goal is to find each value of $x$ that solves the equations.  Engineering applications often need to solve {\it  differential} equations such as $$f'(x) = x f(x) \;\;\; \mbox{or} \;\;\; f''(x) = x^3 \;\;\; \mbox{or} \;\;\; f''(x) + f(x) = 0$$ where the goal is to find each function $f$ that satisfies the equation.  In the real world, solutions to differential equations might represent the shape of an airplane wing or the temperature distribution of an engine block.
\begin{annotation}
\endnote{I am regularly discussing the link between the mathematics we are doing, which may seem abstract, and the engineering applications that they will see in future courses. At the same time, I am encouraging them to take the upper-division courses in differential equations, numerical methods, foundations and analysis to strengthen their mathematics.  I give specific examples of engineers who graduated with a second degree in mathematics and had starting salaries averaging \$20,000 more than the average starting salary of the engineers without the degree.  Mathematics pays because so few are good at it and if they are good at it, they should develop and exploit that skill.  Of course, some simply decide that they like the math more than the engineering and come over from the dark side.}
\end{annotation}

\begin{prb}
Show that if $x$ is not 0, then the function $\displaystyle{f(x) = \frac{2}{x}}$ is a solution to the equation  $x^3f''(x) + x^2f'(x) - xf(x) = 0$.
\end{prb}

\begin{prb}
Find one function $f$ that satisfies $f'(x) = 4x^2 - 2x.$
\end{prb}

We have been very careful up to this point in the book to write the equations in {\it functional} notation.  That is, we never wrote, $y=x^2,$ but rather wrote $f(x) = x^2$ to {\it emphasize} that $f$ is the function and it depends on the {\it independent} variable, $x.$  As you will see in other courses, it is very common to write, $y=x^2$ and then follow this with $y'=2x.$ Differential equations are almost always written in this short-hand, omitting the independent variable where possible, and from this point forward, we too will use this short-hand on occasion.

\begin{prb}
\label{l26} Sketch the graph of $y = 2x^3 - 3x^2 - 12x + 2$ and locate (by using the derivative) the points on the graph that have a horizontal tangent line.
\end{prb}

\begin{prb}
Find an equation for the line that is tangent to $\displaystyle{y = \frac{2-x}{2+x}}$ at $(1,\frac{1}{3})$.
\end{prb}

\textbf{Algebra Reminder.}  Any point where a function crosses the x-axis is called an \textbf{x-intercept} of the function, and any point where a function crosses the y-axis is called a \textbf{y-intercept} of the function.  To find an {\it x-intercept} we set $y$ equal to zero and solve for $x.$ To find a  {\it y-intercept} we set $x$ equal to zero and solve for $y.$

\begin{prb}
\label{l23} Find numbers a, b, and c so that the function $f(x) = ax^2 + bx + c$ has an x-intercept of 2, a y- intercept of 2, and a tangent line with slope of 2 at $(2,4a+2b+c).$
\end{prb}

\begin{prb}
\label{l24} Sketch the function $f(x) = ax^2$, where $a > 0$.  Let c and d be two (distinct) real numbers.  Let x be the x-coordinate of the point where the tangent line to f at $(c,f(c))$ intersects the tangent line to f at $(d,f(d))$.  Show that the distance between x and c equals the distance between x and d.
\end{prb}

\begin{prb}
\label{l25} Let $\dsp f(x) = \frac{2}{x}$ and $c$ be any real number.  Find the area of the triangle ($\dsp A=\frac{1}{2}bh$) formed by the tangent line to the graph at $(c,f(c))$ and the coordinate axes.
\end{prb}

\begin{prb}
Prove that if the graph of $h(x) = f(x) - g(x)$ has a horizontal tangent line at $(c,h(c))$, then the tangent line to f at $(c,f(c))$ is parallel to the tangent line to g at $(c,g(c))$.
\end{prb}

\begin{prb}
Show that if the function f is differentiable at x and $g(x) = (f(x))^2$, then $g'(x) = 2f(x)f'(x)$.
\end{prb}

\begin{prb}
Sketch a graph of
 $$f(x) =  \left\{
         \begin{array}{ll}
          x^2 & \mbox{if   }x \leq 1 \\
         \sqrt{x} & \mbox{if   }x  > 1.
    \end{array}
    \right.$$
Use left and right hand derivatives at $x=1$ to determine if $f$ is differentiable at $x=1.$
\end{prb}

\section{Derivatives of the Trigonometric Functions}

When a mathematician says something is ``true,'' s/he means that s/he can prove it from a set of axioms and definitions. This is called \emph{deductive} reasoning.  We used deductive reasoning on Problem \ref{productrule} (product rule) and on Problem \ref{quotientrule} (quotient rule) by proving (deducing) them from the definition of the derivative. When a  scientist says something is true, s/he means s/he believes it to be true based on data.  This is called {\it inductive}
reasoning because s/he {\it induces} the truth from observations. We are going to determine the derivative of $s(x) = \sin(x)$ in two different ways, inductively and deductively.
\begin{annotation}
\endnote{Experience shows that many of my students have a precarious grasp of the trigonometric functions and by this point in the course I have usually given a fifteen-minute lecture on the unit circle and the definitions, domains and ranges of the various trigonometric functions.  If I have not, I certainly do so now and use the unit circle to plot a few trigonometric functions, perhaps cosine and cotangent.}
\end{annotation}

\begin{prb}
\emph{Inductive:} Use the unit circle to fill in the second row of Table \ref{sinetable} and make a very accurate graph of the sine function using this data, labeling each of these points on the graph. Sketch the tangent line to the graph of sine at each of these points.
\end{prb}

\begin{table}[h!]
\caption{The Derivative of $s(x) = \sin(x)$} \label{sinetable}
\begin{center}
\begin{tabular}{|c|c|c|c|c|c|c|c|c|c|}
\hline
$x$ & 0 & $\pi/4$ & $\pi/2$ & $3\pi/4$ & $\pi$ & $5\pi/4$ & $3\pi/2$ & $7\pi/4$ & $2\pi$\\
\hline
$s(x)$ &&&&&&&&&\\
\hline
$s'(x)$ &&&&&&&&&\\
\hline
\end{tabular}
\end{center}
\end{table}

\begin{prb}
Approximate the slope of each tangent line you graphed in the last problem and fill in the third row of Table  \ref{sinetable} with this data.  Sketch this new function, the derivative of sine, by plotting the points $(x,s'(x))$ from the table. Does the graph of $s'$ look familiar?
\end{prb}

Based on the last two problems, we have a guess for the derivative of sine.  Now, we'll see if our guess is correct.\\

\textbf{Trigonometry Reminder -- The Sum and Difference Identities.}  Appendix \ref{apptrig} has most of the common trigonometric identities, in case you need them later.
\begin{enumerate}
\item $\sin(x \pm y) = \sin(x)\cos(y) \pm \cos(x)\sin(y)$
\item $\cos(x \pm y) = \cos(x)\cos(y) \mp \sin(x)\sin(y)$
\end{enumerate}

\begin{prb}
\emph{Deductive:} Compute the derivative of the sine function using Definition \ref{derivdfn} and the Trigonometry Reminder above.
\end{prb}

\begin{prb}
Compute the derivative of $f(x) = \cos(x).$
\end{prb}

\begin{prb}
Compute the derivatives of the tangent, cotangent, secant, and cosecant functions by using the derivatives of sine and cosine along with the quotient rule.
\end{prb}

\begin{prb}
\label{anti2}
Find a function f with derivative $f'(x) = 3\cos(x) - 2\sin(x) + x + 2.$
\end{prb}

\begin{prb}
If $\dsp y = \frac{\cos(x)}{x+3}$, compute $y'$.
\end{prb}

\begin{prb}
If $y = \sin(x)\cos(x),$ compute $y'$.
\end{prb}

\begin{prb}
Suppose that $L$ is a line tangent to $y=\cos(x)$ at the point, $(c,f(c))$ and $L$ passes through the point $(0,0).$  Show that $c$ satisfies the equation $t = -\cot(t).$ That is show that $c = -\cot(c)$.
\end{prb}

\begin{prb}
Find the equation of the line tangent to the curve $y = \tan(x) + \pi$ at the point on the curve where $x = \pi.$  Repeat this exercise for $x = \pi/4$.
\end{prb}

\section{The Chain Rule}

We can take the derivative of $\dsp f(x) = \sqrt{x}$ and $g(x) = 5x+1$ easily,  but to take the derivative of $\dsp h(x) = \sqrt{5x+1},$ we would need to return to the limit definition for the derivative.  We can take the derivative of  $f(x) = \sin(x)$ and g(x) = $x^2$ easily, but to take the derivative of $h(x) = \sin(x^2)$ we would again return to the limit  definition. The \emph{chain rule} remedies this problem by giving a rule for computing the derivatives of functions that are the composition of two functions.\\

\textbf{Algebra Reminder.}  The function $h(x) = \sqrt{5x+1}$  is the composition of two functions $f(x) = \sqrt{x}$ and  $g(x) = 5x + 1$ since we may write $f(g(x)) = f(5x +1) =  \sqrt{5x+1} = h(x).$  The function $h(x) = \sin(x^2)$ is the composition of $f(x) = \sin(x)$ and $g(x) = x^2.$  If $h$ equals $f$ composed with $g$, we write $h = f \circ g$ and $h(x) = f(g(x))$ for all numbers $x$ in the domain of $h.$

\begin{thm}
\label{chainrule}
\textbf{The Chain Rule.} If $f$ and $g$ are differentiable functions and $h(x) = f(g(x))$, then $h'(x) = f'(g(x))\cdot g'(x)$.
\end{thm}

{\it A sort-of proof.} Assume that each  of $f$ and $g$ is a differentiable function and that $h(x) = f(g(x))$. From the limit definition of the derivative of $h$, we know that
$$h'(x) = \lim_{t \rightarrow x} \frac{h(t)-h(x)}{t-x}.$$

Since $h(x) = f(g(x))$,

$$h'(x) = \lim_{t \rightarrow x} \frac{f(g(t))-f(g(x))}{t-x}.$$

Since

$$\frac{g(t)-g(x)}{g(t)-g(x)} = 1 \mbox{ as long as g(t) } \neq \mbox{ g(x) },$$

by using the {\it Product Rule for Limits} we may write,

$$h'(x) =
\lim_{t \rightarrow x} \frac{f(g(t))-f(g(x))}{g(t)-g(x)} \cdot
\frac{g(t)-g(x)}{t-x} = \lim_{t \rightarrow x}
\frac{f(g(t))-f(g(x))}{g(t)-g(x)}  \cdot \lim_{t \rightarrow x}
\frac{g(t)-g(x)}{t-x}.$$

Since $g$ is differentiable, we know by Theorem \ref{diffcont} that $g$ is continuous, so by the definition of continuity, we have $\dsp \lim_{t \rightarrow x} g(t)=g(x).$ Thus, in the first limit, we can replace $t \rightarrow x$ with $g(t) \rightarrow g(x).$

$$h'(x) = \lim_{g(t) \rightarrow g(x)} \frac{f(g(x))-f(g(t))}{g(x)-g(t)} \cdot \lim_{t \rightarrow x} \frac{g(x)-g(t)}{t-x}.$$

The right-hand side of this equation is the product of two limits. The second limit is the definition of $g'(x)$. Substituting $A = g(t)$ and $B = g(x)$ into the first limit, we may write this as,

$$\lim_{A \rightarrow B} \frac{f(B)-f(A)}{B-A}.$$

But this is just $f'(B)$, or $f'(g(x))$, so we have shown that $$h'(x) = f'(g(x))\cdot g'(x).$$
\emph{q.e.d.}

\begin{prb}
Let $h(x) = \sqrt{7+\sin(x)}.$
\begin{enumerate}
\item Find functions f and g such that $h = f \circ g.$
\item Compute $f'.$
\item Compute $f' \circ g.$
\item Compute $g'.$
\item Use the Chain Rule to compute $h'.$
\end{enumerate}

\end{prb}

\begin{prb}
\label{l22} Let $h(x) = \cos(x^2)$.
\begin{enumerate}
\item Find functions f and g such that $h = f \circ g.$
\item Compute $f'.$
\item Compute $f' \circ g.$
\item Compute $g'$.
\item Use the Chain Rule to compute $h'$.
\end{enumerate}

\end{prb}

\begin{prb}
Given that $f(x) = (x^{5/2} - 4x^{1/3}  + 365 )^{42}$, compute $f'.$
\end{prb}

\begin{prb}
If $y = (\cos(x^2) )^2$, compute $y'$.
\end{prb}

\begin{prb}
Let $\dsp y = \left( \frac{1-x^2}{1+x^2} \right)^{10}$ and compute $y'$ using the Chain Rule first and then the Quotient Rule.  Check your answer by rewriting $\dsp y = \frac{(1-x^2)^{10}}{(1+x^2)^{10}}$  and computing $y'$ using the Quotient Rule first and then the Chain Rule.
\end{prb}

\begin{prb}
\label{anti1} Find a function f with derivative $f'(x) = 5x + 3.$
\end{prb}

\begin{prb}
Find the equation of the tangent line to the curve $y = (x + 1/x)^3$ at the point where $x = -1$. Graph the curve and the line.
\end{prb}

\begin{prb}
Assume $a, b, c$ and $d$ are real numbers and $f(w) = a (\cos(wb) )^2 + c(\sin(wd) )^2.$  Compute $f'$.
\end{prb}

\begin{prb}
Find the real number m such that $y = m \cos(2t)$ satisfies the differential equation $y'' + 5y = 3\cos(2t).$
\end{prb}

\begin{prb}
Given that $f'(x) = \sqrt{2x+3},$ $g(x) = x^2 + 2$, and $F(x) = f(g(x))$, compute $F'.$
\end{prb}

\begin{prb}
Given that $\dsp f'(x) = \frac{x}{x^2-1}$ and $g(x) = \sqrt{2x-1}$, compute $F'$ where $F(x) = f(g(x))$.
\end{prb}

\section{Derivatives of the Inverse Trigonometric Functions}

\begin{dfn}
Given a function $f$ with domain D and range R, if there exists a function $g$ with domain R and range D so that for all $x \in D$ and all $y \in R$ we have $$g(f(x))=x \mbox{   and   }  f(g(y))=y,$$ then we call $g$ the \textbf{inverse} of $f$.
\end{dfn}

\begin{expl}
Discuss inverse functions and compute the inverse of $\dsp f(x) = \frac{1}{x-3}$.
\end{expl}

\begin{prb}
Find the inverse of each of the following functions.
\begin{enumerate}
\item $f(x) = \frac{2}{3}x+4$
\item $g(x) = x^3 - 1$
\item $h(x) = \sqrt[3]{x-4}$
\item $i(x) = 4 - x^5$
\end{enumerate}
\end{prb}

\begin{prb}
\begin{enumerate}
\item Does $n(x) = x^2$ have an inverse?  If so what is it?  If not, why not? \item Does $y(x) = \sqrt{x}$ have an inverse?  If so what is it?  If not, why not?
\end{enumerate}
\end{prb}

Notice in the next definition that we restrict the domain of sine to talk about the inverse sine.  This is because if we simply reversed all the coordinates of sine, the resulting graph would not be a function.   There are many ways we could have restricted the domain, so one simply must memorize the domain and range of all the inverse trigonometric functions.

\begin{dfn}
The \textbf{inverse sine} function, denoted by \textbf{$\invsin$} or \textbf{$\arcsin$} is the inverse of the function $f(x)=\sin(x)$ with the domain $[-\pi/2,\pi/2].$
\end{dfn}

The domain of the inverse sine function is the range of the sine function and the range of the inverse sine function is the restricted domain, $[-\pi/2,\pi/2]$, of our sine function. We define all six inverse trigonometric functions in a similar manner and Table \ref{t9} lists these functions along with their respective domains and ranges.

\begin{table}
\caption{Inverse Trigonometric Functions} \label{t9}
\begin{center}
\begin{tabular}{ |c|c|c| }
\hline \multicolumn{1}{|c|}{Inverse Function}&
 \multicolumn{1}{|c|}{Domain} &\multicolumn{1}{c|}{Range}\\
\hline \hline
$\invsin$ & $[-1,1]$ & $[-\pi/2,\pi/2]$  \\
\hline
$\invcos$ & $[-1,1]$ & $[0,\pi]$  \\
\hline
$\invtan$ & $\re$ & $[-\pi/2,\pi/2]$  \\
\hline
$\invcsc$ & $|x|\geq 1$ & $(0,\pi/2] \cup (\pi,3\pi/2]$  \\
\hline
$\invsec$ & $|x|\geq 1$ & $[0,\pi/2) \cup [\pi,3\pi/2)$  \\
\hline
$\invcot$ & $\re$ & $[0,\pi]$  \\
\hline
\end{tabular}
\end{center}
\end{table}


\begin{thm}
\textbf{Inverse Sine Derivative Theorem.}
\begin{annotation}
\endnote{By this section, I have already reviewed the inverse trigonometric functions and promised that there will be a problem on the next test addressing the domain, range and graph of one of them.  While one could make this section discovery-based, the total content that I must cover requires a few concessions.  You'll have noted by now that we proved the power rule only for integers.  Now we walk them through the derivative of the inverse sine function.   One reason I choose to lecture here is that the total number of important ideas that are embedded in this problem is considerable.  Of course, it reinforces the  definition of the inverse trig functions and the chain rule.  It foreshadows implicit differentiation, the idea of taking the derivative of both sides of an equation.  The triangles that we create are important for the trigonometric substitutions coming in the next semester as well.  Thus, as I go over this, I take lots of time to talk about all of the important aspects and explain that this is why they will be held responsible for the derivations of all of the inverse trigonometric functions.}
\end{annotation}
If $f(x) = \invsin(x)$, then $\dsp f'(x) = \frac{1}{\sqrt{1-x^2}}.$
\end{thm}

\textbf{Proof.} Computing the derivative of the inverse sine function is an application of the Pythagorean Theorem and the Chain Rule. Let $$f(x) = \invsin(x).$$ Since $$\sin(f(x)) = \sin(\invsin(x)) = x,$$ we have, $$\sin(f(x))=x.$$ Taking the derivative of both sides yields, $$\cos(f(x)) f'(x) = 1,$$ and solving for $f'(x)$ yields $$f'(x) = \frac{1}{\cos(f(x))}.$$ Because we don't want $f'$ in terms of $f,$ we apply the Pythagorean Theorem.  Draw a right triangle with one angle, $f(x).$ Since $$\sin(f(x))=x = \frac{x}{1}$$ place $x$ on the side opposite the angle $f(x)$ and $1$ on the hypotenuse.   Note that $$\dsp \cos(f(x)) = \frac{\sqrt{1-x^2}}{1} = \sqrt{1-x^2}.$$ Thus $$\dsp f'(x) = \frac{1}{\sqrt{1-x^2}}.$$ \emph{q.e.d.}

\begin{prb}
Compute the derivatives of the remaining five inverse trigonometric functions and memorize them.
\end{prb}

\begin{prb}
Compute the derivative of each of the following functions.
\begin{enumerate}
\item $y(x) =\invsin(2x)$

\item $H(t) = (t-\invcos(t))^{3}$

\item $F(x)=\invcsc(x)\invtan(x)$

\item $\dsp G(z) = \frac{z^{2}-7z+5}{\invcot(z)}$

\item $h(x) = \invtan(\invsin(x))$
\end{enumerate}
\end{prb}

\section{Derivatives of the Exponential and Logarithmic Functions}

We know how to take the derivative of $f(x) = x^2,$ but what about $g(x) = 2^x$ or $h(x) = e^x?$  We wish to be able to differentiate exponential and logarithmic functions.  If you need a review of these functions, then work through Appendix \ref{appexp}.
\begin{annotation}
\endnote{Even as I appreciate the logic and value of early transcendental texts, I strongly prefer teaching the integral definition for the natural log and deriving the properties of the natural log function and the exponential functions from there.  Still, we teach early transcendental functions and this is the most elegant way I know of to define $e$ under that constraint. As always, I present this slowly and interactively urging questions as I go in an attempt to maximize understanding.}
\end{annotation}

\begin{prb}
Sketch graphs of $f(x) = 2^x$ and $g(x) = 3^x$ on the same coordinate axes  labeling several points and being very careful to show which function is above the other at various points on the graph.
\end{prb}

\begin{expl}
Defining $e$ and finding the derivative of the function $e^x$.
\end{expl}

Consider finding the derivative of $f(x) = 2^x.$  Using limits to compute the derivative, we have:
\begin{eqnarray*}
f'(x) & = & \lim_{h \rightarrow 0} \frac{f(x+h)-f(x)}{h}  \;\;\;
\mbox{by definition of derivative} \cr
 & = & \lim_{h \rightarrow 0} \frac{2^{x+h}-2^x}{h} \;\;\;  \mbox{by definition of} \; \; f \cr
 & = &  \lim_{h \rightarrow 0} \frac{2^x(2^h-1)}{h} \;\;\; \mbox{by algebra} \cr
 & = &  \lim_{h \rightarrow 0} 2^x\frac{2^h-1}{h} \;\;\;   \mbox{by algebra} \cr
 & = &  2^x \lim_{h \rightarrow 0} \frac{2^h-1}{h} \;\;\;  \mbox{by the {\it Constant Multiple Rule for limits}} \cr
& = &  2^x f'(0)
\end{eqnarray*}
We would now know the derivative of the function $2^x$ if only we could determine $f'(0) = \dsp{\lim_{h \rightarrow 0} \frac{2^{h}-1}{h}}.$ We can approximate $f'(0) = \dsp{\lim_{h \rightarrow 0} \frac{2^{h}-1}{h}} \approx .69$ by using a limit table. Similarly, if we went through the same process for $g(x) = 3^x$ we would see that $g'(0) = \dsp{\lim_{h \rightarrow 0} \frac{3^{h}-1}{h}} \approx 1.09$ by using a limit table. Since the slope of $f$ at 0 is less than 1 and the slope of $g$ at 0 is greater than one, it makes sense that there should be an exponential function which has slope \emph{exactly} 1 at 0. We define a number, called $e$, to be the number (between 2 and 3) so that $\dsp{\lim_{h \rightarrow 0} \frac{e^{h}-1}{h}} = 1.$  Now, if we let $j(x) = e^x$ then we have
\begin{eqnarray*}
j'(x) & = & \lim_{h \rightarrow 0} \frac{j(x+h)-j(x)}{h}  \;\;\;
\mbox{by definition of derivative} \cr
 & = & \lim_{h \rightarrow 0} \frac{e^{x+h}-e^x}{h} \;\;\;  \mbox{by definition of} \; \; f \cr
 & = &  \lim_{h \rightarrow 0} \frac{e^x(e^h-1)}{h} \;\;\; \mbox{by algebra} \cr
 & = &  \lim_{h \rightarrow 0} e^x\frac{e^h-1}{h} \;\;\;   \mbox{by algebra} \cr
 & = &  e^x \lim_{h \rightarrow 0} \frac{e^h-1}{h} \;\;\;  \mbox{by the {\it Constant Multiple Rule for limits}} \cr
& = &  e^x f'(0)\cr
&=& e^x.
\end{eqnarray*}

This work results in a definition and a theorem.

\begin{dfn}
Let $e$ be the number that satisfies $\dsp \lim_{h \to 0} \frac{e^h-1}{h} = 1$.
\end{dfn}

\begin{thm}
\textbf{Exponential Derivative Theorem.} If $f(x) = e^x$, then $f'(x) = e^x$.
\end{thm}

\begin{dfn}
Let $g(x)=\ln(x)$ be the function that is the inverse of the function $f(x) = e^x$.
\end{dfn}

\textbf{Algebra Reminders}
\begin{enumerate}
\item $f(x)=e^x$ has domain all $x \in \re$ and range all $x > 0$
\item $g(x)=\ln(x)$ has domain all $x > 0$ and range all $x \in \re$
\item $e^{x+y} = e^x e^y$ and ${(e^x)}^{^y} = e^{xy}$ for all $x,y \in \re$
\item $\ln(xy) = \ln(x) + \ln(y)$  for all $x>0$, $y>0$
\item $\ln(x^y) = y\ln(x)$ for all $y \in \re$ and $x>0$
\item $e^{\ln(x)} = x$ for all $x > 0$
\item $\ln(e^x)=x$ for all $x \in \re$
\item $\log_b(x) = \ln(x)/ln(b)$ for all $b \in \re$ (except $b$=1) and for all $x>0$
\end{enumerate}

\begin{prb}
Compute and simplify the derivatives of the following functions:
\begin{enumerate}
\item $g(x) = e^x  + x^2  +  2x$
\item $f(x) = xe^x$
\item $h(x) = x^3/e^x$
\item $f(x) = e^{x^2}$
\item $f(x) = 3x^2e^{x^4}$
\end{enumerate}
\end{prb}

\begin{prb}
Let $g(x) = \ln(x)$.  Here are two clever ways to compute the derivative of $g$; do which ever you like better!
\begin{enumerate}
\item Since we know that $g(e^x)=x,$ differentiate both sides of this equation using the Chain Rule and then solve for $g'(x).$
\item Apply the exponential function to both sides of the equations $g(x)=\ln(x)$, differentiate, and then solve for $g'(x).$
\end{enumerate}
\end{prb}

\begin{prb}
Compute and simplify the derivative of the following functions:
\begin{enumerate}
\item $g(x) = \ln(7x-8)$
\item $f(x) = x^2\ln(x)$
\item $h(x) = x^3/\ln(3x^2 - x^{1/3})$
\end{enumerate}
\end{prb}

\begin{prb} \label{logdiff}
Let $h(x) = 2^x$ and compute $h'$ first by taking the natural log of both sides and then taking the derivative of both sides.
\end{prb}

\begin{prb}
Compute and simplify the derivative of the following functions:
\begin{enumerate}
\item $g(x) = 2^{7x-8}$
\item $f(x) = 2^x - x^2$
\item $h(x) = 3^{x^2}$
\end{enumerate}\end{prb}

\begin{prb}
Use the change-of-base formula, $\log_b(x) = \ln(x)/\ln(b)$, to compute the derivative of $L(x) = \log_3(x^5-2x).$
\end{prb}

\section{Logarithmic and Implicit Differentiation}

We know how to take the derivative of $f(x) = x^2$ and $g(x) = 2^x.$  But what about functions like $h(x) = x^{x}$ where the variable occurs in both the base and the exponent?  Fortunately, the technique you used in Problem \ref{logdiff}, called \textbf{logarithmic differentiation}, and can be applied to this function as well.
\begin{expl}
Compute the derivative of $f(x) = x^{2x}$.
\end{expl}
Define $f$.
    $$f(x)=x^{2x}$$
Apply the natural logarithm to both sides.
    $$\ln(f(x))=\ln(x^{2x})$$
Apply a property of logarithms to the right-hand side.
    $$\ln(f(x))=2x\ln(x)$$
Take the derivative of both sides.
    $$\frac{f'(x)}{f(x)}=2\ln(x)+(2x)\frac{1}{x} = 2\ln(x) + 2$$
Multiply through by $f(x).$
    $$f'(x)=f(x)\left( 2\ln(x)+2 \right)$$
Substitute $f(x) = x^{2x}$ and simplify.
    $$f'(x)=x^{2x}\left( 2\ln(x)+2 \right) = 2x^{2x} \left( \ln(x) + 1 \right)$$

\begin{prb}
Compute the derivative of $y(x) = (3x)^{x+1}.$
\end{prb}

\begin{prb}
Compute $\dsp H'(x)$ for $\dsp H(x) = \left(\sin(x) \right)^{\cos(2x)}.$
\end{prb}

\begin{expl}
Computing derivatives of relations that are not functions.
\end{expl}

\textbf{Implicit Differentiation.} Thus far we have only taken the derivatives of functions, and the derivatives of these functions were also functions. Implicit differentiation will allow us to find an equation that tells us the slopes of the lines tangent to an equation that does not represent a function, for example, a circle.  For a single value of $x$, a circle may have two values for $y$ and thus two tangent lines.  Therefore, our ``derivative'' will not be a function because for a single value of $x$ there may be two lines tangent to the circle.

Let's look at an example.  The circle $x^2 + y^2 = 9$ is a relation (a set of ordered pairs), but {\it not} a function since $(1, 2\sqrt{2})$ and $(1,-2\sqrt{2})$ are points that satisfy the equation but have the same $x$-coordinate and distinct $y$-coordinates.  How can we determine the slope of any tangent line to this equation, recalling that the derivative was, thus far, defined only for functions?  For this problem, we can do it in two ways, straight  differentiation and implicit differentiation.\\

\textbf{Straight Differentiation.}  First, rewrite the equation, $$ x^2 + y^2 = 9,$$ as, $$ y = \pm \sqrt{9-x^2}.$$  This may be written in  what {\it looks} like functional notation as, $$ f(x) = \pm \sqrt{9-x^2},$$ but $f$ is {\it not} a function since for each value of $x \in (-3,3),$ we have two values for $f(x).$  Still, taking the derivative of either $$ f(x) = + \sqrt{9-x^2} \;\;\;\; \mbox{or}  \;\;\;\; f(x) = - \sqrt{9-x^2}$$ is exactly the same except for the sign, so we do both simultaneously: $$ f'(x) = \pm \frac{1}{2} (9-x^2)^{-\frac{1}{2}} \cdot -2x = \pm \frac{x}{\sqrt{9-x^2}}.$$

Implicit differentiation is an alternative way to differentiate such equations by first taking the derivative of  both sides of the equation and then solving the resulting equation for the derivative instead of solving for $y$ and then
taking the derivative.\\

\textbf{Implicit Differentiation.} We again start with $$ x^2 + y^2 = 9.$$ Replacing $y$ with $f(x)$ as we did in our first example yields, $$ x^2 + \left( f(x) \right)^2 = 9.$$ Differentiating both sides using the Power, Chain, and Constant rules yields, $$2x + 2f(x)f'(x) = 0.$$ Solving for $f'(x)$ we have, $$f'(x) = \frac{-2x}{2f(x)} = \frac{-x}{f(x)}.$$  This
does not look the same as the answer we computed using straight differentiation, but if we again solve
$$ x^2 + \left( f(x) \right)^2 = 9$$ for $f(x)$ to get $$ f(x) = \pm \sqrt{9-x^2},$$ and substitute this into our answer
$$f'(x) = \frac{-x}{f(x)}$$ then we have $$f'(x) =  \frac{-x}{\pm \sqrt{9-x^2}} = \pm \frac{x}{\sqrt{9-x^2}}.$$
This agrees with the result we obtained using straight differentiation.

\begin{prb}
Find $y'$ where $x^3 + y^3  = 1$ in two ways.  First, solve for $y$ and differentiate. Second, differentiate and then solve for $y'.$  Show that both answers are the same. Graph the set of points that satisfy $x^3 + y^3  = 1.$ Is this a function?
\end{prb}

\begin{prb}
Find $y'$ for $x^3  + y^3 = 3xy$.
\end{prb}

This problem is based on a true story.  Only the type of invertebrate and the name of the snack food have been changed in order to protect their identities.

\begin{prb}
You sketch an x-y-coordinate plane on your notebook and then note an ant walking around the page nibbling on your Cheeto crumbs.  The ant's position changes with time, so you define a function $p$ so that the position of the ant at time $t$ is given by the function $p(t)=(x(t),y(t))$.  After considerable observation, you observe that the position function satisfies $2(x(t))^2 + 3(y(t))^2 = 100$, the speed at $t=5$ in the $x$-direction is 2 (i.e. $x'(5)=2$), and the $x$-position at time $t=5$ is 6 (i.e. $x(5)=6$).  Compute the ant's $y$ position at time $t=5$ (i.e. $y(5)$) and the ant's speed in the $y$ direction at time $t=5$ (i.e.  $y'(5)$).
\end{prb}

\begin{prb}
Given the curve $x^2 - 4x  + y^2  - 2y = 4,$ find:
\begin{enumerate}
\item the two points on the curve where there are horizontal tangent lines, and
\item the points on the curve where the tangent lines are vertical.
\end{enumerate}
If you feel lucky, try solving this problem without Calculus by completing the square.
\end{prb}

\begin{prb}
Find the equations for the two lines that pass through the origin and are tangent to the curve defined by $x = y^2  + 1.$  Illustrate with a graph of the curve and tangent lines.
\end{prb}

We proved the Power Rule only for functions of the form $f(x) = x^n$ where $n$ is an integer.  However, we have been using it where $n$ was any number, for example $n=1/2$ in the case where $f(x) = \sqrt{x}$.   We will now use the Chain Rule, Implicit and Logarithmic Differentiation to prove that the Power Rule is valid for \emph{rational} exponents.

\begin{prb}
Use implicit differentiation to find $y'$ if $y = x^{4/3}$.
\end{prb}

\begin{prb}
Suppose $\frac{p}{q}$ is a rational number. Show that the derivative of the function $f(x) = x^{p/q}$ is  $f'(x) = \frac{p}{q}x^{\frac{p}{q}-1}$.
\end{prb}

\begin{prb}
Find the second derivative, $y''$, for the curve $x^3 + y^3 = 1.$
\end{prb}

\begin{prb}
Use implicit differentiation to find the equation of the line tangent to the curve  $x^{2/3} - y^{2/3} - y  = 4$ at the point $(8,0).$
\end{prb}

\section{Practice} \label{chap2probs}

We will not present the problems from this section, although you are welcome to ask about them in class.

\begin{enumerate}
\item Graph these functions and their derivatives.

\begin{enumerate}
\item $F(x)=-x^{2}+5$
\item $h(x) = x^3 - 9x$
\item $f(x)=\cos(2x)$
\item $g(t)=|2+3t|$
\end{enumerate}

\item Compute the derivative of each function.

\begin{enumerate}
\item $\dsp g(t)=\sqrt{3t}+\frac{3}{t}-\frac{5}{t^{3}}$
\item $z(t) = (x^2+1)(x^3+2x)$
\item $\dsp y = \frac{2x^3+x}{2-x^2}$
\item $g(t)=(2t^{2}-5t^{8}+4t)^{12}$
\item  $F(z)=\sqrt[3]{2z+7z^{3}}$
\end{enumerate}

\item Compute and simplify the derivatives of these exponential and logarithmic functions.

\begin{enumerate}
\item $a(x) = 12-e^{x^2}$
\item $b(x) = \ln(x^2)$
\item $c(x) = e^{\sqrt{x^2-1}}$
\item $d(x) = x\ln(x^3)$
\item $\dsp g(t)=\frac{t+e^{t}}{3t^{2}-10t+5}$
\item $\dsp f(t) = \frac{1+t}{e^{t}}$
\item $n(x) = \ln(xe^{x^2})$
\end{enumerate}

\item Compute and simplify the derivatives of these trigonometric functions.

\begin{enumerate}
\item $C(\beta )=3\beta \cos(5\beta)$
\item $H(x)=(5x^{2}+x)(4x^{4}+\tan(x))$
\item $o(x) = \csc(x)\cot(x)$
\item $p(x) = \sin(\cos(x))$
\item $r(x) = \left(\sin(x)/x \right)^3$
\item $i(x) = \sin^3(x)/x^3$
\end{enumerate}

\item Compute and simplify the derivatives of these mixed functions.

\begin{enumerate}
\item $g(t) = t\sin(t)e^t$
\item $h(x) = \tan(x^3-3x)$
\item $k(t) = \invcos(x^2+3x)$
\item $m(x) = \invsin(5x)$
\item $\dsp q(t) = e^{\sec(t^2-2t)}$
\item $\dsp s(x) = 3^x + x^3$
\item $\dsp t(y) = 2^{y^2-3y}$
\item $\dsp u(x) = x3^{\sin(x)}$
\item  $\dsp G(x)=(x+1)^{\sin(x)}$
\end{enumerate}

\item Each hyperbolic trigonometric function is defined. Verify that each stated derivative is correct.

\begin{enumerate}
\item Definition: $\dsp \sinh(x)=\frac{e^{x}-e^{-x}}{2} \;\;\;\;\;\;\;\;$ Show: $(\sinh(x))'=\cosh(x)$
\item Definition: $\dsp \cosh(x)=\frac{e^{x}+e^{-x}}{2}\;\;\;\;\;\;\;\;$  Show:  $(\cosh(x))'=\sinh(x)$
\item Definition: $\dsp \tanh(x)=\frac{\sinh(x)}{\cosh(x)} \;\;\;\;$  Show: $(\tanh(x))'=\sech^{2}(x)$
\item Definition: $\dsp \csch(x)=\frac{1}{\sinh(x)} \;\;\;\;$ Show:  $(\csch(x))'=-\csch(x)\coth(x)$
\item Definition: $\dsp \sech(x)=\frac{1}{\cosh(x)}  \;\;\;\;$ Show:  $(\sech(x))'=-\sech(x)\tanh(x)$
\item Definition: $\dsp \coth(x)=\frac{1}{\tanh(x)} \;\;\;\;$  Show:  $(\coth(x))'=-\csch ^{2}(x)$
\end{enumerate}

\item Show that $\invsinh(x)=\ln(x+\sqrt{x^{2}+1}).$
\item Show that $\dsp (\invsinh(x))'=\frac{1}{\sqrt{x^{2}+1}}$.
\item  Compute the slope of the tangent line to $\dsp f(x)=\frac{2x}{2^{x}}$ when $x=0$.
\item  Compute the equation of the tangent line to $g(x)=x\tan(x)$ when $x=\frac{\pi}{3}$.
\end{enumerate}

\vskip .5in
\noindent
\textbf{Chapter 2 Solutions}\\ \\

\begin{enumerate}

\item Graph these functions and their derivatives.

\begin{enumerate}
\item $F'(x)=-2x$
\item $h'(x) = 3x^2-9$
\item $f'(x)=-2\sin(2x)$
\item $g'(t)=3(2+3t)/|2+3t|$
\end{enumerate}

\item Compute the derivative of each function.

\begin{enumerate}
\item $\dsp g'(t)=\frac{\sqrt{3}}{2\sqrt{t}}-\frac{3}{t^2} + \frac{15}{t^4}$
\item $z'(t) = 5x^4+9x^2+2$
\item $\dsp y' = -\frac{2x^4-13x^2-2}{(x^2-2)^2}$
\item $g'(t)=48t^{11}(5t^7-2t-4)^{11}(10t^7-t-1)$
\item  $\dsp F'(z)=\frac{2+21z^2}{3\sqrt[3]{(2z+7z^3)^2}}$
\end{enumerate}

\item Compute and simplify the derivatives of these exponential and logarithmic functions.

\begin{enumerate}
\item $a'(x) = -2xe^{x^2}$
\item $b'(x) = 2/x$
\item $\dsp c'(x) = \frac{xe^{\sqrt{x^2-1}}}{\sqrt{x^2-1}}$
\item $d'(x) = 3 +\ln(x^3)$
\item $ \dsp g'(t)=\frac{(3t^2-16t+15)e^t - 3t^2+5}{(3t^2-10t+5)^2}$
\item $\dsp f'(t) = \frac{-t}{e^{t}}$
\item $n'(x) = 2x + 1/x$
\end{enumerate}

\item Compute and simplify the derivatives of these trigonometric functions.

\begin{enumerate}
\item $C'(\beta )=3( \cos(5\beta) -5\beta \sin(5\beta))$
\item $H'(x)=(10x+1)(4x^4+\tan(x))+(5x^2+x)(16x^3+\sec^2(x))$
\item $\dsp o'(x) = -\frac{\cos^2(x)+1}{\sin^3(x)}$
\item $p'(x) = -\sin(x)\cos(\cos(x))$
\item $\dsp r'(x) = 3\sin^2(x)\frac{x \cos(x)-\sin(x)}{x^4}$
\item $i'(x) = \;\;\; \mbox{ same answer as last one!}$
\end{enumerate}

\item Compute and simplify the derivatives of these mixed functions.

\begin{enumerate}
\item $g'(t) = te^t\cos(t) + (t+1)e^t\sin(t)$
\item $h'(x) = (3x^2-3)\sec^2(x^3-3x)$
\item $\dsp k'(t) = \frac{2x+3}{\sqrt{1-(x^2+3x)^2}}$
\item $\dsp m'(x) = \frac{5}{\sqrt{1-25x^2}}$
\item $q'(t) = (2t-2)e^{\sec(t^2-2t)}\sec(t^2-2t)\tan(t^2-2t)$
\item $s'(x) = 3x^2 + 3^x\ln(3)$
\item $t'(y) = \ln(2)(2y-3)2^{y^2-3y}$
\item $u'(x) = 3^{\sin(x)} + \ln(3)x\cos(x)3^{\sin(x)}$
\item  $G'(x)=(x+1)^{\sin(x)}\left( \cos(x)\ln(x+1) + \sin(x)/(x+1) \right)$
\end{enumerate}

\item No solution.
\item No solution.
\item No solution.
\item  $f'(0) = 2.$
\item $g'(\pi/3) = \sqrt(3) + 4\pi/3$
\end{enumerate}


\chapter{Applications of Derivatives}

``I am always doing that which I can not do, in order that I may learn how to do it.''  - Pablo Picasso\\ \\

Now that we have a solid understanding of what derivatives are and how to compute them for common functions, let's look at some applications.  Our applications will be both mathematical (using derivatives to compute limits and graph functions) and industrial (using derivatives to minimize cost or maximize fuel economy).

\begin{annotation}
\endnote{I often skip the first section in this chapter.  The student who understands Taylor polynomials as I introduce them, by having the students derive the best linear and quadratic approximations, should be able to approximate functions.  Still, engineers may see these techniques, so when time permits I cover it.}
\end{annotation}

\section{Linear Approximations}

Suppose we have a function $f$ and a point $(p,f(p))$ on $f.$  Suppose $L$ is the tangent line to $f$ at that point.  $L$ is the \emph{best linear approximation} to $f$ at  $(p,f(p))$ since $f(p) = L(p)$ and $f'(p) = L'(p)$ and no line can do better than agreeing with $f$ both in the y-coordinate and the derivative.  Therefore the tangent line to $f$ at $p$ is often called the best linear approximation or the \emph{linearization} of $f$ at $p.$

\begin{prb}
Find and sketch the linearization of the function at the indicated point.
\begin{enumerate}
\item $f(x)=3x^{2}-x+2$ at $x=-2$.
\item $g(x)=\sin(x)$ at $x=\pi/4$.
\end{enumerate}
\end{prb}

\begin{prb}
Suppose $f$ is a differentiable function and $(p,f(p))$ is a point on $f.$  Find the equation of the tangent line to $f$ at the point $(p,f(p)).$
\end{prb}

\begin{prb}
Suppose $f(x) = x^3.$  Find a parabola $p(x) = ax^2 + bx + c$ so that $f(1) = p(1)$ and $f'(1) = p'(1)$ and $f''(1) = p''(1).$  We would call $p$ the best quadratic approximation to $f$ at $(1,f(1)).$
\end{prb}

\begin{expl}
Approximating changes using linear approximations.
\end{expl}

Suppose we have an ice cube floating in space that is expanding as cosmic dust forms on its exterior. Suppose the cube has an initial side length of $s=3$ inches and we wish to approximate how much the volume ($\dsp V=s^3$) will change as the length of the side increases by $.1$ inch.  Of course, we can compute the change in volume directly.  But we can also approximate it using the fact that the derivative tells us the rate of change of the volume.   Recall that $$V'(3) \approx \frac{V(3.1) - V(3)}{3.1 - 3}.$$  If we think of the change in volume as $\Delta V = V(3.1) - V(3)$ and the change in side length as $\Delta s = 3.1 - 3$, then $$V'(3) \approx \frac{\Delta V}{\Delta s}.$$  So, $$\Delta V \approx V'(s)  \Delta s = 3(3)^2 * .1 = 2.7$$ and we expect the volume to change by this amount. In a case where you knew the rate of change of the volume, but did not know a formula for volume you could use this to estimate the change in volume over a small change in side length.

\begin{prb}
Compute the exact change in volume of the cube from the previous discussion and compare to our estimate.
\end{prb}

\begin{prb}
Suppose the radius of a circle $C$ is to be increased from an initial value of 10 by an amount $\Delta r = .01$. Estimate the corresponding increase in the circle's area ($\dsp A=\pi r^2$) and compare it to the exact change.
\end{prb}

\begin{prb}
The surface area of a sphere ($S=4\pi r^{2}$) is expanding as air is pumped in. Approximate the change in the surface area as the radius expands from $r=2.5$ to $r=2.6$ and compare this to the exact change.
\end{prb}

\section{Limits involving Infinity and Asymptotes}

In our original discussion of $lim_{x\to a} f(x) = L$ we required that both $a$ and $L$ be real numbers.  We now wish to consider the possibilities where $a= \pm \infty$.  To determine $lim_{x \to \infty} f(x)$ is to ask what is happening to the values of $f(x)$ as $x$ becomes large and positive. One possibility is that $f(x)$ tends toward a real number as $x$ tends toward infinity.  In this case we write $lim_{x\to \infty} f(x) = L$ and $y=L$ is a horizontal asymptote.  Another possibility is that $f(x)$ grows without bound as $x$ tends to infinity.  In this case we write $lim_{x \to \infty} f(x) = DNE(\infty)$ to indicate that the limit does not exist but $f(x)$ tends to infinity.  \emph{A limit does not exist unless it is a real number and infinity is not a number.}

Now let's return to the case where $a$ is a real number.  It is possible that $f(x)$ grows without bound as $x$ approaches $a.$ In this case we write $lim_{x \to a} f(x) = DNE(\infty)$. Now we know that the limit does not exist and $x=a$ is a vertical asymptote.

Our definition for asymptote will be intuitive, like our definition for limits, because we don't give a mathematically valid explanation of what the word ``approaches'' means.  Still, it will serve our purposes.  A more precise definition might appear in a course titled ``Real Analysis.''
\begin{annotation}
\endnote{In what follows, I try to make clear to the students the difference between the intuitive definitions (non-definitions) such as the one for \emph{asymptote} and the truly mathematically valid definitions, such as the one for $lim_{x\to +\infty} f(x)=L$.  While I do not expect every student in the class to appreciate the distinction, I am capable of identifying those that strive for the deeper understanding and thus even if I do not test over the precise definitions, by including them I am able to use them as a litmus test for those future mathematicians in the room.   I am honest with the class and tell them, when they ask, which definitions I expect them to know and which ones are really beyond the scope of the course and added for those seeking a bit extra.  When students attempt to treat infinity as a number, I will discuss briefly the extended reals and then refer them to Appendix \ref{appreal}}.
\end{annotation}

\begin{dfn}
\label{asymptote}
An \textbf{asymptote} of a function $f$ is any curve (or line) that the graph of $f$ approaches. A \textbf{horizontal asymptote} of $f$ is a horizontal line that the graph of $f$ approaches. A \textbf{vertical asymptote} of $f$ is a vertical line that the graph of $f$ approaches.
\end{dfn}

\begin{prb}
Graph $\dsp f(x)=\frac{3x}{x-2}$. What number does $f(x)$ approach as $x$ becomes large and positive? As $x$ becomes large and negative?  What happens to the values of $f(x)$ as $x$ approaches 2 from the left?  From the right?
\end{prb}

\begin{dfn} \label{limhorizontal}
\textbf{Intuitive Definition} We say that $lim_{x\to +\infty} f(x)=L$ provided that as $x$ becomes arbitrarily large and positive, the values of $f(x)$ become arbitrarily closer to $L$.
\end{dfn}

\begin{dfn} \label{liminfinity}
\textbf{Another Intuitive Definition} We say that $lim_{x\to \infty} f(x)=\infty$ provided that as $x$ becomes arbitrarily large and positive the values of $f(x)$ become arbitrarily large and positive.
\end{dfn}

For the problems in this section, you may justify your answer using a graph or some algebra or limit tables or simply a sentence defending your answer, such as ``the denominator gets larger and larger and the numerator is constant, so the limit must be zero.''

\begin{prb} Evaluate each of the following limits.
\begin{enumerate}
\item   $\dsp \lim_{x\to \infty} \frac{2x}{5x+3}$
\item   $\dsp \lim_{x\to \infty } x^{2}+5$
\item $\dsp \lim_{t\to \infty } \frac{4t+cos(t)}{t}$
\end{enumerate}
\end{prb}

\begin{prb} Evaluate each of the following limits or state why they do not exist.
\begin{enumerate}
\item   $\dsp \lim_{t\to -\infty } 2t^{3}+6$
\item   $\dsp \lim_{z\to \infty } \frac{\sin(z)+7z}{1-z^{2}}$
\item $\dsp \lim_{\alpha \to -\infty } 3\sin(\alpha )$
\end{enumerate}
\end{prb}

Here are precise restatements of our intuitive definitions.

\begin{dfn}
\textbf{Precise Restatement of Definition \ref{limhorizontal}} Let $L$ be a real number. We say that $lim_{x\to \infty} f(x)=L$ provided for each positive real number $\epsilon$ there is a positive real number $M$ such that for every $x>M$, we have $|f(x)-L|<\epsilon $.
\end{dfn}

\begin{dfn}
\textbf{Precise Restatement of Definition \ref{liminfinity}} We say that $lim_{x\to \infty} f(x)=\infty$ provided for each positive real number $N$ there is a positive real number $M$ such that for every $x>M$, we have $f(x)>N$.
\end{dfn}

\begin{prb}
Based on the definitions above, write an intuitive definition and a precise definition for $lim_{x\to \infty} f(x)=-\infty$
\end{prb}

\begin{prb}
Suppose $k$ is some real number and $P$ be the second-degree polynomial $P(x) = kx^2$.  Does the $\lim_{x \to \infty} P(x)$ depend on the value of $k$?  What is $\lim_{x\to \infty } P(x)$?  What is $\lim_{x\to -\infty } P(x)$? Suppose $C(x)=kx^3$. What is $\lim_{x\to \infty } C(x)$? What is $\lim_{x\to -\infty } C(x)$?
\end{prb}

\begin{dfn} \textbf{Yet Another Intuitive Definition} \label{limvertical}
We say that $lim_{x\to a} f(x)=\infty$ if when $x$ approaches $a,$ the values of $f(x)$ become arbitrarily large.  \end{dfn}

The next definition makes this idea precise.

\begin{dfn}
\textbf{Precise Restatement of Definition \ref{limvertical}} Let $a$ be a real number. We say that $lim_{x\to a} f(x)=\infty$ provided for each positive real number $N$ there is positive number $\epsilon$ so that for all $x$ satisfying $|x-a|<\epsilon$, we have $f(x)>M$
\end{dfn}

\begin{prb}
Evaluate each of the following limits. Graph each function; include any vertical asymptotes.
\begin{enumerate}
    \item  $\dsp \lim_{t\to -2} \frac{3}{(2+t)^{2}}$
    \item   $\dsp \lim_{x\to 1^{+} } \frac{x^{2}-5x+6}{x^{3}-1}$
\end{enumerate}
\end{prb}

\begin{prb} Evaluate each of the following limits.
\begin{enumerate}

\item   $\dsp \lim_{x\to 0^{-}} \frac{1}{x}$ and $\dsp \lim_{x\to 0^{+}}
\frac{1}{x}$

\item $\dsp \lim_{x\to -4^{-} } \frac{x-1}{x+4}$ and $\dsp \lim_{x\to -4^{+}
} \frac{x-1}{x+4}$

\item $\dsp \lim_{x\to 0 }\sin(\frac{1}{x})$
\end{enumerate}
\end{prb}

\begin{prb}  \label{hor} Find the asymptotes of each function and use them to sketch a rough graph of each.
\begin{enumerate}
\item   $\dsp G(x)=\frac{x-1}{x+4}$
\item   $\dsp F(x)=\frac{3x}{5x-1}$
\item   $\dsp g(x)=e^{-x}-2$
\item   $\dsp h(z)=-\frac{2z+\sin(z)}{z}$
\end{enumerate}
\end{prb}

\begin{prb} \label{vert}
Write a statement that explains the relationship between limits and vertical asymptotes.  Write a statement that explains the relationship between limits and horizontal asymptotes.
\end{prb}


\section{Graphing}

Prior to an understanding of derivatives, graphing was largely a matter of:
\begin{enumerate}
\item plotting points,
\item memorizing the basic shapes of certain types of functions, and
\item trusting the graphs generated by technology.
\end{enumerate}
With algebra and differentiation at our disposal, we can make a precise science out of graphing that will give more information than even our graphing software yields.
\begin{annotation}
\endnote{Graphing was such a revelation to me when I first saw it that I like to give serious attention to graphing of some non-trivial problems.  There are some challenging ones in this section and more challenging ones in the practice problems.  It is not uncommon for me to give a choice of two or three problems and have students turn in a full graph, listing everything they can and explaining all the mathematics behind how they found the information related to the graph.  This may serve as a presentation grade or as points added to a test.}
\end{annotation}

\begin{expl}
I will work a few of these examples to provide intuition for the upcoming definitions.
\begin{enumerate}
\item $p(x) = x^4-13x^2+36$
\item $\dsp q(x)=\frac{3}{x^2+1}$
\item $f(x) = 2x^{5/3}-5x^{4/3}$
\item $\dsp r(x)=\frac{x^2-1}{2x+1}$
\item $\dsp F(t)=t^{2/3}-3$
\item $\dsp h(x)=\frac{3x^2}{x^{2}+1}$
\end{enumerate}
\end{expl}

We list several questions that you should ask when graphing a function.\\

\textbf{Question 1.} What are the intercepts and asymptotes of the function?\\

Recall that an $x$-intercept is a point where the function crosses the $x$-axis (i.e. $y=0$) and a $y$-intercept is a point where the function crosses the $y$-axis (i.e. $x=0$).

Recall that asymptotes (Definition \ref{asymptote}) are functions (usually lines) that $f$ approaches. Limits are helpful in understanding and locating asymptotes. Review Problem \ref{hor} and Problem \ref{vert} when studying asymptotes.\\

\begin{prb}
List any intercepts and asymptotes for:
\begin{enumerate}
\item $\dsp f(x) = \frac{x}{(x-2)(x+3)}$
\item $E(x) = 3-e^{-x}$
\item $\dsp g(x) = \frac{3x}{\sqrt{x^2+3}}$
\end{enumerate}
\end{prb}

\textbf{Question 2.} What are the local maxima and minima of the function?\\

The peaks and the valleys are called local maxima and local minima, respectively.

\begin{dfn}
If f is a function, then we say that the point $(c,f(c))$ is a \textbf{local maximum} for f if there is some interval, (a,b), containing c so that $f(x) < f(c)$ for all $x \in (a,b)$ except $x=c.$  Local minimum is defined similarly.
\end{dfn}

We have seen examples of local minima and maxima.  Problem \ref{absolutederiv} had a minimum at a point where the derivative did not exist.  In problem \ref{l26} we found the local maxima and minima by finding the points where the tangent lines were horizontal.  Hence points on the graph where the derivative is either zero or undefined are important points.

\begin{dfn}
If $f$ is a function, then any value $x$ in the domain of $f$ where $f'(x)$ is zero or undefined is called a  \textbf{critical value} of $f$. We refer to $(x,f(x))$ as a \textbf{critical point} of $f.$
\end{dfn}

Critical points give us candidates for local maximum or minimum, but they do not guarantee a maximum of minimum.  The function $f(x) = x^3 + 2$ has $(0,2)$ as a critical point, but this point is neither a minimum nor a maximum.\\

\begin{prb}
Find the critical values for these functions.
\begin{enumerate}
\item $g(x) = \frac{1}{3}x^3 + \frac{1}{2}x^2 - 12x$
\item $f(x) = \sin(2x)$ on $[0,2\pi]$
\item $s(x) = (x-3)^{1/3}$
\end{enumerate}
\end{prb}

\textbf{Question 3.} Where is the function increasing? Decreasing?\\

Determining where a function is increasing or decreasing is perhaps the most useful graphing tool, so we'll call it \emph{the Fundamental Theorem of Graphing.}

\begin{dfn}
We say that a function $f$ is \textbf{increasing} if, for every pair of numbers $x$ and $y$ in the domain of $f$ with $x<y$, we have $f(x) < f(y).$ Decreasing, non-increasing, and non-decreasing are defined similarly.
\end{dfn}

If the slopes of all tangent lines to the function $f$ are positive over an interval $(a,b)$, then $f$ is increasing on the interval $(a,b)$ and if the slopes of all tangent lines to the function $f$ are negative over an interval $(a,b)$, then $f$ is decreasing on the interval $(a,b)$.

\begin{thm}
\textbf{The Fundamental Theorem of Graphing.} If $f$ is a function and is differentiable on the interval $(a,b)$ and $f'(x) > 0$ for all $x$ in $(a,b)$, then $f$ is increasing on $(a,b)$.
\end{thm}

\begin{prb}
These are the functions you found the critical values for in the last problem.  List the intervals on which these functions are increasing by finding where $f'>0$.
\begin{enumerate}
\item $g(x) = \frac{1}{3}x^3 + \frac{1}{2}x^2 - 12x$
\item $f(x) = \sin(2x)$ on $[0,2\pi]$
\end{enumerate}
\end{prb}

\textbf{Question 4.} Where is the function concave up? Concave down?\\

Sketch $f(x) = x^3-x.$  The graph of $f$ looks somewhat like a parabola opening downward on the interval $(-\infty, 0)$ and somewhat like a parabola opening upward on the interval $(0,\infty)$. We say that $f$ is {\it concave down} on $(-\infty, 0)$ and {\it concave up} on $(0,\infty)$.  The defining characteristic for concavity is that $f$ is concave down when the slopes of the tangent lines are decreasing and $f$ is concave up when the slopes are increasing.

\begin{dfn}
We say that $f$ is \textbf{concave up} on $(a,b)$ if $f'$ is increasing on $(a,b)$ and \textbf{concave down} on $(a,b)$ if $f'$ is decreasing on $(a,b)$.
\end{dfn}

Applying the \emph{Fundamental Theorem of Graphing}, $f$ will be concave up when $f'$ is increasing which will be when $f''$ is positive.\\

\textbf{Question 5.} What are the inflection points for the function?\\

Inflection points are points on the curve in which the function switches either from concave up to concave down, or from concave down to concave up.

\begin{dfn}
We say that $(x,f(x))$ is an \textbf{inflection point} for $f$ if the concavity of $f$ changes sign at $x.$
\end{dfn}

\begin{prb}
Let $\dsp f(x) = \frac{6}{x^2+3}$.  Find the values where $f''=0$.  Find the intervals where $f$ is concave up. List the inflection points.
\end{prb}

\textbf{Question 6.} What is the least upper bound of the function? The greatest lower bound?\\

From the graph of $\dsp f(x) = \frac{6}{x^2+1}$ in the last problem we see that $(0,6)$ is the highest point on the graph, but there is no lowest point since the graph looks like an infinite bell -- it tends towards zero as $x$  takes on large positive or large negative values.  We say that 6 is the \emph{least upper bound of} $f$ and that $f$ \emph{attains this value} since $f(0)=6.$ We say that $0$ is the \emph{greatest lower bound} and that $f$ \emph{does not attain this value} since there is no number $x$ so that $f(x) = 0.$

\begin{dfn}
If there is a number $M$ satisfying $f(x) \leq M$ for all $x$ in the domain of $f$, then we call $M$ an \textbf{upper bound} of $f.$  If there is no number less than $M$ that satisfies this property, then we call $M$ the \textbf{least upper bound}.
\end{dfn}

\begin{prb}
For each function (1) find the least upper bound or state why it does not exist, and (2) find the greatest lower bound or state why it does not exist.
\begin{enumerate}
\item $E(x) = 3-e^{-x}$
\item $\dsp f(x) = x^4-x^2$
\end{enumerate}
\end{prb}

\begin{prb}
Choose {\emph one} of the functions below and attempt to list as much information as possible.  Use this information to help you sketch a very precise graph of the function.  Do not expect to be able to list all of the information for each function.  You won't be able to answer every question for every function.  Graphing is the art of deciding which questions will best help you graph which functions.
\begin{itemize}
\item $x$-intercepts and $y$-intercepts (where $x=0$ or $y=0$)
\item horizonal asymptotes and vertical asymptotes (lines the function approaches)
\item critical points (places where $f'$ is either zero or undefined)
\item intervals over which each function is increasing (intervals where $f'$ is positive)
\item intervals over which each function is decreasing (intervals where $f'$ is negative)
\item intervals over which each function is concave up (intervals where $f''$ is positive)
\item intervals over which each function is concave down (intervals where $f''$ is negative)
\item least upper bound and greatest lower bound for each function (bounds for the range)
\end{itemize}
\begin{enumerate}
\item $f(x)=2x^{3}-3x^{2}$
\item $p(x) =3x^4 + 4x^3 -12x^2$
\item $g(x)=(1-x)^{3}$
\item $r(x) = 3-x^{1/3}$
\item $\dsp F(t)=\frac{2t}{t+1}$
\item $\dsp h(x)=\frac{5}{x^{2}-x-6}$
\item $\dsp g(x) = \frac{x^3+1}{x}$
\item $s(x) =  x-\cos(x)$
\item $T(x) = \tan(2x)$
\end{enumerate}
\end{prb}

\begin{prb}
Construct two functions $f$ and $g,$ each having a least upper bound of $1$, so that $f$ attains its least upper bound, but $g$ does not.
\end{prb}




\section{The Theorems of Calculus}

In this section we will study several important theorems:  the Intermediate Value Theorem, the Extreme Value Theorem, Rolle's Theorem and the Mean Value Theorem.   When a statement is of the form ``if P then Q'', P is called the hypothesis of the theorem and Q is called the conclusion.
\begin{annotation}
\endnote{I often will sketch pictures and illustrate what these theorems mean.  I hold the students responsible for being able to identify the hypothesis and conclusion of each theorem, for being able to determine if a function satisfies the hypothesis of each theorem, and for applications of each theorem.  Again, those who express a deeper interest in understanding these are the ones I recruit to take our transitions and analysis courses.}
\end{annotation}

\begin{dfn}
$C_{[a,b]}$ denotes the set of all functions that are continuous at every point in $[a,b]$ and $C^1_{[a,b]}$ denotes the set of all functions that are differentiable at every point in $[a,b].$  Thus, when we write $f \in C^1_{[a,b]}$ we mean that $f$ is an element of $C^1_{[a,b]}$,  so $f$ is a function and $f$ is differentiable at every point in $[a,b]$.
\end{dfn}

\begin{thm} \textbf{Intermediate Value Theorem.}
If $f \in C_{[a,b]}$ and $f(a) < f(b)$ and $y$ is a number between $f(a)$ and $f(b)$, then there is a number $x \in [a,b]$ such that $f(x) =  y.$
\end{thm}

\begin{prb}
Does $\dsp f(x) = \frac{x^2-1}{x-1}$ satisfy the hypothesis Intermediate Value Theorem on the interval $[-1,3]$?  Does $f$ satisfy the conclusion?
\end{prb}

\begin{prb}
Use the Intermediate Value Theorem with $f(x)= x^3+3x-2$ to show that $x^3+3x-2=0$ has a solution on the interval $[0,1]$.
\end{prb}

\begin{prb}
Does $ t^3 \cos(t) + 6\sin^5(t) - 3 = 2$ have a solution on the interval $[0,2\pi]?$
\end{prb}

\begin{thm} \textbf{Extreme Value Theorem.}
If $f \in C_{[a,b]}$, then there are numbers $c$ and $d$ in $[a,b]$ so that $f(c) \leq f(x)$ for every $x$ in$[a,b]$ and $f(d) \geq f(x)$ for every $x$ in $[a,b]$.
\end{thm}

The Extreme Value Theorem and the Intermediate Value Theorem together give us the important result that if $f \in C_{[a,b]}$ then $f([a,b]) = [m,M]$ where $m$ and $M$ are the absolute minimum and maximum of $f$, respectively.

\begin{prb}
Find the maximum and minimum of each function on the indicated interval.
\begin{enumerate}
\item $f(x) = x^2 -9$ on $[2,3]$
\item $f(x) = e^x$ on $[2,5]$
\item $f(x) = x^3 - x$ on $[-1,5]$
\item $\dsp f(x) = \frac{4\sqrt{x}}{x^2+3}$ on $[0,4]$
\end{enumerate}
 \end{prb}

\begin{thm} \textbf{Rolle's Theorem.} If $f \in C^1_{[a,b]}$ and $f(a)=f(b)$, then there is a point $c$ in the interval $(a,b)$ such that $f'(c)=0$.
\end{thm}

\textbf{Sort-of-Proof.} We have seen that $f$ has an absolute minimum and an absolute maximum on the closed interval $[a,b]$. Suppose an absolute minimum of $f$ occurs at $x_{m}$ and an absolute maximum of $f$ occurs at $x_{M}$; then $f(x_{m}) \leq f(x) \leq f(x_{M})$ for each $x$ in $[a,b]$. If $x_{m}$ is neither $a$ nor $b$, then $f'(x_{m})=0$ since it is an absolute minimum. Similarly, if $x_{M}$ is neither $a$ nor $b$, $f'(x_{M})=0$ since it is an absolute maximum. Otherwise, we assume that $x_{m}=a$ and $x_{M}=b$. Then we have $f(x_{m})\leq f(x)\leq f(x_{M})$ for every $x$ in $[a,b]$ which implies that $f(x_{m})=f(x)=f(x_{M})$ for every $x$ in $[a,b]$. Therefore, $f'(x)=0$ for all $x$ in $[a,b]$. The same argument applies if $x_{m}=b$ and $x_{M}=a$. \emph{q.e.d.}

\begin{prb}
In each case, determine if the function and the given interval $[a,b]$ satisfy the hypothesis of Rolle's Theorem. If so, find all points $c$ in $(a,b)$ such that $f'(c)=0$.
\begin{enumerate}
\item   $f(x)=3x^2-3$ on $[-3,3]$
\item   $F(t)=6t^{2}-t-1$ on $[-4,5]$
\item   $g(\theta )=\cos(\theta )$ on $[-\pi ,3\pi ]$
\end{enumerate}
\end{prb}

\begin{thm}\textbf{Mean Value Theorem.} If $f \in C^1_{[a,b]}$, and then there is a point $c$ in the interval $(a,b)$ such that $\dsp f'(c)=\frac{f(b)-f(a)}{b-a}$.
\end{thm}

\begin{prb}
In each case, determine if the function and the given interval $[a,b]$ satisfy the hypothesis of the Mean Value Theorem. If so, find all points $c$ in $(a,b)$ such that $\dsp f'(c)=\frac{f(b)-f(a)}{b-a}$.
\begin{enumerate}
\item   $f(x)=3x^2-3$ on $[-1,3]$
\item   $F(t)=6t^{2}-t-1$ on $[-4,5]$
\item   $\dsp h(\beta )=\sin(2\beta )$ on $[-\frac{\pi}{4} ,\pi ]$
\item   $\dsp H(y)=5y^{\frac{2}{3}}$ on $[-2,2]$
\end{enumerate}
\end{prb}

We have already discussed this theorem in an applied context.  If we average 80 miles per hour between Houston and St. Louis, then at some point we must have had a velocity of 80 miles per hour. In our applied problem, $f$ represented the distance traveled, $a$ the starting time, $b$ the ending time, $f'$ the velocity, and $c$ the time that must exist when we were traveling at 80 miles per hour.

\begin{prb}
Suppose $f(t)$ is the distance in feet that Ted has traveled down a road at time t in seconds.  Suppose $f(10)=0$ and $f(20)= 880$. Show that Ted traveled at 60 miles per hour at some time during the 10 second period.
\end{prb}

\begin{prb}
A car is stopped at a toll booth. 18 minutes later it is clocked 18 miles away traveling at 60 miles per hour.  Sketch a graph that might represent the car's speed  from time $t=0$ to $t=18$ minutes. Then use the Mean Value Theorem to prove that the car exceeded 60 miles per hour.
\end{prb}

The next problem is a very slight modification of a problem that I shamelessly stole from my friend and colleague, Brian Loft!

\begin{prb}
The first two toll stations on the Hardy Toll Road are 8 miles apart.  Dr. Mahavier's EasyPass says it took 6 minutes to get from one to the other on a Sunday drive last week.  A few days later, a ticket came in the mail for exceeding the 75 mph speed limit.  Can he fight this ticket?   Does Dr. Mahavier leave his GPS on in his phone while driving? Have you read ``1984?''
\end{prb}

\begin{expl}  \textbf{L'H\^opital's Rule.} Consider these three simple limits and discuss the implications.
\begin{enumerate}
\item $\dsp \lim_{x\to\infty} \frac{x^2}{x} = DNE(\infty)$
\item $\dsp \lim_{x\to\infty} \frac{3x}{x} = 3$
\item $\dsp \lim_{x\to\infty} \frac{x}{x^2} = 0$
\end{enumerate}
\end{expl}

Therefore if the top and the bottom of a rational function both tend to infinity, then the limit can be $0$, $\infty$, or any real number!  We call such limits \emph{indeterminate forms} and these were three examples of the indeterminate form $\dsp \frac{\infty}{\infty}$ since both the numerator and the denominator tended to infinity.  Here is the formal definition.


\begin{dfn}
If $f$ and $g$ are two functions such that $\displaystyle{\lim_{x \to c}} f(x) = \infty$ and $\displaystyle{\lim_{x \to c}} g(x) = DNE(\infty)$, then the function $\dsp \frac{f}{g}$  is said to have the \textbf{indeterminate form} $\dsp \frac{\infty}{\infty}$ at $c$.
\end{dfn}

There are formal definitions for other indeterminate forms, but I don't think we need them to
proceed.  The list of indeterminate forms is:  $$\frac{0}{0}, \\\\\\ \frac{\infty}{\infty}, \\\\\\ 0 \cdot \infty, \\\\\\ 0^0, \\\\\\ 0^\infty, \\\\\\ \infty^0, \\\\\\ 1^\infty \mbox{ and } \infty - \infty.$$  One tool that is helpful in evaluating indeterminate forms follows.

\begin{thm} \textbf{L'H\^opital's\ Rule.}
Suppose $f$ and $g$ are differentiable functions on some interval $(a,b)$ and $c$ is a number in that interval.  If $\displaystyle{\lim_{x \to c} f(x) = 0}$ and $\displaystyle{\lim_{x \to c} g(x) = 0}$  and $g'(x) \ne 0$ for $x \ne c$ and $\displaystyle{\lim_{x \to c} \frac{f'(x)}{g'(x)}}$ exists,  then $\displaystyle{\lim_{x \to c} \frac{f(x)}{g(x)} = \lim_{x \to c} \frac{f'(x)}{g'(x)}}$.
\end{thm}

L'H\^opital's Rule also applies for the indeterminate form: $\dsp \frac{\infty}{\infty}$.

%\begin{thm} {\bf{Cauchy's Mean Value Theorem}}
%If $f$ and $g$ are functions on $[a, b]$ such that:
%\begin{enumerate}
%\item $f$ and $g$ are continuous on $[a, b]$, \item $f$ and $g$
%are differentiable on $(a, b)$ and \item $g'(x) \ne 0$ for any $x
%\in (a, b)$,
%\end{enumerate}
%then there is $c \in (a, b)$ such that $\dsp {{f(b)-f(a)} \over
%{g(b)-g(a)}} = {{f'(c)} \over {g'(c)}}$.
%\end{thm}

%\begin{prb}
%Prove Cauchy's Mean Value Theorem. (Hint: Use Rolle's Theorem to
%show that $\dsp g(b) - g(a) \ne 0$. Let $h(x) =
%[f(b)-f(a)]g(x)-[g(b)-g(a)]f(x)$ and show $h$ satisfies the
%hypothesis of Rolle's Theorem.  Apply Rolle's Theorem to the
%function $h$.)
%\end{prb}


%{\bf{Proof}}. We will only prove the case $\displaystyle{\lim_{x
%\to c} f(x) = 0}$, $\displaystyle{\lim_{x \to c} g(x) = 0}$, and
%$\displaystyle{\lim_{x \to c} {{f'(x)} \over {g'(x)}} = L}$. The
%other cases, though harder, can be proved similarly. Define
%$$
%F(x) =
%\begin{cases}
%f(x) & \text{for $x \ne c$} \cr 0 & \text{for $x = c$}
%\end{cases}
%$$

%$$
%G(x) = \begin{cases}
%g(x) & \text{for $x \ne c$} \cr 0 & \text{for $x = c$}
%\end{cases}
%$$

%Then it is very easy to check that the function $F$ and $G$ are
%continuous on $(a, b)$ and $F'(x) = f'(x)$ and $G'(x) = g'(x)$ for
%$x \ne c$. Apply Cauchy's Mean Value Theorem to either the
%interval $[c, x]$ or $[x, c]$. Then there is $w$ between $c$ and
%$x$ such that $\dsp {{F(x) - F(c)} \over {G(x) - G(c)}} = {{F(x)}
%\over {G(x)}} = {{f(x)} \over {g(x)}}={{F'(w)} \over {G'(w)}} =
%{{f'(w)} \over {g'(w)}}$. The result follows from
%$\displaystyle{\lim_{x \to c}{{f(x)} \over {g(x)}}=\lim_{x \to
%c}{{f'(w)} \over {g'(w)}} = \lim_{w \to c}{{f'(w)} \over
%{g'(w)}}}$ because $w$ is between $c$ and $x$.\\

%\noindent
%\textbf{Other Indeterminate Forms}\\

\begin{prb}
Evaluate each of the following limits using L'H\^opital's rule.  For each problem, is there another way you know how to do it?
\begin{enumerate}
\item   $\dsp \lim_{x\to 1} \frac{x^{2}-2x+1}{x^{2}-1}$
\item   $\dsp \lim_{x\to 0 } \frac{\sin(x)}{x}$
\item   $\dsp \lim_{x\to 0 } \frac{6^{x}-1}{x^{2}+2x}$
\item   $\dsp \lim_{x\to -3 } \frac{3x-\frac{2}{x}}{x^{2}-x-20}$
\end{enumerate}
\end{prb}

\begin{prb}
Evaluate each of the following limits if they exist.  If they do not exist, why not?
\begin{enumerate}
\item   $\dsp \lim_{x\to \infty} \frac{2x}{5x+3}$
\item   $\dsp \lim_{x\to \infty } x^{2}+5$
\item $\dsp \lim_{t\to \infty } \frac{4t+\cos(t)}{t}$
(Try it with, and without, L'H\^opital's rule. Read L'H\^opital's rule carefully.)
\item   $\dsp \lim_{x\to \infty } \frac{1-4x+3x^{2}+x^{5}}{x^{2}+x-2}$
\item $\dsp \lim_{x \to 0^+} \frac{x}{\ln(x)}$
\item $\dsp {\lim_{x \to 0^+} \invsin (3x) \cdot \csc(2x)}$
\item $\dsp \lim_{x\to \infty} (\sqrt{x^{2}+1}-x)$
\end{enumerate}
\end{prb}

\begin{prb}
Use common denominators and L'H\^opital's Rules to compute $\dsp \lim_{x\to 0} (\frac{1}{x}-\frac{1}{\sin(x)})$.
\end{prb}

We can use properties of the natural log to resolve limits of the forms, $0^0$, $\infty^0$ and $1^\infty$.

\begin{expl}
Compute $\lim_{x \to \infty} x^\frac{1}{x}$.
\end{expl}

\begin{enumerate}
\item Suppose the answer is $L$:  \ \ $\dsp L = \lim_{x \to \infty} x^{\frac{1}{x}}$.
\item Apply the natural log to both sides: \ \ $\dsp \ln(L) = \ln( \lim_{x \to \infty} x^{\frac{1}{x}})$.
\item Swap the limit and the ln:\ \ $\dsp \ln(L) = \lim_{x \to \infty} \ln( x^{\frac{1}{x}})$. \; Why is this legal?
\item Do some algebra and compute the limit: \ \ $\dsp \ln(L) = \lim_{x \to \infty} \frac{1}{x} \ln x =  \lim_{x \to \infty} \frac{\ln(x)}{x} = \lim_{x \to \infty} \frac{1}{x}  = 0$.
\item Solve for $L$: \ \ $\ln(L) = 0$ so $L = e^0 = 1$.
\end{enumerate}


\begin{prb}
Evaluate these using the clever trick just described.
\begin{enumerate}
\item $\lim_{x\to 0^{+}} x^{x^{2}}$
\item $\lim_{x\to \infty} (1+\frac{1}{x})^{x}$
\end{enumerate}
\end{prb}


\section{Maxima and Minima}

Many problems from industry boil down to trying to find the maximum or minimum of a function.  In the example that follows, if a mathematician can find a way to make a can cheaper than the competition, then the company will save money and the mathematician will get a raise.  One way to save money would be to determine if the dimensions of the cans can be changed to hold the same amount of fluid, but use less material.
\begin{annotation}
\endnote{Whether I introduce this section with a lecture and example depends on the class.   In some classes, progress is occurring at a good rate and no introduction is needed.  In others, progress is slower and I use a motivating example to speed things along.   Finally, some classes have a few exceptional students who can and will forge forward, but a lecture may be needed to level the playing field for the rest of the class.  When I do introduce optimization problems, I start by reminding them that they already know how to find the maxima and minima of functions over closed intervals, as guaranteed by the Extreme Value Theorem.   Then I point out, tongue-in-cheek, that the only reason one really wants a maximum or a minimum is to \emph{make more money}.  I might discuss how we would like the maximum increase in our mutual fund, constrained by minimum volatility.  Or the maximum lifespan of an engine, constrained by minimum cost.  In other words, these are very real problems and if a mathematician can increase the efficiency of an airplane wing by a few thousandths of a percent, s/he can pay her/his salary many times over in fuel savings.   After the pep talk supporting the need for word problems, I launch into my own example of the TedCo Bottling Company.  This one is particularly interesting since this is no longer how they make beer cans.  The industry standard now is to press solid cylindrical slugs of aluminum into cylinders to form the base in one piece.  Still, the idea of minimizing surface area subject to a constant volume is a valid problem and there is actually a beverage company out there that worked this problem in the 80s and changed the dimensions of their cans accordingly, saving money on each can!}
\end{annotation}

\begin{expl} TedCo Bottling Company wants to find a way to create a can of Big Ted that has 355cc of soda but uses less material than a standard can.  This is called a max/min problem because we want to minimize the surface area while holding the volume constant.
\end{expl}

The surface area of the can is given by
$$A = 2\pi r^2 + 2\pi r h$$
where $r$ is the radius and $h$ is the height in centimeters. The volume is given by
$$V = \pi r^2 h$$
Since the volume is 355 cc, we have
$$355 = \pi r^2 h$$
so we can eliminate $h$ in the surface area equation by solving the volume
equation for $h$ and plugging in.
$$h = \frac{355}{\pi r^2}$$
Therefore
\begin{eqnarray*}
A(r) & = & 2\pi r^2 + 2\pi r h \cr
 & = & 2\pi r^2 + 2\pi r \frac{355}{\pi r^2} \cr
 & = &  2 \left( \pi r^2 + 355/r \right)
\end{eqnarray*}
So, all we need is to find the radius that minimizes the function $A.$  Taking the derivative of $A$ and setting it equal to $0$ we have:
$$r = \sqrt[3]{\frac{355}{2\pi}} \approx 3.837$$
We can solve for $h$ by substituting this back into
$$h = \frac{355}{\pi r^2} \approx 7.67$$
Observing that $h$ and $r$ are in centimeters, how does this compare to a standard Coke can?  If we note that
$$h = \frac{355}{\pi r^2} =  \frac 2{r^2} \cdot \frac{355}{2\pi} = \frac{2r^3}{r^2} = 2r,$$
then we see that this is an unusual sized can.  Why don't the rest of the canning companies follow this material saving model?

\begin{prb}
A rock is thrown vertically. Its distance d in feet from the ground at time t seconds later is given by $d = -16t^2 + 640 t$. How high will the rock travel before falling back down to the ground?  What will its impact velocity be?
\end{prb}

\begin{prb}
A rectangular field is to be bounded by a fence on three sides and by a straight section of lake on the fourth side. Find the dimensions of the field with maximum area that can be enclosed with $M$ feet of fence.
\end{prb}

\begin{prb}
The carp population in Table Rock Lake is growing according to the growth function $y = 2000/(1+49 e^{-.3t})$, where y is the number of carp present after t months. Graph this function.  What is the maximum number of carp that the pond will hold?  At what time is the population of carp increasing most rapidly?
\end{prb}

\begin{prb}
A rectangular plot of land is to be fenced, using two kinds of fencing.  Two opposite sides will use heavy-duty fencing selling for \$4 a foot, while the remaining two sides will use standard fencing selling for \$3 a foot. What are the dimensions of the plot of greatest area that can be fenced with M dollars?
\end{prb}

\begin{prb}
You are in your damaged dune buggy in a desert three miles due south of milepost 73 on Route 66, a highway running East and West. The nearest water is on this (presumably straight) highway and is located at milepost 79. Your dune buggy can travel 4 miles per hour on the highway but only 3 miles per hour in the sand. You figure that you have 2 hours and 10 minutes left before you die of thirst. Assuming that you can take any path from your present position to the water, how long will it take you to get to the water and what does your future look like?
\end{prb}

\begin{prb}
A square sheet of cardboard with side length of $M$ inches is used to make an open box by cutting squares of equal size from each of the four corners and folding up the sides. What size squares should be cut to obtain a box with largest possible volume ($\dsp V = length \times width \times height$)?
\end{prb}

\begin{prb}
Geotech Industries owns an oil rig 12 miles off the shore of Galveston.  This rig needs to be connected to the closest refinery which is 20 miles down the shore from the rig.  If underwater pipe costs \$500,000 per mile and above-ground pipe costs \$300,000 per mile, the company would like to know which combination of the two will cost Geotech the least amount of money.  Find a function that may be minimized to find this cost. (This problem also stolen shamelessly from Brian Loft.)
\end{prb}

\begin{prb}
Find the dimensions of the rectangle with maximum area that can be inscribed in a circle of radius M.
\end{prb}

\section{Related Rates}

Often two quantities are related and the rate at which each changes depends on the rate at which the other changes.   Think of a turbo-charged car.  You may push the accelerator to the floor at a constant rate. While the rate of acceleration of the car is related, it will not be constant because as the turbo kicks in the car will accelerate at a much faster rate even though you are still pushing the pedal down at a constant rate.\begin{annotation}
\endnote{As with the optimization problems, I may introduce related rates with an example.  Regardless of whether I do so, at some point I tell the class how to maximize points in solving these problems.   Identify the variables, draw a picture, find the necessary relations between the variables, differentiate and finish off the problem.  As I grade every problem out of four points, I give a point for a picture, a point for correctly defined variables, a third point for the correct relationship between the variables, and a point for finishing off the problem.  Therefore, a student may earn three points merely by setting up the problem correctly.   This encourages them to try to start every problem by reading carefully to determine what the related variables are.  My experience is that the student who can correctly identify the variables and the relationship between them nearly always completes the problem. I particularly like the boat hook example because it illustrates so many things.  The way I set the problem up, the sign of the velocity of the top of the ladder is negative and sign of the velocity of the base is positive, which sparks a good conversation.  The fact that the ladder strikes the floor with infinite velocity explains the really, really loud sound such an event creates and is another example of a limiting process.  Hence, the discussion of this problem is not usually a five-minute run through, but a twenty-minute discussion.}
\end{annotation}

\begin{expl}
A 20 foot boat hook is leaning against a wall and the base is sliding away from the wall at the rate of 5 feet per minute. This moves the top of the hook down the wall. How fast is the top moving when the base is 8 feet from the wall?   How fast is the top moving at the instant when it strikes the floor?
\end{expl}

\noindent
Let
\begin{itemize}
\item $x(t)$ = the distance from the base of the hook to the wall at time $t$ and
\item $y(t)$ = the distance from the top of the hook to the floor at time $t.$
\end{itemize}
So
\begin{itemize}
\item $x'(t)$ = the velocity at which the base of the hook is moving and
\item $y'(t)$ = the velocity at which the top of the hook is moving.
\end{itemize}
Note that $x$ and $y$ are related by the Pythagorean theorem,
$$x^2(t) + y^2(t) = 20^2.$$
Using implicit differentiation yields,
$$2x(t) x'(t) + 2y(t)y'(t) = 0.$$
If we consider $t^*$ to be the time when the base of the pole is 8 feet from the wall,
then $x(t^*) = 8$ and $x'(t^*) = 5.$ Therefore, we have
$$2x(t^*) x'(t^*) + 2y(t^*)y'(t^*) = 0.$$ And solving for $y'(t^*)$ we have
$$y'(t^*) = \frac{-x(t^*)x'(t^*)}{y(t^*)} = \frac{-40}{\sqrt{400-64}} = - 10\sqrt{21} \approx -2.2.$$
Why is $y'(t^*)$ negative?


\begin{prb}
A baseball diamond is a square with sides 90 feet long. A runner travels from first base to second base at 30 ft/sec. How fast is the runner's distance from home plate changing when the runner is 50 feet from first base?
\end{prb}

\begin{prb}
Two planes leave an airport simultaneously. One travels north at 300 mph and the other travels west at 400 mph.  How fast is the distance between them changing after 10 minutes?
\end{prb}

\begin{prb}
If the temperature of a gas is held constant, then Boyle's Law guarantees that the pressure P and the volume V of the gas satisfy the equation PV = c, where c is a constant. Suppose the volume is increasing at the rate of 15 cubic inches per second.  How fast is the pressure decreasing when the pressure is 125 pounds per square inch and the volume is 25 cubic inches?
\end{prb}

\begin{prb}
Sand pouring from a chute at the rate of 10 cubic feet per minute forms a conical pile whose height is always one third of its radius. How fast is the height of the pile increasing at the instant when the pile is 5 feet high?
\end{prb}

\begin{prb}
A balloon rises vertically from a point that is 150 feet from an observer at ground level. The observer notes that the angle of  elevation is increasing at the rate of 15 degrees per second when the angle of elevation is 60 degrees. Find the speed of the balloon at this instant.
\end{prb}

\begin{prb}
A snowball ($\dsp V=\frac{4}{3}\pi r^3$) melts at a rate that, at each instant, is proportional to the snowball's surface area ($\dsp A=4\pi r^2$). Assuming that the snowball is always spherical, show that the radius decreases at a constant rate.
\end{prb}

\begin{prb}
Herman is cleaning the exterior of a glass building with a squeegee. The base of the 10 foot long squeegee handle, which is resting on the ground, makes an angle of A with the horizontal.  The top of the squeegee is x feet above the ground on the building.  Herman pushes the bottom of the squeegee handle toward the wall. Find the rate at which x changes with respect to A when A = 60 degrees. Express the answer in feet per degree.
\end{prb}

\begin{prb}
Do all the practice problems in Chapter \ref{chap3probs}.  We will not present these, but I will happily answer questions about them.
\end{prb}

\section{Practice} \label{chap3probs}

We will not present the problems from this section, although you are welcome to ask about them in class.

\vskip .1in
\noindent
\textbf{Linear Approximations}

\begin{enumerate}
\item Sketch $f(x)  = 1 + \sin(2x).$  Compute and sketch the linear approximation to $f$ at $x=\pi/2.$
\item Ted's hot air balloon has an outer diameter 24 feet and the material is .25 inches thick.  Assuming it
is spherical approximate the volume using linear approximates and compare it to the exact volume.
\end{enumerate}

\noindent
\textbf{Limits and Infinity}
\begin{enumerate}
\item Compute $\dsp \lim_{x \to \infty} \frac{2x-4}{3x-2}.$
\item Compute $\dsp \lim_{x \to 3+} \frac{2x-4}{x-3}$ and $\dsp \lim_{x \to 3-} \frac{2x-4}{x-3}$
\item Compute $\dsp \lim_{x \to \infty} \frac{2x-4}{3x^2-2}.$
\item Compute $\dsp \lim_{x \to \infty} \frac{\sqrt{x^2-3}}{x-1}$ and
$\dsp \lim_{x \to -\infty} \frac{\sqrt{x^2-3}}{x-1}.$
\end{enumerate}

\noindent
\textbf{Graphing}
\begin{enumerate}

\item For each of the following, find the greatest lower bound and the least
upper bound for the function on the given interval.
\begin{enumerate}
\item $f(x)=x^{3}-3x+10$ on $[0,2]$
\item   $\dsp H(x)=x+\frac{1}{x}$ on $[-2, -0.5]$
\item   $\dsp g(t)=\frac{5}{4-t^{2}}$ on $[-6,4]$
\item  $T(x)=\tan(x)$ on $[0,\frac{\pi }{2}]$
\end{enumerate}

\item For each of the following graph $f,$ $f',$ and $f''$ on the same set of axes.
\begin{enumerate}
\item $f(x)=6x^{2}+x-1$
\item $f(x)=1-x^{3}$
\end{enumerate}

\item Sketch an accurate graph of the following functions. List as much of the following information as you can: intercepts, critical points, asymptotes, intervals where $f$ is increasing or decreasing, intervals where $f$ is concave up or concave down, local maxima and minima, inflection points, and least upper bound and greatest lower bound.

\begin{enumerate}
\item $F(t)=4-t^{2}$
\item $f(x)=x^{3}-4x+2$
\item $G(t)=t^{3}+6t^{2}+9t+3$
\item $\dsp E(x) = e^{x^2}$
\item $\dsp F(z)=\frac{1}{1+z^{4}}$
\item $\dsp y = 3x^4 + 4x^3 - 12x^2$
\item $\dsp f(x) =\frac{4}{5} x^5 - \frac{13}{3}x^3 + 3x + 4$
\item $\dsp k(x) = x^{1/3}$  Watch for where $k'$ is undefined.
\item $\dsp g(x)= x(x-2)^{2/3}$
\item $\dsp h(x) = \frac{e^x + e^{-x}}{2}$
\item $\dsp p(x) = x^3-3x^2+3$
\item $\dsp T(x) = \sin(x) + \cos(x)$
\item $\dsp h(x) = \frac{2x^2}{x^2+x-2}$
\item $\dsp r(t) = \frac{\sqrt{3}}{2}t - \sin(t)$
\item $\dsp z =\frac{x^2-x}{x+1}$  Find 1 vertical and 1 slant asymptote.
\item $\dsp y = x^{1/3}(4-x)$  Watch for where $y'$ is undefined.
\item $\dsp y = \frac{x}{1-x^2}$  Find 2 vertical and 1 horizontal asymptotes.
\item $\dsp y =\frac{\sqrt{x^2-3}}{x-1}$  Watch for 2 horizontal and 1 vertical asymptotes.
\item $\dsp z = \frac{x^2+1}{x}$  What function does this graph approximate for large values of x?
\item $\dsp z = \frac{x^4-1}{x^2}$  What function does this graph approximate for large values of x?
\end{enumerate}
\end{enumerate}

\noindent
\textbf{Rolle's Theorem, Mean Value Theorem, and L'H\^opital's Rule}
\begin{enumerate}

\item Verify that $f(x) = \sin(2\pi x)$ satisfies the hypothesis of the Rolle's Theorem on [-1,4]
and find all values $c$ satisfying the conclusion to the Rolle's Theorem.

\item Verify that $f(x) = x^3 - x^2 -x +1$ satisfies the hypothesis of the Mean Value Theorem on [-1,2]
and find all values $c$ satisfying the conclusion to the MVT.

\item Evaluate each of the following limits using L'H\^opital's Rule, if L'H\^opital's rule applies.
\begin{enumerate}
\item $\dsp \lim_{x\to \infty} \frac{4x+3}{10x+5}$
\item $\dsp \lim_{x\to \infty} \frac{2x^{2}-x+3}{3x^{2}+5x-1}$
\item $\dsp \lim_{t\to 0} \frac{1-\cos(t)}{t^{2}+3}$
\item $\dsp \lim_{x\to \infty} \frac{(\ln(x))^{2}}{x}$
\item $\dsp \lim_{x\to 0^{+}} 3x\ln(x)$
\item $\dsp \lim_{x \to 0^+} {{4^{3x} - \cos(2x)} \over{\tan(x)}}$
\item $\dsp \lim_{x \to 2^+} {{\log(x-2)} \over{\csc(x^2 - 4)}}$
\item $\dsp \lim_{x \to 0^+}x^{\tan(x)}$
\item $\dsp \lim_{x \to 0^-} (x+1)^{\csc(x)}$
\item $\dsp \lim_{x \to \infty} \Big({x \over {x+3}} \Big)$
\item $\dsp \lim_{x \to \infty} (3x)^{1 \over x}$
\item $\dsp \lim_{x \to 0^+} (\sec(x))^{1 \over {x^2}}$
\item $\dsp \lim_{x \to 0^+} (e^x+3x)^{1 \over x}$
\item $\dsp{\lim_{x \to {{\pi} \over 2}^+}({1 \over {\cos(x)}} - \tan x)}$
\item $\lim_{x\to 0^{+}} x^{\tan(x)}$
\item $\dsp{\lim_{x \to 0^-} (1-x)^{1\over x}}$
\item $\dsp{\lim_{x \to {{\pi}\over 2}^+} (x - {{\pi} \over 2})^{1- \sin x}}$
\end{enumerate}
\end{enumerate}

\noindent
\textbf{Max/Min Problems}

\begin{enumerate}
\item What is the minimum perimeter possible for a rectangle with area 16 square inches?

\item A right triangle with (constant) hypotenuse $h$ is rotated about one of its legs to sweep out a right circular cone. Find the radius that will create the cone of maximum volume.

\item Find the point $(x,y)$ on the graph of $\dsp x=\frac{1}{y^{2}}$ that is nearest the origin $(0,0)$ and is in Quadrant IV.

\item The cost per hour in dollars for fuel to operate a certain airplane is $\$0.021v^{2}$, where $v$ is the speed in miles per hour. Additional costs are $\$4,000$ per hour.  What is the speed for a $1,500$ mile trip that will minimize the total cost?

\item A rice silo is to be constructed in the rice fields of southeast Texas so that the silo is in the form of a cylinder topped by a hemisphere. The cost of construction per square foot of the surface area of the hemisphere is exactly twice that of the cylinder. Determine the radius and height of the silo if the volume of the silo must be 2000 cubic feet and the cost is to minimized.
\end{enumerate}

\noindent
\textbf{Related Rate Problems}

\begin{enumerate}

\item   A man 6 feet tall walks at a rate of 5 feet per second toward a streetlight that is 16 feet above the ground.  At what rate is the tip of his shadow moving?

\item   An airplane is flying at a constant altitude of 6 miles and will fly directly over your house, where you have radar equipment.  Using radar, you note that the distance between the plane is decreasing at 400 miles/hour.  What is the speed of the plane when the distance from your house to the plane is 10 miles?

\item Suppose that a spherical balloon is inflated at a rate of 10 cubic feet per second. At the instant that the balloon's volume is $972\pi $ cubic feet, at what rate is the surface area increasing?

\item Your snow cone has height 6 inches and radius 2 inches and is draining out through a hole in the bottom at a rate of one half cubic inch per second.  You naturally wonder, what will the rate of change of the depth be when the liquid is 3 inches deep?
\end{enumerate}

\vskip .5in
\noindent
\textbf{Chapter 3 Solutions}\\ \\

\noindent
\textbf{Linear Approximations}

\begin{enumerate}
\item Linear approximation is $y = -2x + 1 + \pi.$
\item Exact volume = 6795.2, approximate volume = 6785.83.
\end{enumerate}

\noindent
\textbf{Limits and Infinity}
\begin{enumerate}
\item $\dsp \frac{2}{3}$
\item $\infty, -\infty$
\item 0
\item $1, -1$
\end{enumerate}

\noindent
\textbf{Graphing}
\begin{enumerate}

\item For each of the following, find the greatest lower bound and the least upper bound for the function on the given interval.
\begin{enumerate}
\item $8, 12$
\item $-2.5, -2$
\item no greatest lower bound or least upper bound, although there is a local min at $(0,5/4)$
\item  greatest lower bound is 0, no least upper bound
\end{enumerate}

\item For each of the following graph $f,$ $f',$ and $f''$ on the same set of axes.
\begin{enumerate}
\item no graphs, but think about the relationships between the three functions
\end{enumerate}

\item Sketch an accurate graph of the following functions...
\begin{enumerate}
\item Type them into Google search or your favorite software to see graphs.
\end{enumerate}
\end{enumerate}

\noindent
\textbf{Rolle's Theorem, Mean Value Theorem, and L'H\^opital's Rule}
\begin{enumerate}

\item Since all trig functions are differentiable everywhere and $f(-1)=0=f(4),$ $f$ satisfies the hypothesis.  $\dsp c = -\frac{3}{4}, -\frac{1}{4}, \frac{1}{4}, \frac{3}{4}, \frac{5}{4}, \frac{7}{4}, \frac{9}{4}, \frac{11}{4}, \frac{13}{4}, \frac{15}{4}$ satisfy the conclusion.

\item Since all polys are differentiable everywhere, $f$ satisfies the hypothesis. $\dsp c = \frac{1 \pm \sqrt{7}}{3}$ both satisfy the conclusion.


\item Evaluate each of the following limits using L'H\^opital's Rule if L'H\^opital's Rule applies.
\begin{enumerate}
\item $\dsp 2/5$
\item $2/3$
\item $0$
\item $0$
\item $0$
\item $3 \ln(4)$
\item $0$
\item $1$
\item $e$
\item $1$
\item $1$
\item $e$
\item $e^4$
\item $0$
\item $1$
\item $e^{-1}$
\item $1$
\end{enumerate}
\end{enumerate}

\noindent
\textbf{Max/Min Problems}

\begin{enumerate}
\item A 4 x 4 square
\item Radius = $\sqrt{2/3} \ h$
\item $\dsp (\sqrt[6]{2},-\frac{1}{\sqrt[3]{2}})$
\item approximately 523 miles/hour
\item Radius = $\sqrt[3]{750/\pi}$
\end{enumerate}

\noindent
\textbf{Related Rate Problems}

\begin{enumerate}
\item 8 ft/sec
\item 500 mi/hr
\item 20/9 square feet/sec
\item $\dsp \frac{3}{8\pi}$ inches per second
\end{enumerate}


\chapter{Integrals}

``We have to believe - before students can believe - that hard work pays off, that effort matters, that success depends not on your genes but on your sweat.  We must convince students that they can do it when they have teachers who insist on quality work and give them extra remedial help when needed.'' -  Gene Bottoms\\ \\

Integrals are used throughout engineering and industry to solve a wide variety of applied problems such as: the amount of work necessary to lift a 50 ton satellite, the total pressure on the Hoover dam, the probability that you will live to an age between 69 and 71, the volume formulae for various objects, the weight of objects that do not have uniform density, the work required to pump water out of a 55 gallon drum and the area of an oil spill.

\section{Riemann Sums and Definite Integrals}

You know how to compute the areas of certain shapes (rectangles, triangles, etc.) using formulas, but what about unusual shapes like the area trapped under a curve and above the x-axis or the area trapped between two curves?   If we have a function that is always above the $x$-axis on the interval $[a,b]$, then Riemann sums give us a way to compute the area that is between the vertical lines $x=a$ and $x=b$, above the $x$-axis, and under the curve.

\begin{expl} \label{intxsquared}
Determine the area under the curve $f(x) = x^2+1,$ above the x-axis, and between the vertical lines $x=0$ and $x=2.$
\begin{annotation}
\endnote{As stated in the introduction, I do not provide graphics to my students.  It is my contention that forcing them to read and interpret the English is a valuable skill and that drawing a picture for them robs them of the opportunity to read carefully and determine the appropriate picture.}
\end{annotation}
\end{expl}

Graph $f$ and subdivide $[0,2]$ into 4 intervals each of width $\frac{1}{2}$ and with bases, $$[0,\frac{1}{2}], [\frac{1}{2},1], [1,\frac{3}{2}], \mbox{ and } [\frac{3}{2},2].$$  Each has width $\frac{1}{2}.$  These intervals are the bases of 4 rectangles. Make the height of each rectangle the value of $f$ at the right end point of the base of that rectangle. Let's call the sum of the areas of these four rectangles $U_4.$  (If you are unfamiliar with summation notation, $\sum$, see Appendix \ref{appsum} for a review.)

The sum of these rectangles is given by:
\begin{eqnarray*}
U_4 & = & \frac{1}{2} \left( (\frac{1}{2})^2 + 1 \right) +
\frac{1}{2} \left( (\frac{2}{2})^2 + 1 \right) + \frac{1}{2}
\left( (\frac{3}{2})^2 + 1 \right) + \frac{1}{2} \left(
(\frac{4}{2})^2 + 1 \right) \cr & = & \frac{1}{2} \left[ \left(
(\frac{1}{2})^2 + 1 \right) + \left( (\frac{2}{2})^2 + 1 \right) +
\left( (\frac{3}{2})^2 + 1 \right) + \left( (\frac{4}{2})^2 + 1
\right) \right] \cr & = & \frac{1}{2} \sum_{i=1}^4 \left(
(\frac{i}{2})^2 + 1 \right)  = \frac{23}{4} = 5.75
\end{eqnarray*}

This estimate is too large because each rectangle we chose had area larger than the portion of the curve it approximated.  Therefore, we make an estimate that is too small and average the two!  An estimate that would be too small would have four rectangles each with the same bases, but with height the value of $f$ at the {\it left} endpoint of its base.  Let's call the sum of the areas of these four rectangles $L_4.$
\begin{eqnarray*}
L_4 & = & \frac{1}{2} \left( (\frac{0}{2})^2 + 1 \right) +
\frac{1}{2} \left( (\frac{1}{2})^2 + 1 \right) + \frac{1}{2}
\left( (\frac{2}{2})^2 + 1 \right) + \frac{1}{2} \left(
(\frac{3}{2})^2 + 1 \right) \cr & = & \frac{1}{2} \left[ \left(
(\frac{0}{2})^2 + 1 \right) + \left( (\frac{1}{2})^2 + 1 \right) +
\left( (\frac{2}{2})^2 + 1 \right) + \left( (\frac{3}{2})^2 + 1
\right) \right] \cr & = & \frac{1}{2} \sum_{i=0}^3 \left(
(\frac{i}{2})^2 + 1 \right) = \frac{15}{4} = 3.75
\end{eqnarray*}

What you have just computed are referred to as {\it upper} and {\it lower Riemann sums}, respectively.  Our estimate based on averaging the two would yield $$\frac{U_4+L_4}{2} = 4.75.$$


\begin{prb}
\label{defint} Let $f(x) = x^2 + 1.$   Divide the interval $[0,2]$ into eight equal divisions,
$$\{0,\frac{1}{4}, \frac{1}{2}, \frac{3}{4}, 1,  \dots , \frac{7}{4}, 2 \}$$
and compute the upper and lower Riemann sums of $f.$
\end{prb}

Subdividing an interval in this way is called  {\it partitioning} the interval and the sub-intervals created via a partition may be of differing widths as indicated in the next definition.

\begin{dfn}
A \textbf{partition of the interval [a,b]} is a collection of points, $\{ x_0, x_1, \dots, x_n \}$ satisfying, $a = x_0 < x_1 < x_2 < \dots < x_{n-1} < x_n = b.$
\end{dfn}

\begin{dfn}
If $f$ is a function defined on the interval $[a,b]$ and $P=\{x_0, x_1, \dots, x_n \}$  is a partition of $[a,b],$ then a \textbf{Riemann sum of f over [a,b]} is defined by
\begin{eqnarray}
R(P,f) = \sum_{i=1}^n (x_i - x_{i-1}) \cdot f(\hat{x_i})
\end{eqnarray}
where $\hat{x_i} \in [x_{i-1},x_i].$
\end{dfn}

In the definition of Riemann sum, $\hat{x}_i$ can be {\it any} value in the interval $[x_{i-1}, x_i].$  But look back at our example.  When we computed $U_4$ we always chose $\hat{x}_i$ so that $f(\hat{x}_i)$ was the {\it maximum} for $f$ on this interval to assure that our approximation was more than the desired area.  When $\hat{x}$ is chosen in this manner, we call our sum an \textbf{upper Riemann sum}. When we computed $L_4$ we always chose $\hat{x}_i \in [x_{i-1}, x_i]$ so that $f(\hat{x}_i)$ was the {\it minimum} for $f$ on that interval to assure that our approximation was less than the desired area.    When $\hat{x}$ is chosen in this manner, we call our sum a \textbf{lower Riemann sum}.

\begin{prb}
Compute upper and lower Riemann sums for $g(x) = x^2 + 3x$ on $[0,1]$ using the partition, $P = \{0,\frac{1}{3}, \frac{2}{3}, 1\}.$
\end{prb}

\begin{prb}
Compute upper and lower Riemann sums for $g(x) = -3x^3 + 3x$ over $[0,1]$ using the partition, $P = \{0,\frac{1}{4}, \frac{1}{2}, \frac{3}{4}, 1 \}.$   The heights of our rectangles do not always occur at the endpoints of the sub-intervals of our partition.
\end{prb}

Returning to $f(x) = x^2+1$ on $[0,2]$, continue to subdivide our interval into $n$ subdivisions of equal length, letting $n = 10, n=100, n=1000, \dots$ and labeling our partitions as $P_{10}, P_{100}, P_{1,000}, \dots.$ If we label our lower sums corresponding to these partitions as $L_{10}, L_{100}, L_{1,000}, \dots$ and label our upper sums as $U_{10}, U_{100}, U_{1,000}, \dots$, then we would have $$L_4 < L_{10} < L_{100} < L_{1,000} < \dots < U_{1,000} < U_{100} < U_{10} <  U_4.$$    Furthermore, $L_{1,000}$ and $U_{1,000}$ should be very close together and very close the area we desire to compute.

By increasing the number of rectangles and taking the limit as the number of rectangles tends to infinity, we will determine the {\it exact} area.  Our notation for this area is:
$$\dsp \mbox{ Area } =\int_0^2 x^2+1 \ dx = \lim_{n \to \infty} U_n = \lim_{n \to \infty} L_n.$$
For our function, these two limits are equal, but there are functions where these limits are not equal.  When they are equal, we say $f$ is {\it integrable} over the interval.  If they are not equal, we say that $f$ is not integrable. We refer to  $\int_0^2 x^2+1 \ dx$ as ``\emph{the definite integral from 0 to 2 of $x^2+1$.}''

\begin{prb}
Give an example of a function and an interval where the function will not be integrable over that interval.
\end{prb}

We now the consider the upper Riemann sum for $f(x) = x^2 + 1,$ above the x-axis, and between the vertical lines $x=0$ and $x=2$ assuming $n$ equal divisions of the interval $[0,2].$  Our partition of $n$ divisions is now, $$P = \big\{ 0 , \frac{2}{n}, \frac{4}{n}, \dots , \frac{2(n-1)}{n}, 2 \big\},$$ and our upper Riemann sum $U_n$ that approximates the area is given by:

\begin{eqnarray}
 U_n &=& \frac{2}{n} \sum_{i=1}^n \left( (\frac{2i}{n})^2 + 1 \right)
\cr &=& \frac{2}{n} \left[ \sum_{i=1}^n \frac{4i^2}{n^2} +\sum_{i=1}^n 1  \right] \cr &=& \frac{2}{n} \sum_{i=1}^n
\frac{4i^2}{n^2} + \frac{2}{n} \sum_{i=1}^n 1 \cr &=&\frac{8}{n^3} \sum_{i=1}^n i^2 + \frac{2}{n} \cdot n \mbox{
by a formula from Appendix \ref{appsum} }  \cr &=& \frac{8}{n^3}\frac{n(n+1)(2n+1)}{6} + 2  \cr &=& \frac{4n(n+1)(2n+1)}{3n^3} + 2 \cr &=& \frac{8n^3 + 12n^2 + 4n}{3n^3} + 2  \cr &=&\frac{8n^3}{3n^3} + \frac{12n^2}{3n^3} + \frac{4n}{3n^3} + 2  \cr &=& \frac{8}{3} + \frac{12}{3n} + \frac{4}{3n^2} + 2
\end{eqnarray}
Now  take the limit of the Riemann sums as the number of divisions tends to infinity to get the \emph{exact} area under the curve.
\begin{eqnarray}
\mbox{ Area } & = &
\int_0^2 x^2+1 \ dx \cr & = &
\lim_{n \rightarrow \infty} U_n  \cr & = &
\lim_{n \rightarrow \infty} \frac{8}{3} + \frac{12}{3n} +
\frac{4}{3n^2} + 2 \cr & = & \lim_{n \rightarrow \infty}
\frac{8}{3} +
       \lim_{n \rightarrow \infty}\frac{12}{3n} +
        \lim_{n \rightarrow \infty} \frac{4}{3n^2} +
        \lim_{n \rightarrow \infty} 2 \cr
& = & \frac{8}{3}  +  2 \cr & = & 4 \frac{2}{3}
\end{eqnarray}
Look back at your answers from Problem \ref{defint} ($U_8$ and $L_8$) and the solutions from our Example ($U_4$ and $L_4$). Just as we expect, $$L_4 < L_8  < 4 \frac{2}{3} < U_8 < U_4.$$ We denote the number $4 \frac{2}{3}$ by \textbf{$\dsp \int_0^2 x^2 + 1 \; dx$ } and call this {\it the definite integral of $x^2 + 1$ from $x=0$ to $x=2.$} What we have just done motivates the following definition of the definite integral.
\begin{dfn}
\label{defint3} If $f$ is a function defined on $[a,b]$ so that $lim_{n \rightarrow \infty} L_n$ and $lim_{n \rightarrow \infty} U_n$  both exist and are equal, then we say $f$ is \textbf{integrable on [a,b]} and  we define \textbf{the definite integral} by $\int_a^b f(x) \; dx = \lim_{n \rightarrow \infty} L_n.$ We refer to $a$ as the \textbf{lower limit of integration} and $b$ as the \textbf{upper limit of integration.}
\end{dfn}

For each of the next few problems, you will require some of the summation formulas from Appendix \ref{appsum}.

\begin{prb}
Repeat this procedure, but compute the area using \textbf{lower} Riemann sums. That is, compute and simplify $L_n$ (the lower Riemann sum of $f(x) = x^2+1$ on the interval $[0,2]$ assuming $n$ equal divisions) and then compute $lim_{n \rightarrow \infty} L_n.$
\end{prb}

\begin{prb}
\label{defint2}
Compute each integral using Definition \ref{defint3} (Riemann sums).
\begin{enumerate}
\item $\dsp \int_0^1 2x \; dx $
\item $\dsp \int_0^1 x^3 \; dx$
\item $\dsp \int_0^2 2x^2 - 3 \; dx$
\end{enumerate}
\end{prb}

\begin{prb}
Compute the area trapped between the curves $y=x$ and $y=x^2$ using Definition \ref{defint3}.
\end{prb}

Which functions are integrable?  That is a question better-suited for a course called \emph{Advanced Calculus} or \emph{Real Analysis}, but let's give the quick-and-dirty answer without proof. By Definition \ref{defint3} a function is integrable if a certain limit exists, but this can be difficult to compute. It can be shown that every differentiable function is continuous and every continuous function is integrable.  Thus, $$f \mbox{ is differentiable } \Rightarrow f \mbox{ is continuous } \Rightarrow f \mbox{ is integrable}.$$  The next two theorems state this formally.

\begin{thm}
\textbf{A Differentiability Theorem.} If $f$ is differentiable on $[a,b],$ then $f$ is continuous on $[a,b].$
\end{thm}

\begin{thm}
\textbf{An Integrability Theorem.} If $f$ is continuous on $[a,b],$ then $f$ is integrable on $[a,b].$
\end{thm}

We defined integrals in terms of Riemann sums and Riemann sums in terms of partitions of the interval. Therefore the integral $\int_a^b f(x) \ dx$ (see Definition \ref{defint3}) is only defined when $a < b.$  The following definitions define definite integrals where the lower limit of integration is larger than the upper limit of integration or the two endpoints of integration are equal.

\begin{dfn}
If $a$ and $b$ are numbers with $a<b$ and $f$ is integrable on $[a,b]$ then:
\begin{enumerate}
\item  $\dsp \int_b^a f(x) \; dx = - \int_a^b f(x) \; dx$ and
\item  $\dsp \int_a^a f(x) \; dx = 0.$
\end{enumerate}
\end{dfn}

The next theorem we discuss can be proved via Riemann sums, but we will leave that for your first course in \emph{Real Analysis} as well.  That's where the real fun begins.

\begin{thm}
\label{intsumrule}
\textbf{Sum Rule for Definite Integrals.} If $c \in [a,b]$, then $\dsp \int_a^b  f(x) \; dx = \int_a^c f(x) \; dx + \int_c^b f(x) \; dx$.
\end{thm}

\begin{prb}
Compute each of the following integrals using Definition \ref{defint3} (Riemann Sums).  This illustrates Theorem \ref{intsumrule}, the Sum Rule for definite integrals.
\begin{enumerate}
\item $\dsp \int_0^1 x^2 \; dx$
\item $\dsp \int_1^2 x^2 \; dx$
\item $\dsp \int_0^2 x^2 \; dx$
\end{enumerate}
\end{prb}


\section{Anti-differentiation and the Indefinite Integral}

We might denote the derivative of the function $f(x) = x^2$ as any of: $$f', \; \; \frac{df}{dx}, \; \;  (x^2)', \; \; \mbox{or} \; \; \frac{d}{dx} (x^2).$$ If $f$ is a function, then any function with derivative $f$ is called an \textbf{anti-derivative} of $f.$ For the \emph{anti-derivative} of $f$ we may write any of: $$ \int f, \; \; \int f(x) \; dx, \; \; \mbox{or} \; \; \int x^2 \; dx.$$ Any  function which has one anti-derivative has infinitely many anti-derivatives. To indicate all anti-derivatives of $f(x) = x^2$ we will write: $$ \int f(x)  \; dx  = \int x^2 \; dx = \frac{1}{3} x^3 + c, \ \ \ c \in \re.$$ We would read this notation as ``{\it the indefinite integral (or anti-derivative) of $x^2$ with respect to $x$ is $\frac{1}{3}x^3+c$ where $c$ represents an arbitrary constant}.'' It is misleading that mathematicians speak of ``the indefinite integral'' when in fact it is not a single function, but a class of functions.  You may interpret $\int$ and ``\emph{dx}'' as representing the beginning and end of the function we wish to find the anti-derivative of.  We call this process \emph{integrating} the function.  The function to be integrated is called the \emph{integrand}.

\begin{prb}
Evaluate the indicated indefinite integrals and then state the Sum Rule for Indefinite Integrals based on your results.
\begin{enumerate}
\item $\dsp \int 3x^4 + 5\; dx$
\item $\dsp \int 3x^4 \; dx  + \int 5 \; dx$
\end{enumerate}
\end{prb}

\begin{prb}
Evaluate the indicated indefinite integrals and then state the Constant Multiple Rule for Indefinite Integrals based on your
results.
\begin{enumerate}
\item $\dsp{ \int 5 \sin(x) \; dx }$
\item $\dsp{ 5 \int \sin(x) \; dx }$
\end{enumerate}
\end{prb}

\begin{prb} Evaluate the following indefinite integrals.
\begin{enumerate}
\item  $\dsp \int (s^3+s)^3\; ds$
\item  $\dsp \int t(t^{1/2}+8)\; dt$
\item  $\dsp \int \frac{x+x^3}{x^5} \; dx$
\item  $\dsp \int 3\cos(x)-2\sin(4x) + e^{4x} \; dx$
\end{enumerate}
\end{prb}

Like differentiation, anti-differentiation becomes more challenging when the \emph{Chain Rule} appears in the problem.

\begin{expl}
Evaluate  $\dsp \int x^2(x^3+10)^9\; dx.$
\end{expl}

Our first guess will be $g(x) = (x^3 + 10)^{10}.$ Then $g'(x) = 10(x^3+10)^9 \cdot 3x^2 = 30x^2(x^3+10)^9.$  Our answer is correct \emph{except} for the 30 in front.  We take care of this by modifying our first guess to $g(x) = \frac{1}{30}(x^3 + 10)^{10}.$ Now $g'(x) = \frac{1}{30}\cdot 10(x^3 + 10)^{10} \cdot 3x^2 = x^2(x^3+10)^9$ and so our second guess is one anti-derivative of the integrand.  Adding ``+ c'' to the end yields all possible anti-derivatives.

\begin{prb} Evaluate the following indefinite integrals.
\begin{enumerate}
\item $\dsp \int(x+2)(3x^2 + 12x - 5)^{121} \; dx$
\item $\dsp \int(x+2)\sqrt[4]{3x^2+12x-5} \; dx$
\item $\dsp \int p^2 \sqrt[3]{p^3+1} \; dp$
\item $\dsp \int \frac{3\cos(\sqrt{t})}{\sqrt{t}} \; dt$
\item $\dsp \int \frac{x}{(3x^2+4)^5} \; dx$
\item $\dsp \int \frac{x}{3x^2+4} \; dx$
\end{enumerate}
\end{prb}

\begin{prb} These look quite similar.  Try both.  What's the difference?
\begin{annotation}
\endnote{ This problem is intended to train them to ``see'' the chain rule.  The first part is clearly integrable because the derivative of cosine appears, while the second is not integrable, because the derivative of $\cos(t^2)$ does not appear.  If there is time available, I ask how they might modify the second one to make it integrable, hoping that they will see that replacing the numerator with $t\sin(t^2)$ would make it an easy problem.}
\end{annotation}
\begin{enumerate}
\item $\dsp \int \frac{\sin(t)}{\cos^2(t)} \; dt$
\item $\dsp \int \frac{\sin(t)}{\cos(t^2)} \; dt$
\end{enumerate}
\end{prb}


\section{The Fundamental Theorem of Calculus}

The \emph{Fundamental Theorem of Calculus} (FTC) is one of the most applied theorems in all of calculus as it enables us to compute an integral {\it without} using Riemann sums.

To prove the FTC it will be helpful for us to understand the \emph{Mean Value Theorem for Integrals}. This theorem holds for any continuous function $f$ even though we state it here only for positive functions, that is, functions whose graphs are above the $x$-axis. This theorem says that if $f$ is a positive function defined on the closed interval $[a,b],$ then there is a rectangle with base the closed interval $[a,b]$ and height $f(c)$ for some $c$ in $[a,b]$ so that the area of this rectangle $f(c)(b-a)$ is equal to the area under the curve.   The height of this rectangle, $f(c)$, is called the average value of the function over the interval.  This is one of those theorems that is ``obvious'' once you draw a careful picture and follows from two powerful theorems that we did not prove, namely the Extreme Value Theorem and the Intermediate Value Theorem.

\begin{thm}
\textbf{The Mean Value Theorem for Integrals.} If $f \in C_{[a,b]}$, then there exists a number $c \in [a, b]$ such that $\dsp \int_a ^b f(x) \; dx = f(c)(b-a)$.
\end{thm}

\begin{prb}
Let $f \in C_{[a,b]}$ with maximum M and minimum m. Use a graph to show that $m(b-a) \le \int_a ^b f(x)\; dx \le M(b-a).$  Now apply the Intermediate Value Theorem to the function $g(x) = f(x)(b-a)$ to prove the Mean Value Theorem for Integrals.
\end{prb}

%\begin{prb}
%Use the Extreme Value Theorem to obtain the maximum M and minimum m of $f$ over $[a,b]$ and show (a good graph will do) that %$m(b-a) \le \int_a ^b f(x)\; dx \le M(b-a).$  Now apply the Intermediate Value Theorem to the function $g(x) = f(x)(b-a)$ to %prove the Mean Value Theorem for Integrals.
%\end{prb}

\begin{prb}
Let $f(x) = 2x^2 - 3$ on the interval $[0,2].$  Compute the value of $c$ from the Mean Value Theorem for Integrals.  You computed $\int_0^2 f(x) \; dx$ in Problem \ref{defint2}.
\end{prb}

\begin{dfn}
If $f \in C_{[a,b]}$, then the \textbf{average value} of $f$ over $[a, b]$ is the number $\dsp { \frac{1}{b-a} \int_a ^b f(x) \; dx }$.
\end{dfn}

\begin{prb} Find the average value of the function over the given interval.
\begin{enumerate}
\item $g(x)=x^2-3x$ over $[0,4]$
\item $f(x) = |x -2| + 3$ over $[0, 3]$
\end{enumerate}
\end{prb}

\begin{thm} \textbf{The\ Fundamental\ Theorem\ of\ Calculus.}
\begin{annotation}
\endnote{I love the proofs of calculus and especially the proof of the Fundamental Theorem.  Throughout calculus, I encourage those who express interest to consider taking Real Analysis to see the beauty that lives behind this course and many of my engineering students add mathematics as a double major or even as a full second degree in order to take this course. However, I make many concessions due to the full syllabus that we have here at Lamar and some semesters I rarely lecture over this proof simply due to time constraints.}
\end{annotation}
\begin{enumerate}
\item Let $f \in C_{[a,b]}$ and let $x$ be any number in $[a, b]$. If $F$ is the function defined by $$\dsp F(x) = \int_a ^x f(t) \; dt,$$ then $$F'(x) = f(x).$$
\item Let $f \in C_{[a,b]}$. If $F$ is any function such that $F'(x) = f(x)$ for all $x$ in $[a, b],$ then $$\dsp \int_a ^b f(x) \; dx = F(b) - F(a).$$
\end{enumerate}
\end{thm}

\textbf{Proof of Part 1.} Let $f \in C_{[a,b]}$ and define the function $F$ so that $F(x)$ is the definite integral of $f$ from $a$ to $x.$ That is,
$$F(x) =  \int_a^x f(t) \; dt \; \mbox{for every} \; x \in [a,b].$$
If $F'$ exists, then
\begin{eqnarray*}
F'(x) = \lim_{h \rightarrow 0} \frac{ F(x+h) - F(x) }{ h }
& = & \lim_{h \rightarrow 0} \frac{ F(x+h) - F(x) }{ h } \cr
& = & \lim_{h \rightarrow 0} \frac{1}{h} \left[ F(x+h) - F(x) \right] \cr
& = & \lim_{h \rightarrow 0} \frac{1}{h} \left[ \int_a^{x+h} f(t) \; dt - \int_a^{x} f(t) \; dt \right] \cr
& = & \lim_{h \rightarrow 0} \frac{1}{h}  \int_x^{x+h} f(t) \; dt.
\end{eqnarray*}
Applying the {\it Mean Value Theorem for Integrals} at this point, we have
$$\frac{1}{h}  \int_x^{x+h} f(t) \; dt  =\frac{1}{h} ( x+h \; - \; x) \cdot f(c)$$
for some value $c$ that is in the interval $[x, x+h]$.  Since this the interval depends on the value of $h,$ we replace $c$ by  $c_h$ to indicate that $c$ depends on $h.$  As $h \rightarrow 0^+$, we see that $x+h \rightarrow x$ so this interval will shrink to the point $x$.    Since $x < c_h < x+h$ and $h \rightarrow 0^+$, we have that  $c_h \rightarrow x$ by Theorem \ref{squeeze} (The Squeeze Theorem). This observation allows us to finish the proof.
\begin{eqnarray*}
F'(x) & = & \lim_{h \rightarrow 0} \frac{1}{h}  \int_x^{x+h} f(t) \; dt  \cr
& = & \lim_{h \rightarrow 0} \frac{1}{h} ( x+h \; - \; x) \cdot f(c_h)  \; \mbox{where}
\;  c_h \in [x,x+h]  \cr
& = & \lim_{h \rightarrow 0} \frac{1}{h} \cdot  h  \cdot f(c_h)  \;   \cr
& = & \lim_{h \rightarrow 0} f(c_h)  \;   \cr
& = & f(x)  \; \mbox{ because} \; c_h \rightarrow x \mbox{ and}  \;f  \;\mbox{is continuous.}   \cr
\end{eqnarray*}
\emph{q.e.d.}

\textbf{Sketch of Proof of Part 2.} Let $f \in C_{[a,b]}$.  Let
$$F(x)  =  \int_a^x f(t) \; dt \; \mbox{for every} \; x \in [a,b].$$
By the previous proof, $F$ is an anti-derivative of $f$.
$$F(b) - F(a) = \int_a^b f(t) \ dt - \int_a^a f(t) \ dt = \int_a^b f(t) \ dt - 0 = \int_a^b f(t) \ dt.$$
It might appear that we are done, but we have not shown that the theorem is true for \emph{any} anti-derivative of $f$, rather we have only shown that it was true for anti-derivatives of the form $F(x) = \int_a^x f(t) \ dt.$  Suppose we have some other anti-derivative, $G.$  Then $G' = f = F'$, so $G'=F'$.  If two functions have the same derivative then they must have the same shape.  Therefore, either they are the same function or one function is just the other function slid up or down along the y-axis.  It can be shown using the Mean Value Theorem if $F'=G',$ then $G(x) = F(x) + k$.  Therefore, for any anti-derivative $G$ we have,
$$G(b) - G(a) = \big(F(b) + k\big) - \big(F(a) + k\big) = F(b) - F(a) = \int_a^b f(t) \ dt.$$\emph{q.e.d.}

\begin{prb} Use the second part of the fundamental theorem.
\begin{enumerate}
\item Compute $\dsp \int_{1}^{3} (3x-1) \; dx$.  Verify your result using simple formulas for area.
\item Compute $\dsp \int_{0}^{2\pi} \cos^5(x) \sin(x) \; dx.$
\item Compute $\dsp \int_{2}^{3} x (10-x^{2})^3  \; dx.$
\item Compute $\dsp \int_{0}^{3} (x^2+2) e^{x^3+6x} \; dx.$
\end{enumerate}
\end{prb}

\begin{prb} Use the first part of the fundamental theorem.
\begin{enumerate}
\item Compute $F'$ where $\dsp F(x) = \int_{1} ^{x} t^2-3t \; dt$
\item Compute $F'$ where $\dsp F(x) = \int_{5}^{x} \frac{4}{t^4+ 1}\; dt$
\item Compute $F'$ where $\dsp F(x) = \int_{3}^{x^2} t (t^2-4)^2 \; dt$  by applying the chain rule to $F = f \circ g$ where $\dsp f(x) = \int_{3}^{x} t (t^2-4)^2 \; dt$ and $g(x) = x^2$.
\end{enumerate}
\end{prb}

\begin{prb}
Let $f(t)= |t^2 - t|$ and let $F$ be a function defined by $\dsp F(x)= \int_{- 4}^x f(t) \; dt$ where $x \in [-4, 4]$. Answer the following questions.
\begin{enumerate}
\item What is the derivative of $F$?
\item On what intervals is $F$ increasing?
\item On what intervals is $F$ decreasing?
\item On what intervals is $F$ concave upward?
\item On what intervals is $F$ concave downward?
\end{enumerate}
\end{prb}

\section{Applications}

Integration has many applications in all the sciences.  Here are a few we will consider.  Integration can be used to find the length of a curve.  Integration can be used to find the area between two curves, which is useful for computing the area of an oil spill as photographed from a plane. It can be used to derive the formulas for the volumes of objects like cones and to determine the volumes of other less ordinary objects.  It can be used to find the work done in stretching a spring or pumping water out of an oil well.    The key to all these problems is to first approximate the solution using a Riemann sum and then to take the limit as the number of divisions tends to infinity in order to obtain an integral that solves the problem.  Restated, the key is the definition of the definite integral,
$$\dsp \int_a^b f(x) \ \ dx = \lim_{N \to \infty} \sum_{i=1}^N f(x_i) (x_i - x_{i-1}).$$\\

\textbf{Area Between Curves}
\begin{annotation}
\endnote{Being on the Gulf Coast, I motivate the area between curves by drawing an oil spill on the board and asking how we might find the volume of this spill in order to determine what supplies our emergency management team needs to send.   Students generate good ideas -- digitize a photo or place a grid of lines over it and count the squares.  I foreshadow the notion of using interpolation to approximate the ``top'' of the spill with a function $f$ and the ``bottom'' of the spill with a function $g$ and then point out that the area of the spill would be the $\int_a^b f(x) \ dx - \int_a^b g(x) \ dx$, and the volume would be the thickness of the spill times this area.   My goal in such applied discussions is that by foreshadowing numerical mathematics such as finding functions that approximate a set of points,  I will pique their interest in future classes.}
\end{annotation}
\\

\begin{expl}
Discuss using area between curves to approximate the volume of an oil spill, followed by computing the area between $y=x^2+1$ and $y=2x+1$ by adding up the areas of rectangles to define an integral, thus illustrating the central theme of the section.
\end{expl}

\begin{prb}
Consider the curves $y=x^3$ and $y=x.$
\begin{enumerate}
\item Graph these curves and shade the regions bounded by these curves.
\item Draw a few rectangles and write a sum that estimates the area between these curves.
\item Set up and evaluate the definite integrals that represent the area of each of the shaded regions.
\end{enumerate}
\end{prb}

\begin{prb} Use techniques in the previous problem to compute the areas of the regions bounded by these curves.
\begin{enumerate}
\item $y = x^2+4$ and $x+y=6$
\item $x=y^2$ and $y=x-2$
\item $x = \cos(y)$ and $x=0$ from $y=0$ to $y=\pi$
\end{enumerate}
\end{prb}

\begin{prb}
Sketch the area bounded by $y=x^2$, $y=0$, and $x=2.$   Compute the area of this region by computing $\dsp \int_0^2 x^2 \ dx.$  Now fill in the blanks to integrate this along the y-axis:  $\dsp \int_0^2 x^2 \ dx = \int_0^4 \_\_\_\_ \ dy.$
\end{prb}

\textbf{Arc Length}\\

Every one who rides a skateboard knows that you can scrub speed off by slaloming back and forth as you proceed down a hill.  In doing so, you travel a greater distance than the speed freak who takes  a straight line.  If we have a road of length $2 \pi$ and one person takes the straight line path while a second follows the sine curve, how much farther does the second person travel?

\begin{prb}
Let $f(x) = \sin(x)$ from $x=0$ to $x=2 \pi.$
\begin{enumerate}
\item Consider the partition of $[0,2\pi],$ defined by $x_0 = 0,\ x_1 = \pi/2,\ x_2 = \pi,\ x_3 = 3\pi/2 , \ x_4 = 2\pi.$ Approximate the length of the curve by summing the lengths of the straight line segments from $\ (x_{i-1}, f(x_{i-1}))$ to $\ (x_{i}, f(x_{i}))$ for $i=1,2,3,4.$
\item Generalize this so that the partition has $N$ divisions and take the limit as $N \to \infty$.
\end{enumerate}
\end{prb}

\begin{dfn}
A function $f$ is said to be {\bf{continuously differentiable}} on $[a, b]$ if its derivative $f'$ is continuous on $[a, b]$. If $f$ is continuously differentiable on $[a,b]$, then its graph on the interval $[a, b]$ is called a {\bf{smooth curve}}.
\end{dfn}

\begin{prb}
Suppose the function $f$ and its derivative $f'$ are continuous on $[a, b]$. Let $s$ be the arc length of the curve $f$ from the point $(a, f(a))$ to $(b, f(b))$.
\begin{enumerate}
\item Let $a = x_0 < x_1 < x_2 < \cdots < x_n = b$ be a partition of $[a, b]$. \\
Show that $\displaystyle{s \approx \sum_{i=1} ^n (x_i-x_{i-1}) \sqrt{1 + \Big(\frac{  f(x_i) - f(x_{i-1})}  {x_i - x_{i-1}}  \Big)^2 }   }$.
\item Show that $\dsp s = \int_a ^b \sqrt{1 + [f'(x)]^2}\; dx$ by using the Mean Value Theorem for differentiation.
\end{enumerate}
\end{prb}

The result you just proved can be summarized in the following theorem.

\begin{thm}
\textbf{Arc Length Theorem.} If $s$ is the arc length of the smooth curve $f$ on the interval $[a, b]$, then $\dsp s = \int_a ^b \sqrt{1 + [f'(x)]^2}\; dx$.
\end{thm}

\begin{prb}
For each of the following curves, sketch its graph and find the length of the arc with the indicated endpoints.
\begin{enumerate}
\item $\dsp y=x^{\frac{3}{2}}$ from the point $(0,0)$ to the point $(4,8).$
\item $\dsp y=\frac{1}{4}x^3+\frac{1}{3}x^{-1}+2$ from the point where $x = 2$ to the point $x = 5.$ \item $x^2=(y+1)^3$from the point where $x = 0$ to the point where $x=3 \sqrt{3}.$ You may integrate with respect to $x$ or $y.$  Which is easier?
\end{enumerate}
\end{prb}

\textbf{Work and Force}\\

\textbf{Hooke's Law} states that if a spring with force constant $k$ is stretched a distance of $x$ units, then the spring will exert a force of $F(x) = -kx$.  The constant $k$ is measured in either Newtons/meter or pounds/foot depending on whether we are using the metric or English system.   The \textbf{work} done by a force acting on an object is defined to be force times the distance traveled (displacement).

\begin{prb}
Imagine a spring with natural length of 6 inches.  A force of 18 pounds stretches the spring 0.5 inches.
\begin{enumerate}
\item Use Hooke's Law to determine the spring's coefficient, k.
\item Let $2 = x_0 < x_1 < x_2 < \dots < x_{N-1} < x_N = 4$ and explain why $\sum_{i=1}^N -k x_i (x_i - x_{i-1})$ approximates the work done in stretching the spring from 8 inches to 10 inches.
\item Compute $\int_2^4 -kx \ dx$ and explain why your answer is exactly the work done in stretching the spring from 8 inches to 10 inches.
\end{enumerate}
\end{prb}

\textbf{Newton's Second Law of Motion} states that force equals mass times acceleration, $F=ma$. In the SI system the density of water is 1000 kg/${\rm{m^3}}$ and the acceleration due to gravity is 9.81 m/${\rm{sec^2}}$. Therefore the weight of 1 ${\rm{m^3}}$ of water =(1000)(9.81) N/${\rm{m^3}}$ = 9810 N/${\rm{m^3}}$. In the British system, the density of water is 1.94 slugs/${\rm{ft^3}}$ and the acceleration due to gravity is 32.2 ft/${\rm{sec^2}}$. Therefore the weight of water is approximately 62.5 lb/${\rm{ft^3}}$.

\begin{prb}
A cylindrical tank is 14 m across the top, 10 m deep and filled to a height of 8 m with water.
\begin{enumerate}
\item Subdivide the water's depth so that $2 = x_0 < x_1 < x_2 < \dots < x_{N-1} < x_N = 10$ and explain why $\dsp 25 \pi (x_i - x_{i-1})$ is the volume of the $i^{th}$ cylindrical slab of water.
\item  Explain why $\dsp 62.5 \cdot 49  \pi  (x_i - x_{i-1})$ is the weight of this cylindrical slab of water.
\item Explain why $\dsp \sum_{i=1}^N 3062.5 \pi x_i  (x_i - x_{i-1})$  approximates the work done to pump the water to the top of the tank.
\item Compute $\dsp \int_2^{10} 3062.5   \pi  x \ dx$ and explain why this is the work required to pump the water.
\end{enumerate}
\end{prb}

\begin{prb}
A tank in the form of an inverted right-circular cone (think ice-cream cone) is 12 m across the top, 10 m deep and filled to a height of 8 m with water.
\begin{enumerate}
\item Subdivide the water's depth so that $2 = x_0 < x_1 < x_2 < \dots < x_{N-1} < x_N = 10$ and compute the volume of the $i^{th}$ approximately cylindrical slab of water as if it were a perfect cylinder.
%$\dsp \pi \big( \frac{3}{5} (10-x_i) \big)^2 (x_i - x_{i-1})$
\item Compute the weight of this approximately cylindrical slab of water.
%$\dsp 62.5 \; \pi \big( \frac{3}{5} (10-x_i) \big)^2 (x_i - x_{i-1})$
\item Write down a Riemann sum that approximates the work done in pumping the water to the top of the tank.
%$\dsp \sum_{i=1}^N 62.5 \; x_i \; \pi \big( \frac{3}{5} (10-x_i) \big)^2 (x_i - x_{i-1})$
\item Write down and compute an integral that represents the work required to pump the water out of the tank.
%$\dsp \int_2^{10} 62.5 \; x \; \pi \big( \frac{3}{5} (10-x) \big)^2 \ dx$
\end{enumerate}
\end{prb}

\begin{prb}
A cable 150 ft long and weighing 5 lb/ft is hanging vertically down into a well. If a metal bucket of 50 lb is suspended from the lower end, find the work done in pulling the cable and the empty bucket up to the top of the top of the well.
\end{prb}
% $50 * 150  + \sum_{i=1}^N 5 x_i (x_i - x_{i-1}}$ = 750 + \int_0^{150} x \ dx$

\begin{prb}
Using the data from the previous problem, if the bucket can hold 20 ${\rm{ft^3}}$ of water, find the work done in pulling the cable and the bucket full of water up to the top of the well.
\end{prb}
% previous answer + 20 ft^3 * 62.5 lbs/ft^3 * 150 ft traveled


%\textbf{Pressure}
%\begin{enumerate}

%\item The pressure exerted by the liquid at the point is $P = \rho gh$ where $\rho$ is the density of the liquid, $g$ is the acceleration due to gravity, and $h$ is the depth of a point below the surface of the liquid.

%\item Suppose $A$ is the area of a flat plate that is placed horizontally under the liquid, and $F$ is the force caused by liquid pressure acting on the upper face of the plate, then $F=PA$, so $F = \rho ghA$.

%\item Suppose a flat plate is submerged vertically into a liquid whose density is $\rho$ and the length of the plate at a depth of $y$ units below the surface of the liquid is $f(y)$ units, where $f$ is assumed to be continuous on $[c, d]$ and $f(y) \ge 0$ on $[c, d]$. The $F$, the force caused by liquid pressure on the plate, can be expressed as $\dsp F= \int_c ^d \rho gyf(y)dy.$
%\end{enumerate}

%\begin{prb}
%The ends of a trough are semicircular in regions, each with radius of 4 ft. Find the force caused by water pressure on one end if the trough is full.  Find the force if the trough is half-full.
%\end{prb}


\textbf{Volumes of Solids}
\begin{annotation}
\endnote{I confess to having skipped this section many semesters when time was tight, even though I really enjoyed and gained a lot from it as a student.}
\end{annotation}
\\

We now develop a technique that will allow us to compute volumes of various three dimensional objects.  A {\it solid} is a three dimensional object along with its interior; think bowling ball, not balloon. When we computed areas, we began with Riemann sums to approximate the area and as the number of rectangles tended to infinity, the integral expression turned out to be nothing more than integrating over the height of the function at each point.  That is, $$\lim_{n \to \infty} \sum_{i=1}^n f(x_i) (x_i - x_{i-1}) = \int_a^b f(x) \ dx.$$  Thus what we integrated was simply the height of the line segment from $(x,0)$ to $(x,f(x))$.   Intuitively, we might think of this as adding up infinitely many line lengths to get the area.  Of course, the only \emph{mathematically valid} way to think of this is as the limit of sums of areas of rectangles. The same idea applies to solids.  To be precise, we need to slice the volume into $n$ pieces and sum the volumes of each of the $n$ pieces.  Then we would take the limit as $n \to \infty.$   Intuitively, we might think of slicing the solid into infinitely many cross sections and adding up the area of all these cross sections.  The volume of a solid is the integral of the function that tells us the area of each cross section. Let's call this the \emph{method of slicing}.

\begin{prb}
Sketch the region bounded by the lines $\dsp y=\frac{1}{2}x$, $y=0$, and $x = 4$ and the solid cone obtained by rotating this region about the x-axis.
\begin{enumerate}
\item Draw a two dimensional slice of this solid parallel to the base with center at the point $(x,0)$.
\item Find the function $A(x) = \dots$ that gives the area of this two dimensional slice.
\item Compute $\dsp \int_0^4 A(x) \ dx$ to determine the volume of your solid.
\end{enumerate}
\end{prb}

\begin{prb}
Sketch a pyramid whose base is a square of length $4$ and whose height is $10.$
\begin{enumerate}
\item Write a function $A(h)= \dots$ that gives the area of the slice that is parallel to the base and is  at height $h.$
\item Compute $\dsp \int_0^{10} A(h) \ dh$ to determine the volume of the pyramid.
\end{enumerate}
\end{prb}

\begin{prb}
Compute the volume $($$\dsp V=\frac{1}{3}\pi r^2 h$$)$ of the cone  with a circular base of radius $r$ and a height of $h.$
\end{prb}

\begin{prb}
Compute the volume $($$\dsp V=\frac{1}{3} b^2 h$$)$ of the pyramid with a square base of side length $b$ and a height of $h.$
\end{prb}


\begin{prb}
Compute the volume of the solid whose base is the disk $x^2 + y^2 \le 4$ and whose cross sections are squares perpendicular to the diameter of the base through $(-2, 0)$ and $(2, 0)$.  Can you sketch this?
\end{prb}

The method of slicing can be used to find the volumes of {\it surfaces of revolution} which are solids created by taking a two dimensional region (an area) and rotating this region about a line to form a three dimensional solid. For example, take the unit circle and spin it about the $x$-axis and you get a sphere along with its interior.  Or, take a circle of radius one centered at (2,0) and spin it around the $y$-axis and you get a donut.

\begin{expl}
Find the volume obtained by rotating the region bounded by $f(x) = x^3-x$ and $y=0$ about the x-axis.
\end{expl}

Sketching this, we get two solids.  Looking at the portion where $x>0$, we see something shaped like a lime with ends at $(0,0)$ and $(1,0)$.  Choose a value for $x$ between $0$ and $1$.  If we think about slicing the lime parallel to the $y$-axis and through the point $(x,0)$, then the cross section (slice) will be a circle and the area of this circle is $A = \pi r^2 = \pi (f(x))^2$ since the radius of the circle is $f(x).$  Therefore by the method of slicing, the volume of the solid $S$ is given by $$V = \pi \int_a ^b [f(x)]^2 \; dx.$$  This method of finding the volume of revolution is called the \textbf{disk method} since intuitively we are adding up the area of an infinite number of disks.

\begin{prb}
Compute the volume of the ``lime'' in the preceding example.
\end{prb}

There is nothing special about the x-axis.

\begin{prb}
Consider the region bounded by $y=x^3$, $x=0$, and $y=27.$   Sketch the solid created by rotating this region about the $y-axis$.  Compute the area function $A(y) = \dots$ that tells the area of one horizontal slice at height $y$ (this would be a circle with center the point $(0,y)$).   Now compute the volume of this solid of revolution, $\dsp {V = \pi \int_0^{27} A(y) \ dy}.$
\end{prb}

\begin{prb}
\begin{enumerate}
\item Sketch the region bounded by the curves $y = x^2 +1$, $x = -1$, $x = 2$, and $y=-1.$
\item Sketch the solid of revolution generated by revolving this region about the line $y = -1.$
\item Let $-1 < x < 2$ and sketch the slice of the solid at $x$ that is perpendicular to the line $y=-1.$
\item Determine the area of this slice, $A(x).$
\item Integrate this area function from $x=-1$ to $x=2$ to determine the volume of the solid.
\end{enumerate}
\end{prb}

\begin{prb}
Find the volume of the solid generated by rotating the region bounded by $x = 0$, $y=6-x$, and $y=3$ around the $x$-axis.
\end{prb}

\begin{prb}
Suppose $a$ and $b$ are positive numbers with $a < b.$ Find the volume of the solid generated by rotating about the $x$-axis the region bounded by: $x = 0$, $y=x$, $y=a$, and $y = b.$
\end{prb}

\begin{prb}
Find the volume of the solid generated by revolving about the $y$-axis the region bounded by the line $y = 2x$ and the curve $y = x^2.$
\end{prb}

Now that we know this method of slicing up the surfaces, one might ask ``Is this the only way we can slice up the surface to find the volume?''  The answer is that there are lots of ways to slice up the surfaces.  Let's look at another way.

\begin{prb}
Find the volume obtained by rotating the region bounded by $f(x) = x-x^4$ and the x-axis about the x-axis.
\end{prb}

\begin{expl}
Find the volume obtained by rotating around the y-axis the region bounded by $f(x) = x-x^4$ and the x-axis.
\end{expl}

To use the disk method, we would need to rewrite the function $f$ which has as its independent variable, $x$, as a function $g$ which has as its independent variable, $y.$  Go ahead and try -- if you can, you can skip the rest of this section!  If you can't, read on!

We will still be using the method of slicing, but we will use a ``circular knife'' like a cookie cutter so that instead of circular slices, we will have cylindrical slices.  We will add up the surface area of each cylindrical slice, just as before we added up the areas of each circular slice.

Choose a value $x$ between $0$ and $1$ and draw the line segment from $(x,0)$ to $(x,f(x)).$  Now rotate this line about the y-axis. Do you see that this is a ``cylindrical slice'' of your solid?  If we consider all such slices with $x$ ranging from $0$ to $1$, then integrating the function that gives the surface area of each slice will (again) yield the volume of our solid.  But what exactly is the surface area of our slice?  The radius of the cylinder is $x$ and the height is $f(x)$ so the area must be $A(x) = 2 \pi x f(x).$  And the  volume will be the integral of this quantity, $$V = 2 \pi \int_0 ^1 x f(x) \; dx = 2 \pi \int_0 ^1 x (x-x^4) \; dx = \frac{\pi}{3}.$$  This method is called the {\it{shell method}} or the {\it cylindrical shell method}.

\begin{prb}
Find the volume of the solid generated by revolving about the $x$ axis the region bounded by the line $y = \sqrt{x}$ the $x$ axis, and the line $x = 4$ using
\begin{enumerate}
\item the shell method, and
\item the disk method.
\end{enumerate}
\end{prb}

Sometimes the best way to get a deep understanding of a concept is to take a simple problem and work it in every possible way.  The next problem is just that sort of problem.

\begin{prb}
Consider the region in the first quadrant bounded by $y=x^2,$ the y-axis, and $y=4.$ Find the volume of the solid obtained by rotating this region about:
\begin{enumerate}
\item the y-axis using the disk method,
\item the y-axis using the shell method,
\item the x-axis using the disk method, and
\item the x-axis using the shell method.
\end{enumerate}
\end{prb}

\begin{prb}
Use any method to find the volume of the solid generated by revolving about the $y$-axis the region bounded by the curves $y = 8-x^2$ and the curve $y = x^2$.
\end{prb}


\textbf{Center of Mass}\\

\begin{dfn}
\textbf{One-dimensional Moments (Discrete Case)} Place $n$ weights with masses $m_1, m_2, \dots, m_n$ at the points $(x_1,0), (x_2,0), \dots, (x_n,0)$ on the x-axis.\\ \\
The \textbf{mass} is
$$M = \displaystyle{\sum_{i=1}^n m_i} \;\;\;,$$
the \textbf{moment of mass} is
$$M_0 = \displaystyle{\sum_{i=1}^n x_im_i} \;\;\;,$$
and the \textbf{center of mass} is
$$\overline{x} = \frac{\sum_{i=1} ^n x_im_i}{\sum_{i=1}^n m_i}\;\;\;,$$ which is the point on the x-axis where everything would balance.
\end{dfn}

\begin{dfn}
\textbf{One-dimensional Moments (Continuous Case).} Let $L$ be the length of a rod whose left hand endpoint is at the
origin of the $x$ axis and let $\rho$ be a continuous function on $[0, L]$ such that $\rho(x)$ is the density at $x \in [0, L]$.\\ \\
The \textbf{mass} is
$$\dsp M = \int_0 ^L \rho(x) \;\ dx \;\;\;,$$
the \textbf{moment of mass} is
$$\dsp M_0 = \int_0 ^L x \rho(x) \; dx\;\;\;,$$
and the \textbf{center of mass} is
$$\overline{x}=\frac{\int_0 ^L x \rho(x) \; dx}{\int_0 ^L \rho(x) \; dx}\;\;\;.$$
\end{dfn}

\begin{prb}
Find the moment of mass and the center of mass of each of the following systems.
\begin{enumerate}
\item $m_1=4$ at $x_1=-3$; $m_2=1$ at $x_2=-1$; $m_3=2$ at $x_3 =3$; $m_4 =1$ at $x_4=-5$; $m_5=5$ at $x_5=-4$; $m_6=6$ at $x_6=1$
\item A rod whose length is 5 meters with linear density $3x+2$ kg/m where $x=0$ is the left-hand endpoint of the rod.
\end{enumerate}
\end{prb}

\begin{dfn}
\textbf{Two-dimensional Moments (Discrete Case).} Assume that the thickness and the weight of a sheet of material are negligible and there are $n$ particles with mass $m_1, m_2, \cdots m_n$ located at $(x_1, y_1),$ $ (x_2, y_2), \cdots,$ $(x_n, y_n)$ on the $xy$ plane. Then the \textbf{total mass} of the system ($M$), the \textbf{total moment with respect to the $x$ axis} ($M_x,$),  the \textbf{total moment with respect to the $y$ axis} ($M_y$), and  the \textbf{center of mass of the system} ($(\overline{x}, \overline{y})$), are defined by
$$\dsp M= \sum_{i=1} ^n m_i \;\;\;,$$
$$\dsp (M_x,M_y) = (\sum_{i=1}^n y_i m_i, \sum_{i=1}^n x_i m_i)\;\;\;, \mbox{ and }$$
$$\dsp (\overline{x},\overline{y}) = (\frac{M_y}{M}, \frac{M_x}{M})\;\;\;.$$
\end{dfn}

\begin{dfn}
\textbf{Two-dimensional Moments (Continuous Case).} A \textbf{lamina} is a two dimensional material of continuously
distributed mass. A \textbf{homogeneous lamina} is a lamina with constant area density. Assume that a lamina $L$ is bounded by the curve $y=f(x)$, the $x$ axis, and the lines $x = a$ and $x = b$ such that $f$ is continuous on $[a, b]$, $f(x) \ge 0$ for all $x$ in $[a, b]$, and the \textbf{area density} function $\rho$ is also continuous on $[a, b]$. Then the \textbf{total mass} of the system ($M$), the \textbf{total moment with respect to the $x$ axis} ($M_x,$),  the \textbf{total moment with respect to the $y$ axis} ($M_y$), and  the \textbf{center of mass of the system} ($(\overline{x}, \overline{y})$), are defined by
$$\dsp M= \int_a ^b \rho(x) f(x) \; dx\;\;\;,$$
$$\dsp (M_x,M_y) = \left( {1 \over 2} \int_a ^b \rho(x)[f(x)]^2 \; dx, \; \int_a ^b x \rho(x) f(x) \; dx \right)\;\;\;, \mbox{ and }$$
$$\dsp (\overline{x},\overline{y}) = (\frac{M_y}{M},\frac{M_x}{M})\;\;\;.$$
\end{dfn}

\begin{prb}
Find the total mass and center of mass of the collection of objects:  $m_1=4$ at $(-1, 3)$; $m_2=1$ at $(0, 0)$; $m_3=2$ at $(2,-4)$; $m_4 =1$ at $(4, -3)$.
\end{prb}

\begin{prb}
Find the total mass and center of mass for a lamina bounded by the curve $y=x^2-4$ and the $y=3x+6$ with the area density function $\rho(x)=x+1$.
\end{prb}


\section{Practice} \label{chap4probs}

We will not present the problems from this section, although you are welcome to ask about them in class.

\vskip .1in
\noindent
\textbf{Riemann Sums}

\begin{enumerate}
\item Sketch $f(x)=x^{2}+2$ on $[0,3].$  Subdivide the interval into 6 subintervals of equal length and compute the upper and lower Riemann sums of $f$ over this partition.
\item Use the definition of the integral (as the limit of Riemann sums) to compute the area under $f(x)=x^{2}+2$ on $[0,3].$
\item Approximate the area under $f(x)=-\sqrt{5-x}$ on $[1,2]$ with $n=4$ using upper and lower sums.
\end{enumerate}

\noindent
\textbf{Definite and Indefinite Integrals}

\begin{enumerate}
\item Compute the following indefinite integrals.
    \begin{enumerate}
    \item $\dsp{\int 2x+ \frac{3}{4} x^2 \; dx}$
    \item $\dsp\int 1+e^s \; ds$
    \item $\dsp \int \sin(x) \; dx$
    \item $\dsp \int \cos(y) \; dy$
    \item $\dsp \int \pi \cos(x) + \pi \; dx$
    \item $\dsp \int \sin(2t) \; dt$
    \item $\dsp{\int 5\sqrt{x}-\frac{1}{x} \; dx}$
    \item $\dsp \int (x^2 + 1) \; dx$
    \item $\dsp \int (x^2 + 1)^2 \; dx$
    \item $\dsp \int x(x^2 + 1)^2 \; dx$
    \item $\dsp \int 2x\cos(x^2) \; dx$
    \item $\dsp{ \int \frac{5}{t} +\frac{1}{5t}+5^{t}\; dt}$
    \item $\dsp \int \sin^4(t)\cos(t) \; dx$
    \item $\dsp \int (5x^3 - 3x^2)^8(15x^2 - 6x) \; dx$
    \item $\dsp \int (y^2-2y)^2(5y^4-2) \; dy$
    \item $\dsp \int t^2(t^3 - 3t) \; dt$ \item $\int -\pi x + e \; dx$
    \end{enumerate}
\item Compute the following definite integrals.
    \begin{enumerate}
    \item $\dsp \int_{0}^{2} 6-2x \; dx$
    \item $\dsp \int_{2}^{5} x^{2}-6x+10 \; dx$
    \item $\dsp \int_{-1}^{3} 4x^{3}-x+2 \; dx$
    \end{enumerate}
\item Evaluate each integral.
    \begin{enumerate}
    \item $\dsp \int 2x+1 \ dx$
    \item $\dsp \int \frac{1}{3}t^{3}-\sqrt{t} \ dt$
    \item $\dsp \int 4(\sec^{2}(x)-\cos(x)) \ dx$
    \item $\dsp \int z\sqrt{z}+\frac{4}{z^{2}}+\sqrt[3]{4z} \ dz$
    \item $\dsp \int \frac{x+1}{\sqrt{2x}} \ dx$
    \item $\dsp \int \sin(t)+\sqrt{\frac{3}{t}} \ dt$
    \item $\dsp \int_{}^{} x^{2}\sin(x^{3}) \ dx$
    \item $\dsp \int_{}^{} \cos^{5}(x)\sin(x) \ dx$
    \item $\dsp \frac{1}{2} \int_{}^{} 2x\sqrt{1-x^{2}} \ dx$
    \item $\dsp \int \frac{z+1}{^{3}\sqrt{3z^{2}+6z+5}} \ dz$
    \item $\dsp \int_{-1}^{0} (\cos(2t)+t) \ dt$
    \item $\dsp \int_{0}^{\frac{\pi}{4}} \sin(\theta )-\cos(\theta ) \ d\theta $
    \item $\dsp \int_{-2}^{1} (5-x)^{3} \ dx$
    \item $\dsp \int_{\pi /4}^{\pi /2} \frac{\cos(x)}{\sin^{2}(x)} \ dx$
    \item $\dsp \int_{-\sqrt{3}}^{2} 4x\sqrt{x^{2}+1} \ dx $
    \end{enumerate}
\item Evaluate each of the following integrals, utilizing the suggested substitution.
    \begin{enumerate}
    \item $\dsp \int \frac{x}{\sqrt{x+5}} \; dx$;  \;\;\; $u(x) = \sqrt{x+5}.$
    \item $\dsp \int x \sqrt[3]{x-1} \; dx$;  \;\;\; $u(x) = \sqrt[3]{x-1}.$
    \item $\dsp \int \frac{x^3}{\sqrt{4-x^2}} \; dx$;  \;\;\; $u(x) = \sqrt{4-x^2}.$
    \item $\dsp \int \frac{x^3}{\sqrt{x^2+1}} \; dx$;  \;\;\; $u(x) = \sqrt{x^2+1}.$
    \item $\dsp \int \sqrt{9-x^2} \; dx$;  \;\;\; $x =  3\sin(u).$
    \item $\dsp \int \frac{1}{4-x^2} \; dx$;  \;\;\; $x = 2\sin(u).$
    \item $\dsp \int \frac{1}{1+x^2} \; dx$;  \;\;\;  $x = \tan(u).$
    \end{enumerate}
\end{enumerate}

\newpage
\noindent
\textbf{Average Value, Mean Value, Fundamental Theorem and Arclength}

\begin{enumerate}

\item Compute the average value of $f(x) = x^3$ over $[-3,1].$

\item Compute the average value $f(x) = |x|$ over $[-1,1].$

\item Let $f(x) = x^2 + 2x-4$ be the continuous function on $[0,4]$. Find $c$ in $[0,4]$ as stated in the Mean Value Theorem for integrals.

\item Let $F(x) =\dsp \int_{1 \over 2}^{x} \cos(t) \; dt$.  Compute the integral and then take the derivative of your result.

\item Compute $\dsp {{d} \over {\; dx}} \int_{1 \over 2}^{x} t^3 - 4t + e^t \; dt$

\item Use the chain rule to compute $\dsp {{d} \over {\; dx}}     \int_{9}^{\sqrt{x}} {e^{t^2}} \; dt$

\item Find the arc length of $y^2=4x^3$ from $(0,0)$ to $(4,16).$

\end{enumerate}

\noindent
\textbf{Area and Volume}

\begin{enumerate}

\item Find the area of the region bounded by $y=x^2$, $x=-2$, $x=-1$, and the $x$-axis.

\item Find the area of the region bounded by  $y=2|x|$, $x=-3$, $x=1$, and the $x$-axis.

\item Find the area of the region bounded by $y=x^2-4$ and the $x$-axis.

\item Find the area of the region bounded by $y=x^2$ and $y=-x^2+6x$.

\item Find the area of the region bounded by $y=x^3$ and $y=4x$.

\item Find the area of the region bounded by $x=y^2$ and $x-2y=3$.

\item Is there a region bounded by $x-3=y^2-4y$ and $x=-y^2+2x+3$?

\item Find the area of the region bounded by $x=2y-y^2$ and $x=-y$.

\item Find the area of the region bounded by $y=x^3+3x^2-x+1$ and $y=5x^2+2x+1$.

%\item Find the volume of the solid that lies between planes perpendicular to the $x$-axis at $x = -1$ and $x=1$ between these planes are squares whose diagonals run from the semi-circle $y=-\sqrt{1-x^2}$ to the semi-circle $y = \sqrt{1-x^2}$.

%\item The base of a solid is the triangular region with vertices $(0,0)$, $(1,0)$, and $(0,1)$. Suppose each slice of the solid is perpendicular to this triangular base such that the bottom side of the slice is parallel to the line segment joining $(1,0)$ and $(0,1)$. Find the volume of the solid.

%\item Find the volume of the solid generated by revolving about the line $x=-1$ the region bounded by the curve $(x+1)^2=20-4y$ and the lines $x=-1$, $y=1$, and to the right of $x=-1$. \vskip

%\item Find the volume of the solid generated by revolving about the $x$-axis the region bounded by the curve $y=|x-1|+2$ and the line $y=4$.

\item Find the volume of the solid generated by revolving the region bounded by the curve $y = \sqrt{x}$, the $x$-axis, and the line $x=4$ revolving about the $x$-axis using (a) the disk method and (b) the shell method.

\item Find the volume of the solid generated by revolving about the $y$-axis the region bounded by the line $y=2x$ and the curve $y=x^2$.

%\item Find the volume of the solid generated by revolving about the line $y=-2$ the region bounded by the curve $y=|x-1|+2$ and the line $y=4$.

\item Find the volume of the solid generated by revolving about the line $x=6$ the region bounded by the line $y=2$ and the curve $y=x^2$.

%\item Find the volume of the solid generated by revolving about the line $x = -2$ the region bounded by the curve $y=|x-1|+2$ and the line $y=4$.


\item Compute the volume of the solid whose base is the disk $x^2+y^2 \le 4$ and whose cross sections are equilateral triangles perpendicular to the base.

\item  Verify the formula for the volume of a sphere by revolving the region bounded by the circle $x^2 + y^2=R^2$ about the $y$-axis with the disk method.

\item Find the volume of the solid generated by revolving the region bounded by the curve $y=(4x-1)^{1/3}$, the $x$-axis, and the line $x=4$, revolving about the $x$-axis using the (a) cylindrical shell method and (b) the disk method.

\item Find the volume of the solid generated by revolving the region bounded by the curve $y^2-y=x$ and the $y$-axis, about $x$-axis.

\item Find the volume of the solid generated by revolving the region bounded by the curve $y=8-x^2$, $y=x^2$, about the $y$-axis.

\item Find the volume of the solid generated by revolving the region by the curves $y=|x|+1$ and $y=2$, about the $x$-axis.

\end{enumerate}

%\item Find the volume of the solids generated revolving the region bounded by the curve $y=\sqrt{x}$, $y=1$, and $x=0$ about
%    \begin{enumerate}
%    \item the $x$-axis
%    \item the $y$-axis
%    \item the line $x=1$
%    \item the line $y=1$
%    \item the line $x=2$
%    \item the line $y=4$
%    \end{enumerate}

\noindent{Work}

\begin{enumerate}
\item Suppose we have a spring which is 6 inches at rest and 10 pounds of force will stretch it 4 inches.  Write out and compute an integral that equals the work required to stretch the spring 6 inches from its resting state.
\item Find the work required for a crane to pull up 500 feet of cable weighing 2 lbs/ft with an 800 pound wrecking ball on it.
\item Suppose we have a 1 m deep aquarium with a base of 1 m by 2 m which is completely full of water.  How much work is required to pump half of the water out?
\end{enumerate}

\noindent
\textbf{Center of Mass}
\begin{enumerate}

\item  Let the length of a rod be 10 meters and the linear density of the rod $\rho(x)$ be written in the form $\rho(x)=ax+b$ with $x=0$ representing the left end of the rod and $x=10$ representing the right end of the rod. If the density of the rod is $2 kg/m$ at the left end and 17 kg/m at the right end, find the mass and the center of mass of the rod.

\item Find the centroid (center of mass with of an object with constant density) of the region bounded by $y=x^3$ and $y=x$ in the first quadrant.  What would the center of mass be if we considered only the third quadrant?  What if we considered both the first and third quadrant?


\end{enumerate}

%\item A force of 10 N stretches a spring of natural length 5 meters to an additional 0.5 meter. Find the work done in
%stretching from
%    \begin{enumerate}
%    \item its natural length to  6 meters;
%    \item 6 meters to 7 meters.
%    \end{enumerate}

%\item A hemispherical tank, placed so that the top is a circular region of radius 4 ft, is filled with water to a depth of 3 ft. Find the work done in pumping the water to the top of the tank.

%\item A trough full of water is 12 ft long, and its cross section is in the shape of an isosceles triangle 2 ft wide across the top and 1 ft high. Find the work done pumping all the water to a level 3 ft above the top of the tank.

%\item A bucket weighing 30 lb containing 10 cu ft of water is attached to the lower end of a chain 80 ft long and weighing 2 ft/lb that is hanging in a deep well.
%    \begin{enumerate}
%    \item Find the work done in raising the bucket to the top of the well.
%    \item If water is leaking out of the bucket at a constant rate and has all leaked out just as soon as the bucket is at the     top of the well. Find the work done in raising the bucket to the top of the well
%    \end{enumerate}

%\item The ends of a trough are in the shape of equilateral triangles of length 2 ft. Find the force caused by water pressure on one end if the trough is full.

%\item The face of a gate of a dam is in the shape of an isosceles trapezoid 4 meters wide at the top and 6 meters wide at the bottom, and 5 meters high. If the top of the gate is 50 meters below the surface of the water, find the force caused by water pressure on the gate.

%\item A rectangular fish tank with interior dimensions of 4 ft by 6 ft and height 3 ft is filled by water to within 1 inch to the top. Find the fluid force against each side of the fish tank.

%\item A plate in the shape of a region bounded by the parabola $x^2=4y$ and the line $y=4$ is placed into a liquid such that the vertex of the parabola is the bottom and the line is on top and is the surface of the liquid.
%    \begin{enumerate}
%    \item If the density of the liquid is a constant $\rho$, find the fluid force against the plate.
%    \item If the plated is positioned 5 units below the surface of the liquid, find the fluid force again the plate.
%    \end{enumerate}


\vskip .5in
\noindent
\textbf{Chapter 4 Solutions}\\ \\

\noindent
\textbf{Riemann Sums}

\begin{enumerate}
\item Lower = 12.875, Upper = 17.375, Average = 15.125
\item 15
\item Upper $\approx$ -1.836, Lower $\approx$ -1.903, Average $\approx$ -1.869
\end{enumerate}

\noindent
\textbf{Definite and Indefinite Integrals}

\begin{enumerate}
\item Compute the following indefinite integrals.
    \begin{enumerate}
    \item $\dsp{x^2 + \frac{1}{4}x^3 + c}$
    \item $\dsp s + e^s + c$
    \item $\dsp -\cos(x)+c$
    \item $\dsp \sin(y)+c$
    \item $\dsp \pi \sin(x) + \pi x +c$
    \item $\dsp -\frac{1}{2}\cos(2t)+c$
    \item $\dsp \frac{10}{3}x^{3/2} - \ln(x) +c$
    \item $\dsp \frac{1}{3}x^3+x+c$
    \item $\dsp \frac{1}{5}x^5 + \frac{2}{3}x^3+x+c$
    \item $\dsp \frac{1}{6}(x^2+1)^3 + c$
    \item $\dsp \sin(x^2)+c$
    \item $\dsp 5\ln(t) + \frac{1}{5}\ln(t) + \frac{1}{\ln(5)} 5^t+c$
    \item $\dsp \frac{1}{5} \sin^5(t) +c$
    \item $\dsp \frac{1}{9} (5x^3-3x^2)^9 +c$
    \item $\dsp \frac{5}{9}y^9 - \frac{5}{2}y^8 + \frac{20}{7}y^7 - \frac{2}{5}y^5 + 2y^4 - \frac{8}{3}y^3 +c$
    \item $\dsp \frac{1}{6}t^6-\frac{3}{4}t^4+c$
    \item $\dsp -\frac{\pi}{2}x^2+ex+c$
    \end{enumerate}
\item Compute the following definite integrals.
    \begin{enumerate}
    \item $8$
    \item $6$
    \item $84$
    \end{enumerate}
\item Evaluate each integral.
    \begin{enumerate}
    \item $\dsp x+{x}^{2} + k$
    \item $\dsp \frac{1}{12}{t}^{4}-\frac{2}{3}{t}^{3/2} + k$
    \item $\dsp 4 ( \tan(x)  - \sin(x) ) +k$
    \item $\dsp \frac{2}{5}\sqrt{z^5}-4\,{z}^{-1}+\frac{3}{16} \frac{3}{\sqrt[3]{4}\sqrt[3]{z^2}} + k$
    \item $\dsp \frac{1}{3}\sqrt {x} \left( 3+x \right) \sqrt {2} + k$
    \item $\dsp -\cos(t) + 2\sqrt(3t) + k$
    \item $\dsp -\frac{1}{3}\cos \left( {x}^{3} \right)  + k$
    \item $\dsp -\frac{1}{6} \left( \cos \left( x \right)  \right) ^{6} + k$
    \item $\dsp -\frac{1}{3} \left( 1-{x}^{2} \right) ^{3/2} + k$
    \item $\dsp \frac{1}{4} \left( 3\,{z}^{2}+6\,z+5 \right) ^{2/3} + k$
    \item $\dsp \frac{1}{2}\sin \left( 2 \right) -\frac{1}{2} + k$
    \item $1-\sqrt{2} + k$
    \item $\dsp {\frac {2145}{4}} + k$
    \item $\sqrt{2}-1 + k$
    \item $\dsp {\frac {20}{3}}\,\sqrt {5}-{\frac {32}{3}} + k$
    \end{enumerate}
\item Evaluate each of the following integrals, utilizing the suggested substitution.
    \begin{enumerate}
    \item $\dsp 2/3\,\sqrt {x+5} \left( -10+x \right) + k$
    \item $\dsp {\frac {3}{28}}\, \left( x-1 \right) ^{4/3} \left( 3+4\,x \right)  + k$
    \item $\dsp -1/3 \sqrt{4-x^2} (x^2+8) + k$
    \item $\dsp 1/3\,\sqrt {{x}^{2}+1} \left( -2+{x}^{2} \right)  + k$
    \item $\int 1/2\,x\sqrt {9-{x}^{2}}+9/2\,\arcsin \left( 1/3\,x \right) + k $
    \item $\dsp 1/4\,\ln  \left( x+2 \right) -1/4\,\ln  \left( x-2 \right) + k$
    \item $\arctan \left( x \right) + k $
    \end{enumerate}

\end{enumerate}

\newpage
\noindent
\textbf{Average Value, Mean Value, Fundamental Theorem and Arclength}

\begin{enumerate}

\item $-5$

\item $\dsp \frac{1}{2}$

\item $\dsp \frac{\sqrt{93}-3}{3} \approx 2.215$

\item $F'(x) =\cos(x)$

\item $F'(x) = x^3 - 4x + e^x \; dt$

\item $\dsp F'(x) = \frac{ e^x}{2\sqrt{x}}$

\item $\dsp \frac{ 74 \sqrt{37}-2}{27} \approx 16.6$

\end{enumerate}

\noindent
\textbf{Area and Volume}

\begin{enumerate}

\item $\dsp \frac{7}{3}$

\item $10$

\item $\dsp \frac{32}{3}$

\item $9$

\item $8$

\item $\dsp \frac{32}{3}$

\item $no$

\item $\dsp \frac{9}{2}$

\item $\dsp \frac{71}{6}$

\item $8 \pi$

\item $\dsp \frac{8 \pi}{3}$

\item $32 \sqrt{2} \pi$

\item $\dsp \frac{32\sqrt{3}}{3}$

\item  $\dsp \frac{4\pi r^3}{3}$

\item no answer -- attempt to estimate whether your answer makes sense
\item no answer -- attempt to estimate whether your answer makes sense
\item no answer -- attempt to estimate whether your answer makes sense
\item no answer -- attempt to estimate whether your answer makes sense

\end{enumerate}

\noindent
\textbf{Work}

\begin{enumerate}
\item  15/4 ft lbs
\item  650,000 ft lbs
\item  2450 J
\end{enumerate}

\noindent
\textbf{Center of Mass}

\begin{enumerate}

\item $95$ and approximately $6.3$

\item $\dsp ( \frac{16}{105}, \frac{4}{15} )$

\end{enumerate}


\chapter{Techniques of Integration}

``That student is taught the best who is told the least.'' - R. L. Moore\\ \\

Change of variable (substitution), partial fractions, integration by parts, and trigonometric substitutions are four techniques of integration that can be applied to evaluate indefinite integrals.  More than one technique will often work on the same problem.

\section{Change of Variable Method}

The integrand can often be rewritten so that it is easier to see how one of our derivative theorems can be applied to find the antiderivative of the integrand.   This method is called either a \emph{change of variable} or \emph{substitution} because you replace some part of the integrand with a new function.
\begin{annotation}
\endnote{Before calculators could compute 90\% of integrals, techniques of integration were important as ways to quickly solve problems for engineers.  Now, I tend to focus more on understanding the notation and, when possible, the derivation of the techniques.  For example, I like students to see that u-substitution often ``hides'' the chain rule.  Therefore, I work out the following example in class and then do the same problem using u-substitution.  Some students will have seen u-substitution and I'm certainly happy to allow them to use a method they know.  Others will prefer the change of variable where the chain rule is fully displayed.  Showing the problems as a change of variable allows the deep-thinking student to see what is hiding behind the notation of u-substitution.  When I first taught this, I was quite surprised at the number of bright students who shunned u-substitution for the more difficult demonstration of the chain rule.  More detail on my perspective on how to do this carefully and correctly is outlined in the paper, ``Calculus: The importance of precise notation,� co-authored with W. S. Mahavier. PRIMUS, Volume 18, Issue 4 July 2008 , pages 349 - 360. An even better article that I discovered later is ``Teaching Integration by Substitution,'' David Gale, American Mathematical Monthly, Vol. 101, No. 6 (Jun. - Jul., 1994, pp. 520-526).}
\end{annotation}

\begin{expl}
Consider $\dsp \int \frac{x}{\sqrt{x+5}} \; dx.$  If we let $u(x) = \sqrt{x+5}$, then $\dsp u'(x) = \frac{1}{2\sqrt{x+5}}$ so $$\int \frac{x}{\sqrt{x+5}} \; dx = 2\int (u^2(x)-5)u'(x) \; dx =  2\int u^2(x)u'(x) - 5u'(x) \; dx = \frac{2}{3} u^3(x) - 10u(x) = \dots.$$
\end{expl}

\begin{prb}
Evaluate the indefinite integral, $\dsp \int x^2(2x^3 + 10)^{15} \; dx,$ using the following steps.
\begin{enumerate}
\item Let $u(x)=2x^3+10$ and compute $u'.$
\item Show that $\dsp \int x^2(2x^3 + 10)^{15} \; dx =  \frac{1}{6} \int ( u(x) )^{15} u'(x) \; dx$.
\item Evaluate this indefinite integral and  then replace $u(x)$ with $2x^3+10$ in your answer to eliminate the function $u$ from your answer.
\item  Verify your answer by taking the derivative.
\end{enumerate}
\end{prb}

\begin{prb}
\label{rational} Evaluate each of the following integrals.
\begin{enumerate}
\item $\dsp \int (2x+7) \root 3 \of {x^2 + 7x +3} \; dx$
\item Three parts to this one!
\begin{enumerate}
\item $\dsp \int \sqrt{2 - x} \; dx$
\item $\dsp \int x \sqrt{2 -x} \; dx$
\item $\dsp \int x^2 \sqrt{2 - x} \; dx$
\end{enumerate}

\item $\dsp \int \frac{{(\ln x)}^4}{x} \; dx$

\item $\dsp \int {5x^2}{4^{x^3}} \; dx$

\item $\dsp \int_0 ^\frac{\pi}{8} \sin^5(2x) \cos(2x) \; dx$

\item Three parts to this one!
\begin{enumerate}
\item $\dsp \int\frac{x}{(x^2+2)^2} \; dx$
\item $\dsp \int\frac{x}{\sqrt[4]{x^2+2}} \; dx$
\item $\dsp \int \frac{x}{x^2+2} \; dx$
\end{enumerate}
\end{enumerate}
\end{prb}


\section{Partial Fractions}

This section focuses on integrating rational functions. The method of \emph{partial fractions} may be used on rational functions to break one rational function into a sum of two or more rational functions, each of which has a lower degree denominator and is easier to anti-differentiate.  When a rational function has a numerator with degree greater than the degree of the denominator, one can long divide to rewrite the integrand as the sum of a polynomial and another rational function with numerator of lower degree than the degree of the denominator.  Sometimes the rational function displays features of our derivative theorems and thus is already in a form that is integrable.

\begin{prb}
Since the degree of the numerator is larger than the degree of the denominator, use long division to evaluate $\dsp {\int \frac{x^2-3}{x+1} \ dx}$.
\end{prb}

\begin{prb}
You will integrate this rational function by breaking it into two rational functions and integrating each of those functions separately (partial fractions).
\begin{enumerate}
\item Find numbers $A$ and $B$ so that $\dsp{ \frac{1}{x^2-1} = \frac{A}{x-1}+\frac{B}{x+1} }.$
\item Evaluate $\dsp {\int \frac{1}{x^2-1} \ dx = \int \frac{A}{x-1} \ dx + \int \frac{B}{x+1} \ dx = \dots}$.
\end{enumerate}
\end{prb}

\begin{prb}
Evaluate $\dsp \int \frac{4x}{x^2 - 2x-3}\; dx$ via partial fractions.
\end{prb}


\begin{prb}
Evaluate $\dsp \int \frac{3x^2-2x+1}{x^2 -2x-3}\; dx.$  This is called an improper fraction because the degree of the numerator is greater than or equal to the degree of the denominator.  In such cases, long divide first.
\end{prb}

\begin{prb}
Evaluate  $\dsp \int \frac{x^2+x-2}{x(x+1)^2}\; dx$ via partial fractions by first finding constants, $A, B$, and $C$ so that $\dsp \frac{x^2+x-2}{x(x+1)^2} = \frac{A}{x} + \frac{B}{x+1}+\frac{C}{(x+1)^2}$.
\end{prb}

\begin{prb}
Evaluate $\dsp \int \frac{5x^2+3}{x^3 + x}\; dx$ via partial fractions by first finding constants $A$, $B$, and $C$ so that $\dsp \frac{5x^2+3}{x^3 +x} = \frac{A}{x}+\frac{Bx+C}{x^2+1}$.
\end{prb}

\begin{prb}
Evaluate these three indefinite integrals that look similar, but are not!
\begin{enumerate}
\item  $\dsp \int \frac{5}{x^2-1} \; dx$
\item  $\dsp \int \frac{5x}{x^2-1} \; dx$
\item  $\dsp \int \frac{5}{x^2+1} \; dx$
\end{enumerate}
\end{prb}

\section{Integration by Parts}

Integration by parts is an application of the product rule -- essentially the product rule in reverse!  It is typically used when the integrand is the product of two functions, one which you can easily anti-differentiate and one which becomes simpler (or at least no more complicated) when you differentiate.

\begin{prb}
Find two functions $f$ and $g$ so that $\dsp \int f(x) \cdot g(x) \; dx \neq \int f(x) \; dx \cdot \int g(x) \; dx$.
\end{prb}

\begin{expl}
Evaluate the integral $\dsp \int x \ln (x) \; dx$.
\end{expl}
We begin by guessing a function which has $x \ln(x)$ as a part of its derivative when we use the product rule.  From the product rule for derivatives we have,
$$ (x^{2} \ln x)'  = 2x \ln x + x^{2} (\frac{1}{x}).$$
Integrating both sides yields,
$$\dsp \int (x^{2} \ln x)'  \; dx = \int 2 x \ln x \; dx  + \int x  \; dx.$$
Now we have,
$$\dsp  x^2 \ln x  + k_1= 2 \int x \ln x \; dx + \frac{1}{2} x^2 + k_2$$
Where $k_1$ and $k_2$ are constants. Solving for the integral we wanted yields,
$$\dsp \int x \ln x  \; dx = \frac{1}{2} x^2 \ln x - \frac{1}{4} x^2 + \frac{1}{2}(k_1-k_2).$$

\begin{prb} Use the idea from the example to compute the following anti-derivatives.
\begin{enumerate}
\item $\dsp \int x^{45} \ln x \; dx$
\item $\dsp \int x e^x \; dx$
\item $\dsp \int x \cos(x) \; dx$
\end{enumerate}
\end{prb}

\begin{prb}
\label{ibt}
Let $f$ and $g$ be differentiable functions. Starting with the product rule for derivatives, show that
\begin{annotation}
\endnote{Given that the indefinite integral was defined to be some anti-derivative of the integrand, this is a lie -- they are not necessarily equal, but rather differ by at most a constant.}
\end{annotation}
$$\int f(x)g'(x) \; dx = f(x) \cdot g(x) -  \int g(x) f'(x) \; dx.$$
\end{prb}

\begin{prb}
Evaluate each indefinite integral.
\begin{enumerate}
\item $\dsp \int x^{2} e^x \; dx$
\item $\dsp \int x^{2} \sin(x) \; dx$
\item $\dsp \int e^x \cos x \; dx$
\item $\dsp \int (\ln x)^2 \; dx$
\item $\dsp \int \sec^3 x \; dx$
\end{enumerate}
\end{prb}

\begin{prb} Compute the derivatives of each.
\begin{enumerate}
\item $s(x) = \invsin(x)$ by differentiating the equation $\sin(s(x))=x$ and solving for $s'(x)$
\item $c(x) = \invsec(x)$
\item $t(x) = \invtan(x)$
\end{enumerate}
\end{prb}

Now we address the integrals of the inverse trigonometric functions. Use integration by parts since you know the derivatives of each function.

\begin{prb}
Compute the following indefinite integrals via integration by parts.
\begin{enumerate}
\item $\dsp \int \invsin(x) \; dx$ using $\invsin(x) = \invsin(x) \cdot 1.$
\item $\dsp \int \invtan(x) \; dx$
\item $\dsp \int \invsec(x) \; dx$
% TED last one hard!   either trig sub with u = sec(x) or multiply by 1:  x + \sqrt{x^2-1}  / (x + \sqrt{x^2-1})
\end{enumerate}
\end{prb}

\begin{thm}
\textbf{Integration by Parts.} If each of $f$ and $g$ are functions that are differentiable on the interval $[a,b]$, then
\begin{annotation}
\endnote{Again, given that the indefinite integral was defined to be some anti-derivative of the integrand, this is a lie -- they are not necessarily equal, but rather differ by at most a constant.}
\end{annotation}
$$\int f(x)g'(x) \; dx = f(x) \cdot g(x) -  \int g(x) f'(x) \; dx$$ and
$$\int_a^b f(x)g'(x) \; dx = f(x) \cdot g(x) |_a^b -  \int_a^b g(x) f'(x) \; dx$$
\end{thm}

Writing $f(x)$ as $u$ and $g(x)$ as $v$ and using Leibnitz notation, this formula is often written as:
$$\int u \ dv= uv - \int v \ du.$$

\section{Integrals of Trigonometric Functions}

When integrating trigonometric functions you will often use a combination of substitution, integration by parts, and trigonometric identities. Appendix \ref{apptrig} has the trigonometric identities that will be necessary for this section: the Pythagorean, double angle, half angle, and product identities.

\begin{prb}
Evaluate $\dsp \int \cos^3(x) \; dx$ by using the fact that $\cos^{3} = \cos^2 \cdot \cos$ and the Pythagorean identity for $\cos^2.$
\end{prb}

\begin{prb}
Evaluate $\dsp \int \cos^5 (x) \; dx$ by using the fact that $\cos^5= \cos^4 \cdot
\cos$
\end{prb}

\begin{prb}
Evaluate $\dsp \int \cos^{2n+1} (x) \; dx$ and $\dsp \int \sin^{2n+1} (x) \; dx$ where $n$ is a positive integer.
\end{prb}

\begin{prb}
Evaluate $\dsp \int \sin^4 (x) \; dx$ using the half-angle identity for $\sin.$
\end{prb}

\begin{prb}
Evaluate each of the following integrals.
\begin{enumerate}
\item $\dsp \int \sin^3 (x) \cos^2(x) \; dx$
\item $\dsp \int \sin^3 (x) \cos^5 (x) \; dx$
\item $\dsp \int \sin^4(x) \cos^2 (x) \; dx$
\item $\dsp \int \sin(3x) \cos(6x) \; dx$
\end{enumerate}
\end{prb}

\begin{prb}
Evaluate $\dsp{ \int \csc(x) \; dx }$ by first multiplying by $\dsp{ \frac{\csc(x) - \cot(x)} {\csc(x) - \cot(x)} }.$
\end{prb}

\begin{prb}
Evaluate $\dsp \int \tan^3(x) \sec^4(x) \; dx$ in the two different ways indicated.
\begin{enumerate}
\item Use $\tan^3 \cdot \sec^4 = \tan^2 \cdot \sec^3 \cdot \tan \cdot \sec$ and the Pythagorean identity for $\tan^2.$
\item Use $\tan^3 \cdot \sec^4 = \tan^3 \cdot \sec^2 \cdot \sec^2$ and the Pythagorean identity for $\sec^2.$
\end{enumerate}
\end{prb}

\begin{prb}
Evaluate each of the following integrals.
\begin{enumerate}
\item $\dsp \int \cot^5(x) \; dx$
\item $\dsp \int_{\frac{3\pi}{4}}^{\frac{5\pi}{4}} \sec^4(x) \; dx$
\item $\dsp \int \cot^3(x) \csc^5(x) \; dx$
\item $\dsp \int \tan^2(x) \sec^4(x) \; dx$
\item $\dsp \int \tan^4(x) \sec^3(x) \; dx$
\end{enumerate}
\end{prb}

\begin{prb}
\textbf{Not for presentation.}  Integrate the first, second, third, and fourth power of each of the six trigonometric functions.  It is a well-documented fact that one of these (24) problems has appeared on at least one test in every Calculus II course since the days of Newton and Leibnitz.
\end{prb}

\section{Integration by Trigonometric Substitution}

We have already used substitution (change of variables) to transform difficult integration problems into easier ones.  Trigonometric substitutions are an extension of this same theme.

\begin{prb}
Derive the formula ($\dsp A=\pi r^2$) for the area of a circle of radius $r$ using integration as follows.
\begin{enumerate}
\item Compute the definite integral $\dsp \int_{-3}^3 \sqrt{9-x^2} \; dx$ by making the substitution, $x(t) = 3\sin(t).$
\item Show that the area inside of a circle $x^2+y^2=r^2$ is $\pi r^2$ by setting up and evaluating an appropriate definite integral using the substitution $x(t) = r\sin(t)$.
\end{enumerate}
\end{prb}

\begin{prb}
Evaluate each of the following integrals.
\begin{enumerate}
\item $\dsp \int \frac{x^2}{\sqrt{x^2+4}}\; dx \;$ by using the substitution $x = 2 \tan (\theta)$
\item $\dsp \int_0^4 5x \sqrt{16-x^2} \; dx$
\item $\dsp \int_2^3 \frac{1}{{4x^3 \sqrt{x^2 - 1}}} \; dx \;$ by using the substitution
$x = \sec (\theta)$
\item $\dsp \int \frac{1}{\sqrt{x^2-2x + 5}} \; dx \;$ by completing square and letting $u=x-1$
\item $\dsp \int \frac{2x+5}{x^2+4x+1} \; dx \; $
%by rewriting the integrand as $\dsp {{2x+4} \over {x^2 + 4x+1}}+ {{1} \over {x^2+4x+1}}$
\end{enumerate}
\end{prb}

\begin{dfn}
If $a$ and $b$ are non-zero numbers, then the set of points in the plane satisfying $$\frac{x^2}{a^2} + \frac{y^2}{b^2} = 1$$ is an ellipse centered at the origin.
\end{dfn}

\begin{prb}
Consider the ellipse, $\dsp \frac{x^2}{25} + \frac{y^2}{16} = 1$. Set up and evaluate a definite integral that represents the area (or half the area) inside this ellipse.
\end{prb}

\begin{prb}
Solve each of the following problems.  Assume that $a$ is a real number.
\begin{enumerate}
\item Evaluate $\dsp \int \frac{1}{\sqrt{a^2 - x^2}} \; dx$
\item Evaluate $\dsp \int \frac{1}{x^2 + a^2} \; dx$ \item
Evaluate $\dsp \int \frac{1}{x\sqrt{x^2 - a^2}} \; dx$
\end{enumerate}
\end{prb}

\begin{prb}
Verify the formula $\dsp A=\pi ab$ for the area of the ellipse $\dsp \frac{x^2}{a^2}+\frac{y^2}{b^2} = 1$ where $a$ and $b$ are posiive numbers.
\end{prb}

\section{The Hyperbolic Functions}

Similar to the trigonometric functions in Euclidean trigonometry, the following functions are important in {\bf{hyperbolic trigonometry}}.  These occur in engineering problems because they appear as solutions to differential equations modeling electrical circuits and they can be used to represent the curve that results when hanging a wire between two poles (think telephone lines).
\begin{annotation}
\endnote{I often assign this section as homework.  It is not clear to me that it is worth time to do it in class.}
\end{annotation}

\begin{dfn}
The \textbf{hyperbolic sine and hyperbolic cosine} functions are denoted by \textbf{sinh} and \textbf{cosh} and defined by:
$$\sinh(x) = \frac{e^x - e^{-x}}{2} \mbox{ and } \cosh(x) = \frac{e^x +e^{-x}}{2}.$$
The \textbf{hyperbolic tangent, hyperbolic cotangent, hyperbolic secant, and hyperbolic cosecant} functions are denoted by \textbf{tanh}, \textbf{coth}, \textbf{sech}, \textbf{csch} and defined by:
$$\dsp \tanh(x) = \frac{\sinh(x)}{\cosh(x)}, \;\; \coth(x) = \frac{\cosh(x)}{\sinh(x)}, \;\; \dsp \sech(x) = \frac{1}{\cosh(x)}, \;\; \mbox{ and } \;\; \dsp \csch(x) = \frac{1}{\sinh(x)}.$$
\end{dfn}

\begin{prb}
Assume that $x$ is a number and show that $\cosh^2(x) - \sinh^2(x) = 1$.
\end{prb}

\begin{prb}
Assume that $x$ is a number and show that each of the following is true.
\begin{enumerate}
\item $\dsp \frac{d}{dx} \sinh(x) = \cosh(x)$
\item $\dsp \frac{d}{dx} \tanh(x) = \sech^2(x)$
\end{enumerate}
\end{prb}

While many inverse functions (such as $invtan$) cannot be written out explicitly as a function of $x$ only, all of the inverse hyperbolic trigonometric functions can be.

\begin{prb}
Rewrite $\invcosh$ as a function of $x$ and without using any hyperbolic functions by solving $\dsp y = \frac{e^x+e^{-x}}{2}$
for $x.$
\end{prb}

\begin{prb}
Show that $\dsp \frac{d}{dx} \invsinh (x)=\frac{1}{\sqrt{x^2 +1}}$ by letting $y = \invsinh(x),$ applying implicit differentiation to $\sinh(y) = x,$ and using the identity $\cosh^2(x) - \sinh^2 (x) = 1$.
\end{prb}

\begin{prb}
Show that $\dsp \frac{d}{dx} \invsech (x)= -\frac{1}{x \sqrt{1-x^2}}$ \; for \; $0 < x \leq 1.$
\end{prb}

\section{Improper Integrals}

In our definition of the definite integral $\dsp \int_a ^b f(x) \; dx$, it was assumed that $f$ was defined on all of $[a, b]$ where $a$ and $b$ were numbers.  We wish to extend our definition to consider the possibilities where $a$ and $b$ are $\pm \infty$ and where $f$ might not be defined at every point in $[a,b].$  When any of these possibilities occur, the integral is called an \textbf{improper integral}.

\begin{prb}
Let $\dsp f(x) = \frac{1}{x^2}.$
\begin{enumerate}
\item Sketch a graph of $f.$
\item Compute $\dsp \int_1^{10} f(x) \; dx, \int_1^{50} f(x) \; dx,$ and $\dsp \int_1^{100} f(x) \; dx.$
\item Compute $\dsp \lim_{N \to \infty} \int_1^{N} f(x) \ dx.$
\end{enumerate}
\end{prb}

\begin{dfn}
If $a$ is a number and $f$ is continuous for all $x \ge a$, then we define the \textbf{improper integral} $\dsp \int_a^{\infty} f(x) \; dx$ to be $lim_{N \to \infty} \int_a^{N} f(x) \ dx.$ If this limit exists, we say the integral \textbf{converges} and if it does not exists, we say the integral \textbf{diverges}.
\end{dfn}

\begin{prb}
Write down definitions for $\dsp \int_{-\infty}^a f(x) \; dx $ and $\dsp \int_{-\infty}^{\infty} f(x) \; dx.$
\end{prb}

\begin{prb}
Let $\dsp f(x) = \frac{4}{(x-1)^2}.$
\begin{enumerate}
\item Sketch the graph of $f.$
\item Evaluate $\dsp \int_2^{100} f(x) \; dx$
\item Evaluate $\dsp \int_2^{100,000} f(x) \; dx$
\item Evaluate $\dsp \int_2^{\infty} f(x) \; dx$
\item Evaluate $\dsp \int_a^{\infty} f(x) \; dx$ where $a > 1$.
\end{enumerate}
\end{prb}

\begin{prb}
Sketch $\dsp f(x) = \frac{1}{x-1}$ and evaluate $\dsp \int_2^{\infty} f(x) \; dx$.
\end{prb}

\begin{prb}
Sketch $\dsp f(x) = \frac{1}{4+x^2}$ and evaluate $\dsp \int_{-\infty}^{\infty} f(x) \; dx.$
\end{prb}

\begin{prb}
Evaluate $\dsp \int_{-\infty}^{-2} \frac{5}{x \sqrt{x^2-1}} \; dx$.
\end{prb}

\begin{dfn}
If $f$ is continuous on the interval $[a, b)$, $\displaystyle{\lim_{x \to b^-} |f(x)| =  \infty}$, then we define
$\dsp \int_a^{b} f(x) \ dx = {\lim_{c \to b^-} \int_a ^c f(x) \; dx}$.
\end{dfn}

\begin{prb}
Definitions.
\begin{enumerate}
\item Suppose $f$ is continuous on the interval $(a, b]$ and $\displaystyle{\lim_{x \to a^+} |f(x)| =  \infty}$.  Write a definition for $\dsp \int_a^{b} f(x) dx$.
\item Suppose $f$ is continuous on the interval $[a, b]$ except at $c$ where $a < c < b.$  Suppose $\dsp \lim_{x \to c-} = \dsp \lim_{x \to c+} f(x) =  \pm \infty$. Write a definition for $\dsp \int_a^{b} f(x) dx$.
\end{enumerate}
\end{prb}

\begin{prb}
Evaluate each of the following improper integrals.
\begin{enumerate}
\item $\dsp \int_0^2 \frac{1}{(x-2)^2} \ dx$ \item $\dsp \int_3^5 \frac{x}{4-x} \; dx$
\item $\dsp \int_0^\frac{\pi}{2} \cot x \; dx$ \item $\dsp \int_0^e \frac{\ln x}{x} \; dx$
\item $\dsp \int_1^{\infty} \frac{\; dx}{-2x \sqrt{x^2-1}}$
\end{enumerate}
\end{prb}

\section{Practice} \label{chap5probs}

We will not present the problems from this section, although you are welcome to ask about them in class.

\vskip .1in
\noindent
Here's a strategy for tackling indefinite integrals.

\begin{itemize}
\item  Is it easy?  I.e. can we just guess a function $f$ so that $f'$ is the integrand?
\item  Is there a function $f$ in the integrand along with the derivative $f'$ so that this is a reverse
chain rule/subtitution problem, for example, $\dsp \int x^2 \sin(4-x^3) \; dx$?
\item  Is there a substitution such as $u = x+5$ for $\dsp \int \frac{x}{\sqrt{x+5}} \; dx$ or $x = \tan(t)$ for $\dsp \frac{x^2}{\sqrt{x^2+4}} \; dx?$
\item  Can we rewrite the integrand using algebra, a trigonometric identity, long division, or partial fractions to make an easier problem?
\item Is the integrand of the form $u \cdot v'$ for some functions $u$ and $v$ so that I can apply integration by parts?
\item Is there a power or a product of powers of trigonometric functions that I know to apply certain identities to? \item Is there some clever trick that I just {\it know} to apply to this integral such as when I see this particular problem? For example:
\begin{itemize}
\item $\dsp{ \int \cos^2(x) \; \; dx}$ Use the half-angle formula.
\item $\dsp{ \int \sec(x) \; \; dx}$  Multiply by $\dsp{ \frac{\sec(x) +\tan(x)} {\sec(x) + \tan(x)})}$.
\end{itemize}
\end{itemize}


\begin{enumerate}

\item Integrate by substitution.
\begin{enumerate}

\item  $\dsp \int x^2 \sqrt{x^3-3} \ dx $
\item  $\dsp \int \frac{(\sqrt{x} + \pi)^4}{\sqrt{x}} \ dx $
\item  $\dsp \int x\sqrt{x+5} \ dx $
\item  $\dsp \int  \sin(42x^5-6) 37x^4 \ dx $
\item Evaluate $\int x \sin(x^2) \cos(x^2) \; dx$ using $u(x) = \sin(x^2).$ Now evaluate $\int x \sin(x^2) \cos(x^2) \; dx$ using $u(x) = \cos(x^2).$ Are your answers the same? Why or why not?

\end{enumerate}

\item Integrate by parts.
\begin{enumerate}
\item  $\dsp \int xe^{-x} \ dx $
\item  $\dsp \int x\sin(2x)   \ dx $
\item  $\dsp \int x^2 \sinh(x)   \ dx $ where $\dsp \sinh(x) = \frac{e^x-e^{-x}}{2}$ and $\dsp \cosh(x) = \frac{e^x+e^{-x}}{2}$
\end{enumerate}

\item Integrate by partial fractions.
\begin{enumerate}
\item  $\dsp \int \frac{2}{x^2-1}   \ dx $
\item  $\dsp \int \frac{4x^2+13x-9}{x^3+2x^2-3x}  \ dx $
\item  $\dsp \int \frac{6x-11}{(x-1)^2}   \ dx $
\item  $\dsp \int \frac{-19x^2+50x-25}{x^2(3x-5)}  \ dx $
\end{enumerate}

\item Integrate by trigonometric identities.
\begin{enumerate}
\item  $\dsp \int \sin^3(x)    \ dx $
\item  $\dsp \int \cos^4(x)  \ dx $
\item  $\dsp \int \cos^3(x)\sin^6(x)   \ dx $
\item  $\dsp \int  \tan^3(x)\sec^5(x)  \ dx $
\item  $\dsp \int \sin^2(x)\cos^2(x)  \ dx $
\end{enumerate}

\item Integrate by trigonometric substitution.
\begin{enumerate}
\item  $\dsp \int \frac{\sqrt{x^2-9}}{x}    \ dx $
\item  $\dsp \int \frac{1}{4+x^2}     \ dx $  Either $x=2\tan(t)$ or  $x = 2\sinh(t)$) works.
\item $\dsp \int \frac{x^2}{4-x^2}   \ dx $
\item  $\dsp \int \frac{1}{x^3\sqrt{x^2-25}}   \ dx $
\end{enumerate}

\item Integrate these Improper Integrals
\begin{enumerate}
\item $\dsp \int_0^{\infty} {{3x} \over {4+x^2}}\; dx$
\item $\dsp \int_0^{\infty} {{7x} \over {x^4 + 9}}\; dx$
\item $\dsp \int_{-1}^{\infty} {{1} \over {x^2+2x+5}}\; dx$
\item $\dsp \int_0^1 {{-2} \over {x^3-5x^2}}\; dx$
\item $\dsp \int_{- \infty}^{\infty} {{2^x} \over {3+2^x}}\; dx$
\item $\dsp \int_0^e {{\ln(x)} \over {3x}}\; dx$
\item Use the arc length formula to verify that the circumference the circle $x^2+y^2=R^2$
is $2\pi R$. Where did you use the concept of improper integrals in this problem?
\end{enumerate}

\item Evaluate the indefinite integrals using any method.

\begin{enumerate}
\item $\dsp \int \sqrt{x-7} \; dx \;\;\;$  $\dsp \int x\sqrt{x-7} \;
dx\; \; \;$  $\dsp \int x^2\sqrt{x-7} \; dx$
\item $\dsp \int {{4t^2} \over {\sqrt{8-t^3}} } \; dt$
\item $\dsp \int{{3 \cot(t) + 2 \csc(t)} \over {\sin(t)} } \; dt$
\item $\dsp \int {{\sec^2 \sqrt{x+5}} \over {\sqrt{x+5}}} \; dx$
\item $\dsp \int_4^5 {{1} \over {\sqrt{x-4}}}\; dx$
\item $\dsp \int \cos (ax+b) \; dx$ Assume $a$ and $b$ are constants.
\item $\dsp \int {3^{4x}} \; dx$
\item $\dsp \int {b^{kx}} \; dx$ Assume $k$ is a constant.
\item $\dsp \int (\sec(e^{2x}) + \csc(e^{2x}))^2 e^{2x} \; dx$
\item $\dsp \int{{2-e^{3x}} \over {e^x}}\; dx$
\item $\dsp \int{{e^{2x}} \over {e^x - 5}}\; dx$
\item $\dsp \int{{1} \over {3 \sqrt{x}(4+ \sqrt{x})^4}}\; dx$
\item $\dsp \int_0^{\infty} xe^{-2x}\; dx$
\item $\dsp \int {{\ln^3(4x)} \over {5x}}\; dx$
\item $\dsp \int x \sqrt{2x+1} \; dx$
\end{enumerate}

%TED Add some definite integrals in the mix and some practice with hyperbolics

\item Evaluate the indefinite integrals using any method.
\begin{enumerate}
\item $\dsp \int {{2x-1} \over {x^3-5x^2-6x}}\; dx$
\item $\dsp \int \sin^3(x) \; dx$
\item $\dsp \int {{x^2-x-6} \over{x^3-1}}\;dx$
\item $\dsp \int (16-x^2)^{3/2}\; dx$
\item $\dsp \int {{3x+1} \over {x^2-4}}\; dx$
\item $\dsp \int \csc^6(x) \; dx$
\item $\dsp \int {{2x^3 -4x^2+x+ {{21} \over 8}} \over {2x^2-x-2}}\; dx$
\item $\dsp \int {{7x^2-3x-1} \over {x^3-x^2-x-2}}\; dx$
\item $\dsp \int \cos^2(x) \; dx$
\item $\dsp \int {4 \over {x \sqrt{x^4 + 9}}} \; dx$
\item $\dsp \int \sin^2(x) \cos^2(x) \; dx$
\item $\dsp \int \sin^3(x) \cos^4(x) \; dx$
\item $\dsp \int {{2x-1} \over {x^3 - 4x^2}}\; dx$
\item $\dsp \int \cot^5(x) \csc^2(x) \; dx$
\item $\dsp \int \tan^3(x)\; dx$
\item $\dsp \int \tan^4(x) \sec^4(x) \; dx$
\item $\dsp \int {2 \over {x^2 \sqrt{x^2 - 4}}}\;
dx$
\end{enumerate}

\item Evaluate the indefinite integrals using any method.

\begin{enumerate}
\item $\dsp \int {3 \over {\sqrt{2x-x^2}}} \; dx$
\item $\dsp \int {{2x^3-4x^2+3x+1} \over {2x^2-x-2}} \; dx$
\item $\dsp \int {{7x^2-3x-1} \over {x^3-x^2-x-2}} \; dx$
\item $\dsp \int x \sin(3x) \; dx$
\item $\dsp \int x^2 4^x \; dx$
\item $\dsp \int x^5 \sqrt{x^2 + 4} \; dx$ Does $u=\sqrt{x^2 + 4}$ or $u = x^2+4$ work better?
\item $\dsp \int {{\sqrt x} \over {1+ x^{1/3}}}\; dx$ (Let $x = t^6$)
\item $\dsp \int \log(x) \; dx$
\item $\dsp \int \invcos(x) \; dx$
\item $\dsp \int \sqrt{x^2-x-6} \; dx$ \item $\dsp \int x^3 \ln(x) \; dx$
\item $\dsp \int {6 \over {\sqrt{6x+x^2}}} \; dx$
\item $\dsp \int e^x \cos(2x) \; dx$
\item $\dsp \int \cos(x) \ln(\sin(x)) \; dx$
\end{enumerate}

\item Assume that $x$ is a number and show that each of the following is true.
\begin{enumerate}
\item $1- \tanh^2(x) = \sech^2(x)$
\item $1- \coth^2(x) = -\csch^2(x)$
\end{enumerate}

\item Assume that $x$ is a number and show that each of the following is true.
\begin{enumerate}
\item $\dsp \frac{d}{dx} \cosh(x) = \sinh(x)$
\item $\dsp \frac{d}{dx} \cotanh(x) = -\csch ^2(x)$
\item $\dsp \frac{d}{dx} \sech(x) = -\sech(x) \cdot \tanh(x)$
\item $\dsp \frac{d}{dx} \csch(x) = -\csch(x) \cdot \coth (x)$
\end{enumerate}

\item Rewrite $\invsinh$ as a function of $x$ and without using any hyperbolic functions by solving $\dsp y = \frac{e^x-e^{-x}}{2}$ for $x.$

\item Verify each of the following.  Why are there restrictions on the domains?
\begin{enumerate}
\item $\dsp \frac{d}{dx} \invcosh (x) = \frac{1}{\sqrt{x^2 -1}}$ \; for \; $x > 1$
\item $\dsp \frac{d}{dx}  \invtanh (x) = \frac{1}{1-x^2}$ \; for \; $|x|<1$
\item $\dsp \frac{d}{dx} \invcoth (x) = \frac{1}{1-x^2}$ \; for \; $|x| > 1$
\item $\dsp \frac{d}{dx} \invcsch (x) = -\frac{1}{|x| \sqrt{1+x^2}}$ \; for \; $x \ne 0$
\end{enumerate}



\end{enumerate}

\vskip .5in
\noindent
\textbf{Chapter 5 Solutions}\\ \\

\begin{enumerate}

\item Integrate by substitution.
\begin{enumerate}

\item  $\dsp 2/9\, \left( {x}^{3}-3 \right) ^{3/2}$
\item  $\dsp 2/5(\sqrt {x} +\pi)^5$
\item  $\dsp 2/15\, \left( x+5 \right) ^{3/2} \left( -10+3\,x \right) $
\item  $\dsp -{\frac {37}{210}}\,\cos \left( 42\,{x}^{5}-6 \right) $
\item  $\dsp 1/4\, \left( \sin \left( {x}^{2} \right)  \right) ^{2} $, $-\dsp 1/4\, \left( \cos \left( {x}^{2} \right)  \right) ^{2} $

\end{enumerate}

\item Integrate by parts.
\begin{enumerate}
\item  $\dsp -( 1+x  ) {e}^{-x} $
\item  $\dsp 1/4\,\sin \left( 2\,x \right) -1/2\,x\cos \left( 2\,x \right)$
\item  $\dsp {x}^{2}\cosh \left( x \right) -2\,x\sinh \left( x \right) +2\,\cosh  \left( x \right)$
\end{enumerate}

\item Integrate by partial fractions.
\begin{enumerate}
\item  $\dsp -\ln |x-1| + \ln |x+1| $
\item  $\dsp 2\,\ln  \left( x-1 \right) -\ln  \left( x+3 \right) +3\,\ln  \left( x \right) $
\item  $\dsp 5\, \left( x-1 \right) ^{-1}+6\,\ln  \left( x-1 \right)$
\item  $\dsp \frac{2}{3} \ln  \left( 3\,x-5 \right) -5\,{x}^{-1}-7\,\ln  \left( x \right) $
\end{enumerate}

\item Integrate by trigonometric identities. The answers are computer generated, so may not be in simplest form.

\begin{enumerate}
\item  $\dsp -\cos(x) + \frac{1}{3}\cos^3(x)$
\item  $\dsp \frac{1}{4} \cos^3(x) \sin(x) + \frac{3}{8}\cos(x) \sin(x) + \frac{3}{8}x$
\item  $\dsp  -\frac{1}{9} \sin^9(x) + \frac{1}{7} \sin^7(x)$
\item  $\dsp  \frac{1}{7} \sec^7(x) - \frac{1}{5} \sec^5(x)$
\item  $\dsp -\frac{1}{4} \cos^3(x)  \sin(x) +\frac{1}{8} \cos(x) \sin(x) + \frac{1}{8}x $
\end{enumerate}

\item Integrate by trigonometric substitution.
\begin{enumerate}
\item  $\dsp \sqrt{x^2-9}-3 \arctan \big( \frac{\sqrt{x^2-9}}{3} \big)$
\item  $\dsp \frac{1}{2}\arctan \left( x/2 \right)$
\item $\dsp -x-\ln  \left( x-2 \right) +\ln  \left( x+2 \right) $
\item  $\dsp {\frac {1}{50}}\,{\frac {\sqrt {{x}^{2}-25}}{{x}^{2}}}-{\frac {1}{250} }\,\arctan \left( 5\,{\frac {1}{\sqrt{{x}^{2}-25}}} \right) $
\end{enumerate}

\item Integrate these Improper Integrals
\begin{enumerate}
\item diverges
\item $\dsp \frac{7\pi}{12}$
\item $\dsp \frac{\pi}{4}$
\item diverges
\item diverges
\item diverges
\item the integrand is undefined at $\pm R$
\end{enumerate}

\item Evaluate the indefinite integrals using any method.

\begin{enumerate}
\item $\dsp 2/3\, \left( x-7 \right) ^{3/2}$, $2/15\, \left( x-7 \right) ^{3/2}$,  $\left( 14+3\,x \right)
{\frac {2}{105}}\, \left( x-7 \right) ^{3/2} \left( 392+84\,x+15\,{x}^ {2} \right)$
\item $\dsp -\frac{8}{3}\sqrt{8-t^3}$
\item $\dsp -3\, \left( \sin \left( t \right)  \right) ^{-1}-2 \cot(t)$
\item $\dsp 2 \tan(\sqrt{x+5})$
\item $2$
\item $\dsp {\frac {\sin \left( ax+b \right) }{a}}$
\item $\dsp 1/4\,{\frac {{3}^{4\,x}}{\ln  \left( 3 \right) }}$
\item $\dsp {\frac {{b}^{kx}}{k\ln  \left( b \right) }}$
\item $\dsp 1/2 \tan(e^{2x}) + \ln(tan(e^{2x})) - 1/2 \cot(e^{2x})$
\item $\dsp  -(2 e^{-x}+e^{2x}/2)$
\item $\dsp e^x + \ln(e^x-5)$
\item $\dsp-2/9\, \left( 4+\sqrt {x} \right) ^{-3} $
\item $\dsp \frac{1}{4}$
\item $\dsp 1/20\, \left( \ln  \left( 4\,x \right)  \right) ^{4}$
\item $\dsp 1/15\, \left( 2\,x+1 \right) ^{3/2} \left( -1+3\,x \right)$
\end{enumerate}


\item No more solutions!  Either use an integrator on the web or use a symbolic calculator.

\item No solutions to hyperbolic problems because all are to verify something is true so you know if you got the right answer!

\end{enumerate}

\chapter{Sequences \& Series}

``All the instruments have been tried save one, the only one precisely that can succeed: well-regulated freedom.'' - Jean-Jacques Rousseau\\ \\

Intuitively speaking, a {\it sequence} may be thought of as a list of numbers and a {\it series} may be thought of as the sum of a sequence of numbers. The sequence below is approaching $0.$  Does the series below add up to a number, or as we keep adding terms, do the sums tend to infinity?

$$\mbox{Sequence: } 1, 1/2, 1/3, 1/4, \dots$$
$$\mbox{Series: } 1 + 1/2 + 1/4 + 1/8 + \dots$$

\section{Sequences}

\begin{dfn}
A \textbf{sequence} is a function with domain a subset of the natural numbers and range a subset of the real numbers.
\end{dfn}

\begin{expl} We will denote our sequences by listing elements of the range of the sequence.
\begin{itemize}
\item $\dsp{\frac{1}{2} \;, \frac{1}{4} \; , \frac{1}{8}}\; , \dots$ denotes the range of the sequence $\dsp f(n) = \frac{1}{2^n}, \; n = 1,2,3,\dots$.
\item $a_1, a_2, a_3, \dots$ denotes the range of the sequence $f(n) = a_n, \; n = 1,2,3,\dots$.
\item $\dsp \{x_n \}_{n=1}^{\infty}$ denotes the range of the sequence $f(n) = x_n, \; n = 1,2,3,\dots$.
\item $\dsp \{x_n=\frac{n}{n+1} \}_{n=1}^{\infty}$ denotes the range of the sequence $\dsp f(n) = \frac{n}{n+1}, \; n = 1,2,3,\dots$.
\end{itemize}
We mathematicians abuse the definition of function by referring to $y=x^3$ as a function, when in fact the function is the collection of ordered pairs generated by this equation.  In a similar abuse, while $\{x_n \}_{n=1}^{\infty}$ or $x_1, x_2, x_3, \dots$ represents the elements of the range of the sequence, we refer to it as though it were the sequence itself.
\end{expl}

\begin{dfn}
The sequence $\{x_n \}_{n=1}^{\infty}$ is said to be \textbf{bounded above} if there is number $M$ such that $x_n \le M$ for all $n \ge 1$. If such a number exists, then $M$ is called an \textbf{upper bound} for the sequence.  \textbf{Bounded below} and \textbf{lower bound} are defined in the analogous way.  The sequence is \textbf{bounded} if it is bounded above and bounded below.
\end{dfn}

\begin{prb}
Show that the sequence, $\dsp \{x_n = \frac{1}{n} \}_{n=1}^{\infty}$ is bounded above and below.
\end{prb}

\begin{dfn}
The sequence $\{x_n \}_{n=1}^{\infty}$ is \textbf{increasing} (often called strictly increasing) if $x_n < x_{n+1}$ for all natural numbers $n$. The sequence $\{x_n \}_{n=1}^{\infty}$ is \textbf{non-decreasing} if $x_n \le x_{n+1}$ for all natural numbers $n$. \textbf{Decreasing} and \textbf{non-increasing} are defined similarly. A sequence is said to be \textbf{monotonic} if it is either increasing, decreasing, non-increasing, or non-decreasing.
\end{dfn}

\begin{dfn}
The \textbf{factorial} function is defined for all non-negative integers as follows:
\begin{enumerate}
\item $0!=1$
\item $n! = n \cdot (n-1)!$ \; for \; $n=1,2,3,\dots$.
\end{enumerate}
\end{dfn}

The first few factorials are:
\noindent
$$ 0! = 1, \ \ \  1! = 1, \ \ \ 2! = 2 \cdot 1 = 2, \ \ \ 3! = 3 \cdot 2 \cdot 1 = 6, \ \ \ 4! = 4 \cdot 3 \cdot 2 \cdot 1 = 24, \dots$$

\begin{prb}
For each of the following, prove that it is monotonic or show why it is not monotonic.
\begin{enumerate}
\item $\dsp \{\frac{n+3}{4n+2} \}_{n=1}^{\infty}$
\item $\dsp \{ \sin({\frac{n \pi}{3}}) \}_{n=1}^{\infty}$
\item $\dsp \{\frac{(n-1)!}{2^{n-1}} \}_{n=1}^{\infty}$
\end{enumerate}
\end{prb}

\begin{dfn}
Given the sequence, $\{x_n \}_{n=1}^{\infty},$ the number $G$ is called the \textbf{greatest lower bound} of the sequence if $G$ is a lower bound for the sequence and no other lower bound is larger than $G.$  The number $L$ is called the \textbf{least upper bound} if $L$ is an upper bound and no other upper bound of the sequence is less than $L.$
\end{dfn}

\begin{prb}
Find the least upper bound and the greatest lower bound for these sequences.
\begin{enumerate}
\item $\dsp \{ \frac{n}{n+1} \}_{n=1}^\infty$
\item $\dsp \{ \frac{2^n}{n!} \}_{n=2}^\infty$
\item $\dsp \{ \frac{2n+1}{n} \}_{n=2}^\infty$
\end{enumerate}
\end{prb}

\begin{prb}
For each sequence, write out the first five terms.  Is there is a number that the values of $x_n$ approach as $n \rightarrow \infty$? If so, what is it.  If not, why not?
\begin{enumerate}
\item $\dsp\{ x_n = \frac{1}{n} - \frac{1}{n-1}\}$ \; for \; $n=2,3,4, \dots$
\item $\dsp\{ x_n = (-1)^n (2-\frac{1}{n})\}$ \; for \; $n=1,2,3,\dots$
\item $\dsp\{\frac{n^2}{n!}\}_{n=5}^\infty$
\end{enumerate}
\end{prb}

\begin{dfn}
Given a sequence, $\{x_n \}_{n=k}^{\infty}$, we say that the sequence {\bf{converges}} to the number $L$ if for every positive number $\epsilon$, there is a natural number $N$ so that for all $n \ge N,$  we have $|x_n - L| < \epsilon.$
\end{dfn}

If $\{x_n \}_{n=k}^{\infty}$ converges to $L,$ we may write  ``$\displaystyle{\lim_{n \to \infty} x_n = L}$'' or
``$\displaystyle{x_n \to L}$ as $n \to \infty.''$   We call a sequence \textbf{convergent} if there is a number to which it converges, and \textbf{divergent} otherwise.  A divergent sequence might tend to $\pm \infty$ or might ``oscillate.'' In either case, there is no single number that the range of the sequence approaches.

\begin{prb}
\label{fiven} Consider the sequence, $\dsp \frac{5n+4}{3n+1}.$ Let $\epsilon = 0.00002$ and find the smallest natural number $N$ so that $\dsp |\frac{5n+4}{3n+1} - \frac{5}{3}| < \epsilon$ for every $n \ge N$.  Repeat for $\epsilon = 0.00001$.
\end{prb}

\begin{prb}
Consider the sequence, $\dsp \frac{n^2}{n^2+1}.$ Let $\epsilon = .0004$ and find the smallest natural number $N$ such that $\dsp |\frac{n^2}{n^2+1} - {1}| < \epsilon$ for all $n \ge N$. Repeat for $\epsilon = .0002$.
\end{prb}

\begin{prb}
Let $\epsilon$ represent a positive number less than one.  Find the smallest natural number $N$ such that $\dsp |\frac{5n+4}{3n+1} - \frac{5}{3}| < \epsilon$ for every $n \ge N$.  The number $N$ will depend on $\epsilon$ since a smaller value for $\epsilon$ may result in a larger value for $N.$
\end{prb}

\begin{prb}
Let $\{x_n\}_{n=1}^{\infty}$ be the sequence so that $x_n = 0$ if $n$ is odd and $x_n=1$ if $n$ is even.
\begin{enumerate}
\item Show that the sequence is bounded.
\item Show that the sequence does not converge to $0$.
\item What would we need to prove in order to show that the sequence is divergent?
\end{enumerate}
\end{prb}

\begin{prb}
Let $\epsilon$ represent a positive number less than one.  Find the smallest positive integer $N$ such that $\dsp |\frac{n^2}{n^2+1} - {1}| < \epsilon$ for all $n \ge N$.
\end{prb}

\begin{prb}
Show that if $\{ a_n \}_{n=1}^\infty$ is a convergent sequence then it is a bounded sequence.
\end{prb}

\begin{dfn}
The sequence $\{x_n \}_{n=k}^{\infty}$ is called a \textbf{Cauchy sequence} if for every $\epsilon > 0$, there is a natural number $N$ so that $|x_n - x_m| < \epsilon $ for all $m, n \ge N$.
\begin{annotation}
\endnote{How much to include...  Certainly Cauchy sequences are not a necessary part of Calculus II, but I couldn't resist putting them in.  We often skip them or perhaps a few students take the challenge.  Again, the mix of easy and hard problems is for the sake of identifying and recruiting potential majors.}
\end{annotation}
\end{dfn}

\begin{prb}
Show that if $a$ and $b$ are numbers then $|a+b| \le |a| + |b|$.
\end{prb}

\begin{prb}
Show that if $\{ a_n \}_{n=1}^\infty$ is a convergent sequence then it is a Cauchy sequence.
\end{prb}

It is also true that if $\{ a_n \}_{n=1}^\infty$ is a Cauchy sequence then it is a convergent sequence.  That's a bit more interesting problem than we have time to explore -- take \emph{Analysis}.

\begin{dfn}
The sequence $\{x_n \}_{n=k}^{\infty}$ is called a \emph{C} sequence if for every $\epsilon > 0$, there is natural number $N$ such that $|x_N - x_{n}| < \epsilon $ for all $n \ge N$.
\end{dfn}

\begin{prb} Two parts.
\begin{enumerate}
\item Suppose that $\{ a_n \}_{n=1}^\infty$ is a \emph{C} sequence.  Is it a Cauchy sequence?
\item Suppose that $\{ a_n \}_{n=1}^\infty$ is a Cauchy sequence.  Is it a \emph{C} sequence?
\end{enumerate}
\end{prb}

An \textbf{axiom} is a statement that we use without proof.  All of mathematics is built on axioms since one must make certain assumptions just to get started.  There are entire branches of mathematics devoted to determining what of the mathematics that we use now would still be true if we changed the underlying axiom system that mathematicians (more or less) universally agree on.

The \textbf{Completeness Axiom} says that if we have a set of numbers and there is some number that is greater than every number in our set, then there must be a \emph{smallest} number that is greater than or equal to every number in the set.
\begin{annotation}
\endnote{To explain the Completeness Axiom, I use a simple tool taken from Gordon Johnson.  Suppose we hold up in our hands an imaginary x-axis and we break it at the number 3.  Where is 3?  Is 3 in the piece of the x-axis that is in our left hand or in our right hand?  Suppose it is in our left hand.  What number does the part of the x-axis in our right hand start at?   This gives an intuitive way to think about the fact that some sets that are bounded above have a maximum element (the part of the axis in our left hand has a largest element). Yet some sets that are bounded below (the part of the axis in our right hand) do not have a minimum element, even though there is a number that is the largest number less than every number in the set.}
\end{annotation}

\begin{axm}
{\bf{Completeness Axiom}} If a non-empty set $S$ of real numbers has an upper bound, then $S$ has a least upper bound.
\end{axm}

\begin{thm} \label{monotonic}
\textbf{Monotonic Convergence Theorem.} Every  bounded monotonic sequence is convergent.
\end{thm}

%\textbf{Proof.} Ask me if you are interested. \emph{q.e.d.}

%Suppose $\{x_n \}_{n=k}^{\infty}$ is a monotonic sequence.  Therefore
%the sequence is either increasing, decreasing, non-increasing, or non-decreasing.
%Case 1. Suppose that it is increasing.  Since this sequence is bounded we
%know by the Completeness Axiom that it has a least upper bounded.  Let's call
%that number $L$. Let $\epsilon$ be a positive number.  Since $L - \epsilon$ is
%less than $L$, then $L - \epsilon$ cannot be an upper bound of the sequence.
%Therefore, for some positive integer $N$, $L - \epsilon \le
%x_N$. On the other hand, $x_n \le L$ for all integers $n \ge k$
%so $L - \epsilon < x_N \le x_n \le L < L + \epsilon$ for $n
%\ge N$ and the result follows. QED. (QED = quod erat demonstrandum =
%that which was to be demonstrated).

\begin{prb}
Show that $\dsp \{ \frac{2^n}{n!} \}_{n=1}^{\infty}$ is convergent.
\end{prb}

The next theorem formalizes the relationship between sequences and limits of functions.

\begin{thm}
Let $\{x_n \}_{n=1}^{\infty}$ be a sequence and $f$ be a function defined for all $x \ge 1$ such that $f(n)=x_n$ for all $n=1,2,3.\dots$.  The only difference between the function $f$ and the function $x$ (remember, a sequence is a function) is that the domain of $f$ includes real numbers and the domain of $x$ is only natural numbers.
\begin{enumerate}
\item If $\displaystyle{\lim_{x \to \infty} f(x) = L}$, then $\displaystyle{\lim_{n \to \infty} x_n = L}$.
\item If $\displaystyle{\lim_{x \to \infty} f(x) = \pm \infty}$, then $\displaystyle{\lim_{n \to \infty} x_n = \pm \infty}$.
\item If $\displaystyle{\lim_{n \to \infty} x_n = L}$ and $\displaystyle{\lim_{n \to \infty} y_n = M}$, then
\begin{enumerate}
\item $\displaystyle{\lim_{n \to \infty} (x_n \pm y_n) = L \pm M}$; \item $\displaystyle{\lim_{n \to \infty} (x_n \cdot y_n) = LM}$;
\item $\displaystyle{\lim_{n \to \infty} \frac{x_n}{y_n} = \frac{L}{M}}$ for $M \ne 0$.
\end{enumerate}
\end{enumerate}
\end{thm}

\begin{prb}
Consider the sequence $\dsp x_n = \frac{1}{n}$ for all $n=1,2,3,\dots.$ Sketch and define a function, $f$, so that $f$ agrees with $x_n$ at all the natural numbers, but $\dsp \lim_{x \to \infty} f(x) \ne 0.$
\end{prb}

\begin{prb}
Evaluate each of the following limits, whenever it exists.
\begin{enumerate}
\item $\displaystyle{\lim_{n \to \infty} \frac{n^2 + 2n + 3}{3n^2 + 4n -5}}$ \item $\displaystyle{\lim_{n \to \infty} \frac{6n + 100}{n^3 + 4}}$
\item $\displaystyle{\lim_{n \to \infty} \frac{\log_3 (n)}{n}}$ \item $\displaystyle{\lim_{n \to \infty} (1+\frac{3}{n})^n}$
\end{enumerate}
\end{prb}

\section{Series with Positive Constant Terms}

\begin{expl}
Bouncing balls.
\end{expl}

Suppose we drop two balls at the same time from a height of one foot and the first one bounces to the height of one foot with each bounce. (Question for my Mechanical Engineers.  Why is it that if you invent this ball, you can retire?)  The second ball bounces to half the height that it fell from each time.  What distance does each ball travel?  Answer:

$$\mbox{Ball 1:  } 1 + 1 + 1 + 1 + 1 + \dots$$
and
$$\mbox{Ball 2:  }  1 + \frac{1}{2} + \frac{1}{2} + \frac{1}{4} + \frac{1}{4} + \dots = 1 + 1 + \frac{1}{2} + \frac{1}{4} + \frac{1}{8} + \frac{1}{16} + \dots.$$
Such sums are called {\it series} or {\it infinite series} and the individual numbers that are added are called the {\it terms} of the series.  The first sum does not ``add up,'' so we'll say it \emph{diverges}. The second sum does ``add up,'' so we'll say it \emph{converges}. To see that the second sum converges, let $$S_1 = 1, \; S_2 = 1 + 1, \; S_3 = 1 + 1 + \frac{1}{2}, \; S_4 = 1 + 1 + \frac{1}{2} + \frac{1}{4}, \; \dots$$ and consider the limit of this new sequence that we have created. The new sequence we have generated, $S_1, S_2, S_3, \dots$ is called the {\it sequence of partial sums} of the series and looks like this
$$S_1 = 1, \; S_2 = 2, \; S_3 = 2\frac{1}{2}, \; S_4 = 2\frac{3}{4},\dots$$
\begin{prb}
Write down the simplest expression you can find for the $N^{th}$ partial sum, $S_N,$ for the series $1+1+1+ \dots$ and compute $\dsp
\lim_{N \to \infty} S_N.$
\end{prb}

\begin{prb}
Write down the simplest expression you can find for the $N^{th}$ partial sum, $S_N,$ for the series $\dsp 1 + 1 + \frac{1}{2} + \frac{1}{4} + \frac{1}{8} + \dots$ and compute $\dsp \lim_{N \to \infty} S_N.$
\end{prb}

This notion of adding up partial sums of the series is the motivation for the definition of a series. We say that an infinite series {\it converges} if the limit of the partial sums exists.

\begin{dfn}
Given a sequence $\{ a_n \}_{n=k}^\infty$, where $k$ is a positive integer or zero, we define the \textbf{$N^{th}$ partial sum} of this sequence to be  the number $\displaystyle{S_N = \sum_{n=k}^{N} a_n}.$ We denote the limit of the sequence of partial sums by the notation, $\displaystyle{\sum_{n=k}^{\infty} a_n} = \lim_{N \to \infty} S_N,$ and call this the {\it infinite series} associated with the sequence.
\end{dfn}

\begin{dfn}
Given a sequence, $\dsp \{a_n\}_{n=k}^{\infty},$ the series $\displaystyle{\sum_{n=k}^{\infty} a_n}$ is said to
{\bf{converge}} if $\displaystyle \lim_{N \to \infty} S_N$ exists and is said to \textbf{diverge} if $\displaystyle \lim_{N \to \infty} S_N$ does not exist.
\end{dfn}

\begin{prb}
Let $\dsp a_0 = 1-\frac{1}{2}$, $\dsp a_1 = \frac{1}{2} - \frac{1}{3}$, $\dsp a_2 = \frac{1}{3} - \frac{1}{4}$, $\cdots$.
\begin{enumerate}
\item Find a formula for the $n^{th}$ term of the sequence, $a_n.$
\item Compute the limit of the terms of the sequence,  $a_n.$
\item Find a formula for the $N^{th}$ partial sum of the sequence, $S_N.$
\item Compute the limit of the partial sums,
$\displaystyle{\lim_{N \to \infty} S_N = \sum_{n=1}^{\infty} a_n}$.
\end{enumerate}
\end{prb}

We are about to develop many theorems (called ``tests'') to tell us if a series converges or diverges.  Like the various techniques of integration, each test is easy to apply, but figuring out which test to use on a particular series may be challenging.

In our bouncing ball example, you found a formula for the $n^{th}$ partial sum, $S_N$ and showed that $\dsp \lim_{N \to \infty} S_N$ was a real number. Since the sequence of partial sums converged, we say that our series converges. For this example, we were able to write down an exact expression for the $N^{th}$ partial sum and we were able to compute the limit.  In many cases, we will be able to determine whether a particular series converges, but not know from the test what it converges to.  Knowing that the series converges is valuable, because as long as we \emph{know} that the it converges then we can use our calculators or computer algebra systems to add enough terms to get as close to the answer as we want.

\begin{prb}
Let $\dsp a_0 = 3$, $\dsp a_1 = 3 (\frac{1}{2})$, $\dsp a_2 = 3(\frac{1}{2})^2$, $\cdots$.
\begin{enumerate}
\item Find a formula for the $n^{th}$ term, $a_n$.
\item Compute the limit of the terms, $\displaystyle{\lim_{n \to \infty} a_n}$.
\item Find a formula for the $N^{th}$ partial sum, $S_N$.
\item Compute the limit of the partial sums, $\displaystyle{\lim_{N \to \infty} S_N}$ or
$\displaystyle{\sum_{n=0}^{\infty} a_n}$.
\end{enumerate}
\end{prb}

\begin{prb}
Let $\dsp a_0 = 5$, $a_1 = \frac{5}{3}$, $a_2 = \frac{5}{9}$, $\cdots$.
\begin{enumerate}
\item Find a formula for the $n^{th}$ term, $a_n$.
\item Compute the limit of the terms, $\displaystyle{\lim_{n \to \infty} a_n}$.
\item Find a formula for the $N^{th}$ partial sum, $S_N$.
\item Compute the limit of the partial sums, $\displaystyle{\lim_{N \to \infty} S_N}$ or
$\displaystyle{\sum_{n=0}^{\infty} a_n}$.
\end{enumerate}
\end{prb}

\begin{prb}
Let $\dsp a_0 = a$, $a_1 = ar$, $a_2 = ar^2$, $\cdots$.
\begin{enumerate}
\item Find a formula for the $n^{th}$ term, $a_n$.
\item Compute the limit of the terms, $\displaystyle{\lim_{n \to \infty} a_n}$.
\item Find a formula for the $N^{th}$ partial sum, $S_N$.
\item Compute the limit of the partial sums, $\displaystyle{\lim_{N \to \infty} S_N}$ or
$\displaystyle{\sum_{n=0}^{\infty} a_n}$ where $r<1.$ (What happens when $r = 1$ or $r > 1$?).
\end{enumerate}
\end{prb}

In many series problems, it is useful to write out a few terms to see if there is something common that you can factor out
of the sum.  This is trick 49.2.b section 8 from the intergalactic ``Math Book'' that they give out when you get your Ph.D.

\begin{prb}
Determine whether these geometric series converge and if so to what number.
\begin{enumerate}
\item $\dsp{ \sum_{n=3}^\infty  \frac{3}{4^n}  }$ \item $\dsp{
\sum_{n=3}^\infty  \frac{2^n}{3^{n+4}}  }$ \item $\dsp{
\sum_{n=2}^\infty  \frac{4^n}{3^n}  }$ \item $\dsp{
\sum_{n=3}^\infty  \frac{\pi}{e^{n-1}}  }$
\end{enumerate}
\end{prb}

If you only do one problem in Calculus, do this one.
\begin{annotation}
\endnote{Most of my students, and perhaps even a few of yours, have a definite interest in money.  In the book ``How to Make Money in Wall Street,'' Louis Rukeyser wrote something like ``While it has been stated that money can't buy happiness, I'm confident that you can find other things to spend it on.''  Because of this interest, we spend some time talking about investing and how a deep understanding of series is a good start for a career in financial mathematics.  Students (and friends) are often shocked at the interest paid on a house or car note.   Today more than ever, the complex trading algorithms are based on non-trivial mathematics stemming from the fields of calculus, probability and statistics.  As investing is a personal hobby, I can speak with some knowledge about the average mathematical ability of the brokers I have worked with over the years and the need for mathematical literacy in the field.}
\end{annotation}

\begin{prb}
If a banker tells you that you will earn 6\% interest compounded monthly, then you will actually earn $\frac{6}{12}$\% each month, which is better than 6\% annually since you get interest on your interest.  Suppose you save \$300 per month and earn $\frac{6}{12}$\% interest on the amount of money in the bank at the end of each month.  How much money do you have in 30 years?  How much was savings?  How much was interest?
\end{prb}


\begin{thm}
\textbf{The $n^{th}$ Term Test.} If $\displaystyle{\sum_{n=k}^{\infty} a_n}$ converges, then $\dsp{ \lim_{n \to \infty} a_n = 0}.$
\end{thm}

This test is essentially useless as written, but it's contrapositive is quite useful.  The \textbf{contrapositive} of the statement ``if P then Q'' is ``if not Q then not P.''   To make this simple, the converse of ``if you are good, then I will take you for ice cream'' is ``if I did not take you for ice cream, then you were not good.''   Great.  Now I'm hungry.  We will look at a proof of the $n^{th}$ term test later.

\begin{prb}
Assume that $a_n > a_{n+1} > 0$ for all $n=1,2,3,\dots.$  Prove that if $\dsp{ \lim_{n \to \infty} a_n = L > 0},$ then the series, $\displaystyle{\sum_{n=k}^{\infty} a_n}$ diverges.  This is a special case of the $n^{th}$ term test where the terms are positive.
\end{prb}

\begin{prb}
In your mechanical engineering lab, you create a ball with a coefficient of elasticity so that if you dropped the ball from a height of one foot then the ball would bounce exactly to height $\dsp \frac{1}{2n}$ feet on the $n^{th}$ bounce.  How far does the ball travel?
\end{prb}

The $n^{th}$ term test told us that if the terms don't approach zero, then the series diverges.  This does not imply that if the terms do approach zero, then the series converges.  The following problem illustrates two important points.  First it shows that a series can have terms that approach zero, but still not converge. Second it shows a way to help determine whether a series converges by making a comparison between the series and a closely related integral.

\begin{prb} \label{harmonicseries}
\textbf{The Harmonic Series.}
\begin{annotation}
\endnote{Depending on the strength of the class, I may simply work this one out because the problem illustrates so well the intuition that leads to the integral test.  If you would like your kids to read a nice play on harmonics, consider http://scholarship.claremont.edu/jhm/vol3/iss2/11/}
\end{annotation}
Let $\dsp a_1 = 1, a_2 = \frac{1}{2}, a_3 = \frac{1}{3}, \dots$.

\begin{enumerate}
\item Find $\displaystyle{\lim_{n \to \infty} a_n}$.
\item Let $\dsp f(x) = \frac{1}{x}$ be defined on $[1, \infty)$ and sketch a graph of both $f$ and our sequence, $a_n = \frac{1}{n}$ for $n=1,2,3,\dots$ on the same coordinate axes.
\item Sketch rectangles (Riemann sums) whose areas approximate the area under the curve, $f$.
\item Use your picture to show that $\displaystyle{S_N = \sum_{n=1}^{N-1} a_n \ge \int_1^N \frac{1}{x} \; \; dx}$.
\item Show that $\dsp \int_1^\infty \frac{1}{x} \; \; dx$ diverges, so $\dsp \sum_{n=1}^{\infty} \frac{1}{n}$ diverges.
\end{enumerate}
\end{prb}

\begin{prb}
Let $\dsp a_n = \frac{1}{n^2}$ for $n=1,2,3,\dots$.
\begin{enumerate}
\item Find $\displaystyle{\lim_{n \to \infty} a_n}$.
\item Use a graph to show that for each $N,$ $$\displaystyle{S_N = \sum_{n=1}^N \frac{1}{n^2} \le 1 + \int_1^N \frac{1}{x^2} \; \; dx \le 1 + \int_1^{\infty} \frac{1}{x^2} \; \; dx}$$\ so that the sequence of partial sums $\{S_n \}_{n=1}^{\infty}$ is bounded above.
\item Conclude that $\displaystyle{\sum_{n=1}^{\infty} a_n}$ converges because $\{S_n \}_{n=1}^{\infty}$ satisfies the hypothesis of Theorem \ref{monotonic}.
\end{enumerate}
\end{prb}

\begin{thm} \textbf{Integral Test.}
Let $f$ be a continuous function such that $f(x) \ge 0$ and $f$ is decreasing for all $x \ge 1$ and $f(n) = a_n$ for all $n=1,2,3,\dots$.   Then $$\sum_{n=1}^{\infty} a_n \mbox{ converges if and only if } \int_1^\infty f(x) \;\; dx \mbox{ converges }.$$
\end{thm}

\begin{prb}
Apply the integral test to each series to determine whether it converges or diverges.
\begin{enumerate}
\item $\dsp{ \sum_{n=4}^\infty  \frac{1}{4n^3}  }$ \item $\dsp{\sum_{n=2}^\infty  \frac{3}{n^2}  }$
\item $\dsp{ \sum_{n=1}^\infty  \frac{5}{n^{1.001}}  }$
\item $\dsp{ \sum_{n=1}^\infty  \frac{4}{\sqrt{2n}}  }$
\end{enumerate}
\end{prb}

\begin{prb}
\label{inttest}
Prove the integral test as follows.  Let $f$ be a continuous function such that $f(x) \ge 0$ and is decreasing for all $x \ge 1$, and let $f(n)=a_n$. Suppose $\dsp{ \int_1^\infty f(x) \ dx}$ converges.
\begin{annotation}
\endnote{Again, depending on the strength of the class, I may simply work this one out.}
\end{annotation}
\begin{enumerate}
\item   Partition the interval $[1, N]$ into $N-1$ equal intervals and sketch the rectangles as you did in Problem \ref{harmonicseries}.
\item Express the total area of the rectangles in sigma notation; express the total area of the circumscribed rectangles in sigma notation.
\item Establish these two inequalities geometrically. $$\displaystyle{\sum_{n=1}^N a_n \le a_1 + \int_1^N f(x) \; dx} \; \mbox{ and } \displaystyle{\int_1^N f(x) \; \; dx \le \sum_{n=1}^N a_n}$$
\item Prove that $\displaystyle{\sum_{n=1}^{\infty} a_n}$ converges by showing that $S_n$ is bounded above and increasing.
\end{enumerate}
\end{prb}

\begin{thm} \textbf{$p$-Series\ Test.} The series $\displaystyle{\sum_{n=1}^{\infty} \frac{1}{n^p}}$ converges if $p > 1$ and diverges if $p \le 1$.
\end{thm}

\begin{prb}
Use the integral test to prove the p-series test for the case when $p>1$.
\end{prb}

\begin{prb}
Determine whether each series converges.  If it does, use Problem \ref{inttest} to give an upper bound for the infinite sum.
\begin{enumerate}
\item $\dsp{ \sum_{n=4}^\infty \frac{5}{n^4} }$
\item $\dsp{\sum_{n=1}^\infty \frac{1}{2n^3} + \frac{2}{n^2} }$
\item $\dsp{\sum_{n=1}^\infty \frac{1}{\sqrt{2n^3}} }$
\end{enumerate}
\end{prb}

We now have two types of series for which we can easily determine convergence or divergence: {\it geometric series} and {\it p-series}.  We will use these to determine the convergence or divergence of many other series. If a series with positive terms is less than some convergent series, then it must converge. On the other hand, if a series is larger than a series with positive terms that diverges then it must diverge.

\begin{prb}
Consider the series $\displaystyle{\sum_{n=1}^{\infty} \frac{1}{n^3+2}}.$
\begin{enumerate}
\item Show that for all $n \ge 1,$ we have $\dsp \frac{1}{n^3+2} > 0.$
\item Show that for all $n \ge 1,$ we have $\dsp \frac{1}{n^3+2} < \frac{1}{n^3}.$
\item Why is this series monotonic and bounded and therefore convergent?
\end{enumerate}
\end{prb}

\begin{thm} \textbf{Direct Comparison Test.}  Suppose that each of $\displaystyle{\sum_{n=k}^{\infty} a_n}$  and
$\displaystyle{\sum_{n=k}^{\infty} b_n}$ are series such that $0 \leq a_n \le b_n$ for all integers $n \ge k$.
\begin{enumerate}
\item If $\displaystyle{\sum_{n=k}^{\infty} b_n}$ converges, then $\displaystyle{\sum_{n=k}^{\infty} a_n}$ converges.
\item If $\displaystyle{\sum_{n=k}^{\infty} a_n}$ diverges, then $\displaystyle{\sum_{n=k}^{\infty} b_n}$ diverges.
\end{enumerate}
\end{thm}

\begin{prb}
Apply the direct comparison test to each of these series.
\begin{enumerate}
\item $\dsp{ \sum_{n=2}^\infty  \frac{1}{3^n+5}  }$
\item $\dsp{ \sum_{n=2}^\infty  \frac{1}{3^n-5}  }$
\item $\dsp{\sum_{n=3}^\infty  \frac{3}{n^2 \ln(n)}  }$
\item $\dsp{\sum_{n=3}^\infty  \frac{e}{5(n^5 + 32)}  }$
\end{enumerate}
\end{prb}

\begin{expl}
Limit Comparison Test
\end{expl}

Consider the series, $\displaystyle{\sum_{n=1}^{\infty} \frac{n}{2^n (3n-2)}}.$  The terms converge to zero, so the $n^{th}$ term test tells us nothing. It is not a geometric series either.  It's not obvious what other series we might use the direct comparison test with, although that is a possibility. Still, each term of this series is the product of $\dsp{ \frac{1}{2^n}}$ and $\dsp{ \frac{n}{3n-2}}$ and we know that $\dsp \lim_{n \to \infty} \frac{n}{3n-2} = \frac{1}{3},$  so for large values of $n,$ the $n^{th}$ term is approximately equal to $\dsp{ \frac{1}{2^n} \cdot \frac{1}{3} }$.  Hence, our series behaves (for large $n$) a lot like a geometric series and that's a good intuitive reason to believe it converges.  When we have a series like this that ``looks like'' a series that we know converges, in this case $\dsp \sum_{n=1}^\infty \frac{1}{3}\frac{1}{2^n}$, then if the limit of the ratio of the terms of the two series equals a positive number, then the series either both converge or both diverge.  The next theorem makes this precise.

\begin{thm} \textbf{Limit Comparison Test.} Let $\{a_n \}_{n=k}^{\infty}$ and $\{b_n \}_{n=k}^{\infty}$ be sequences of positive terms such that $\displaystyle{\lim_{n \to \infty}\frac{a_n}{b_n} = c}$ where $c$ is a positive real number (i.e. $c=\infty$ is not allowed). Then the series $\displaystyle{\sum_{n=k}^{\infty} a_n}$ and $\displaystyle{\sum_{n=k}^{\infty} b_n}$ either both converge or both diverge.
\end{thm}

\begin{prb}
Apply the limit comparison test to each series to determine if it converges or diverges.
\begin{enumerate}
\item $\dsp{ \sum_{n=2}^\infty  \frac{1}{3^n-5}  }$
\item $\dsp{ \sum_{n=3}^\infty  \frac{3}{n + \pi}  }$
\item $\displaystyle{\sum_{n=1}^{\infty} \sin \Big(\frac{1}{n}\Big)}$ %limit comparison with 1/n
\item $\displaystyle{\sum_{n=1}^{\infty} \Big(\frac{n+1}{2n+3}\Big) \Big(\frac{3^n}{5^n + 1}\Big)}$
\end{enumerate}
\end{prb}

\begin{prb}
In the real world of computer science and operations research, the ``average'' of a series is an important notion. Suppose we have a sequence, $\{ a_n \}_{n=1}^\infty$ where $0 \leq a_n \leq 1$ for all positive integers, $n.$
\begin{annotation}
\endnote{A friend of mine who works in the defense industry called me up because his team needed an answer to this question.  I'd tell you what is was related to, but then I would have to kill you.}
\end{annotation}
Consider the sequence given by $$A_N = \frac{1}{N} \sum_{n=1}^N a_n.$$
\begin{enumerate}
\item Suppose $a_1=a_2=a_3=\dots = 1$ Does the sequence $\{ A_N \}_{N=1}^\infty$ converge?  To what?
\item Suppose $a_1=1, a_2 =0, a_3= 1, \dots$ Does the sequence $\{ A_N \}_{N=1}^\infty$ converge?  To what?
\item Do you think that  $\{ A_N \}_{N=1}^\infty$ converges no matter what $a_1, a_2, \dots$ is?
If so give an argument, if not, give an example for $a_1, a_2, \dots$ and
show that for this sequence the $A_n$'s do not converge.
\end{enumerate}
\end{prb}
% solution: Let a1=0, so A1=0.  Let a2=a3=...=a10=1, so A10=9/10.  Let a11=...=a100=0, so A100=9/100.
% Let a101=...=a1000=1 so A1000=909/1000.   Thus the A_Ns can be made to get arbitrarily close to 0 and
% arbitrarily close to 1 alternating back and forth so that the A_Ns diverge.

\begin{prb}
Determine the convergence or divergence of the following series by using any of the previously stated tests.
\begin{enumerate}
\item $\displaystyle{\sum_{n=1}^{\infty} \frac{2}{3^n + n}}$
\item $\displaystyle{\sum_{n=1}^{\infty} \frac{n^2-n +1}{2n^2 -n - 3}}$
\item $\displaystyle{\sum_{n=4}^{\infty} \frac{-2}{n^2-4n+3}}$
\item $\displaystyle{\sum_{n=4}^{\infty} \frac{1}{n!}}$  %How to do this?  Compare to 1/n^2 intuitively?
\item $\displaystyle{\sum_{n=1}^{\infty} \cos(n \pi)}$
\end{enumerate}
\end{prb}

\begin{prb}
Determine the convergence or divergence of each of the following series.  If any of them converge, compute the number to which it coverges.
\begin{enumerate}
\item $\displaystyle{\sum_{n=1}^{\infty}{ {{|\sin(\frac{n \pi}{2})}|} \over {2^n}}}$ %geometric
\item $\displaystyle{\sum_{n=1}^{\infty} (2n+1)^\frac{1}{n}}$  %nth term test
\item $\displaystyle{\sum_{n=2}^{\infty} \frac{n^2+2n+1}{3n-4}}$ %nth term test
\item $\displaystyle{\sum_{n=1}^{\infty} \frac{(-1)^{n+1}}{e^n}}$
% for last one, compare to geometric with all positive terms to get convergence; to find
% what it converges to, break into even and odd and use geometric on those
\end{enumerate}
\end{prb}

\section{Alternating Series}

We now consider series where every other term of the series is positive and every other term of the series is negative.  To keep things simple, we will write each term of an alternating series as the product of two parts: the positive part, $a_n, n=1,2,3,\dots$ and the alternating part, $(-1)^n$ or $(-1)^{n+1}$.

\begin{dfn}
If $a_n > 0$ for all positive integers $n$, then the series $\displaystyle{\sum_{n+1}^{\infty} (-1)^n a_n}$ and the series
$\displaystyle{\sum_{n=1}^{\infty} (-1)^{n+1} a_n}$ are called \textbf{alternating series}.
\end{dfn}

\begin{expl}
The alternating Harmonic Series.
\end{expl}

Recall the Harmonic Series, $\dsp{\sum_{n=1}^\infty \frac{1}{n} },$ where the terms approached zero, but the series still diverged.  We can modify this series to create the alternating series
$$S=\dsp{ \sum_{n=1}^\infty (-1)^{n+1} \frac{1}{n} } = 1 - \frac{1}{2} + \frac{1}{3} - \frac{1}{4} + \frac{1}{5} - \cdots.$$
By grouping the terms,
$$0 \leq (1 - \frac{1}{2}) + (\frac{1}{3} - \frac{1}{4}) + \cdots
= S = 1 - (\frac{1}{2} - \frac{1}{3}) - (\frac{1}{4} - \frac{1}{5}) - \cdots \leq 1$$
so this series is between $0$ and $1$. The even partial sums can be written as:
$$S_2 = 1 - \frac{1}{2}, \; \; S_4 = (1 - \frac{1}{2}) + (\frac{1}{3} - \frac{1}{4}), \; \; S_6 = \dots$$
so
$$S_{2N} = (1 - \frac{1}{2}) + (\frac{1}{3} - \frac{1}{4}) + \cdots + (\frac{1}{2N-1} - \frac{1}{2N}).$$
This implies that the sequence $\{S_{2n} \}_{n=1}^{\infty}$ is increasing and bounded above by 1, so the sequence of even terms converges to some number, let's call it $L.$ A similar argument shows that the odd terms form a decreasing sequence that is bounded below, so the odd terms converge to some number, call it $G.$  Draw these terms on a number line and you will see that
$$S_2 < S_4 < \dots < L \leq G < \dots < S_5 < S_3 < S_1.$$
This means that $$|L-G| \leq |S_{N} - S_{N+1}| = 1/N.$$
Since $\lim_{N\to \infty} 1/N = 0$ we have that $L=G$ so this alternating series converges.\\

\textbf{Conclusion.} This alternating series has the property that if we take the absolute value of the terms then the resulting series is the Harmonic Series which diverges.  But the alternating series itself converges. We call this \emph{conditionally convergent} and we classify all alternating series as either \textbf{divergent} (series and absolute value of series diverge), \textbf{conditionally convergent} (series converges, but absolute value of series diverges), or \textbf{absolutely convergent} (series converges and absolute value of series converges).

\begin{dfn}
The series $\displaystyle{\sum_{n=k}^{\infty} a_n}$ is said to be {\bf{absolutely convergent}} if the series
$\displaystyle{\sum_{n=k}^{\infty} |a_n|}$ converges and the series $\displaystyle{\sum_{n=k}^{\infty} a_n}$ is said to be
{\bf{conditionally convergent}} if it converges but the series $\displaystyle{\sum_{n=k}^{\infty} |a_n|}$ diverges.
\end{dfn}

Here are two tests for checking the convergence of an alternating sequence.  The first says that if the sum of the absolute value of the terms converge, then the alternating series also converges. The second test gives us another test for when an alternating series converges \emph{and} gives us a powerful tool for saying how accurate the $N^{th}$ partial sum is.  This can tell us how many terms we need to add up in order to get an accurate approximation to the infinite sum.

\begin{thm} \textbf{Absolute Convergence Test.}  Let $\{a_n \}_{n=k}^{\infty}$ be a sequence of positive terms.  If $\displaystyle{\sum_{n=k}^{\infty} a_n}$ converges, then the alternating series $\displaystyle{\sum_{n=k}^{\infty} (-1)^n a_n}$ and $\displaystyle{\sum_{n=k}^{\infty} (-1)^{n-1} a_n}$ both converge.
\end{thm}


\begin{thm} \textbf{Alternating Series Test.} Let $\{a_n \}_{n=k}^{\infty}$ be a non-increasing sequence of positive terms such that $\displaystyle{\lim_{n \to \infty} a_n = 0}$. Then
\begin{enumerate}
\item the alternating series $\displaystyle{\sum_{n=k}^{\infty}
(-1)^n a_n}$ and $\displaystyle{\sum_{n=k}^{\infty} (-1)^{n-1}
a_n}$ both converge and
\item if $S$ is the sum of the series and $S_N$ is a partial sum, then  $|S_N - S| \le a_{N+1}$.
\end{enumerate}
\end{thm}

\begin{prb}
Determine the convergence or divergence of each of the following alternating series. If the series converges, then find a positive integer  $N$ so that $S_N$ will be within .01 of the series.
\begin{enumerate}
\item $\displaystyle{\sum_{n=1}^{\infty} \frac{(-1)^n n}{n^2+1}}$
\item $\displaystyle{\sum_{n=1}^{\infty} \sin \Big(\frac{(2n-1) \pi}{2} \Big)}$
\item $\displaystyle{\sum_{n=1}^{\infty} \frac{(-1)^n}{\sqrt[3] n}}$
\item $\displaystyle{\sum_{n=1}^{\infty} (-1)^n \frac{2^n}{n^2}}$
\item $\displaystyle{\sum_{n=1}^{\infty} \Big(\frac{-1}{n}\Big)^{n+1}}$
\end{enumerate}
\end{prb}

Recall that it is possible for a series to have positive terms that approach zero, but the series still diverges -- for example, the Harmonic Series.  It turns out that you can have an alternating series where the limit of the terms is zero, but the alternating series still diverges.   The next problem asks for you to find one!

\begin{prb}
Construct a sequence of non-negative terms, $\{ a_n \}_{n=1}^\infty,$ so that the $\lim_{n \to \infty} a_n = 0$ and $\dsp \sum_{n=1}^\infty (-1)^{n} a_n$ diverges.
\end{prb}
%0, 1/2, 0, 1/3, 0, 1/4, 0, 1/5....  so the *alternating* sum becomes
% 0 - 1/2 + 0 - 1/3 + 0 - 1/4 +... = - (1/2 + 1/3 + 1/4 + 1/5 + ...)

\begin{prb}
Write out the first few terms for  $\sum_{n=1}^\infty (-1)^n$.   Does it converge or diverge?  Why?
\end{prb}

\begin{prb}
Consider the sequence of non-negative terms where $a_{2n-1} = n$ and $a_{2n} = n + \frac{1}{(n+1)^2}$.  Determine whether $\sum_{n=1}^\infty (-1)^n a_n$ converges or diverges by writing out both the first 6 terms  $a_1, a_2, a_3, a_4, a_5, a_6$ and then the first six partial sums $S_1, S_2, S_3, S_4, S_5, S_6$.
\end{prb}

The integral, direct comparison and the limit comparison tests can only be used for series with positive terms.  The next few theorems provide us with tests that we can use regardless of whether the terms are positive or not.  First, let's recall the \textbf{$n^{th}$ Term Test} which said that if $\displaystyle{\sum_{n=1}^{\infty} a_n}$ converges, then $\dsp{ \lim_{n \to \infty} a_n = 0}.$  Let's prove that this works for any series regardless of whether the terms are positive or not.   If the series converges, let's call it $S=\displaystyle{\sum_{n=1}^{\infty} a_n}$.  Since it converges, $S = \lim_{n \to \infty} S_n$.   But if this is the case, then isn't it true that $S = \lim_{n \to \infty} S_{n-1}$?  From this we have $$0=S - S =  \lim_{n \to \infty} S_{n} - \lim_{n \to \infty} S_{n-1} = \lim_{n \to \infty} \big( S_n - S_{n-1} \big) = \lim_{n \to \infty} a_n.$$  Since we didn't use anywhere that the terms were positive, the $n^{th}$ term test may be applied to any series whether the terms are positive or not. \emph{q.e.d.}

The next two tests, the \textbf{Ratio Test} and the \textbf{Root Test} also work on series regardless of whether the terms are positive or not.

\begin{thm} \textbf{Ratio Test.} Let $\{a_n \}_{n=k}^{\infty}$ be a sequence such that $\displaystyle{\lim_{n \to \infty} |\frac{a_{n+1}}{a_n}| = r}$
(where we allow the case $r=\infty$). Then
\begin{itemize}
\item the series $\displaystyle{\sum_{n=k}^{\infty} a_n}$ converges absolutely if $r < 1,$
\item the series $\displaystyle{\sum_{n=k}^{\infty} a_n}$ diverges if $r > 1$ or $r = \infty,$  and
\item the test fails if $r=1.$
\end{itemize}
\end{thm}

\textbf{Sketch of Proof of Ratio Test.}  This is a quite advanced proof, worthy of a course in Analysis.  If you read it and are curious, I'll happily discuss it or fill in the missing details.   Let's consider the case where $\displaystyle{\lim_{n \to \infty} |\frac{a_{n+1}}{a_n}| = r}$ and $r < 1$. Since $\dsp |\frac{a_{n+1}}{a_n}|$ is getting closer and closer to $r$ then we can choose a positive integer $N$ large enough to assure that $\dsp |\frac{a_{N+1}}{a_N}| < 1$.  Therefore we can find a positive real number $R$ so that $|a_{N+1}| < R|a_N|$.  A little bit of algebra shows that for all positive integers $k$ we have $|a_{N+k}|<R^k|a_N|$.   This implies that $\dsp{\sum_{k=1}^{\infty} |a_{N+k}|}$ converges and we can conclude that $\dsp{\sum_{n=1}^{\infty} a_n}$ converges absolutely.  \emph{q.e.d.}

\begin{prb}
Apply the ratio test to each series in an attempt to determine the convergence or divergence of each series.  If the ratio test fails, apply another test to determine convergence or divergence.
\begin{enumerate}
\item $\dsp{ \sum_{n=2}^\infty  \frac{n^2}{n!}  }$
\item $\dsp{ \sum_{n=3}^\infty  \frac{2^n}{n^3}  }$
\item $\dsp{ \sum_{n=3}^\infty  \frac{\ln(n)}{n}  }$
\end{enumerate}
\end{prb}

\begin{prb}
Using any test you like, determine whether the series is absolutely convergent, conditionally convergent, or divergent.
\begin{enumerate}
\item $\displaystyle{\sum_{n=1}^{\infty} \frac{{(-1)}^n}{\ln(n+1)}}$
\item $\displaystyle{\sum_{n=1}^{\infty} (-1)^n \frac{n!}{3^{n+1}}}$
\item $\displaystyle{\sum_{n=1}^{\infty} (-1)^n \frac{n^2+3}{4n^3}}$
\end{enumerate}
\end{prb}

Here is one last test that the pundits believe you should be aware of.

\begin{thm} \textbf{Root\ Test.} Let $\{a_n \}_{n=k}^{\infty}$ be a sequence such that $\displaystyle{\lim_{n \to \infty} \sqrt[n] {|a_n|}=r}$ (where we
allow the case $r=\infty$). Then
\begin{itemize}
\item the series $\displaystyle{\sum_{n=k}^{\infty} a_n}$ converges absolutely if $r < 1$, \item the series $\displaystyle{\sum_{n=k}^{\infty} a_n}$ diverges if $r > 1$ or $r = \infty,$ and \item the test fails if $r=1.$
\end{itemize}
\end{thm}


\section{Power Series}

From the first semester of calculus, we know that the best linear ($1^{st}$ degree polynomial) approximation to the function, $f(x) = e^x$ at the point $P=(2,e^2)$ is the tangent line to the function at this point.  Let's call that approximation $L$ and note that $L(2) = f(2)$ and $L'(2) = f'(2)$.  That is, $L$ and $f$ intersect at $P$ and $L$ and $f$ share the same derivative at $P$. What is the best quadratic ($2^{nd}$ degree polynomial) approximation? If the best first degree approximation to the curve agrees at the point and in the first derivative, then the best second degree approximation should agree with the function at the point, in the first derivative, and in the second derivative.

\begin{expl}
Compute best linear and quadratic approximations to $f(x) = x^3-16x$ at $x=1$.
\begin{annotation}
\endnote{Depending on the class, I might work this example or I might not. If I do, it serves as an introductory lecture to the concept behind Power Series.   I typically compute the first degree approximation, $L,$ to $f(x) = x^3 - 16x$ at $x=2.$  Then I would ask what $f$ and $L$ have in common until we agreed that $f(2)=L(2)$ and $f'(2)=L'(2)$.  Next I would ask what a second degree approximation, $Q,$ would have in common until we agreed that $f(2) = Q(2)$, $f'(2) = Q'(2)$, and $f''(2) = Q''(2).$  We might even solve for $Q$.  If time allowed, I would point out that we already know that $f(x) = 1/(1-x)$ has an expansion for $0<x<1$ with $f(x) = \sum x^n$ (geometric series) and so $g(x) = 1/(1-x^2) = \sum (x^2)^n$, so some rational functions have infinite polynomial expansions.   I would conclude the discussion by pointing out that the key questions we ask are, ``When can we approximate a function at a point by an infinite polynomial?'' and ``What is the largest set of numbers over which this infinite polynomial (series) converges?''.}
\end{annotation}
\end{expl}

\begin{prb}
\label{taylorex}
Let $f(x) = e^x.$  Find the best linear approximation, $L(x) = mx+ b,$ to $f$ at $(2,e^2).$  Find the best quadratic approximation, $Q(x) = ax^2 + bx +c,$ to $f$ at $(2,e^2).$  Graph all three functions on the same pair of coordinate axes.
\end{prb}

\begin{prb}
Let $f(x) = \cos(x).$  Find the best linear (L, $1^{st}$ degree), quadratic (Q, $2^{nd}$ degree), and quartic (C, $4^{th}$ degree) approximations to $f$ at $(\pi, \cos(\pi)).$  Sketch the graph of $f$ and all three approximations on the same pair of coordinate axes. Compute and compare $\dsp f(\frac{3\pi}{2})$, $\dsp L(\frac{3\pi}{2})$, $\dsp Q(\frac{3\pi}{2})$, and $\dsp C(\frac{3\pi}{2}).$
\end{prb}

From our work on geometric series, we know that
$$\displaystyle{\sum_{n=0}^{\infty} t^n = \frac{1}{1-t}}$$
for $|t| < 1$.  Therefore the function $ \dsp g(t) = \frac{1}{1-t} \mbox{ where } |t| < 1$ can be written as a series,
$$\displaystyle{g(t) = \frac{1}{1-t} = 1 + t + t^2 + t^3 + \cdots = \sum_{n=0}^{\infty} t^n}.$$
Replacing $t$ by $-x^2$, we have
$$\dsp \frac{1}{1+x^2}=1-x^2+x^4- \cdots + (-1)^{n-1} x^{2n-2} + \cdots.$$

Based on these two observations, we see that we can write at least two rational functions as infinite series (or infinite polynomials).  And we can write every polynomial as an infinite series, since
$$f(x) = a_0 + a_1x + a_2x^2 + \cdots + a_Nx^N = \sum_{i=0}^{\infty} a_ix^i$$
where $a_i = 0$ for $i > N$.  Our goal is a systematic way to write any differentiable function (like $sin$ or $cos$
or $\ln$) as an infinite series.\\

\begin{dfn}
$0^0 = 1$
\end{dfn}

\textbf{Notation.} To keep from confusing the powers of $f$ with the derivatives of $f$, we use $f^{(n)}$ to represent $f$ if $n=0$ and the $n^{th}$ derivative of $f$ if $n \ge 1.$  Therefore when $n \geq 1$,  $f^{(n)}(c)$ means the $n^{th}$ derivative of $f$ evaluated at the number $c$.

\begin{prb}
Let $\displaystyle{f(x) = \sum_{i=0}^{\infty} a_ix^i}.$ Write out $S_6$, the $6^{th}$ partial sum of this series.  Compute $S_6'$ and $S_6''$. Compute $S_6'(0)$ and $S_6''(0).$
\end{prb}

\begin{prb}
Let $\displaystyle{f(x) = \sum_{i=0}^{\infty} a_ix^i}.$  Compute $f'$ and $f''.$  Compute $f(0), f'(0), f''(0), \dots$?  If $n$ is any positive integer, what is the $n^{th}$ derivative at $0$, $f^{(n)}(0)$?
\end{prb}

\begin{dfn}
If $f$ is a function with $N$ derivatives, then the \textbf{$N^{th}$ degree Taylor polynomial of $f$ expanded at $c$} is defined by $$T_N(x) = \sum_{n=0}^{N} \frac{f^{(n)}(c)}{n!}(x-c)^n.$$
\end{dfn}

\begin{expl}
Compute the Taylor series polynomials of degree 1, 2, 3 and 4 for $f(x) = 1/x$ expanded at 1.  Compute the Taylor series polynomial for the same function expanded at 1.
\end{expl}

\begin{prb}
Suppose that $\dsp T(x) = \sum_{n=0}^{\infty} \frac{f^{(n)}(c)}{n!}(x-c)^n.$ Write out the first few terms of this series and compute $T'(c)$, $T''(c)$, $T'''(c)$, \dots.
\end{prb}

\begin{dfn}
The series in the last problem is called the \textbf{Taylor Series for $f$ expanded at c}.  When $c=0$ this is called the \textbf{McLaurin Series for $f$}.
\end{dfn}

\begin{prb}
Let $f(x) = e^x.$  Compute the first, second, and third degree Taylor polynomials for $f$ expanded at 2. Compare to the result of problem \ref{taylorex}.
\end{prb}

\begin{expl}
Compute the Taylor series for $f(x)=1/x$ at $c=1$ and use the Ratio Test to determine the interval of convergence.  Also check convergence at both endpoints of the interval of convergence.
\end{expl}

\begin{prb}
For each function, write the infinite degree Taylor polynomial as a series.
\begin{enumerate}
\item $f(x) = e^x$ expanded at 0
\item $f(x) = \sin(x)$ expanded at 0
\item $\dsp f(x) = \frac{\cos(x)}{x}$ expanded at 0 (Divide the series for $\cos(x)$ by $x$.)
\end{enumerate}
\end{prb}

\begin{prb}
Let $f(x) = \ln(x).$  Compute the first, second, and third degree Taylor polynomial for $f$ expanded at 1. Compute $|f(2.5) - T_N(2.5)|$ for $N=1,2,3.$
\end{prb}

We now can write certain functions as infinite series, but we know that not every infinite series converges. Therefore, our expressions only make sense for the values of $x$ for which the series converges. For this reason, every time we write down an infinite series to represent a function, we need to determine the values of $x$ for which the series converges.  This is called interval of convergence (or domain) of the series.

\begin{dfn}
A \textbf{{ power series expanded at $c$} } is any series of the form $\displaystyle{\sum_{n=0}^{\infty} a_n(x-c)^n}$.  Taylor and McLaurin series are special cases of power series.  The largest interval for which a power series converges is called its \textbf{{interval of convergence}}.  Half the length of this interval is called the \textbf{{radius of convergence}}.
\end{dfn}

\begin{prb}
Use the ratio test to find the interval and radius of convergence for each power series.
\begin{enumerate}
\item $\dsp \sum_{n=0}^{\infty} \frac{x^n}{n+1}$
\item $\dsp \sum_{n=0}^{\infty} \frac{x^n}{n!}$
\item $\dsp \sum_{n=0}^{\infty} n!(x+2)^n$
\end{enumerate}
\end{prb}

The next theorem says that the interval of convergence of a power series expanded at the point $c$ is either (i) only the one point, $c$ (bad, since this means the series is useless), (ii) an interval of radius $R$ centered at $c,$ (good, and the endpoints might be contained in the domain), or (iii) all real numbers (very good).

\begin{thm}
\textbf{Power Series Theorem.} For the power series $\displaystyle{\sum_{n=0}^{\infty} a_n(x-c)^n}$, one and only one of the following is true.
\begin{enumerate}
\item The series converges only at $c$.  (bad)
\item There exists a positive number $R$ called the \textbf{radius of convergence} such that $\displaystyle{\sum_{n=0}^{\infty} a_n(x-c)^n}$ converges absolutely for $|x - c| < R$. (good)
\item The series converges absolutely for all $x$. (great)
\end{enumerate}
\end{thm}

When we write the power series for a function (like $e^x$) and it converges for all values of $x$ then we simply have a different way to write the function that turns out to be both computationally useful, because of convergence, and theoretically useful.  Speaking loosely, $e^x$ is just an infinite polynomial -- how cool is that?!

\begin{prb}
Find the interval of convergence for each of the following power series.
\begin{enumerate}
\item $\dsp \sum_{n=0}^{\infty} \frac{(x-3)^n}{\ln(n+2)}$
\item $\dsp \sum_{n=1}^{\infty} \frac{(-x)^n}{n}$
\item $\dsp \sum_{n=0}^{\infty} \frac{\log(n+1)}{(n+1)^4} (x+1)^n$
\item $\dsp  \sum_{n=0}^{\infty} \frac{n^3}{3^n} (x+2)^n$
\end{enumerate}
\end{prb}

The following theorem says that if you have a power series, then its derivative is exactly what you want it to be!  Just take the derivative of the series term by term and the new infinite series you get is the derivative of the first and is defined on the same domain.

\begin{thm}
\textbf{Power Series Derivative Theorem.} If $\displaystyle{f(x) = \sum_{n=0}^{\infty} a_n(x-c)^n}$ for all $x$ in the interval of convergence $(c-R, c+R)$, then
\begin{enumerate}
\item $f$ is differentiable in $(c-R, c+R)$,
\item $\displaystyle{f'(x) = \sum_{n=1}^{\infty} na_n(x-c)^{n-1}}$ for all $x$ in $(c-R, c+R)$, and
\item the interval of convergence of $f'$ is also $(c-R, c+R)$.
\end{enumerate}
\end{thm}

\begin{prb}
Suppose that $\displaystyle{f(x)=\sum_{n=0}^{\infty} a_n(x-c)^n}$ with radius of convergence $r > 0$. Show that $\dsp a_n$ must equal $\dsp \frac{f^{(n)}(c)}{n!}$  for $n = 0, 1, 2,$ and $3.$
\end{prb}

How does the calculator on your phone approximate numbers like $\sqrt{5}$ or $\sin(\pi/7)$?  It uses Taylor series.  For example, if we want to approximate $\sqrt{5}$ accurate to 3 decimal places, we could compute the Taylor series for $f(x) = \sqrt{x}$ expanded about $c=4$ because 4 is the integer closest to 5 for which we actually know the value of $\sqrt{4}$.   Then we can use this polynomial to approximate $\sqrt{5}.$   But there's a problem.  While the Taylor series approximates the function, it's pretty hard to store an infinite polynomial in a phone.  So, how can we determine what degree Taylor polynomial we need to compute in order to assure a given accuracy?  We look at the remainder of the series.   For any positive integer, $N,$ $f$ can be rewritten as the sum of the Taylor polynomial of degree $N$ plus the remainder of the series by writing,
$$f(x) = \sum_{n=0}^N \frac{f^{(n)}(c)}{n!} (x-c)^n + \sum_{n=N+1}^\infty \frac{f^{(n)}(c)}{n!} (x-c)^n$$ or
$$f(x) = T_N(x) + R_N(x),$$
where
$$T_N(x) = \sum_{n=0}^N \frac{f^{(n)}(c)}{n!} (x-c)^n \mbox{ is the } N^{th} \mbox{ degree Taylor polynomial}$$
and
$$R_N(x)=\sum_{n=N+1}^\infty \frac{f^{(n)}(c)}{n!}(x-c)^n \mbox{ is called the } N^{th} \mbox{ degree Taylor remainder}.$$
Since $f(x) = T_N(x) + R_N(x),$ for all $x$ in the interval of convergence, the error for any $x$ in the interval of
convergence is just $| f(x) - T_N(x)| = |R_N(x)|.$    If we could just approximate the size of the remainder term, we would know how accurate the $N^{th}$ degree Taylor polynomial was.  The next theorem says that the remainder, even though it is an infinite sum, can be written completely in terms of the $(N+1)^{st}$ derivative, so if we can find a bound for the $(N+1)^{st}$ derivative then we can get a bound on the error of the Taylor series.

\begin{thm} \label{taylorerror}
\textbf{Taylor Series Error Theorem.} Let $f$ be a function such that $f^{(i)}$ is continuous for each $i=0, 1, 2, \cdots, N$ on $[a, b]$ and $f^{(N+1)}(x)$ exists for all $x$ in $(a, b)$.  Then $\dsp R_N(x)=\frac{f^{(N+1)}(k)}{(N+1)!}(x-c)^{N+1}$ for some $k$ between $x$ and $c$.
\end{thm}

\begin{prb}
Let $f(x) = \sqrt{x}$ and compute $T_3(x)$, the third degree Taylor series for $f$ expanded at 4.  Estimate the error $|f(5)-T_3(5)|$ by using Theorem \ref{taylorerror} to find the maximum of $|R_3(5)|$.  Now use your calculator to compute $|f(5)-T_3(5)|$ and see how this compares to your error estimate.
\end{prb}

\begin{prb}
Let $f(x) = \sin(x)$ and compute $T_3(x)$, the third degree McLaurin series $f.$  Estimate the error $|f(\frac{\pi}{7})-T_3(\frac{\pi}{7})|$ by using Theorem \ref{taylorerror} to find the maximum of $|R_3(\frac{\pi}{7})|$.  Now use your calculator to compute $|f(\frac{\pi}{7})-T_3(\frac{\pi}{7})|$ and see how this compares to your error estimate.
\end{prb}

\section{Practice} \label{chap6probs}

We will not present the problems from this section, although you are welcome to ask about them in class.

\vskip .1in
\noindent
\textbf{Sequences}

\begin{enumerate}
\item Write the first five terms of the sequence in simplest form.
\begin{enumerate}
\item $\dsp \{5 - {{2n} \over 3} \}_{n=2}^{\infty}$
\item $\dsp \{\cos(\frac{n \pi}{4}) \}_{n=0}^{\infty}$
\item $\dsp \{{{3^n + 2} \over {2^n}} \}_{n=1}^{\infty}$
\end{enumerate}

\item Write a formula for each sequence.  I.e. $x_n = \_\_\_$ for $n = 1,2,3,\dots$.
\begin{enumerate}
\item $\dsp {3 \over 5}, {6 \over {25}}, {9 \over {125}}, {{12} \over {625}},  \cdots$
\item $\dsp -1, 1, -1, 1, -1, \cdots$
\item $\dsp {1 \over 2}, -{2 \over 3}, {3 \over 4}, -{4 \over 5}, {5 \over 6}, \cdots$
\end{enumerate}

\item What is the least upper bound and the greatest lower bound of each given sequence?  Prove it.
\begin{enumerate}
\item $\dsp \{ \frac{2n-2}{n} \}_{n=1}^\infty$
\item $\dsp \{ \frac{2n+5}{3n+1} \}_{n=2}^\infty$
\end{enumerate}

\item Is the sequence monotonic or not?  Prove it.
\begin{enumerate}
\item $\dsp \{ \frac{n}{n^2+1} \}_{n=1}^\infty$
\item $\dsp \{\sin(\frac{2n \pi}{3}) \}_{n=1}^{\infty}$
\item $\dsp \{\frac{n-2}{n} \}_{n=1}^{\infty}$
\end{enumerate}

\item Use limits to find whether or not the sequence converges.
\begin{enumerate}
\item $\dsp x_n = {{n+2} \over {n^2+3n+2}}$
\item $\dsp x_n = n \sin \Big({1 \over n} \Big)$
\item $\dsp x_n = \Big({{1} \over {n}} \Big)^n$
\item $\dsp x_n = \Big(1 - {1 \over n} \Big)^n$  Remember, the natural log function $\ln$ is your friend.
\end{enumerate}

\item Use the definition of limit to find the smallest $N$ so that
$\dsp \Big|{3 \over n}-0 \Big| < .001$ for $n \ge N$.

\item  Assume $\epsilon$ is a small positive number and use the definition of limit to
find the smallest $N$ in terms of $\epsilon$ so that $\dsp \Big|{{n+3} \over {2n}}- {1 \over 2}
\Big| < \epsilon$ for $n \ge N$.
\end{enumerate}

\noindent
\textbf{Series}\\

\begin{enumerate}

\item List, in simplest form, the first three partial sums for the series and write a
formula for the $N^{th}$
partial sum.
\begin{enumerate}
\item $\dsp 1 + \frac{1}{3} + \frac{1}{9} + \frac{1}{27} + \frac{1}{81} + \dots$
\item $\dsp 3 + \frac{3}{5} + \frac{3}{25} + \frac{3}{125} + \dots$
\end{enumerate}

\item What does the geometric series converge to?
\begin{enumerate}
\item $\dsp \sum_{n=1}^{\infty} {3 \over {4^n}}$
\item $\dsp \sum_{n=1}^{\infty} \Big({9 \over {10}} \Big)^n$
\end{enumerate}

\item Use the fact that the Harmonic Series diverges to argue that $\dsp \sum_{n=3}^\infty \frac{5}{3n}$ diverges.

\item Apply the integral test to determine if the series converges.
\begin{enumerate}
\item $\dsp \sum_{n=2}^{\infty} {{2n} \over {(n^2-3)^2}}$
\item $\dsp \sum_{n=2}^{\infty} {{\ln(n)} \over {n}}$
\end{enumerate}

\item Determine if these series converge via one of the $n^{th}$ term test, integral
test, p-series test, comparison test, or limit comparison test.

\begin{enumerate}
\item $\dsp \sum_{n=2}^{\infty} {2 \over {1-n^2}}$
\item $\dsp \sum_{n=1}^{\infty} {1 \over {4n^2+2n}}$
\item $\dsp \sum_{n=0}^{\infty} \Big({{n+1} \over {n+3}}\Big)$
\item $\dsp \sum_{n=1}^{\infty} \Big(\sin \Big({{\pi} \over 4} \Big) \Big)^n$
\item $\dsp \sum_{n=0}^{\infty} {{n+2} \over {3n+4}}$
\item $\dsp \sum_{n=0}^{\infty} {1 \over {3n}} + {1 \over {2^n}}$
\item $\dsp \sum_{n=1}^{\infty} {3 \over {2^n}} + {4 \over {3^n}}$
\item $\dsp \sum_{n=2}^{\infty} 4+{1 \over {n^2}}$
\item $\dsp \sum_{n=1}^{\infty} {1 \over n}- {1  \over {n+2}}$
\item $\dsp \sum_{n=0}^{\infty} {e^{-5n}}$
\item $\dsp \sum_{n=1}^{\infty} {{3n} \over {n^4 + 16}}$
\item $\dsp \sum_{n=3}^{\infty} \sqrt{{1} \over {3n+4}}$
\item $\dsp \sum_{n=2}^{\infty} {{2} \over {n^2-1}}$
\item $\dsp \sum_{n=1}^{\infty} {1 \over {n^n}}$
\item $\dsp \sum_{n=1}^{\infty} {2 \over {\sqrt{n^2+2n}}}$
\item $\dsp \sum_{n=1}^{\infty} {{2} \over {n3^n}}$
\item $\dsp \sum_{n=2}^{\infty} {1 \over {n^2+5n+6}}$
\item $\dsp \sum_{n=2}^{\infty} {1 \over {3^n-1}}$
\item $\dsp \sum_{n=2}^{\infty} {{\ln(n)} \over {n^3}}$
\item $\dsp \sum_{n=2}^{\infty} {{\ln(n^2)} \over {n^2}}$
\end{enumerate}

\item Determine whether the alternating series is convergent or divergent.
\begin{enumerate}
\item $\dsp \sum_{n=2}^{\infty} (-1)^n{{2} \over {n^2-1}}$
\item $\dsp \sum_{n=0}^{\infty} (-1)^n {{3n} \over {4^n}}$
\item $\dsp \sum_{n=1}^{\infty} (-1)^n{{4^n} \over {n^2}}$
\item $\dsp \sum_{n=3}^{\infty} {{(-5)^{n-2}} \over {3^{n+1}}}$
\item $\dsp \sum_{n=2}^{\infty} (-1)^n \Big( {1 \over {\log(n)}} \Big)$
\item $\dsp \sum_{n=0}^{\infty} (-1)^n \Big({{3} \over {n+3}}\Big)$
\item $\dsp \sum_{n=0}^{\infty} \cos(n \pi)$
\item $\dsp \sum_{n=1}^{\infty} (-1)^n{{3^n} \over {3n}}$
\item $\dsp \sum_{n=1}^{\infty} (-1)^n \Big({1\over {\sqrt n}} \Big)$
\item $\dsp \sum_{n=1}^{\infty} (-1)^n{3 \over {n^{10}}}$
\item $\dsp \sum_{n=2}^{\infty} (-1)^n\Big(4 + {{1} \over {n^2}} \Big)$
\item $\dsp\sum_{n=1}^{\infty} {{\sec(n \pi)} \over {n}}$
\end{enumerate}

\item Determine if it converges absolutely,converges conditionally, or diverges.
\begin{enumerate}
\item $\dsp \sum_{n=2}^{\infty} {{(-4)^n} \over {3^{n+1}}}$
\item $\dsp \sum_{n=2}^{\infty} (-1)^n{2 \over {n-1}}$
\item $\dsp \sum_{n=2}^{\infty} (-1)^n{{1} \over {n(\log_4 n)^2}}$
\item $\dsp \sum_{n=2}^{\infty} \Big({{-5} \over {8}}\Big)^n$
\item $\dsp \sum_{n=1}^{\infty} (-1)^n {{n^n} \over{n!}}$
\item $\dsp \sum_{n=2}^{\infty} (-1)^n {{n!} \over{n^n}}$
\item $\dsp \sum_{n=1}^{\infty} \cos(n \pi)$
\item $\dsp \sum_{n=2}^{\infty} (-1)^n{{3n} \over {n^2-1}}$
\item $\dsp \sum_{n=1}^{\infty} {{3n} \over {(-3)^n}}$
\item $\dsp\sum_{n=1}^{\infty} (-1)^n {{\sqrt{n}} \over {n+4}}$
\item $\dsp \sum_{n=1}^{\infty} {{n^2} \over {(-2)^n}}$
\item $\dsp \sum_{n=1}^{\infty} {{n^e} \over {(-e)^n}}$
\end{enumerate}

\item For each of the following power series, determine its interval of convergence.
\begin{enumerate}
\item $\dsp \sum_{n=1}^{\infty} {{x^n} \over {n+3}}$
\item $\dsp \sum_{n=0}^{\infty} (-x)^n{1 \over {4^n}}$
\item $\dsp \sum_{n=1}^{\infty} (-1)^n{{3^n} \over {n^2}}x^n$
\item $\dsp \sum_{n=0}^{\infty} {{(x-5)^n} \over {3^{n+1}}}$
\item $\dsp \sum_{n=1}^{\infty} {{x^n} \over {\sqrt{n^2+1}}}$
\item $\dsp \sum_{n=1}^{\infty} {{x^n} \over {\log(n+1)}}$
\item $\dsp \sum_{n=0}^{\infty} {{(x+3)^n}\over{n^n}}$
\item $\dsp \sum_{n=0}^{\infty} {{\cos(n \pi)x^n}\over {n!}}$
\item $\dsp \sum_{n=1}^{\infty} {{(x+4)^n} \over{(n+4)4^n}}$
\item $\dsp \sum_{n=1}^{\infty} \Big(1+ {1 \over n} \Big)^{2n} x^n$
\item $\dsp \sum_{n=0}^{\infty} n^{2n}x^n$
\item $\dsp \sum_{n=1}^{\infty} {{(x-2)^n} \over {2n+2}}$
\end{enumerate}

\item For each of the following functions, find its Taylor or MacLaurin series
and state the radius of convergence.
\begin{enumerate}
\item $\dsp f(x) = -{1 \over {x^2}}$; $c=1$
\item $\dsp f(x) = \sin(2x)$; $c = 0$
\item $\dsp f(x) = 3^x$; $c = 1$
\item $\dsp f(x) = e^x$; $c = 2$
\item $\dsp f(x) = \cos(x)$; $c= {{\pi} \over 2}$
\item $\dsp f(x) = \ln(1+x)$; $c = 0$
\end{enumerate}

\item Use multiplication, substitution, differentiation, and integration whenever possible to find a series representation of each of the following functions.
\begin{enumerate}
\item $\dsp f(x) =x^2 \cos(x)$ Multiply $x^2$ times the Taylor series for cosine.
\item $\dsp f(x) ={{\sin(x^2)} \over x}$
\item $\dsp f(x) = 5xe^{3x}$
\item $\dsp f(x) = {2 \over {(1-x)^2}}$
\end{enumerate}

\end{enumerate}

\vskip .5in
\noindent
\textbf{Chapter 6 Solutions}\\ \\

\textbf{Sequences}

\begin{enumerate}
\item Write the first five terms of the sequence in simplest form.
\begin{enumerate}
\item $\dsp \frac{11}{3}, \frac{9}{3}, \frac{7}{3}, \frac{5}{3}, \dots$
\item $\dsp 1, \frac{\sqrt{2}}{2}, 0, \frac{-\sqrt{2}}{2}, -1, \dots$
\item $\dsp \frac{5}{2}, \frac{11}{4}, \frac{29}{8}, \dots$
\end{enumerate}

\item Write a formula for each sequence.  I.e. $x_n = \_\_\_$ for $n = 1,2,3,\dots$.
\begin{enumerate}
\item $\dsp x_n = \frac{3n}{5^n}  \cdots$
\item $\dsp (-1)^n \cdots$
\item $\dsp (-1)^{n+1} \frac{n}{n+1} \cdots$
\end{enumerate}

\item What is the least upper bound and the greatest lower bound?  Prove it.
\begin{enumerate}
\item GLB = 0, LUB = 2
\item GLB = 2/3 , LUB = 9/7
\end{enumerate}

\item Is the sequence monotonic or not?  Prove it.
\begin{enumerate}
\item decreasing
\item not monotonic
\item increasing
\end{enumerate}

\item Use limits to find whether or not the sequence converges.
\begin{enumerate}
\item converges to 0
\item converges to 1
\item converges to 0
\item converges to $e^{-1}$
\end{enumerate}

\item N = 3001

\item  N = the first positive integer larger than $\dsp \frac{3}{2\epsilon}$
\end{enumerate}

\newpage
\noindent
\textbf{Series}\\

\begin{enumerate}

\item List, in simplest form, the first three partial sums for the series and write a
formula for the $N^{th}$
partial sum.
\begin{enumerate}
\item $\dsp 1 , \frac{4}{3} , \frac{13}{9} , \frac{40}{27} , \dots,  S_n= \frac{1-3^n}{1-3}\frac{1}{3^{n-1}} = \frac{1 - (\frac{1}{3})^n}{1-\frac{1}{3}}$
\item $\dsp 3 , 3\frac{6}{5} , 3\frac{31}{25} , \dots, S_n = 3 \frac{1-5^n}{(1-5)5^{n-1}}= 3\frac{1 - (\frac{1}{5})^n}{1-\frac{1}{5}}$
\end{enumerate}

\item What does the geometric series converge to?
\begin{enumerate}
\item 1
\item 9
\end{enumerate}

\item $\dsp \sum_{n=3}^\infty \frac{5}{3n} = \frac{5}{3} \sum_{n=3}^\infty \frac{1}{n} > \sum_{n=3}^\infty \frac{1}{n}$ which diverges

\item Apply the integral test to determine if the series converges.
\begin{enumerate}
\item converges
\item diverges
\end{enumerate}

\item Determine if these series converge via one of the $n^{th}$ term test, integral
test, p-series test, comparison test, or limit comparison test.

\begin{enumerate}
\item converges by limit comparison with $\dsp \frac{1}{n^2}$
\item converges by limit comparison with $\dsp \frac{1}{n^2}$
\item diverges, $n^{th}$ term test
\item converges, geometric series
\item diverges, $n^{th}$ term test
\item separate the two series, one harmonic (diverges), one geometric (converges)
\item separate the two series, both geometric (converges)
\item separate the two series, one constant (diverges by $n^th$ term test), one converges by integral or p-series test
\item limit comparison with $\dsp \frac{1}{n^2}$
\item converges, geometric series
\item limit comparison with $\dsp \frac{1}{n^3}$
\item limit comparison with $\dsp \frac{1}{\sqrt{n}}$
\item converges by limit comparison with $\dsp \frac{1}{n^2}$
\item converges by comparison to $\dsp \frac{1}{n^2}$
\item converges by limit comparison with $\dsp \frac{1}{n}$
\item converges by comparison to $\dsp \frac{1}{n^2}$
\item converges by comparison to $\dsp \frac{1}{n^2}$
\item converges by limit comparison with $\dsp \frac{1}{3^n}$
\item converges by comparison to $\dsp \frac{1}{n^2}$
\item converges, integral test
\end{enumerate}

\item Determine whether the alternating series is convergent or divergent.
\begin{enumerate}
\item converges
\item converges
\item diverges, terms don't tend to zero
\item diverges, rewrite it so that you can see that $a_n = (5/3)^n$
\item converges
\item converges
\item diverges
\item diverges
\item converges
\item converges
\item diverges, terms don't tend to zero
\item converges, rewrite it without a trig function
\end{enumerate}

\item Conditionally Convergent, Absolutely Convergent, or Divergent?
\begin{enumerate}
\item D
\item CC
\item AC, to get absolutely convergent, convert $\log_4$ to $\ln$ and use integral test
\item AC
\item D
\item AC, ratio test + that clever natural log trick for limits
\item D
\item CC
\item AC
\item CC
\item this is a neat problem
\item this is a really neat problem
\end{enumerate}

\item Intervals of convergence.
\begin{enumerate}
\item $(-1,1)$
\item $(-4,4)$
\item $(-1/3,1/3)$
\item $(2,8)$
\item $(-1,1)$
\item $(-1,1)$
\item no interval, diverges for all $x \neq -3$
\item no interval, diverges for all $x \neq 0$
\item $(-8,0)$
\item $(-1,1)$
\item this is a neat problem
\item $(1,3)$
\end{enumerate}

\item Taylor Series (just the first few terms)
\begin{enumerate}
\item $\dsp -1+2\, \left( x-1 \right) -3\, \left( x-1 \right) ^{2}+4\, \left( x-1 \right) ^{3}-5\, \left( x-1 \right) ^{4}+6\, \left( x-1 \right) ^{5}$
\item $\dsp 2x -8x^3/3! +32 x^5/5! - 128 x^7/7! \dots$
\item $\dsp 3+3\,\ln  \left( 3 \right)  \left( x-1 \right) + \frac{3}{2} \left( \ln  \left( 3 \right)  \right) ^{2} \left( x-1 \right) ^{2}+ \frac{1}{2} \left( \ln  \left( 3 \right)  \right) ^{3} \left( x-1 \right) ^{3}+\frac{1}{8}  \left( \ln  \left( 3 \right)  \right) ^{4} \left( x-1 \right) ^{4}+$\\  $\dsp \frac{1}{40} \left( \ln  \left( 3 \right)  \right) ^{5} \left( x-1 \right) ^{5}$
\item $\dsp {e^{2}}+{e^{2}} \left( x-2 \right) + \frac{1}{2}{e^{2}} \left( x-2 \right) ^ {2}+\frac{1}{6}{e^{2}} \left( x-2 \right) ^{3}+\frac{1}{24}{e^{2}} \left( x-2  \right) ^{4}+{\frac {1}{120}}\,{e^{2}} \left( x-2 \right) ^{5}$
\item $\dsp -x+\frac{\pi}{2} +{\frac {1}{6}} \left( x-\frac{\pi}{2} \right) ^{3}-{\frac {1}{120}} \left( x-\frac{\pi}{2} \right) ^{5}$
\item $\dsp x-{\frac {1}{2}}{x}^{2}+{\frac {1}{3}}{x}^{3}-{\frac {1}{4}}{x}^{4}+{ \frac {1}{5}}{x}^{5}$
\end{enumerate}

\item Use multiplication, substitution, differentiation, and integration whenever possible to find a series representation of each of the following functions.
\begin{enumerate}
\item $\dsp \sum_{n=0}^\infty (-1)^n \frac{x^{2(n+1)}}{(2n)!}$
\item $\dsp \sum_{n=0}^\infty (-1)^n \frac{x^{4n+1}}{(2n+1)!}$
\item $\dsp 5\sum_{n=0}^\infty  \frac{3^n}{n!}x^{n+1}$
\item $\dsp 2 \sum_{n=0}^\infty (n-1)r^n$  You know the series for $\dsp \frac{1}{1-r}.$
\end{enumerate}

\end{enumerate}


\chapter{Conic Sections, Parametric Equations, and Polar Coordinates}

``Learning is not compulsory... neither is survival.'' - W. Edwards Deming

\section{Conic Sections}

In your mathematical life, you have no doubt seen circles, ellipses, parabolas, and possibly even hyperbolas.   Here we make precise the geometric and algebraic definitions of these objects.
\begin{annotation}
\endnote{My goal for the material on conic sections is to make them aware of and provide a taste of the link between the algebraic definitions and the geometric definitions.}
\end{annotation}

\begin{dfn}
Consider two identical, infinitely tall, right circular cones placed vertex to vertex so that they share the same axis of symmetry (two infinite ice cream cones placed point-to-point).  A \textbf{conic section} is the intersection of this three dimensional surface with any plane that does not pass through the vertex where the two cones meet.
\end{dfn}

These intersections are called circles (when the plane is perpendicular to the axis of symmetry), parabolas (when the plane is parallel to one side of one cone), hyperbolas (when the plane is parallel to the axis of symmetry), and ellipses (when the plane does not meet any of the three previous criteria). This is a geometric interpretation of the conic sections, but each conic sections may also be represented by a quadratic equation in two variables.

\begin{prb}
\textbf{Algebraically.} Graph the set of all points $(x,y)$  in the plane satisfying $Ax^2+Bxy+Cy^2+Dx+Ey+F=0$ where:
\begin{enumerate}
\item $A=B=C=0$ ,$D = 1$, $E = 2$ and $F = 3$.
\item $A=1/16$, $C=1/9$, $B=D=E=0$ and $F=-1$.
\item $A=B=C=D=E=0$ and $F = 1$.
\item $A=B=C=D=E=F=0$.
\end{enumerate}
\end{prb}

Every conic section may be obtained by making appropriate choices for $A$, $B$, $C$, $D$, $E$, and $F$. Now let's return to the definition you might have seen in high school for a circle.

\begin{dfn}
Given a plane, a point $(h,k)$ in the plane, and a positive number $r,$ the {\bf{circle}} with \textbf{center} $(h,k)$ and \textbf{radius} $r$ is the set of all points in the plane at a distance $r$ from $(h,k)$.
\end{dfn}

\begin{prb}
Determine values for $A, B, \dots,E$ and $F$ so that $Ax^2+Bxy+Cy^2+Dx+Ey+F=0$ represents a circle. A single point. A parabola.
\end{prb}

\begin{prb}
Use the method of completing the square to rewrite the equation $4x^2+4y^2+6x-8y-1=0$ in the form $(x-h)^2 + (y-k)^2 = r^2$ for some numbers $h$, $k$, and $r$.  Generalize your work to rewrite the equation $x^2+y^2+Dx+Ey+F=0$ in the same form.
\end{prb}

\begin{dfn}
The \textbf{distance between a point $P$ and a line $L$} (denoted by d(P,L)) is defined to be the length of the line segment perpendicular to $L$ that has one end on $L$ and the other end on $P$.
\end{dfn}

\begin{dfn}
Given a point $P$ and a line $L$ in the same plane, the \textbf{parabola} defined by $P$ and $L$ is the set of all points in
the plane $Q$ so that $d(P,Q) = d(Q,L).$  The point $P$ is called the \textbf{focus} of the parabola and the line $L$ is called the {\bf{directrix}} of the parabola.
\end{dfn}

%TED We need a good physics problem here like showing that a paraboloid reflects
% light rays emanating at the focus in parallel directions.

\begin{prb}
Parabolas.
\begin{enumerate}
\item Let $p$ be a positive number and graph the parabola that has focus $(0, p)$ and directrix $y=-p$
\item Let $(x, y)$ be an arbitrary point of the parabola with focus $(0, p)$ and directrix $y=-p$. Show that $(x,y)$ satisfies $x^2=4py$.
\item Let $(x, y)$ be an arbitrary point of the parabola with focus $(p, 0)$ and directrix $x=-p$. Show that $(x,y)$ satisfies $y^2=4px$.
\item Graph the parabola, focus, and directrix for the parabola $y=-(x-2)^2+3$.
\end{enumerate}
\end{prb}


\begin{prb}
Find the vertex, focus, and directrix of each of the parabolas defined by the following equations.
\begin{enumerate}
\item $y=3x^2-12x+27$
\item $y=x^2+cx+d$ (Assume that $c$ and $d$ are numbers.)
\end{enumerate}
\end{prb}

\begin{dfn}
Given two points, $F_1$ and $F_2$ in the plane and a positive number $d,$ the {\bf{ellipse}} determined by $F_1, F_2$ and $d$ is the set of all points $Q$ in the plane so that $d(F_1,Q) + d(F_2,Q) = d.$  $F_1$ and $F_2$ are called the {\bf{foci}} of the ellipse. The \textbf{major axis} of the ellipse is the line passing through the foci.  The \textbf{vertices} of the ellipse are the points where the ellipse intersects the major axis.  The \textbf{minor axis} is the line perpendicular to the major axis and passing through the midpoint of the foci.
\end{dfn}

\begin{prb}
Let $F_1=(c,0), F_2=(-c,0),$ and $d=2a.$  Show that if $(x,y)$ is a point of the ellipse determined by $F_1, F_2$ and $d$, then $(x,y)$ satisfies the equation $\dsp \frac{x^2}{a^2}+\frac{y^2}{b^2}=1$ where $c^2 = a^2 - b^2.$ What are the restrictions on $a, b,$ and  $c$?
\end{prb}

\begin{prb}
Ellipses.
\begin{enumerate}
\item Graph the ellipses defined by $\dsp \frac{x^2}{25}+\frac{y^2}{9}=1$ and $\dsp \frac{(x+1)^2}{25}+\frac{(y-2)^2}{9}=1$.  List the foci for each, the vertices for each, and the major axis for each. Compare the two graphs and find the similarities and differences between the two graphs.
\item For the ellipses defined by $\dsp \frac{x^2}{a^2}+\frac{y^2}{b^2}=1$ and $\dsp \frac{(x-h)^2}{a^2}+\frac{(y-k)^2}{b^2}=1$, list the foci for each, the vertices for each, and the major axis for each.
\end{enumerate}
\end{prb}

\begin{dfn}
Given two points, $F_1$ and $F_2$ in the plane and a positive number $d,$ the \textbf{hyperbola} determined by $F_1, F_2$ and $d$ is the set of all points $Q$ in the plane so that $| d(F_1,Q) - d(F_2,Q) | = d.$  $F_1$ and $F_2$ are called the {\bf{foci}} of the hyperbola.
\end{dfn}

\begin{prb}
Let $F_1=(c,0), F_2=(-c,0),$ and $d=2a.$  Show that the equation of the hyperbola determined by $F_1, F_2$ and $d$ is given by $\dsp \frac{x^2}{a^2}-\frac{y^2}{b^2}=1$ where $c^2 = a^2 +  b^2.$ What are the restrictions on $a, b,$ and $c?$
\end{prb}

\begin{prb}
Graph two hyperbolas (just pick choices for $a$ and $b$). Solve the equation $$\dsp \frac{x^2}{a^2}-\frac{y^2}{b^2}=1$$ for $y$. Show that as $x \to \infty$ the hyperbola approaches the lines $L(x) = \pm \frac{b}{a} x;$  i.e., show that the asymptotic behavior of a hyperbola is linear.
\end{prb}

\begin{prb}
Hyperbolas.
\begin{enumerate}
\item Graph $\dsp \frac{x^2}{25}-\frac{y^2}{9}=1$ and $\dsp \frac{(x+1)^2}{25}-\frac{(y-2)^2}{9}=1$. Compare the two graphs and find the similarities and differences between the two hyperbolas. Find the foci and the equations of the asymptotes of these hyperbolas.
\item Find the vertices and the asymptotes of $\dsp \frac{x^2}{a^2}-\frac{y^2}{b^2}=1$ and $\dsp \frac{(x-h)^2}{a^2}-\frac{(y-k)^2}{b^2}=1$.
\end{enumerate}
\end{prb}

\begin{prb}
Use the method of completing the square to identify whether the set of points represents a circle, a parabola, an ellipse, or a hyperbola.  Sketch the conic section and list the relevant data for that conic section from this list: center, radius, vertex (vertices), focus (foci), directrix, and asymptotes.
\begin{enumerate}
\item $x^2-4y^2-6x-32y-59=0$
\item $9x^2+4y^2+18x-16y=11$
\item $x^2+y^2+10 = 6y -4 +4x$
\item $x^2+y^2 +6x-14y=6$
\item $3x^2-2x+y=5$
\end{enumerate}
\end{prb}

%TED I have not worked on this section at all...

%\indent For a given $xy$-coordinate system, suppose that a new $uv$-coordinate has the same origin $O$ as the $xy$-coordinate system and the angle from the positive $x$-axis, when measured in the counter-clockwise direction, is $\phi$. Let $(x, y)$ and $(u, v)$ be the same point $P$ on these two coordinate systems and let $r$ be the distance between $P$ and $O$, and let $\theta$ be the angle, measured in the counter-clockwise direction, from the positive $x$-axis to the ray $\overrightarrow{OP}$.\\

%From trigonometry, $\dsp \sin (\theta) ={y \over r}$, $\dsp \cos (\theta) = {x \over r}$, $\dsp \sin(\theta - \phi) = {v \over r}$, and $\dsp \cos (\theta - \phi) = {u \over r}$. Thus $\dsp u = r \cos(\theta - \phi) = r \cos (\theta) \cos (\phi) + r \sin (\theta) \sin (\phi) = x \cos (\phi) - y \sin (\phi)$. Similarly, $v = -x \sin (\phi) + y \cos (\phi)$. Solving these equations for $x$ and $y$ yields, $x = u \cos (\phi) - v \sin (\phi)$ and  $y = u \sin (\phi) + v \cos (\phi)$.\\

%\begin{thm}
%Show that if $B \ne 0$, the equation $Ax^2+Bxy+Cy^2+Dx+Ey+F=0$ can be transformed into the equation $A^*u^2 + C^*v^2+D^*u+E^*v+F^*=0$ such that $A^*$ and $C^*$ are not both zero, by rotation of axes through an angle $\phi$ for which $\dsp \cot (2 \phi) = {{A - C} \over B}$. (Hint: Use $x = u \cos (\phi) - v \sin (\phi)$; $y= u \sin (\phi) + v \cos (\phi)$ to re-write the original equation into $A^*u^2 + B^*uv+ C^*v^2+D^*u+E^*v+F^*=0$ and show that $A^* = A \cos^2 (\phi) + B \sin (\phi) \cos (\phi) + C \sin^2 (\phi)$; $B^* = 0$; $C^* = A \sin^2 (\phi) - B \sin (\phi) \cos (\phi) + C \cos^2 (\phi)$.)
%\end{thm}

%\begin{prb}
%For the equation $24xy -7y^2 + 36 = 0$, remove the $xy$ term by using a rotation of axes as described in the previous problem. Identify the conic section. Draw a sketch of the graph and show both sets of axes.
%\end{prb}


\section{Parametric Equations}

Parametric equations represent a mathematical way to model the position of an object in space at a given time, for example, a satellite traveling through space or a fish swimming in the ocean. In this section, we restrict our study to objects traveling in the plane. In the next chapter we will extend this notion to three-space.
\begin{annotation}
\endnote{I'll confess to getting carried away with the proofs in this section.  While I always \emph{knew} the formulas in the calculus books relating the derivatives of the functions represented in parametric form to the derivatives of the functions represented in rectangular coordinates, I'd never tried to rigorously make an argument justifying them.  When I did so, and saw how beautiful it was, I couldn't help but write it up.  Therefore the proofs in this section are hardly necessary, but if there is time and a student expresses interest, they certainly give one an understanding of the importance of precisely defining your notation.}
\end{annotation}

\begin{expl}
Define $c(t) = (t-1, 2t+4)$, sketch it, eliminate the parameter, sketch that and discuss the information that is lost in this process. Define two functions $x$ and $y$ on the interval $[0,2\pi]$ by $x(t)=r \cos (t)$ and $y(t)=r \sin (t)$. Now define $C$ on the same interval by $C(t) = \big(x(t), y(t)\big)$.  If we assume that $t$ represents time, then we may consider $C(t)$ to represent the position of some object at that time. $C(t) = \big(x(t), y(t)\big)$ is called a {\bf{parametric equation}} with {\bf{parameter}} $t$ and defines a function from $[0,2\pi]$ into $\re^2.$   $C$ is a way to represent the circle centered at $(0, 0)$ with radius $r$ since for any $t$ we have $\big(x(t)\big)^2 + \big(y(t)\big)^2 = r^2 \cos^2 (t) + r^2 \sin^2(t) = r^2$. In this mathematical model, each point on the circle models the position of the object at a specific time.
\end{expl}

\begin{dfn}
If each of $f$ and $g$ is a continuous function, then the curve in the plane defined by the range of $C,$ where $C(t) = \big(f(t), g(t)\big),$ is called a \textbf{parametric curve} or a \textbf{planar curve}.
\end{dfn}

\begin{prb}
Sketch each pair of planar curves by plotting points.  Compare.
\begin{enumerate}
\item $C(t) = \big(t, \sin(t)\big)$ and $D(t) = \big(\sin(t),t\big)$ with $0 \le t \le 2\pi$
\item $C(t) = \big(2\sin(t), 3\cos(t)\big)$ and $D(t) = \big(2\cos(t), 3\sin(t)\big)$ with   $0 \le t \le 2\pi$
\end{enumerate}
\end{prb}

\begin{prb}
Sketch the planar curve $C(t) = (3t+2, 9t^2 + 1).$  Sketch the tangent line to this curve when $t=1$.   Write the equation of this tangent line as a function $y=mx + b$ and as a planar curve, $D(t) = ( \_\_\_ \; , \_\_\_ ).$
\end{prb}

\begin{dfn}
If $C(t) = \big(f(t),g(t)\big)$ is a parametric equation where both of $f$ and $g$ are differentiable at the point $t$, then the \textbf{derivative} of $C$ is defined by $C'(t) = \big(f'(t),g'(t)\big).$
\end{dfn}

\begin{prb}
Let $C(t) = \big(2t+3, 4(2t+5)^2\big)$.
\begin{enumerate}
\item Sketch a graph of $C$ between the times $t=0$ and $t=2.$
\item Compute $C'(t)$ and $C'(1).$
\end{enumerate}
\end{prb}

\begin{prb}
Let $C(t) = (2t+3, 4(2t+5)^2).$
\begin{enumerate}
\item Let $x = 2t+3$ and $y= 4(2t+5)^2$ and eliminate the parameter, $t,$ to write this in terms of $x$ and $y$ only.
\item Compute the derivative of $y$ with respect to $x$ and compute $y'$ when $x=5.$
\item Compute the derivative of $y$ with respect to $t$ and compute $y'$ when $t=1.$
\item Compute the derivative of $x$ with respect to $t$ and compute $x'$ when $t=1.$
\item Observe that the derivative of $y$ with respect to $x$ at $x=5$ equals the quotient of the derivative of $y$ with respect to $t$ at $t=1$ and the derivative of $x$ with respect to $t$ when $t=1$.
\end{enumerate}
\end{prb}

The preceding problem illustrates an important theorem that relates the derivatives of two different, but related, functions:
\begin{enumerate}
 \item the derivative of the parametric curve $c=\big( f,g \big)$ and
 \item the derivative of the function $y$ that we obtain when we eliminate the parameter.
\end{enumerate}

\begin{thm}\label{paradiff}
\textbf{Parametric Derivative Theorem.} If $c = \big(f,g\big)$ and $y$ is the function obtained by eliminating the parameter, then $\dsp y'(f(t)) = \frac{g'(t)}{f'(t)}$. Since $f(t)$ represents our $x$ coordinate, many books would write this as $\dsp y'(x) =  \frac{g'(t)}{f'(t)}$.
\end{thm}

To justify this theorem, we offer three levels of ``proof.''  An intuitive argument, a slightly more precise argument, and an actual proof.

\textbf{Intuitively.} Suppose we have a parametric equation, $C(t) = \big(f(t),g(t)\big)$ where $f$ and $g$ are differentiable functions.  If we let $x = f(t)$ and $y=g(t)$ then $y'(x)$ is the rate of change of $y$ with respect to $x.$  The rate of change of $y$ with respect to time is $g'(t)$ and the rate of change of $x$ with respect to time is $f'(t).$ The rate of change of y over the rate of change of x is $\dsp  y'(x)  = \frac{g'(t)}{f'(t)}.$\\ \emph{q.e.d.}

\textbf{More precisely.} Suppose we have a parametric equation, $$C(t) = \big(f(t),g(t)\big)$$ where $f$ and $g$ are differentiable functions.   If we can eliminate the parameter $t$ (as we did in the last problem), then we can write a relationship that  yields a differentiable function, $y$ so that for every $t,$ $$y\big(f(t)\big) = g(t).$$ Then using implicit differentiation, we can see that $$y'\big(f(t)\big) \cdot f'(t) = g'(t).$$ Dividing through by $f'(t)$ we have the desired result, $$ y'\big(f(t)\big) = \frac{g'(t)}{f'(t)}.$$ If we put $x = f(t)$, then we have $$y'(x) = \frac{g'(t)}{f'(t)}.$$\emph{q.e.d.}\\

Before proceeding to give a true proof of this theorem, we need one more tool, which the next problem illustrates.\\

\textbf{Notation.}  The expression ${}^{-1}$ means {\it inverse} when applied to a function (see the left-hand side of the equation in part 2 of Problem \ref{p336}) and {\it reciprocal} when applied to a number (see the right-hand side of the equation in part 2 of Problem \ref{p336}). When a function has an inverse, we say it is invertible.

\begin{prb} \label{p336}
Let $f(x) = x^3 - 1$ and $(x,y) = (3, 26).$
\begin{enumerate}
\item Compute $f'$ and $f^{-1}.$ \item Show that $\dsp{\big(f^{-1}\big)'(26) = \big( f'(3) \big)^{-1}}.$
\end{enumerate}
\end{prb}

\begin{thm} \label{invdiff}
\textbf{Inverse Derivative Theorem.} Let $f$ be a function that is differentiable, and has an inverse that is differentiable.  If $y = f(x)$ then $\dsp{ (f^{-1})'(y) = \big(f'(x) \big)^{-1}}.$
\end{thm}

\begin{prb}
Prove Theorem \ref{invdiff} by computing the derivative of $\dsp{ f^{-1}\big(f(x)\big) = x.}$
\end{prb}

Now that we have Theorem \ref{invdiff} we can give a full proof of Theorem \ref{paradiff}.

\textbf{A Proof.} Suppose we have a parametric equation, $$C(t) = \big(f(t),g(t)\big)$$ where $f$ and $g$ are differentiable functions whose derivatives are continuous. Suppose further that there is a value $t^*$ where $f'(t^*) \neq 0.$  Now, since $f'(t^*) \neq 0$ it is either positive or negative.  Let's assume it's positive. (We also need to do the case where it is negative, but it turns out that the argument is almost identical, so we omit it.)  Since $f'$ is continuous and $f'(t^*)>0$, there must be some open interval $(a,b)$ containing $t^*$ where $f'(t) > 0$ for all $t$ in $(a,b).$ Now, since $f'$ is positive on this interval $f$ must be increasing on this interval and therefore $f$ must be invertible on this interval.  Suppose $(t,x)$ is a point of $f$ with $t$ in $(a,b).$ Thus, $t = f^{-1}(x)$ and $x$ is in $\big(f(a),f(b)\big).$  Now, let's define a new function, $y$ by $$y(x) = g\big(f^{-1}(x)\big).$$ Computing the derivative via the chain rule as we did in the last `proof' yields, $$y'(x) = g'\big(f^{-1}(x)\big) \cdot (f^{-1})'(x).$$ Applying Theorem \ref{invdiff}, we know that $$\dsp{ \big(f^{-1}\big)'(y) = \big( f'(x) \big)^{-1}}$$ so we have $$y'(x) = g'\big(f^{-1}(x)\big) \cdot \big(f^{-1}\big)'(x) = g'(t) \cdot \frac{1}{f'(t)} = \frac{g'(t)}{f'(t)}.$$\emph{q.e.d.}

\begin{prb}
Assume that $C, f,$ and $g$ are as above and compute $y''(x)$ to show that\\ $\dsp{ y''(x) = \frac{g''(t) f'(t) - g'(t) f''(t)}{\big(f'(t)\big)^3} }.$  What we really want to compute is the derivative of the function $y'$ and we know that $\dsp{ y'\big(f(t)\big) = \frac{g'(t)}{f'(t)}},$ so compute the derivative of both sides of this equation with respect to $t$.
\end{prb}

The next theorem summarizes our results about derivatives of parametric equations.

\begin{thm}
\textbf{Parametric Second Derivative Theorem.} Suppose we have a parametric equation, $C(t) = \big(f(t),g(t)\big)$ where $f$ and $g$ are differentiable functions.  Then, at any value of $t$ where $f'(t) \ne 0$ if we let $x=f(t)$ and $y=g(t)$ we have:
\begin{enumerate}
\item $\dsp{ y'(x) = \frac{g'(t)}{f'(t)}}$ and
\item $\dsp{ y''(x) = \frac{g''(t) f'(t) - g'(t) f''(t)}{\big(f'(t)\big)^3} }.$
\end{enumerate}
\end{thm}


This theorem is often written in textbooks using Leibniz notation as:
\begin{enumerate}
\item $\dsp{ \frac{dy}{\; dx} = \frac{ \frac{dy}{dt} }{ \frac{dx}{dt} } }$
\item $\dsp \frac{d^2y}{\; dx^2} = \frac{\frac{dy'}{dt}}{\frac{dx}{dt}}$ where $\dsp y' = \frac{dy}{dx}$ and $\dsp \frac{dx}{dt} \ne 0$.
\end{enumerate}


\begin{prb}
Let $C(t) = (t^2, t^3).$  Eliminate the parameter to write this parametric equation as a function $y$ in terms of $x.$  Compute each of $y'$ and $y''$ at $t=2$ both directly and by applying the theorem.
\end{prb}

\begin{thm} \label{arclength}
\textbf{Arc Length Theorem.} Suppose we have a parametric equation, $C(t) = \big(f(t),g(t)\big)$ where $f$ and $g$ are differentiable functions with continuous derivatives for all $a \leq t \leq b$.   If the curve $C$ is traversed exactly once as $t$ increases from $a$ to $b$, then the curve's \textbf{arc length} is given by $\dsp L = \int_a^b \sqrt{ \big(x'(t)\big)^2 + \big(y'(t)\big)^2 } \; dt$.
\end{thm}

This is one of the theorems in calculus with a very nice straightforward proof.  If I don't remember to show you the proof, ask!

\begin{prb}
Let $C(t) = (2t, t^2)$ from $t=0$ to $t=2$.  Use the arc length formula to determine the length of this curve.  Now, eliminate the parameter to write this parametric equation as a function $f$ in terms of $x.$  Use the formula we had for the arc length of a function, $L = \int_a^b \sqrt{1 + \big(f'(x)\big)^2} \ dx,$ to verify your answer.
\end{prb}

\begin{prb}
Let $C(t) = \big(5\cos(t)-\cos(5t), 5\sin(t)-\sin(5t)\big).$  Compute the arc length of the range of $C$ over the interval $[0,2\pi]$.
\end{prb}
% This is the length of the curve drawn by a circle of radius 1
% rolling around a circle of radius 5.

\section{Polar Coordinates}

Like the Cartesian (rectangular) coordinate system that you are familiar with, the polar coordinate system is a way to specify the location of points in the plane. Some curves have simpler equations in polar coordinates, while others have simpler equations in the Cartesian coordinate system.

\begin{expl}
Convert a few points from polar coordinates to rectangular coordinates and vice-versa -- perhaps $(3,4)$, $(0,1)$, $(5\pi/6, 2)$, and $(3\pi/2,-3)$.  Convert a few equations from polar to rectangular coordinates and vice-versa -- perhaps $5x-3y=9$, $3x^2=y$, $r = 2\sin(\theta)$, and $r=4$.
\end{expl}

\begin{dfn}
Given a point $P$ in the plane we may associate $P$ with an ordered pair $(r, \theta)$ where $r$ is the distance from $(0,0)$ to $P$ and $\theta$ is the angle between the $x-$axis and the ray $\overrightarrow{OP}$ as measured in the  counter-clockwise direction. The pair $(r, \theta)$ is called a \textbf{polar coordinate representation} for the point $P$.
\end{dfn}

\textbf{Warning.}  There are many polar coordinate representations for a point.  For example, if $(r, \theta)$ is a polar coordinate representation for some point in the plane, then by adding $2\pi$ to the angle, $(r, \theta + 2\pi)$ is a second polar coordinate representation for the same point.  We also allow $r \leq 0$ so that by negating $r$ and adding $\pi$ to the angle, $(-r, \theta + \pi)$ is a third polar coordinate representation for the same point.

\begin{prb}
Plot the following points in the polar coordinate system and find their coordinates in the Cartesian coordinate system.
\begin{enumerate}
\item $(1, \pi)$
\item $\dsp \Big(3, \frac{5 \pi}{4}\Big)$
\item $\dsp \Big(-3, \frac{\pi}{4}\Big)$
\item $\dsp \Big(-2, \frac{- \pi}{6}\Big)$
\end{enumerate}
\end{prb}

\begin{prb} \label{polar1}
Suppose $(r, \theta)$ is a point in the plane in polar coordinates and $(x,y)$ are the coordinates in the Cartesian coordinate system.  Write formulas for $x$ and $y$ in terms of $r$ and $\theta$.
\end{prb}

\begin{prb} \label{polar2}
Suppose $(x,y)$ is a point in the plane in Cartesian coordinates.  Write formulas (in terms of $x$ and $y$) for
\begin{enumerate}
\item the distance $r$ from the origin to $(x,y)$ and
\item the angle between the positive x-axis and the line containing the origin and $(x,y)$.
\end{enumerate}
\end{prb}

\begin{prb}
Graphing.
\begin{enumerate}
\item Graph the curve $y = \sin(x)$ for $x$ in $[0, 2 \pi]$ in the Cartesian coordinate system.
\item Graph the curve $r = \sin(\theta)$ for $\theta$ in $[0, 2\pi]$ in the polar coordinate system.
\end{enumerate}
\end{prb}

\begin{prb}
Each of the following equations is written in the Cartesian coordinate system. Use the relationships you found in Problem \ref{polar1} to write each in polar coordinates.
\begin{enumerate}
\item $2x-3y=6$
\item $x^2+y^2=9$
\item $25x^2+4y^2=100$
\item $x = 4y^2$ \item $xy=36$
\end{enumerate}
\end{prb}

\begin{prb}
Each of the following equations is written in the polar coordinate system.  Use the relationships you found in Problem \ref{polar2} to write each in Cartesian coordinates.
\begin{enumerate}
\item $r = 9 \cos (\theta)$
\item $r^2 = 2 \sin (\theta)$
\item $\dsp r = \frac{1}{3 \cos (\theta) - 2 \sin (\theta)}$
\item $r = 8$ \item $\dsp \theta = \frac{\pi}{4}$
\end{enumerate}
\end{prb}

\begin{prb}
Graph, in a polar coordinate system, the curve defined by each of the following equations.
\begin{enumerate}
\item $r=1+2 \cos (\theta)$
\item $r = 2 + 2 \sin (\theta)$
\item $r = 3 \sin (2 \theta)$
\item $r = 2 \cos (3 \theta)$
\item $r^2 = 4 \sin (2 \theta)$ (If you know how to use a graphing calculator to plot polar curves, try it.)
\end{enumerate}
\end{prb}

\begin{prb}
Find the points of intersection of the graphs of the given polar equations. (When I set them equal without graphing them, I always miss at least one point of intersection.)
\begin{enumerate}
\item $r = 4 - 4 \cos (\theta)$ and $r = 4 \cos (\theta)$
\item $r = 2 \cos (2 \theta)$ and $r = \sqrt{3}$
\item $r = \sin (\theta)$ and $r = \sqrt{3} \cos (\theta)$
\item $r = 4 + 2 \cos (\theta)$ and $r = 5$ \item $r = 1-2 \cos (\theta)$ and $r = 1$
\end{enumerate}
\end{prb}


\begin{expl}
Derive the formula for the arc length of a function starting with the distance formula.  Derive the formula for the arc length of a planar curve defined by a parametric equation.
\end{expl}

\begin{expl}
Compute the arc length (using the equation they will derive next) for the spiral $r=\theta$ from $\theta=0$ to $\theta = 2\pi$.
\end{expl}

\begin{prb}
If we display the independent variable $\theta$, then Cartesian and polar coordinates are related by $x(\theta) =  r(\theta) \cos (\theta)$ and $y(\theta) = r(\theta) \sin (\theta)$.  Use the formula for the arc length of a parametric curve $$\dsp L = \int_a^b \sqrt{\big(x'(t)\big)^2 + \big(y'(t)\big)^2} \; dt$$ to derive the formula for the arc length of a curve in polar coordinates $$\dsp L=\int_{\alpha}^{\beta} \sqrt{\big(r(\theta)\big)^2+\big(r'(\theta)\big)^2} \; d\theta.$$
\end{prb}

\begin{prb}
Use the arc length formula from the previous problem and the half-angle identity to find the length of the cardioid defined by $r = 3(1 + \cos (\theta))$.
\end{prb}

\begin{prb}
Recall that $\dsp {{dy} \over {\; dx}} = {{{dy} \over {d \theta}} \over {{dx} \over {d \theta}}}.$ Use this to find the tangent line to the curve in a polar plane $r = 2 + 2 \sin (\theta)$ at $\theta = \pi/4.$
\end{prb}

\begin{expl}
Compute the area of one of the four ``petals'' of $r=3\sin(2\theta).$  Generalize to demonstrate a method for finding the area trapped inside polar functions.

Let $r=f(\theta)$ be a continuous and non-negative function in the interval $[\alpha, \beta]$ and $A$ be the area of the region bounded by the graph of $f$ and the radial lines $\theta = \alpha$ and $\theta = \beta$.   To find $A$, partition the interval $[\alpha, \beta]$ into $n$ subintervals $[\theta_0,\theta_1]$, $[\theta_1, \theta_2]$, $\cdots$,  $[\theta_{n-1}, \theta_n]$ of equal lengths where $\alpha = \theta_0 < \theta_1 < \cdots < \theta_n = \beta$. Since the area of a circular sector of radius $r$ and angle $\theta$ is $\dsp {1 \over 2}r^2 \theta$, the area of each of our sectors is $\dsp{1 \over 2} (\theta_i - \theta_{i-1})^2$ so the total area is approximately the sum of these sectors.  Thus $A = \displaystyle{\lim_{n \to \infty} {1 \over 2} \sum_{i=1}^n \big[f(\theta_i)\big]^2 \big(\theta_{i}-\theta_{i-1}\big) ={1 \over 2} \int_{\alpha}^{\beta} [f(\theta)]^2 d \theta}$.
\end{expl}

\begin{prb}
Find the area of each region in a polar plane.
\begin{enumerate}
\item the region bounded by the graph of $r = -2 \cos (\theta)$
\item the region bounded by one loop of the graph of the equation $r = 4 \cos (2 \theta)$
\item the region bounded by the inner loop of the curve $r=2+3 \cos (\theta)$
\item the region bounded by the intersection of the two graphs $r=4 \sin (2\theta)$ and $r = 4 \cos (2 \theta)$
\item the region that lies inside the cardioid $r = 2 - 2 \cos (\theta)$ and outside the circle $r=1$
\end{enumerate}
\end{prb}

\section{Practice} \label{chap7probs}

We will not present the problems from this section, although you are welcome to ask about them in class.

\vskip .1in
\noindent
\textbf{Conics}

\begin{enumerate}
%\item Suppose the center of a circle is on the $y$-axis
%and the circle also passes through the point $(-a, b)$. If the
%slope of the tangent line at $(-a,b)$ to the circle is $-m$, find
%the center and radius of the circle.
%\item Let
%$P=(4,3)$ be a point in the exterior of the circle $x^2+y^2=4$.
%Suppose $A=(a_1, a_2)$ and $B=(b_1, b_2)$ are points the circle
%such that $\ell_{[A,P]}$ and $\ell_{[B,P]}$, where $\ell_{[A,P]}$
%is the line containing $A$ and $P$ and $\ell_{[B,P]}$ containing
%$B$ and $P$, are tangent lines to the circle through $P$ at the
%point of contact $A$ and $B$. Find the points $A$ and $B$.
\item Complete the square for the parabola, $y=4x^2-8x+7$. Sketch the parabola,
labeling the vertex, focus, and directrix. Compute the area of the region bounded by the
parabola, the x-axis, the vertical line through the focus, and $\dsp x={3 \over 2}$.
%\item  For the parabola,
%$y=4x^2-8x+7$, find the length of the arc from the vertex of the
%parabola to the point of intersection of the parabola and the
%horizontal line through the focus of the parabola which is to the
%right of the vertex.
\item  For the ellipse
$3x^2+2y^2-12x+4y+2=0$, find
\begin{enumerate}
\item its center, vertices, foci, and the extremities of the minor axis;
\item an equation of the tangent to the ellipse at $(0,-1)$.
\end{enumerate}
%\item Suppose $A$, $B$, and $C$ are positive
%numbers and $A > B$. For the ellipse $Ax^2+By^2=C$, find
%    \begin{enumerate}
%    \item its foci
%    \item the area bounded by the ellipse.
%    \end{enumerate}
\item For the ellipse $\dsp {{x^2} \over 9}+{{y^2} \over 4}=1$,
\begin{enumerate}
\item set up a definite integral to evaluate the length of the
arc of the ellipse from $(3,0)$ to $(-3,0)$;
\item find an approximation of the arc length in part 1.
\end{enumerate}
\item For the hyperbola
$9y^2-4x^2+16x+36y-16=0$,
    \begin{enumerate}
    \item find its center, vertices, foci, and equations of asymptotes;
    \item draw a sketch of the hyperbola and its asymptotes.
    \end{enumerate}
\item For each of the following equations,
identify whether the curve is a parabola, circle, ellipse, or
hyperbola by removing the $xy$ term from the equation by rotation
of the axes.
\begin{enumerate}
\item[a.] $x^2+2xy+y^2-4x+4y=0$ \item[b.] $x^2+2xy+y^2+x-y-1=0$
\item[c.] $24xy-7y^2-1=0$ \item[d.] $8x^2-12xy+17y^2+24x-68y-32=0$
\end{enumerate}
\end{enumerate}

\noindent
\textbf{Parametrics}

\begin{enumerate}
\item Find $\dsp{{dy} \over {\; dx}}$ and $\dsp {{d^2y} \over {dx^2}}$ for each of the
following parametric curves.
\begin{enumerate}
\item $x=a \cos(t)$, $y =b \sin(t)$
\item $x= \invsin(t)$, $y = \invtan(t)$
\item $x=t^ee^t$, $y = t \ln(t)$ Only compute dy/dx for this one.
\end{enumerate}


\item Let ${\cal C}$ be the curve defined by $x=
\sin(t)+t$, $y=\cos(t)-t$, $t$ is in $\dsp \Big[-{{\pi} \over
2},{{\pi} \over 2} \Big]$. Find an equation of the tangent to
${\cal C}$ at the point $(0,1)$.

\item Suppose we have an object traveling in the plane with position at time $t$ of
$x=6-t^2$ and $y=t^2+4$.  When is the object stopped?  When is it moving left? Right?
Rewrite the parametric equations as a function by eliminating time.

\item Find the length:
\begin{enumerate}
\item $x=e^t \cos(t)$ and $y=e^t \sin(t)$ from $t=1$ to $t=4$
\item $x=4 \cos^3(t)$ and $y=4 \sin^3(t)$ from $t=\pi/2$ to $t= \pi$
\end{enumerate}

\item Find the points of intersection of the
curves ${\cal C}_1$ with parametric equations $x=t+1$, $y=2t^2$
and ${\cal C}_2$ with parametric equations $x=2t+1$, $y=2t^2+7$. (Actuarial Exam)

\item  Let ${\cal C}$ be the curve defined by $x= 2t^2+t-1$, $y=t^2-3t+1$,
$-2<t<2$. Find an equation of the tangent to ${\cal C}$ at the
point $(0,5)$. (Actuarial Exam)
\end{enumerate}

\noindent
\textbf{Polars}

\begin{enumerate}

\item For each of the following problems, find a polar
equation of the graph having the given equation in the rectangular
coordinate system.
\begin{enumerate}
\item $4x-3y=7$
\item $xy=1$
\item $x^3+y^3=3xy$
\end{enumerate}

\item For each of the following problems, find
an equation in the rectangular coordinate system having the given
polar equation.
\begin{enumerate}
\item $\dsp r = {{2} \over {1- 3\cos(\theta)}}$
\item $r= 3 \sin(\theta)$
\item $r^2=6 \cos(2 \theta)$
\item $r=5 \sec(\theta)$
\end{enumerate}

\item For each of the following problems, find the points
of intersection of the graphs of the given pairs of polar curves.
\begin{enumerate}
\item $r=4 \cos(2 \theta)$; $r=2$
\item $r= 1 - \cos(\theta)$; $r= \cos(\theta)$
\end{enumerate}

\item For each of the following problems, find
the area of the region enclosed by the graph of the given polar
equation.
\begin{enumerate}
\item $r=4+4 \cos(\theta)$
\item $r=4- \sin(\theta)$
\item $r=2 \sin(3 \theta)$
\end{enumerate}

\item For each of the following problems, find
the area of the intersection of the region enclosed by the graphs
of the two given polar equations.
\begin{enumerate}
\item $r=4-2 \cos(\theta)$; $r=2$
\item $r=2 \sin(2\theta)$; $r= 2 \cos(\theta)$
\end{enumerate}

\item For each of the following problems, find
the area of the inner loop of the limacon.
\begin{enumerate}
\item $r=1-2 \cos(\theta)$
\item $r=3-4 \sin(\theta)$
\end{enumerate}

\end{enumerate}

\vskip .5in
\noindent
\textbf{Chapter 7 Solutions}\\ \\

\noindent
\textbf{Conics}

\begin{enumerate}
%\item Suppose the center of a circle is on the $y$-axis
%and the circle also passes through the point $(-a, b)$. If the
%slope of the tangent line at $(-a,b)$ to the circle is $-m$, find
%the center and radius of the circle.
%\item Let
%$P=(4,3)$ be a point in the exterior of the circle $x^2+y^2=4$.
%Suppose $A=(a_1, a_2)$ and $B=(b_1, b_2)$ are points the circle
%such that $\ell_{[A,P]}$ and $\ell_{[B,P]}$, where $\ell_{[A,P]}$
%is the line containing $A$ and $P$ and $\ell_{[B,P]}$ containing
%$B$ and $P$, are tangent lines to the circle through $P$ at the
%point of contact $A$ and $B$. Find the points $A$ and $B$.
\item $y = 4 (x-1)^2 + 3$, directrix $y=2\frac{15}{16}$, focus $ = (1, 3\frac{1}{16})$,
vertex $ = (1,3)$, and area $ = 1\frac{2}{3}$
%\item  For the parabola,
%$y=4x^2-8x+7$, find the length of the arc from the vertex of the
%parabola to the point of intersection of the parabola and the
%horizontal line through the focus of the parabola which is to the
%right of the vertex.
\item  center $= (2,-1)$, vertices $ = (2, -1 \pm \sqrt{6})$, foci $= (2, -1 \pm \sqrt{2})$,
minor extremities $ = (0,-1)$ and $(4,-1)$, equation of tangent at $(0,-1)$ is $x=0$
%\item Suppose $A$, $B$, and $C$ are positive
%numbers and $A > B$. For the ellipse $Ax^2+By^2=C$, find
%    \begin{enumerate}
%    \item its foci
%    \item the area bounded by the ellipse.
%    \end{enumerate}
\item $\dsp \frac{2}{3}\int_{-3}^3 \sqrt{1 + \frac{4}{9}(9-x^2)} \; dx = 5\invsin(\frac{2\sqrt{5}}{5}) \approx 7.54$
\item center $= (2,-2)$, vertices $=(2,0)$ and $(2,-4)$, asymptotes $y = \pm \frac{2}{3}(x-2)-2$,
foci $=(2, -2 \pm \sqrt{13})$
\item no solutions
\end{enumerate}

\noindent
\textbf{Parametrics}

\begin{enumerate}

\item
\begin{enumerate}
\item $\dsp \frac{dy}{dx} = \frac{b \cos(t) }{-a \sin(t)}$, $\dsp \frac{d^2y}{dx^2} = \frac{b}{a^2 \sin^3(t)}$
\item $\dsp \frac{dy}{dx} = \frac{\sqrt{1-t^2}}{1+t^2}$, $\dsp \frac{d^2y}{dx^2} = \frac{t(t^2-3)}{(1+t^2)^2}$
\item $\dsp \frac{dy}{dx} = \frac{\ln(t) + 1}{t^{e-1}e^{t+1} + t^e e^t}$
\end{enumerate}

\item $\dsp y = -\frac{1}{2} x + 1$

\item stopped at $t=0$, moving left at all positive times,
right at all negative times, $y=-x+10$ for $x \leq 6$

\item arc lengths
\begin{enumerate}
\item $\sqrt{2}(e^4-e^1)$
\item $-6$
\end{enumerate}

\item  $\dsp x = 1 \pm \sqrt{14/3}$

\item $\dsp y=\frac{5}{3} x + 5$

\end{enumerate}

\noindent
\textbf{Polars}

\begin{enumerate}
\item
\begin{enumerate}
\item $\dsp r = \frac{7}{4\cos(\theta) - 3\sin(\theta)}$
\item $\dsp r = \frac{1}{\pm \sqrt{\cos(\theta)\sin(\theta)}}$
\item $\dsp r = \frac{3\cos(\theta)\sin(\theta)}{\cos^3(\theta)+\sin^3(\theta)}$
\end{enumerate}

\item
\begin{enumerate}
\item $\dsp y = 2\sqrt{2x^2+3x+1}$
\item $\dsp x^2 + y^2 = 3y$
\item $\dsp x^4 + 2x^2y^2 + y^4 = 6x^2 - 6y^2$
\item $x=5$
\end{enumerate}

\item
\begin{enumerate}
\item $0 \pm \pi/6, \pi/2 \pm \pi/6, \pi \pm \pi/6, 3\pi/2 \pm \pi/6$
\item $\pm \pi/3$
\end{enumerate}

\item
\begin{enumerate}
\item $24\pi$
\item $33\pi/2$
\item $2\pi$
\end{enumerate}

\item
\begin{enumerate}
\item $6\pi$
\item $2(3\pi-8)/3$
\end{enumerate}

\item $\dsp \frac{\pi - 3\sqrt{3}}{2}$

\end{enumerate}


\chapter{Vectors and Lines}

``The only way of discovering the limits of the possible is to venture a little way past them into the impossible.''  - Arthur C. Clarke\\ \\

Welcome to calculus in three dimensions.  The beauty of this material is how closely it parallels your first semester of calculus. After a brief introduction to the coordinate plane, you learned how to graph lines and parabolas.  After a brief introduction to three-space, we will be graphing planes and paraboloids.  Just as we defined continuity in terms of limits in Calculus I, we will define continuity of functions of several variables in terms of limits in this course. Lines were important as they allowed you to define tangent lines to functions and the derivative.  Tangent planes to functions and surfaces will aid our definition of derivative.   Once you understood the derivative, you used it to find maxima and minima of real valued functions and we will use the derivative of functions in three-space to find maxima and minima in our applications as well.  Two of the most central ideas of your first calculus course were the chain rule and the fundamental theorem of calculus.  We will extend your notion of the chain rule and the fundamental theorem in this course. Only near the end of the course will the work we do not have a parallel to your first course.  At the end, we'll cover Green's and Gauss' Theorems, which are necessary tools in physics and engineering, but which had no parallel in Calculus I.
\begin{annotation}
\endnote{On the first day of Calculus III, we define vector addition, scalar multiplication, points in $\re^3$ and vectors in $\re^3$, just as they are defined below. We then give a parametric equation for a circle and discuss how we might write $y=x^2$ (or any other function) as a parametric curve in one or more ways. Then we give two points in the plane, ask the students to create a parametric equation for the line between two points and have one or more students place their (hopefully different) solutions on the board.  We discuss whether they are correct and how different equations may produce the same set of points.  Once this is done, I have them convert the parameterized lines to slope-intercept form.  This takes the entire first period and gets several students to the board in a relaxed setting, which is my primary goal for the first day.  I send them home to read the notes and start working the problems for presentation on the next day, promising that I'll post a syllabus soon.  Because we use a web site for all communication with the class, that is where I'll post the course notes, syllabus, and other details.}
\end{annotation}

\begin{dfn}
$\nat$ is the set of all \textbf{Natural Numbers}.
\end{dfn}

\begin{dfn}
$\re$ is the set of all \textbf{Real Numbers}.
\end{dfn}

We will also use the notation, ``$\in$,'' to mean ``is an element of.'' Thus, ``$x \in \re$'' means ``$x$ is an element of $\re,$'' or ``x is a real number.''  Similarly, $x,y \in \re$ means ``x and y are real numbers.'' In Calculus I and II, you lived in two-space, or $\re^2 = \{ (x,y) : x,y \in \re \}.$ Now you have graduated to 3-space!

\begin{dfn}
Three dimensional space (Euclidean 3-space or ${\re} ^3$), is the set of all ordered sequences of 3 real numbers.  That is
$${\re}^3\ =\ \{(x_{1}, x_{2}, x_{3}):x_{1}, x_{2}, x_{3} \in {\re}\}.$$
\end{dfn}

Of course, there is no reason to stop with the number 3.  More generally, $n$-space or $\re^n$ is the set of all ordered
sequences of $n$ real numbers, but we will spend most of our time concerned with only $\re, \re^2, \re^3,$ and occasionally, $\re^4.$    Euclidean 4-space is handy since one might want to consider an object or shape in 3-space that is moving with respect to time, thus adding a $4^{th}$ dimension.

We will write elements in $\re^3$ just as letters; hence, by $x \in \re^3$ we mean the element, $x = (x_1, x_2, x_3)$ where $x_1,x_2,x_3 \in \re.$ The \textbf{origin} is the element, $o = (0,0,0).$

\begin{dfn}
If $x, y \in {\re}^3$, then $\oa{xy}$ is the directed line segment from $x$ to $y.$  We abbreviate $\oa {ox}$ by $\oa x.$  Directed line segments are referred to as \textbf{vectors}.
\end{dfn}

Physicists and mathematicians often speak of a vector's {\it magnitude} and {\it direction}.  Given a vector, $\oa x,$ by \textbf{magnitude (or norm)}  we mean the distance between the point, $x,$ and the origin, $o$.  By \textbf{direction} we mean the direction determined by the directed line segment $\oa x$ that has base at $o$ and tip at the point, $x.$  When we say to ``sketch the vector $\oa x$'' we mean to draw the directed line segment from the origin to the point, $x.$

We are making a distinction between {\it points} and {\it vectors}. Points are the actual elements of 3-space and vectors are directed line segments.  The word \textbf{scalar} will be used to refer to real numbers (and later in your mathematical career as complex numbers or elements of any field).  The word, \textbf{point} may be used to mean a real number, an element of $\re^2$, an element of $\re^3,$ etc.

Having carefully made clear the distinction between \emph{point} and \emph{vector} you will have to work hard to keep me honest; I tend to use the two more or less interchangeably.

\begin{dfn}
If $x, y \in \re^3,$ with $x =(x_{1}, x_{2}, x_{3})$  and $y=(y_{1}, y_{2}, y_{3})$, and $\alpha \in {\re}$ then:
\begin{itemize}
\item $x+y=(x_{1}+y_{1},\  x_{2}+y_{2},\  x_{3}+y_{3})$ \hspace{0.1 in} ``Addition in $\re^3$''
\item $\alpha x= (\alpha x_{1},\ \alpha x_{2},\  \alpha x_{3})$ \hspace{0.1 in} ``Scalar Multiplication in $\re^3$''
\end{itemize}
\end{dfn}

Now, to be precise, we should also define {\it vector addition} and {\it scalar multiplication for vectors}. Since the definition is identical (except for placing arrows above the $x$ and $y$) we omit this.  As you can see, always making a distinction between a {\it point} and a {\it vector} can be cumbersome.

\begin{prb}
Let $\oa{x}=\oa{(1,2)}$ and $\oa{y}=\oa{(5,2)}$ and sketch $\oa{x},\ \oa{y},\  -\oa{x},\  2\oa{y}.$
\end{prb}

Consider the two vectors, $\oa {xy}$ and $\oa y - \oa x.$ Both vectors have the same direction and the same magnitude.
They are different because the vector $\oa y - \oa x$ has its base at the origin and its tip at the point $y-x$ while
$\oa {xy}$ has its base at $x$ and its tip at $y.$

\begin{prb}
Sketch $\oa{x}+\oa{y}, \oa{x}-\oa{y}$ and $\oa{xy}$ where $\oa{x}=\oa{(2,3)}$ and $\oa{y}=\oa{(4,2)}.$
\end{prb}

\begin{dfn}
Let $n \in {\nat}.$ A  function from ${\re}$ to ${\re}^n$ is called a \textbf{parametric} curve.
\end{dfn}

We will be concerned primarily with vector valued functions where the range is $\re^2$ or $\re^3.$  If $n=2$, then such functions are also called {\it planar curves} and if $n=3$ they are called {\it space curves}.  We'll put a vector symbol, $\oa{}$, over such functions to remind us that the range is not a real number but a point in ${\re}^n.$

\begin{prb}
Let $\oa l(t)=(2t,\ 3t)$.  Sketch $\oa l$ for all $t \in \re.$ If $t$ represents time and $\oa l(t)$ represents the position of a llama at time $t$, then how fast is the llama traveling?
\end{prb}

\begin{prb}
Let $\oa l(t)= (1,2)+(4,5)t$.  Sketch $\oa l$ for all $t \in \re$ and give the speed of a lemur whose position in the plane at time $t$ seconds is given by $\oa l(t).$
\end{prb}

\begin{prb}
What is the distance between the point $(1,2,3)$ and the origin? What is the distance between $(1,2,3)$ and $(4,5,6)$?
\end{prb}

\begin{prb}
Sketch $\oa r(t)=t(1,2,3)+(1-t)(3,4,-5)$ for all $t \in \re$. Compute $\oa r(0)$ and $\oa r(1).$
\end{prb}

\begin{prb}
What is the distance between ${x}=(x_{1}, x_{2}, x_{3})$ and ${y}=(y_{1}, y_{2}, y_{3})$?
\end{prb}

\begin{prb}
Let $x = (2,5,-1)$ and $y = (1,2,4).$
\begin{enumerate}
\item Write an equation, $\oa l,$ for the line in ${\re}^3$ passing through $x$ and $y$ with $\oa l(0) = x$ and
$\oa l(1) = y.$
\item Write an equation, $\oa m,$ for the line in ${\re}^3$ passing through $x$ and $y$ so that
$\oa m(0) = x$ and the speed of an object with position determined by the line is twice the speed of an object with
position determined by the line in part 1 of this problem.
\end{enumerate}
\end{prb}

\begin{prb}
Find infinitely many parametric equations for the line passing through $(a,b,c)$ in the direction $\oa{(x,y,z)}.$
\end{prb}

\begin{prb}
Plot some points in order to graph $\oa f(t) = (\sin(t),\cos(t),t)$ for $t \in [0,6\pi].$ How would you write the set representing the range of $\oa f?$
\end{prb}

In Calculus I and II you studied functions from $\re$ to $\re,$ for example $f(x)=x^3,$ and functions from $\re$ to ${\re}^2,$ for example $\oa r(t)=(\cos(t),\sin(t))$.  Now we have added functions from ${\re}$ to ${\re}^3$ such as the examples in the previous few problems. We wish to add functions from $\re^2$ to $\re$ to our list next. Such functions are called \emph{real-valued functions of several variables.}
\begin{annotation}
\endnote{Here we discuss the domains and ranges of various ``types'' of functions including examples of all the types of function we have seen such as functions from $\re \rightarrow \re$, from $\re^2 \rightarrow \re$, from $\re \rightarrow \re^2$.  We discuss the coordinate planes, $x=0$, $y=0$, and $z=0$ along with translations of these planes such as $x=1$, $z=3$, and $y=-4$. Then we graph $f(x,y) = 2x + y^2$ by graphing the slices $x=-2$, $x=0$, $x=2$, and $y=0$. We'll note that this might be written as $z - 2x - y^2 = 0$ which really means the set, $\dsp \{ (x,y,z) \in \re^3 : z - 2x - y^2 = 0 \}$.  I try to be very precise when I lecture, demonstrating the independent variables and writing things out completely. We compose this function with $l(t) = (0,0) + (1,1)t$ and discuss that at any time $t$, a bug standing at position $l(t)$ might see her friend directly above (or below) her at height, $f(l(t))$ and position $(t,f(l(t)))$.  Such discussions are my attempt at giving a conceptual intuition to the many types of functions and surfaces that will arise throughout the course. Because most of my students are engineers who will see ample applications, we don't spend lots of time assigning ``real-world'' applications.  We do spend  time discussing the types of real-world problems to which the mathematics we do can be applied.}
\end{annotation}

\begin{dfn}
A \textbf{real valued function of several variables} is a function $f$ from ${\re}^n \to {\re}$.  We will primarily consider $n=2$ and $n=3$ in this course.
\end{dfn}

\begin{expl}
Functions we will encounter and their names.
\end{expl}
\begin{enumerate}
\item real valued functions defined on $\re$\\
$f: \re \to \re$,  $f(t) = t^2 + e^t$
\item parametric curves (also called vector valued functions) defined on $\re$
\begin{enumerate}
    \item planar curves\\
    $\oa f : \re \to \re^2$, $\oa f(t) = (e^{2t+1}, 3t+1)$
    \item space curves\\
    $\oa f : \re \to \re^3$, $\oa f(t) = (2t+1, (3t+1)^3, 4t+1)$
\end{enumerate}
\item real-valued functions of several variables, or {\it multivariate} functions
\begin{enumerate}
\item real valued functions of two variables\\
    $f: \re^2 \to \re$, $f(x,y) = x^2 + \sqrt{y}$
\item vector valued functions of three variables\\
    $f: \re^3 \to \re^2$, $f(x,y,z) = (2xy, 3x + 4y^2)$
\item vector valued functions of several variables\\
    $f: \re^n \to \re^m$ where $m,n \ge 2$
\end{enumerate}
\end{enumerate}

\begin{prb}
Graph the set of all points $(x,y,z)$ in ${\re}^3$ that satisfy $x+y+z=1.$
\end{prb}

\begin{prb}
Graph the set of all points $(x,y,z)$ in ${\re}^3$ that satisfy $z=4.$
\end{prb}

\newpage
\begin{dfn}
If $\oa{x}$ is a vector in ${\re}^3$,  then $\mid \oa{x}\mid = \dsp{\sqrt{x_{1}^2+x_{2}^2+x_{3}^2}}$. This is called the \textbf{magnitude}, \textbf{length}, or \textbf{norm} of $\oa x.$
\end{dfn}

We defined the norm on vectors, but the same definition is valid for {\it points} in $\re^3.$
\begin{annotation}
\endnote{We define the dot product of two vectors and discuss the difference between the words that we will inevitably use interchangeably, orthogonal and perpendicular.  We give the simple example that if $x=(-1,1)$ and $y=(1,1)$ then they have zero dot product, hence are orthogonal.  Of course, a graph of $x$ and $y$ as vectors along with simple trigonometry shows that they are also perpendicular. More discussion on the relationship between dot products and projections will be forthcoming.  In the problems, students are asked to prove that two orthogonal vectors are in fact perpendicular.}
\end{annotation}

\begin{thm}
\label{cos}
\textbf{Law of Cosines.}  Given any triangle with sides of lengths $a,b,$ and $c,$ and having an angle of measure $\alpha$ opposite the side of length $a,$ the following equation holds: $a^2=b^2 + c^2-2 \ b \ c \ \cos(\alpha)$.
\end{thm}

\begin{prb}
Sketch in ${\re}^2$ the vectors $\oa{(1,2)}$ and $\oa{(3,5)}$, and find the angle between these vectors by using the law of cosines.
\end{prb}

\begin{prb}
Sketch in ${\re}^3$ the vectors $\oa{(1,2,3)}$ and $\oa{(-2,1,0)}$, and find the angle between these vectors by using the law of cosines.
\end{prb}

\begin{dfn}
If $\oa{x}=\oa{(x_{1}, x_{2}, x_{3})}$ and $\oa{y}=\oa{(y_{1}, y_{2}, y_{3})}$, then the \textbf{dot product} of $\oa{x}$ and $\oa{y}$ is defined by $\oa{x}\cdot \oa{y}=x_{1}y_{1}+x_{2}y_{2}+x_{3}y_{3}$
\end{dfn}

Again, we defined the dot product on vectors, but the same definition is valid for {\it points} in $\re^3.$

\begin{dfn}
We say the vectors $\oa{x}$ and $\oa{y}$ are \textbf{orthogonal} if $\oa{x}\cdot \oa{y}=0$.
\end{dfn}

\begin{prb}
\label{normxminusy}
Show that if $\oa x$ and $\oa y$ are vectors in $\re^3$ then $$\mid \oa x-\oa y\mid^2\ =\ \mid \oa x\mid ^2-2 \oa x \cdot \oa y + \mid \oa y \mid ^2.$$
\end{prb}

\begin{prb}
Find two vectors orthogonal to $\oa{(1,2)}$.  How many are there?
\end{prb}

\begin{prb}
Find three vectors orthogonal to $\oa{(1,2,3)}$.  How many are there?
\end{prb}


\begin{prb}
\label{uvcos}
Use Theorem \ref{cos} and Problem \ref{normxminusy} to show that if $\oa{x}, \oa{y} \in {\re}^3$ and $\theta$ is the angle between $\oa{x}$ and $\oa{y}$, then $\oa{x}\cdot \oa{y}\ =\ \mid \oa{x}\mid\ \mid \oa{y}\mid \cos \ \theta$.
\end{prb}

\begin{prb}
Find two vectors orthogonal to both $\oa{(1,4,3)}$ and $\oa{(2,-3,4)}$. Sketch all four vectors.\begin{annotation}
\endnote{Because we allow students to look at other materials, it is common for a student to find a vector orthogonal to two vectors by using the cross product which we have not developed yet. Typically, they compute the determinant of the ``matrix'' with rows $(\oa{i}, \oa{j}, \oa{k})$, $\oa{u}$, and $\oa{v}$. When this happens, we use the problem as an opportunity to show  the class how to compute determinants of 2x2 and 3x3 matrices.  Some of my students have had linear algebra, but many have not.  Next we set up the equations one would need to solve in order to demonstrate how the previous problem could be worked  by brute force.  And I say that while we can use this trick, we won't know (as mathematicians) that it is valid until we resolve Problem \ref{cross} that shows how to find a vector perpendicular to two vectors.  The key point is that in defining the cross product to be the vector orthogonal to $\oa u$ and orthogonal to $\oa v$ with length the area of the  parallelogram with sides $\oa u$ and $\oa v$ is more or less an arbitrary choice to force a unique solution. We could have chosen any reasonable third condition (rather than the area of the parallelogram) to guarantee a unique solution.  I'm  attempting to tie together the fundamental ideas of linear algebra with the problem of finding a vector orthogonal to $\oa{u}$ and $\oa{v}$.\\

Around this point in the course, we expand on the meaning of the dot product by showing that if $\oa x$ and $\oa y$ are vectors and $\oa y$ has unit length, then $\oa x \cdot \oa y$ is the length of the projection of $\oa x$ onto $\oa y$ by using $\oa x \cdot \oa y = |\oa x| |\oa y| \cos{\theta}$ and $\cos{\theta} = opp/hyp$. We do the same computation where $\oa y$ is not a unit vector.  The point is that the dot product has an important meaning associated with projections.  We will use this very important fact in demonstrating how a vector line integral is used to compute the work in moving a particle through a vector field as well as with Green's Theorem and Gauss' Divergence Theorem.  I attempt to make every fact that we study early in the course a recurring theme throughout the course and we attempt to relate it to other subjects.  Of course projections, cross products, and determinants in this setting are powerful motivations for a deep understanding of linear algebra.  If they have taken linear, then it is an argument for taking the second semester; if not, it is a basis for understanding the first semester.}
\end{annotation}
\end{prb}


\begin{prb}
Show that if two non-zero vectors $\oa{x}$ and $\oa{y}$ are orthogonal then the angle between them is $90^{\circ}$. Hence any two orthogonal vectors are perpendicular vectors.
\end{prb}

\begin{prb}
Given the vectors $\oa{u},\oa{v} \in \re^3,$  find the area of the parallelogram with sides $\oa{u}$ and $\oa{v}$ and diagonals $\oa{u+v}$ and $\oa {uv}$.  The vertices of this parallelogram are the points:  the origin, u, v, and u+v.
\end{prb}

\begin{prb}
\label{cross}
Assume $\oa{u},\ \oa{v}\in {\re}^3$.  Find a vector $\oa{x}=(x,y,z)$ so that $\oa{x} \perp \oa{u}$ and $\oa{x}\ \perp \oa{v}$ and $x+y+z=1$.
\begin{annotation}
\endnote{This is a bit of nasty algebra, but foreshadows the cross product.  Since the last equation is arbitrary, we will get a multiple of the cross product, not the cross product itself.}
\end{annotation}
\end{prb}

\begin{prb}
Prove or give a counter example to each of the following where $\oa u, \oa v \in \re^3$ and $c \in \re$:
\begin{enumerate}
\item $\oa{u}\cdot \oa{v} \ =\  \oa{v}\cdot \oa{u}$
\item $\oa{u} (\oa{w} \cdot \oa{v})\ =\  (\oa{u} \cdot \oa{w}) (\oa{u} \cdot \oa{v})$
\item $c(\oa{u}\cdot \oa{v})\ =\ (c\oa{u})\cdot \oa{v}\ =\ \oa{u}\cdot (c\oa{v})$
\item $\oa{u} + (\oa{v} \cdot \oa{w})\ =\  (\oa{u} + \oa v) \cdot (\oa{u} + \oa{w})$
\item $\oa{u}\cdot \oa{u}\ =\  \mid \oa u\mid^2$
\end{enumerate}
\end{prb}

\begin{prb}
\label{sketch}
Let $f(x,y)=x^2+y^2$.  Sketch the intersection of the graph of $f$ with the planes: $z=0;\  z=4;\  z=9;\  y=0;\  y=-1;\  y=1;\ x=0;\ $  $x=-1;\  x=1$. Now sketch all of these together in one 3-D graph.
\begin{annotation}
\endnote{Here a discussion arose about the difference between a surface and a function. I told them that a surface is a two dimensional topological manifold; unfortunately, three out of four of those words are beyond the scope of the course --  take topology! I told them to think of a surface as a set of points in three-space where at every point one could place a tangent plane; think smooth and differentiable. We are most interested in being sure we understand when a given set of  points in three-space is a function.  Backing up to two-space, we discussed the example of a circle in the plane as an \emph{equation} that is not a \emph{function}, re-emphasizing that $x^2 + y^2 = 4$ means $\dsp \{ (x,y) \in \re^2 : x^2 + y^2 = 4 \}$.  We observe that if we have an equation in three variables, $x$,$y$, and $z$ such as $x^2 + y^2 + z^2 = 9$ then solving for $z$ may or may not result in a function, just as solving the circle $x^2+y^2=9$ for $y$ does not yield a function.  Of course, we know of (and sketch) other examples of surfaces by considering $x^2+y^2=9$ and $2x-3y=5$ in three-space as cylinders and planes that are not functions.}
\end{annotation}
\end{prb}

In the previous problem, the intersection of the graph with $z=0, z=4, \mbox{ and } z=9$ are called {\it level curves} because each represents the path you would take if you walked around the graph always remaining at a certain height or level.  Recall the Chain Rule from Calculus I.

\begin{thm}
\textbf{Chain Rule.} If $f: \re \to \re$ and $g: \re \to \re$ are differentiable functions, then $(f \circ g)(t) = f(g(t))$ and $(f \circ g)'(t) = f'(g(t))g'(t)$
\end{thm}

\begin{prb}
\label{comp1}
Let $g(x,y)=x^2+y^3$ and $\oa l(t)=(0,1)t$.  Compute $g\circ \oa l$. Graph $g,$ $\oa l,$ and $g\circ \oa l$.
\end{prb}

\begin{prb}
\label{comp2}
Compute $(g\circ \oa l)'(t)$  and $(g\circ \oa l)'(2)$. What is the significance of this number with respect to your graphs from the previous problem?
\end{prb}

Here is a reminder from Calculus II of the definition of arc length.

\begin{dfn}
\label{arclengthdfn}
If $f:\ \re \to \re$ is a function which is differentiable on $[a,\ b]$, then the \textbf{arc length of f on \bm$[a,\ b]$} is $\dsp{\int_{a}^{b}{\sqrt{1+(f'(x))^2}\ dx}}.$ If $\oa c:{\re}\to{\re}^2$ is a vector valued function which is differentiable on $[a,\ b]$ so that $\oa c(t) = (x(t),y(t))$, then the \textbf{arc length of $\oa c$ on $[a,\ b]$} is $\dsp{\int_a^b{\sqrt{(x'(t))^2+(y'(t))^2}\ dt}}$.
\end{dfn}

%borrowed problem!!!!!!!!!!
\begin{prb}
A man walks along a path on the surface $f(x,y)=4-2x^2-3y^2$ from one point on the x-axis to a second point on the x-axis, always remaining directly above the x-axis.  Graph the path and write an integral expression for the distance he walked and compute the distance he walked.
\end{prb}

%borrowed problem!!!!!!!!!!
\begin{prb}
A lady walks along the surface from the previous problem staying exactly 3 units above the $xy$-plane.  Write an integral expression for the distance she walks if she starts and stops at $(0, \frac{1}{\sqrt{3}},3)$ and never retraces her steps?
\end{prb}

\begin{prb}
\label{comp3}
Redo Problem \ref{comp1} and Problem \ref{comp2} with $\oa l(t)=(-1,1)t$.
\end{prb}

\begin{prb}
\label{comp4}
Find the slope of the line tangent to $f(x,y)=x^3+3y^2$ at $(1,2,13)$ that lies above the line $\oa l(t)=(1,2)+(1,1)t$.
\end{prb}

\begin{prb}
Given $\oa a=\oa{(4,3)}$, $\oa b=\oa{(1,-1)}$, and $\oa c=\oa{(6,-4)}$, determine the angle between $\oa{ba}$ and  $\oa{bc}$.
\end{prb}

In the next problem  the notation, $| \cdot |,$ is used for both the absolute value (on the left side of the equation) and the norm (on the right side of the equation).  Is this bad notation?  Consider the definition for the norm, that
$$| (x_1, x_2) | = \sqrt{ x_1^2 + x_2^2}.$$ Suppose we take the norm of a vector in $\re^1$, such as $(x_1).$  Then, $$| (x_1) | = \sqrt{ x_1^2 } = \mbox{ the absolute value of the number } \  x_1.$$ Thus, the absolute value of $x$ {\it is} the norm of $x$ so you have been studying norms since high school (elementary school?) without knowing it!

\begin{prb}
Show that if $\oa{u}$ and $\oa{v}$ are vectors in $\re^2$ then $\mid \oa{u}\cdot \oa{v}\mid \ \leq \mid \oa{u} \mid \mid \oa{v}\mid$.  Two approaches follow:
\begin{enumerate}
\item Let $\oa{u} = (u_1,u_2)$ and $\oa{v} = (v_1,v_2)$.  Substitute into both sides and simplify.
\item Observe that for any constant $k$, $(\oa{u} - k \oa{v})\cdot(\oa{u} - k \oa{v}) \geq 0$.  Simplify this expression and use the quadratic formula.
\end{enumerate}
\end{prb}

The next result is known as the {\it Triangle Inequality} and it states essentially that the shortest distance between two points is the straight line.  Look at a graph of $\oa u, \oa v, \oa{u+v},$ and $\oa{u(u+v)}.$  If you travel from the origin, along the vector $\oa u$ and then along the vector $\oa{u(u+v)}$, then you have traveled further than if you traveled along the vector $\oa{u+v}.$

\begin{prb}
\textbf{Triangle Inequality.} Show that if each of $\oa{u}$ and $\oa{v}$ are vectors in $\re^2$ \\ then $\mid \oa{u}+ \oa{v}\mid \ \leq \mid \oa{u}\mid +\mid\oa{v}\mid$.
\end{prb}

\section{Practice} \label{chap8probs}

We will not present the problems from this section, although you are welcome to ask about them in class.


\vskip .1in
\noindent
\textbf{Vectors}

\begin{enumerate}
\item Let $\oa{x} = (-1, 3) $ and $\oa y = (4,1) $ and
graph $\oa x, \oa y, \oa {xy},$ and $\oa{x-y}.$
\item  Find the norm of each vector in the previous problem.
\item Graph $\oa{(1,-2,1)}$ and $\oa{(-2,4,3)}$ and find the angle
between them.
\end{enumerate}

\noindent
\textbf{Lines}

\begin{enumerate}
\item Sketch the vectors  $x=(1,3,5)$ and $y = (2,4,-3)$  and the
line through these points. \item Find a parametric equation,
$\oa{l},$ for the line in the previous problem so that
$\oa{l}(0)= x$ and $\oa{l}(1) = y.$ \item  Plot $\oa l(t) = (2,
-1, 4) + (1,0,0)t.$  Assuming this represents the position of an
object, compute the speed of the object. \item Find an equation
for the position of the object that has: the same speed as the
object in the previous problem, the same position at time 0, and
is travelling in the opposite direction.
\end{enumerate}

\noindent
\textbf{Functions}

\begin{enumerate}
\item Sketch a graph of $f(x,y) = 2x^2 + 3y^2.$
\item Sketch the intersection of $f$ from the previous problem with the
plane, $z=4.$
\item Sketch the function $g(x,y) = x^3 + y^2.$  Be sure and label a few
points.
\item Compute the composition $g \circ \oa{l}$ where $g$ is from the
previous problem and $\oa l(t) =  (2, -1) + (1,0)t.$
\item Compute $(g \circ l)'(3)$ and indicate its meaning.
\item Sketch $l(t) = (0,1) + (1,0)t$ for $t \ge 0.$
\item Sketch $c(t) = \big( 3\cos(t) , 3\sin(t) \big)$ for $0 \le t \le 2\pi.$
\item Sketch $e(t) = \big( 4\cos(t) , 2\sin(t) \big)$ for $0 \le t \le 2\pi.$
\item Graph $\oa r(t) = ( \cos(t), 2t, \sin(t) )$ for $ t \in
[0, 6\pi];$ if $\oa r(t)$ is the position of an object at time
$t$ then show that the speed of the object is constant.
\end{enumerate}


\vskip .5in
\noindent
\textbf{Chapter 8 Solutions}\\

\noindent
\textbf{Vectors}
\begin{enumerate}
\item a parallellogram with diagonals
\item $\| x \| = \sqrt{10}$ and $\|y\|=\sqrt{17}$
\item approximately 122 degrees
\end{enumerate}


\noindent
\textbf{Lines}
\begin{enumerate}
\item
\item $\oa l (t) = (1,3,5) + (1,1, -8)t$ ($\infty$ possible solns)
\item $\| \oa l'(t) \| = 1$
\item $ \oa p (t) = (2,-1,4)-(1,0,0)t$ ($\infty$ possible solns)
\end{enumerate}

\noindent
\textbf{Functions}
\begin{enumerate}
\item a squished paraboloid
\item an ellipse
\item a water slide
\item $(t+2)^3 + 1$
\item 75, the slope of the line (in three-space) that lies above the
line $\oa l$ and is tangent to the graph of $g$ at the point $(5,-1,126)$
\item $l$ = line
\item $c$ = circle, radius = 3
\item $e$ = ellipse
\item $s$ = spiral; a spring wrapped around the y-axis in three-space;
speed is $\sqrt{5}$
\end{enumerate}



\chapter{Cross Product and Planes}


``Happiness lies in the joy of achievement and the thrill of creative effort.'' - Franklin D. Roosevelt\\ \\

In Problem \ref{cross}, you were given two vectors and asked to find a vector that was perpendicular to both.  Because there are infinitely many vectors perpendicular to any two given vectors, we added another condition ($x_1+x_2+x_3=1$) so that the answer would be unique.  This problem was a warm-up for the definition of the {\it cross product}.

\begin{dfn}
The \textbf{cross product} of two vectors is the vector that is perpendicular to both of them and has length that is the area of the parallelogram defined by the two vectors.
\end{dfn}

If $\oa u = (u_1,u_2,u_3)$ and $\oa v = (v_1,v_2,v_3)$, then the cross product of $\oa u$ and $\oa v$ may be computed by
one of two methods:\\

\textbf{Method One.}
\hskip 1.in $\oa u \times \oa v = (u_{2}v_{3}-u_{3}v_{2},\ u_{3}v_{1}-u_{1}v_{3},\  u_{1}v_{2}-u_{2}v_{1})$ \\

For the second method, you need a tool from linear algebra, determinants. If I have not done so already, ask me during class to show you how to compute the determinant of 2 by 2 and 3 by 3 matrices.\\

\textbf{Method Two.} If we define $\oa i = \oa{(1,0,0)}, \oa j = \oa{(0,1,0)},$ and $\oa k = \oa{(0,0,1)}$, then we may compute the cross product as:
$$ \oa u \times \oa v = det
    \begin{pmatrix}
        \oa i & \oa j & \oa k \\
        u_1 & u_2 & u_3 \\
        v_1 & v_2 & v_3
    \end{pmatrix}. $$

\begin{prb}
Prove or give a counter example for each statement, assuming $\oa u = (u_1,u_2,u_3), \oa v = (v_1,v_2,v_3), \oa w = (w_1, w_2, w_3)$ and $k \in \re.$
\begin{enumerate}
\item $k(\oa{u}\times \oa{v})=(k\oa{u})\times \oa{v} = \oa{u}\times (k\oa{v})$
\item $\oa{u}\times \oa{v} = - \big( \oa{v}\times \oa{u} \big)$
\item $\oa{u}\times \oa{u} = \oa{u}\times \oa 0 = \oa 0\times \oa{u} = \oa{0}$
\item $(\oa{u}\times \oa{v})\cdot \oa{w} = \oa{u}\cdot (\oa{v}\times \oa{w})$
\item $k + ( \oa u \times \oa v) = (k + \oa u) \times (k + \oa v)$
\item $\oa{u}\times (\oa{v}+\oa{w}) = (\oa{u}\times \oa{v})+(\oa{u}\times \oa{w})$
\item $\oa{u}\times (\oa{v} \cdot \oa{w}) = (\oa{u}\times \oa{v})\cdot(\oa{u}\times \oa{w})$
\item $\oa{u}\times (\oa{v}\times \oa{w}) = (\oa{u}\cdot \oa{w})\oa{v}-(\oa{u}\cdot \oa{v})\oa{w}$
\end{enumerate}
\end{prb}

One might think of planes in ${\re}^3$ as analogous to lines in ${\re}^2$. This is because a line is a one-dimensional object in $\re^2$ -- that is it has dimension one less than the dimension of the space.  In $\re^3$ a plane is two-dimensional -- it has dimension one less than the dimension of the space.
\begin{annotation}
\endnote{We give both algebraic and geometric developments for planes.\\

Algebraic: We launch from lines, noting that the slope-intercept form, $y = mx + b$ doesn't give us all lines, while $ax +by = c$ does.  We graph one simple line $2x - 3y = 9$ and graph the $x-$ and $y-$intercepts.  Moving to three dimensions, we note that $z = ax + by + c$ won't give us all planes, but $ax + by + cz = d$ will. And we need only plot three simple points, the $x-$, $y-$, and $z-$ intercepts to sketch most planes.\\

Geometric:  The slope of a line gives us the orientation of the line and a point on the line uniquely determines the line among all lines with that slope. With planes, the perpendicular to a plane gives us its orientation in three-space but we still need a point to nail down which plane we are discussing, just as we did with lines.  We typically work one example, finding a plane orthogonal to (-2,-2,2) and containing (2,3,5) by sketching the orthogonal, the point, and the plane.  We observe that any point $(x,y,z)$ on the plane must satisfy $((x,y,z)-(2,3,5)) \cdot (-2,-2,2) = 0$ and we simplify this expression to place the equation in standard form, observing that this yields the algebraic form we discussed earlier.  I'll query them as to what they observe about the coefficients and see if they notice that the coefficients are indeed a scalar multiple of the orthogonal to the plane.  If not, we'll discover it soon enough. From this example, we see what it means for a vector to be perpendicular to a plane. We say a vector is perpendicular to a plane when it is perpendicular to every vector that lays in the plane.}
\end{annotation}

\begin{dfn}
Given $a,b, c \in \re$ where $a$ and $b$ are not both zero, the \textbf{line} determined by $a,b,$ and $c$ is the collection of all points $(x,y) \in \re^2$ satisfying $ax+by=c.$
\end{dfn}

Given this definition we can define a {\it plane} in the same manner.

\begin{dfn}
Given $a,b, c, d \in \re$ where $a,b$ and $c$ are not all zero, the \textbf{plane} determined by $a,b,c$ and $d$ is the collection of all points $(x,y,z) \in \re^3$ satisfying $ax+by +cz=d.$
\end{dfn}

If \emph{algebraically} we think of a plane as all $(x,y,z)\in {\re}^3$ satisfying $ax+by+cz=d$ where not all of $a,b$ and $c$ are zero, then \emph{geometrically} we can think of a plane as uniquely determined by a vector and a point where the plane is perpendicular to the vector and contains the point.  We need both because there are infinitely many planes  perpendicular to a given vector, but knowing one point in the plane uniquely determines the plane.

\begin{prb}
Show that $(3,-2,5)$  is perpendicular to the plane $3x -2y +5z = 7$ by choosing two points, $x=(x_1,x_2,x_3)$ and $y=(y_1,y_2,y_3),$ in the plane and showing that $(3, -2, 5) \cdot \oa{x-y} = 0.$
\end{prb}

\begin{prb}
Show that $(a,b,c)\in {\re}^3$ is orthogonal to the plane $ax+by+cz=d$ by showing that if $x=(x_1,x_2,x_3)$ and $y=(y_1,y_2,y_3)$ are two points on the plane, then $(a, b, c) \cdot \oa{x-y} = 0.$
\end{prb}

\begin{prb}
Determine whether these two planes are parallel.
\begin{enumerate}
\item $2x-3y+\frac{5}{2}z=9$
\item $x-\frac{3}{2}y+\frac{5}{4}z=12$
\end{enumerate}
\end{prb}

\begin{prb}
Write in standard form (ax+by+cz=d) the equation of a plane $\perp$ to the first plane from previous problem and containing the point $(9,2,3)$.
\end{prb}

\begin{prb}
Find all the planes parallel to the plane $x+y-z=4$ and at a distance of one unit away from the plane. When does the ``distance between two planes'' make sense?
\begin{annotation}
\endnote{Just as occurs with the formula for computing the cross product of two vectors, often a student will use a formula from a book for the distance between two planes. If this occurs, I praise the student for getting this formula for us to use and agree that we can use it from now on.  Again, I emphasize that it might be wrong (perhaps Wikipedia has a typo?) and that until we prove it, we won't know if it is valid.  Then we'll address how we might find the distance without a formula and that by doing it this way, we can take a step toward deriving the formula.  There are a few nice ways to do this. I like to create a parametric line that is orthogonal to both planes and passes through a point in one plane at time $t=0$. Then we can determine the time at which the line intersects the other plane and use the formula for the distance between two points to find the distance.  This method easily generalizes to derive the formula.  Of course, one can use projections and  trigonometry as well.}
\end{annotation}
\end{prb}

\begin{prb}
Find both angles between these two planes:
\begin{enumerate}
\item $2x-3y+4z=10$
\item $4x+3y-6z=-4$
\end{enumerate}
\end{prb}

\begin{prb}
Find the equation of the plane containing $(2,3,4)$, $(1,2,3)$ and $(6,-2,5)$.
\end{prb}

For the next two problems, there is a formula on the web or in some book, but it's probably wrong because of a typographical
error. Find a vector $\perp$ to both planes, determine the equation of a parametric line, $\oa l$, passing through both planes.  Find the points where $\oa l$ intersects each plane.  Find the distance between these points.

\begin{prb}
Find the distance between the two planes, $x+y+z=1$ and $x+y+z=2$.
\end{prb}

\begin{prb}
Find the distance between the two planes, $3x-4y+5z=9$ and $3x-4y+5z=4$.
\end{prb}

\begin{prb}
Find an equation for the distance between two planes, $ax + by + cz = e$ and $ax + by + cz = f.$
\end{prb}

%HINT:  If u \perp P1 and v \perp P2 then u x v is parallel to
%both P1 and P2.

\begin{prb}
Find the equation of the plane containing the line $\oa l(t)=(1+2t,\ -1+3t,\ 4+t)$ and the point $(1,-1,5)$.
\end{prb}

The next problem asks for the intersection of two planes. What are all the possibilities for the intersection of any two planes? One possibility is a line.  To find the equation of the line, there are a couple of ways we could think about this. First, we could find two points in the intersection and then find the equation of that line.  Or we could observe that the intersection of two planes must be contained in each plane and thus is parallel to both planes.  How can we find a  vector that is parallel to both planes?

\begin{prb}
Find the intersection of the two planes $3x-2y+6z=1$ and $3x-4y+5z=1.$
\end{prb}

\section{Practice} \label{chap9probs}

We will not present the problems from this section, although you are welcome to ask about them in class.

\vskip .1in
\noindent
\textbf{Cross Products}

\begin{enumerate}
\item Compute the cross product of $\oa u = (3,2,-1)$ and $\oa v = (2,\pi,-3)$ and state the significance of the resulting vector.
\item Find two vectors of unit length orthogonal to both $\oa a = (4,-3,2)$ and $\oa b = (2,5,-3).$
\end{enumerate}

\noindent
\textbf{Planes}

\begin{enumerate}

\item Find the equation of a plane parallel to $3x +2z = 4 - y$ and containing $(2,4,-2).$
\item Find the equation of the line that represents the intersection of $3z - 4x + 2y = 4$ and $2z - x = 6 - y.$
\item Find an equation for the plane containing $(1,2,3), (2,3,4),$ and $(-1,1,1).$  Find another equation for the same plane.
%\item Find intersection of $2x-3y+5z=4$ and $2x+5y-10z=5$.
\item Find the distance between the two planes: $2x-3y+5z=5$ and $2x-3y+5z=15$.
\end{enumerate}

\noindent
\textbf{Limits}

\begin{enumerate}
\item Consider the function defined by
$f(x,y)=\left\{ \begin{array}{ll}
\dsp{\frac{x^2}{x^2+y}} & x^2+y \neq 0\\
0 & (x,y)=(0,0)
\end{array} \right.$\\
Does the limit exist as $(x,y) \to (0,0)$ along $y=kx$ exist for every $k \in \re?$  Does the limit exist as $(x,y) \to (0,0)$ along $y=kx^2$ exist for every $k \in \re?$ Is the function $f$ continuous at $(x,y)=(0,0)$?\\
\end{enumerate}


\vskip .5in
\noindent
\textbf{Chapter 9 Solutions}\\

\noindent
\textbf{Cross Products}
\begin{enumerate}
\item $(\pi - 6, 7, 3\pi - 4)$
\item $\dsp \pm \frac{1}{\sqrt{933}}( -1, 16, 26)$
\end{enumerate}

\noindent
\textbf{Planes}

\begin{enumerate}
\item $3x+y + 2z = 6$
\item $(4,10,0) + (-2, -10,4)t$ (infinitely many other ways to write this one line; check yours by substituting it into each plane)
\item $z-x=2$
\item $\dsp \frac{10}{\sqrt{38}}$
\end{enumerate}

\noindent
\textbf{Limits}

\begin{enumerate}
\item No, for a chosen value of $k$, the limit along the path
$y=kx^2$ would be $\frac{1}{1+k}.$  Therefore the limit along
different parabolic paths (different values of $k$) would
yield different results.
\end{enumerate}


\chapter{Limits and Derivatives}

``Real knowledge is to know the extent of one's ignorance.''  - Confucius\\ \\

For the next two definitions, suppose that $x, y,$ and $z : \re \to \re$ are differentiable functions.

\begin{dfn}
The \textbf{limit of the vector valued function of one variable} $\oa f(t) = (x(t), y(t), z(t)),$ as $t$ approaches
$a$ is defined by $$\lim_{t\to a} \  \oa f(t)= \left( \lim_{t\to a}\ x(t), \lim_{t\to a}\ y(t),\lim_{t\to a}\ z(t) \right)$$ as long as each of these limits exists.  If any one of the limits does not exist, then the limit of $\oa f$ does not exist at $a$.
\end{dfn}

\begin{dfn}
The \textbf{derivative of the vector valued function of one variable} $\oa f(t) = (x(t), y(t), z(t)),$ is defined by
$$\oa f'(t)=(x'(t), y'(t), z'(t))$$ as long as each of the derivatives exists.  If any one of the derivatives does not exist, then the derivative of $\oa f$ does not exist.
\end{dfn}

\begin{prb}
Let $\oa f(t)=\dsp{\left(t^2-4,\  \frac{\sin(t)}{t},\  \frac{4t^3}{e^t} \right) }$. Compute $\dsp{\lim_{t\to 0}\ \oa  f(t)}$.
\end{prb}

\begin{prb}
Compute $\oa f'$  and $\oa f'(0)$ for $\oa f$ from the previous problem.
\begin{annotation}
\endnote{Once this problem has been presented, we give the geometrical interpretation of $f'(0)$ as the tangent to the parametric curve.  Parametric curves and polar coordinates were introduced in the second semester of calculus, so here we might remind them that if $c(t) = (x(t),y(t))$ is a planar curve, then $c'(t)=(x'(t),y'(t))$ is  tangent to $c$ and each of $(-y'(t),x'(t))$ and $(y'(t),-x'(t))$ are normals.  I attempt Socratic, question-and-answer lectures.  ``What do you suppose $f'(0)$ means?''   (After listening and waiting for a long time, if there is no  response...) ``What did $f'(0)$ tell us about $f(x)= x^2+1$?''   If a series of questions won't generate success, then sketching the elliptical example $c(t) = (3\sin(t),4\cos(t))$ and reminding them of the definition of the derivative $c'(t) = \lim_{h \to 0} \frac{c(t+h)-c(t)}{h}$ can help motivate just why $c'$ is tangent to $c$.}
\end{annotation}
\end{prb}

We now wish to develop the rules for limits and derivatives that parallel the rules from calculus in one dimension.  Of course, we know you have not {\it forgotten} any of these rules, so the ones we develop should look familiar!  The good news is that the really hard work in proving these was done in the first semester of calculus and thus the work here is more notational than mathematical!

\begin{prb}
\label{limfg}
Let $\oa f(t) = (t^2,t^3-1,\sqrt{t-1})$ and $\oa g(t)= (2-t^2,t^3,\sqrt{t+1})$.
\begin{enumerate}
\item Compute $\dsp \lim_{t\to 2}\ \oa f(t).$
\item Compute $\dsp \lim_{t\to 2}\ \oa g(t).$
\item Compute $\dsp \lim_{t\to 2}\ \oa f(t) + \lim_{t\to 2}\ \oa g(t).$
\item Compute $\dsp \lim_{t\to 2}\  [\oa f(t)+ \oa g(t)].$
\item What can you conjecture about $\dsp \lim_{t\to a}\  [\oa f(t)+ \oa g(t)]$ for arbitrary choices of $a, \oa f,$ and $\oa g$?
\end{enumerate}
\end{prb}

\begin{prb}
State 5 rules for limits of the vector valued functions, $\oa f, \oa g: \re \to \re^3$ that parallel the limit rules from Calculus I and prove one of these conjectures. You may grab a book or look on the web to remind you of the rules from  Calculus I.
\end{prb}

\begin{prb}
Compute $\oa f'(2)$ and $\oa g'(2)$ where $\oa f$ and $\oa g$ are from Problem \ref{limfg}.  Compute $\oa{(f+g)}'(2)$.
What can you conjecture about $\oa{(f+g)}'(t)$ for arbitrary choices of $\oa f$ and $\oa g$?
\end{prb}


\begin{prb}
State 5 rules for derivatives of vector valued functions that parallel the derivative rules from Calculus I.  Prove one of these conjectures. You may grab a book or look on the net to remind you of the rules from Calculus I.
\end{prb}

You may assume that which ever one you do not prove will end up on the next test.  Yes, it is a well known fact that all calculus teachers can read students' minds.  How else would we always be able to schedule our tests on the same days as your physics tests?\\

\textbf{Limits of Functions of Several Variables} \\

Recall from Calculus I the various ways in which you computed limits.  If possible, you substituted a value into the  function.  If not, perhaps you simplified the function via some algebra or computed a limit table or graphed the function.  Or you might have applied L`H\^opital's Rule. Your instructor probably used the Squeeze Theorem to obtain the result that $\dsp \lim_{t \to 0} \frac{\sin(t)}{t} = 1.$ Recall that if the left hand limit equaled the right hand limit then the limit existed.

In three dimensions, the difficulty is that there are more paths to consider than merely left and right.  For the limit to exist at a point $(a,b)$, we need that the limit as $(x,y)$ approaches $(a,b)$ exists regardless of the path we take as  we approach $(a,b)$. We could approach $(a,b)$ along the x-axis for example, setting $y=0$ and taking the limit as $x \to 0$.  Or we could take the limit along the line, $y=x$.  The limit exists if the limit as $(x,y) \to (a,b)$ along every possible path exists. We will see an example where the limit toward $(a,b)$ exists along every straight line, but does not exist along certain non-linear paths!
\begin{annotation}
\endnote{To introduce limits, we start with $f(x) = |x|/x$ from Calculus I and consider the left- and right-hand limits. Before talking about three dimensions, I'll ask them if anyone has seen a house where if you approach it from one direction you enter on one floor, but if you enter from another direction, you enter on a different floor.   Usually someone knows someone who lives in a house on a hill where such construction is standard.   If so, they can visualize different paths that lead to different heights at the origin.  Then we consider the problem in three-space
$$\dsp{\lim_{(x,y)\to (0,0)} \frac{x^2-y^2}{x^2+y^2}}$$ by computing the limits along the lines $x=0$, $y=0$, and $y=x.$ We address how, just as in Calculus I, the limit exists only if the limit exists along every possible path and we foreshadow upcoming problems by warning them that there are problems where the limits exist and agree along every straight line path, but there is a non-linear path along which we get a limit that does not agree.  Because they don't have the tools to check every possible path, on tests I'll only ask them to compute limits along specific paths and make their best guess as to whether a limit exists or does not exist.}
\end{annotation}

\begin{dfn}
If $(a,b)\in {\re}^2$  and $L\in \re$ and $f: \re^2 \to \re$ is a function, then we say that $$\dsp{\lim_{(x,y)\to (a,b)} f(x,y) = L}$$ if $f(x,y)$ approaches $L$ as $(x,y)$ approaches $(a,b)$ along \textbf{every} possible path.
\end{dfn}

\begin{prb}
Sketch $f(x,y) = x^2 + y^2,$ indicate the point $\big( 2,3,f(2,3) \big)$ and compute $\dsp{\lim_{(x,y)\to (2,3)}  x^2+y^2}.$
\end{prb}

\begin{prb}
Use any free web-based software to graph the function from the previous example near $(0,0).$  Print and use a high-lighter to mark the paths $x=0$, $y=0$, and $y=x$. (Google Chrome on a Windows box will graph functions like $z=x^2+y^2$ just by typing it into the search bar!)
\end{prb}

%vp
\begin{prb}
Convert the previous problem to polar coordinates via the substitution $x=r\,\cos(\theta)$ and $y=r\,\sin(\theta)$ and then compute the limit as $r \to 0.$
\end{prb}

\begin{prb}
Let $\dsp f(x,y) = \frac{x+y^2+2}{x-y+2}.$
\begin{enumerate}
\item Graph $f$ using any software and state the domain.
\item Compute $\dsp{\lim_{(x,y) \to (-1,2)} f(x,y)}.$
\end{enumerate}
\end{prb}

Recall your Calculus I definition of continuity for functions of one variable.

\begin{dfn}
A function $f: \re \to \re$ is continuous at $a \in \re$ if $\lim_{x \to a} f(x) = f(a).$
\end{dfn}

This definition says that for $f$ to be continuous at $a$ three things must happen.  First, the function must be defined at $a$. This means that $a$ must be in the domain of the function so that $f(a)$ is a number.  Second, the limit of the function as we approach $a$ must exist.  And third, $f(a)$ must equal the limit of $f$ at $a$. The same statement defines continuity for all functions.

\begin{dfn}
A function $f: \re^n \to \re^m$ is continuous at $a \in \re^n$ if $\lim_{x \to a} f(x) = f(a).$
\end{dfn}

\begin{prb}
Consider $\dsp{\lim_{(x,y)\to (0,0)} \frac{x^4-y^4}{x^2+y^2}}.$
\begin{enumerate}
\item Compute this limit along the lines: $x=0$, $y=0$, $y=x$, and $y=-x$.
\item Convert to polar coordinates and check the limit.
\item Graph using any software.
\item Why isn't this function continuous at $(0,0)$?
\item How can you modify $f$ in such a way as to make it continuous at $(0,0)$?
\end{enumerate}
\end{prb}

\begin{prb}
Determine whether $\dsp{\lim_{(x,y)\to (0,0)} \frac{xy+y^3}{x^2+y^2}}$ exists and if so, state the limit.
\end{prb}

%HHH p. 55
\begin{prb}
Determine whether the function $f$ is continuous at $(x,y)=(0,0)$ by considering the paths $y=kx^2$ for several choices of k.
$f(x,y)=\left\{ \begin{array}{ll}
\dsp{\frac{x^2y}{x^4+y^2}} & (x,y)\neq (0,0)\\
0 & (x,y)=(0,0)
\end{array} \right.$\\
\end{prb}

\textbf{Directional Derivatives}\\

Suppose we have the function $f(x,y)=x^2+y^2$ and we are sitting on that function at some point $(a,b,f(a,b))$ other than $(0,0,0)$. Then there are many directions we can walk while remaining on the surface. Depending on the direction of our path, the rate of increase of our height, or slope of our path, may vary.  Some paths will move us uphill and others downhill.  Suppose while we sit at the point, $(a,b,f(a,b)),$ we decide to walk in a direction that will not change the $y$ coordinate, but only changes the $x$ coordinate. Thus, we are walking on the surface and staying within the plane, $y=b.$   Walking in this way, we could go in one of two directions.  Either we could go in the direction that increases $x$ or decreases $x.$   Let's go in the direction that increases $x.$  Now, consider the tangent line to the curve at this point that lies in the plane, $y=b.$  As we take our first step along the curve our rate of increase in  height, $z,$ will be the same as the slope of that tangent line.  This slope is the {\it directional derivative of the function at the point $(a,b)$ in the $x$ direction}.  If we had decided to fix $x=a$ and walk in the direction that increases $y$, then  the slope of the line tangent to the function and in the plane $x=a$ is the {\it directional derivative of $f$ at $(a,b)$ in the $y$ direction.}
\begin{annotation}
\endnote{This lecture ties together many concepts that have been developed (limits, the use of vectors as directions, composition of functions, and parametric curves) while refreshing concepts from Calculus I such as  the limit definition of the derivative.

Letting $g(x) = x^3$ we use the limit definition of the derivative to compute the slope of the tangent line to $g$ at $(2,8)$.  Letting $f(x,y) = x^3+y^2$ we use the limit  $$\lim_{h \to 0} \frac{f(2+h,0) - f(2,0)}{h}$$ to compute the slope of the line tangent to $f$ at the point $(2,0,8)$ and in the plane $y=0$.  We illustrate this with a careful graph,  observing that it is the same as our previous result for $g$  because we are computing the slope of a line that is tangent to a slice of $f$ that is precisely the graph of $g$. We then use limits to compute the directional derivative of $f$ in the direction $(-1,0)$ (the negative x-direction) observing that the result has the same absolute value, but the opposite sign of our previous results.   This is a departure from Calculus I where we did not talk about the derivative at $x$ in the left or right directions. There was only \emph{one} derivative at any point $x$ at which it was differentiable. We then compute the slope of the tangent line in the direction $(0,1)$ (the positive $y-$direction) at the same point.  We conclude with the general definition for the directional derivative of $f$ at vector $u$ in direction $v$, defining the various notations for functions of two variables:
\begin{enumerate}
\item the directional derivative of $f$ in the direction $v$ at $u$ is $\dsp D_v f(u) = \lim_{h \to 0} \frac{f(u+hv) - f(u)}{h}$
\item directional derivative in the x-direction is $\dsp D_{(1,0)} f(u) = df/dx = f_x = f_1$
\item directional derivative in the y-direction is $\dsp D_{(0,1)} f(u) = df/dy = f_y = f_2$
\item gradient of $f$ at $u$ is $\dsp \nabla f(u)  = ( f_1, f_2, f_3, ... , f_n)$
\end{enumerate}
To toss in a tad of history, we mention that this idea is generalizable to spaces other than $\re^n$ and is called the Gateaux Derivative and that there are numerous notions of differentiability, including the Frechet derivative.  A function that is Frechet differentiable is also Gateaux differentiable since Frechet equates to having, in some sense, a ``total''  derivative, while Gateaux only indicates that we have directional derivatives in every direction.  If I have honors  students, these can make good projects for study.}
\end{annotation}

The next definition formalizes this discussion and the problem immediately following it is an example that will make these notions of directional derivative precise!

\begin{dfn}
\label{ddx}
If f: ${\re}^2\to \re$ is a function and $(a,b)$ is in the domain of $f$, then the \textbf{derivative of $f$ in the $(1,0)$
direction at $(a,b)$} is the slope of the line tangent to $f$ at the point $(a,b,f(a,b))$ and in the plane, $y=b.$
\end{dfn}

When it exists, the {\it derivative of $f$ with respect to $x$ at $(a,b)$} can be computed via the limit: $$f_{x}(a,b)=\dsp{\lim_{h\to 0}
\frac{f(a+h,b)-f(a,b)}{h}}$$ or, since $y$ is being held constant and $x$ is changing, one may just compute the derivative of $f$ as if $x$ is the variable and $y$ is a constant.

\textbf{Notation.}  Suppose that $f: {\re}^2\to \re$ is a function as in the previous definition.  There are many phrases and notations used to denote the \textbf{derivative of $f$ in the $(1,0)$ direction at $(a,b)$}. For example,
\begin{itemize}
\item $f_{1}$ -- the \textbf{derivative of f with respect to the first variable}
\item $f_{x}$ -- the \textbf{the derivative of f with respect to x}
\item $\dsp{\frac{\partial f}{\partial x}}$ -- the \textbf{partial derivative of f with respect to x}
\end{itemize}
Similarly, $f_{2}$, $f_{y}$, and $\dsp{\frac{\partial f}{\partial y}}$  would denote the same concepts where the derivative was taken in the $(0,1)$ direction.

\begin{prb}
Let $f(x,y) = x^2 + y^2$ and $(a,b)=(2,-3).$
\begin{enumerate}
\item Sketch $f$ and sketch the line tangent to $f$ at the point $(2,-3,f(2,-3))$ that is in the plane, $y=-3.$
\item Let $g(x) = f(x,-3)$ and compute $g'(2).$
\item What part of the graph of $f$ is the graph of $g?$
\end{enumerate}
\end{prb}

\begin{prb}
Let $\dsp f(x,y) = \frac{y}{x}.$ Compute $f_x(1,2)$ using the limit described following Definition \ref{ddx}.
\end{prb}

\begin{prb}
Compute $f_{x}$ and $f_{y}$ for each function.
\begin{enumerate}
\item $f(x,y)=x^3-4x^2$
\item $f(x,y)=e^{xy^2}$
\item $f(x,y)=\dsp{\frac{x^2}{\sin(xy)}}$
\item $f(x,y)=e^{x^2y}\sin(x-y)$
\end{enumerate}
\end{prb}

\begin{dfn}
Just as in Calculus I, ``second derivatives'' are merely derivatives of the first derivatives.  Thus $f_{xx}=(f_{x})_{x}$. I.e. $f_{xx}$ is the derivative of $f_{x}$ with respect to x. Similarly, $f_{yy}=(f_{y})_{y}$, $f_{xy}=(f_{x})_{y}$ and $f_{yx}=(f_{y})_{x}$.
\end{dfn}

Other standard notations are:
$$f_x = \frac{\partial f}{\partial x}, \  \  \ f_y = \frac{\partial f}{\partial y}, \  \  \ f_{xx} = \frac{\partial^2  f}{\partial x^2}, \  \  \ f_{yy} = \frac{\partial^2 f}{\partial y^2}, \  \  \ \mbox{ and } \  \  \ f_{xy} = \frac{\partial^2 f}{\partial x \partial y}$$


\begin{thm}
\textbf{Clairaut's Theorem.}  For any function, f, whose second derivatives exist, we have $f_{xy}=f_{yx}$.
\end{thm}

\begin{prb}
Find $f_{xx}, f_{xy}, f_{yx}$, and $f_{yy}$ for each function.
\begin{annotation}
\endnote{While this problem seems trivial, it is a spring board that we use to discuss where all these derivatives show up in physical applications, Maxwell's equation, Laplace's equation, etc.}
\end{annotation}
\begin{enumerate}
\item $f(x,y)=e^{x^2+y^2}$
\item $f(x,y)=\sin(xy+y^3)$
\item $f(x,y)=\sqrt{3x^2-2y^3}$
\item $f(x,y)=\dsp{\cot\left( \frac{x}{y}\right) }$
\end{enumerate}
\end{prb}

The study of partial differential equations  is the process of finding functions that satisfy some equation that has derivatives with respect to multiple variables.  For example {\it Laplace's Equation} is the equation $u_{xx}+u_{yy} +u_{zz}=0$ and solutions give us information about the steady state of heat flow in a three dimensional object.

\begin{prb}
Which of these functions satisfy $u_{xx}(x,y)+u_{yy}(x,y)=0$ for all $(x,y) \in \re^2$?
\begin{enumerate}
\item $u(x,y)=x^2+y^2$
\item $u(x,y)=x^2-y^2$
\item $u(x,y)=x^3+3xy^2$
\item $u(x,y)=\ln(\sqrt{x^2+y^2})$
\item $u(x,y)=\sin(x)\cosh(y)+\cos(x)\sinh(y)$
\item $u(x,y)=e^{-x}\cos(y)-e^{-y}\cos(x)$
\item Find a solution to this equation other than the ones listed above.
\end{enumerate}
\end{prb}

In the last two problems, we studied functions of two variables and we defined derivatives in each direction, the $x$ direction and the $y$ direction.  If $f$ were a function of three variables, then there would be partial derivatives with respect to each of $x, y$, and $z.$ Let's extend the notion of the partial derivatives with respect to $x$ and $y$ to functions with domain $\re^n$ where $n>2.$  If $f : \re^n \to \re$, then the domain of $f$ is $\re^n$ so there is a partial derivative of $f$ with respect to the first variable, the second variable, and so on, up to the partial derivative of $f$ with respect to the $n^{th}$ variable.  We use the notation, $f_1, f_2, f_3, \dots, f_n$  or $\frac{\partial f}{\partial x_1}, \frac{\partial f}{\partial x_2}, \frac{\partial f}{\partial x_3}, \dots , \frac{\partial f}{\partial x_n}$ to denote the derivative of $f$ with respect to each variable.
\begin{annotation}
\endnote{While it may seem repetitive, we find that an occasional review of the various types of functions we have studied along with their derivatives is needed.  We know and emphasize that there is really only one definition of a function so we are  really only talking about functions with different domains and ranges.  If this leads to a discussion, we'll define the following. Given two sets, $A$ and $B$, a relation on $A \times B$ is a subset of $A \times B$. A function on $A \times B$ is a relation on $A \times B$ in which no two elements of the relation have the same first coordinates.  While accurate, this curt, systematic explanation does not make the language (real-valued, parametric, vector valued, space curve, planar curve, etc.) any easier so we consider these examples,
\begin{enumerate}
\item $f(x) =\sin(x^3)$
\item $g(t) = ( 4\cos(t), 3\sin(t))$
\item $f(x,y) =\sin^2(xy) - \sqrt{\cos(x^2)}$
\item $f(x,y) = ( \cos(xy),\sin(xy) )$
\item $3x^2 + 5y^2 + 6z^2 = 1$ (not a a function!)
\end{enumerate}
discussing the domain and range of each.}
\end{annotation}

\begin{prb}
Let $f(x_1,x_2,x_3,x_4,x_5) = x_3\sqrt{(x_1)^3+(x_2)^2} + x_4 e^{x_5 x_3}.$ Compute the five partial derivatives, $f_1, f_2, \dots, f_5.$
\end{prb}

This allows us to define the derivative for functions of $n$ variables.

\begin{dfn}
If $f:{\re}^n \to \re$ is a function and each partial of $f$ exists, then the \textbf{gradient} of f is the function $$\nabla f:{\re}^n \to {\re}^n$$ and is defined by $$\nabla f= (f_{1}, f_{2}, f_{3}, \ldots , f_{n}) =(\frac{\partial f}{\partial x_1},\frac{\partial f}{\partial x_2}, \frac{\partial f}{\partial x_3}, \dots , \frac{\partial f}{\partial x_n}).$$
%In ${\re}^2$, $\nabla f=(f_{x}, f_{y})$, or evaluated at (x,y), $\nabla f(x,y)=(f_{x}(x,y),f_{y}(x,y))$
\end{dfn}

\begin{prb}
For each function below, state the domain of the function, compute the gradient and state the domain of the gradient.
\begin{enumerate}
\item $f(x,y)=x^3y^2-x^2y^3$
\item $g(x,y,z)=zx^3-3xyz+\ln(x^2yz^3)$
\end{enumerate}
\end{prb}

For any function of two variables, $f,$ let's assume once again that we are at a point on the function, $(a,b,f(a,b)).$  We know that the slope of the line tangent to $f$ at $(a,b,f(a,b))$ and above the line $y=b$ is $f_x(a,b)$.  And we know that the slope of the line tangent to $f$ at $(a,b,f(a,b))$ and above the line $x=a$ is $f_y(a,b)$.  Now consider  a line in the $xy$-plane passing through $(a,b)$ in some direction $\oa{(c,d)}$ that is not parallel to either the $x$ axis or the $y$ axis. What would the slope of the line tangent to $f$ at $(a,b,f(a,b))$ and above this line be? From the point $(a,b)$ there are infinitely many directions that we might travel, not just the directions parallel to the $x$ and $y$ axes.  We can define such a direction from $(a,b)$ by a vector, $\oa{(c,d)}.$  The slopes of the lines tangent to $f$ at $(a,b,f(a,b))$ in the direction $\oa{(c,d)}$ are called the {\it directional derivatives} of $f$ at $(a,b)$ in the direction $\oa{(c,d)}.$  The partial derivatives of $f$ with respect to $x$ and $y$ are your first examples of directional derivatives where the directions were $\oa i = (1,0)$ and $\oa j = (0,1).$  Here is the formal definition of the directional derivative.

\begin{dfn}
\label{dd}
Let $f: \re^n \to \re$ be a function.  The \textbf{directional derivative} of $f$ at $\oa{u}$ in the direction $\oa{v}$ is given by: $$D_{\oa{v}}f(\oa{u})= \dsp{\lim_{h \to 0} \frac{f(\oa{u}+h\oa{v})-f(\oa{u})}{h}},$$  where  $\oa{v}$ must be a unit vector.
\end{dfn}

Of course, not every limit exists, so directional derivatives may exist in some directions but not others.

\begin{prb}
\label{ddp}
Using the definition just stated, compute the directional derivative of $f(x,y)=4x^2+y$ at the point $\oa u = (1,2)$ in the direction $\oa v$ for each $\oa v$ defined below.
\begin{annotation}
\endnote{As soon as this problem goes on the board, we state and give an example of Theorem \ref{dot} below.  The point of the previous problem is to demonstrate that if we compute the directional derivatives without using unit vectors for the direction then we don't actually get the slope of the tangent line.  Even though $(1,1)$ and $(\frac{\sqrt{2}}{2},\frac{\sqrt{2}}{2})$ represent the same direction, the directional derivative would depend on the magnitude of the direction vector, which is why the definition requires that we use a unit vector.}
\end{annotation}

\begin{enumerate}
\item $\oa{v}=(0,1)$
\item $\oa{v}=(1,0)$
\item $\oa{v}=(1,1)$
\item $\oa{v}=(\frac{\sqrt{2}}{2},\frac{\sqrt{2}}{2})$
\end{enumerate}
\end{prb}

\textbf{Important.}  In the previous problem, you got different answers for the directional derivatives in parts 3 and 4 even though those represent the same direction.  For this reason, it is a convention to always use a {\it unit} vector for the direction when computing directional derivatives.

\begin{prb}
Use Definition \ref{dd}, Definition \ref{ddx}, and the discussion immediately following Definition \ref{ddx} to show that if $\ f:{\re}^2\to \re$ and $\oa{i}=(1,0)$, then $D_{\oa{i}}f(\oa{v})=f_{x}(\oa{v})$.
\end{prb}

\textbf{Non-Definition.} An analytical  definition of \textbf{f is differentiable} at $u$ is beyond the scope of this course, but a geometrical definition is not.  In two dimensions (Calculus I), f was differentiable at $a$ if there was a \textbf{tangent line} to f at $(a,f(a))$. In three dimensions (Calculus III), f is differentiable at $\oa{u}$ if there is a \textbf{tangent plane} to f at $(u,f(u)).$

\begin{dfn} \label{nhood}
An $\epsilon$-neighborhood of $u \in \re^n$ is the set of all points with a distance from $u$ of less than $\epsilon$.  I.e. $N_{\epsilon}(u)=\{v \in \re^n:\ \mid u-v \mid < \epsilon \}$.
\end{dfn}

How will we be able to tell when a function is ``nice,'' that is, when a function has a derivative?

\begin{thm}
\textbf{A Differentiability Theorem.} If $\nabla f$ exists at $u$ and at all points in some $\epsilon$-neighborhood of $u$, then f is differentiable at $u$.
\end{thm}

\begin{prb}
Is $\dsp f(x,y) = y^3(x-\frac{1}{2})^\frac{2}{3}$ differentiable at $(1,2)?$
\end{prb}

\begin{prb}
For each of the following problems, either prove it or give a counterexample by finding functions and variables for
which it does not hold.  Assume $f:{\re}^2\to \re$ and $g:{\re}^2\to \re$ are differentiable.  Assume $x, y \in \re^2$ and $c \in \re.$  Assume $x, y, x + y$ are in the domain of $f$ and $g.$
\begin{enumerate}
\item $\nabla f(x+y)= \nabla f(x)+\nabla f(y)$
\item $\nabla (f+g)(x)=\nabla f(x)+\nabla g(x)$
\item $\nabla (cf)(x)=c\nabla f(x)$
\item $\nabla f(cx)=c\nabla f(x)$
\item $\nabla f(cx)=\nabla f(c)\ \nabla f(x)$
\item $\nabla (f\cdot g)(x)= \nabla f(x)g(x)+f(x) \nabla g(x)$
\item $\dsp \nabla (\frac{f}{g})(x) = \frac{  \nabla f(x)g(x)-f(x) \nabla g(x) }{ g^2 }$
\end{enumerate}
\end{prb}

As in Calculus I, it is very nice to know when and where a function is continuous.  The following theorem answers that question in both cases.

\begin{thm}
\textbf{A Continuity Theorem.} From Calculus I, if $f: \re \to \re$ is differentiable at $x$ then f is continuous at $x.$  In Calculus III, if $f : \re^n \to \re$ is differentiable at $x$, then f is continuous at $x$.
\end{thm}

The next theorem is a very important one.  In Calculus I, we first computed derivatives using the definition and then proved rules to help us differentiate more complex functions; we do the same here.  We won't prove this theorem, but it gives a very easy way to compute directional derivatives by using the dot product and the gradient of the function.
\begin{thm}
\label{dot}
\textbf{Directional Derivative Theorem.} $D_{\oa{v}}f(u)=\nabla f(u)\cdot \oa{v}$ for any $u,\ \oa{v} \in {\re}^3$ where $\mid \oa{v}\mid = 1$.
\end{thm}

\begin{prb}
Redo Problem \ref{ddp} using Theorem \ref{dot}.
\end{prb}

To date we have studied the derivatives of functions from $\re$ to $\re,$ from $\re$ to $\re^2,$ and from $\re^2$ to $\re.$ Of course there is nothing special about $\re^2$ here. We might as well have studied $\re^n$ as all the derivatives would follow the same rules. Now let's consider the derivative of a function, $f : \re^2 \to \re^2.$

\begin{dfn}
If $f : \re^2 \to \re^2$  is any vector valued function of two variables defined by $f(x,y) = \big( u(x,y) \ , \
v(x,y) \big)$, then the derivative of $f$ is given by
$$ Df =
\begin{pmatrix}
u_x(x,y) & u_y(x,y)  \cr  v_x(x,y) &  v_y(x,y).
\end{pmatrix}
$$
\end{dfn}

\begin{prb}
Compute the derivative of $\dsp f(x,y) = \big( x^2\sin(xy) \ , \ \frac{e^y}{\tan(x)} \big).$
\end{prb}

Because of the number of different domains and ranges of functions we are studying, we have several variations of the chain rule.   Before we begin, I would like to take this opportunity to apologize for the number of notations used by mathematicians, physicists, and engineers for derivatives, partial derivatives, total derivatives, gradients, Laplacians, etc.  There are a number of notations and all are convenient at one time or another.  I attempt to adhere for the most part to the functional notation for derivatives ($f_1, f_x, f',$ etc.), but Leibniz notation, $\frac{\partial f} {\partial x}$, is a convenient notation as well. Table \ref{derivnot} illustrates my preferred notations, where $L(\re^2, \re^2)$ denotes the set of all 2x2 matrices.

\begin{table}
\centering
\caption{Derivative Notation}  \label{derivnot}
\vskip 5pt
\begin{tabular}{ |c|c| }
\hline
\multicolumn{1}{|c|}{Function}
&\multicolumn{1}{c|}{Derivative}\\
\hline
\hline
$f: \re \to \re$ & $f' : \re \to \re$  \\
\hline
$f: \re^2 \to \re$ & $\nabla f : \re^2 \to \re^2$  \\
\hline
$f: \re^2 \to \re^2$ & $Df : \re^2 \to L(\re^2, \re^2)$  \\
\hline
\end{tabular}
\end{table}

The chain rule you learned in Calculus I, which is stated next, can be generalized to take the derivative of the  composition of any of the functions we have been studying.  That is, given any two functions with domains so that their composition actually makes sense and so that they are differentiable at the appropriate places, we can compute their derivative using the same chain rule that you learned in Calculus I with one minor change. When the domains and ranges of the functions change, the derivatives change.  Thus, in the following statement, depending on the domain of $f$, sometimes $f'$ means the derivative of a parametric curve, but sometimes it means the gradient of $f$, $\nabla f$, and sometimes it means the matrix, $Df$, of derivatives of $f$.  The same holds for the derivative of $g$.  And finally, the symbol $\cdot$ might mean multiplication or the dot product or matrix multiplication.   You'll know from context which one. The point is to realize that no matter how many different notational ways we have of writing the chain rule, it always boils down to this one.
\begin{annotation}
\endnote{We spend a fair amount of time demonstrating the chain rule in various forms via examples, emphasizing that while it may be written differently due to the different notations that we use for the derivative of  functions with differing domains, there is really only one chain rule:  $(f \circ g)' = (f' \circ g) \cdot g'$ where the  $'$ and $\cdot$ must be interpreted correctly depending on the domains of $f$ and $g$. And each time the chain rule is used in class, we reiterate this theme. First we do a Calculus I example, letting $h(x) = \cos(\sqrt{x^3-3})$ and breaking $h$ into $f \circ g$ to clearly demonstrate the chain rule. Second, we do a more complicated composition, letting $f(x,y) = e^{xy}$ and $\oa g(t) = (t^2+1, 3t^3)$ and computing $(f \circ \oa g)'$ in two ways.  We compose the two functions and compute the derivative and then we use Theorem \ref{cr2}. In one final example, we demonstrate both ways, letting $f(x,y) = \sin(x^2+y)$ and $\oa g(s,t) = (s^2t, s+t)$ and computing $\nabla (f \circ \oa g )$ by first composing and then taking the derivative and by applying Theorem \ref{cr3}.}
\end{annotation}

\begin{thm}
\textbf{Chain Rule.} \label{cr1} If $f, g: \re \to \re$ are differentiable functions, then $$(f \circ g)' = (f' \circ g) \cdot g'$$ or with the independent variable displayed, $$\big(f \circ g\big)'(x) =  f'\big(g(x)\big) \cdot g'(x).$$
\end{thm}

Next we state the chain rule for differentiating functions that are the composition of a function of two variables with a planar curve.  Notice that this theorem is {\it exactly} the same as the original theorem, but restated for functions with different domains. Because $f : \re^2 \to \re$ we replace the $f'$ from the previous theorem with $\nabla f$ and because $\oa g : \re \to \re^2$ we replace the $g'$ in the previous theorem with $\oa g'.$  Thus the theorem still says (in English) that ``the derivative of ($f$ composed with $g$)  is the (derivative of $f$) evaluated at $g$ times (the derivative of $g$).''

\begin{thm}
\label{cr2} \textbf{Chain Rule.}  If $f:{\re}^2\to \re$ and $\oa g:\re \to {\re}^2 $ are both differentiable, then
$$(f \circ \oa g)' = \big((\nabla f)\circ \oa g\big)\cdot \oa g'.$$ Writing this with the independent variable $t$ in place we could write: $$ \big(f\circ \oa g\big)'(t)= \Big(\nabla f\Big) \big(\oa g(t)\big)\cdot \oa g'(t).$$
\end{thm}

On the right hand side of the last line of the Chain Rule, we have the composition of $\nabla f$ with $g(t)$.  Because we are multiplying vectors, ``$\cdot$'' represents dot product and not multiplication.

\begin{prb}
Let $f(x,y) = x^2 - 3y^2$ and $\oa g(t) = (2,3) + (4,5)t.$ Compute $(f \circ \oa g)'$ both by direct composition and by
using Theorem \ref{cr2}.
\end{prb}

\begin{prb}
Compute $(w \circ \oa g)'$ where $\dsp w(x,y) = e^x\sin(y) - e^y\sin(x)$ and $\oa g(t) = (3,2)t.$  Write a complete sentence that says what line $\big(w \circ \oa g\big)'(-1)$  is the slope of.
\end{prb}

\begin{prb}
Redo the following problems using Theorem \ref{cr2} and paying special attention to the use of unit vectors.
\begin{enumerate}
\item Problem \ref{comp1} and Problem \ref{comp2}.
\item Problem \ref{comp3}.
\item Problem \ref{comp4}.
\end{enumerate}
\end{prb}

Next we state the chain rule for differentiating functions that are the composition of a function of several variables with a function from the plane into the plane. Notice that this theorem is {\it exactly} the same as the original theorem, but restated for functions with different domains. Because $f : \re^2 \to \re$ we replace the $f'$ from the original theorem with $\nabla f$ and because $\oa g : \re^2 \to \re^2$ we replace the $g'$ in the previous theorem with $D\oa g.$  Thus the theorem still says that ``the derivative of ($f$ composed with $g$)  is the (derivative of $f$) evaluated at $g$ times (the derivative of $g$).'' Because $D \oa g$ is a matrix, the right hand side of this is now a vector times a matrix.
\begin{annotation}
\endnote{After we have seen several chain rule problems at the board, we attempt to take some of the mystery out of the notation. Suppose $L: \re^2 \to \re^2$ and $f: \re^2 \to \re$.  Then the derivatives of $f$ with respect to the first and second variables are: $$ \big(f(L(s,t))\big)_s =  \nabla f(L(s,t)) \cdot  L_s(s,t) \mbox{ and } \big(f(L(s,t))\big)_t =  \nabla f (L(s,t)) \cdot L_t(s,t).$$

If you don't like subscripts representing the derivatives, we can eliminate these.

$$ \frac{\partial}{\partial s} \big(f(L(s,t))\big) = \big( \nabla f(L(s,t))\big) \cdot \frac{\partial}{\partial s}L(s,t) \mbox{ and } \frac{\partial}{\partial t} \big(f(L(s,t))\big) = \big( \nabla f(L(s,t))\big) \cdot \frac{\partial}{\partial t}L(s,t)$$

Eliminating the independent variables, and using the Leibniz notation, we have:
$$\frac{\partial f}{\partial s} = (\frac{\partial f}{\partial x}, \frac{\partial f}{\partial y}) \cdot (\frac{\partial x}{\partial s}, \frac{\partial y}{\partial s}) = \frac{\partial f}{\partial x} \frac{\partial x}{\partial s} + \frac{\partial f}{\partial y} \frac{\partial y}{\partial s}$$
and
$$\frac{\partial f}{\partial t} = (\frac{\partial f}{\partial x}, \frac{\partial f}{\partial y}) \cdot (\frac{\partial x}{\partial t}, \frac{\partial y}{\partial t})  =  \frac{\partial f}{\partial x} \frac{\partial x}{\partial t} +
\frac{\partial f}{\partial y} \frac{\partial y}{\partial t} $$}
\end{annotation}

\begin{thm}
\textbf{Chain Rule.}
\label{cr3}
If $f:{\re}^2\to \re$ and $\oa g:\re^2 \to {\re}^2 $ are both differentiable, then $$\nabla (f \circ \oa g) = \Big( \big(\nabla f\big) \circ \oa g\Big) \cdot D\oa g.$$ Writing this with the independent variables displayed, $$\nabla \Big(f\big(g(s,t)\big)\Big) = \Big( \nabla f\Big)\big(g(s,t)\big) \cdot \big(D \oa g\big)(s,t).$$
\end{thm}

\begin{prb}
Let $f(x,y) = 2x^2 - y^2$ and $\oa g(s,t) = (2s+5t,3st).$ Compute $\nabla (f \circ \oa g)$ in two ways.  First, compute by composing and then taking the derivative.  Second, apply Theorem \ref{cr3}.
\end{prb}

\begin{prb}
Let $w(x,y) = \ln(x+y) - \ln(x-y)$ and $g(s,t) = (te^s, - e^{st}).$ Compute $\nabla (w \circ \oa g)$ in two ways.  First, compute by composing and then taking the derivative.  Second, apply Theorem \ref{cr3}.
\end{prb}

\begin{prb}
Prove or give a counterexample to the statement that $f'(\oa l(t)) = (f(\oa l(t)))'$ for all differentiable functions $f: \re^2 \to \re$ and $\oa l: \re \to \re^2.$
\end{prb}

The next problem tells us something important. If you are sitting on some function in three-space and you are trying to decide what direction you should travel to go uphill at the steepest possible rate, then the gradient tells us this direction.

\begin{prb}
Use Problem \ref{uvcos} and Theorem \ref{dot} to show that $D_{\oa{v}}f(\oa{u})$ is largest when $\oa{v}=\nabla f(\oa{u})$ and smallest when $\oa{v}=-\nabla f(\oa{u})$.
\end{prb}

\textbf{Surfaces.} Earlier we gave a list of the types of functions we have studied so far. Of course, there is only one definition for a function, so we are really talking about functions with different domains and ranges as was illustrated by the need for different chain rules for functions with different domains.  In linear algebra, we see functions with domain the set of matrices (the determinant function) and $\dsp T(f) = \int_0^1 f(x) \ dx$ is a function from the set of all continuous functions into the real numbers.

In earlier courses, you studied not only functions, but {\it relations} such as $x^2 + y^2 = 1.$  What was the difference?
Well, a function has a unique $y$ for each $x$ while a relation may have several $y$ values for a given $x$ value.  In three dimensions a function will have a unique $z$ for a given coordinate pair, $(x,y).$  When one has multiple $z$ values for a given $(x,y)$ value, we call it a {\it surface.} We have thus far studied mostly functions from $\re^2$ to $\re$
such as $f(x,y)=x^2+y^2$ or $f(x,y)=ye^{x^2}$, but surfaces are equally important. Surfaces are to functions in three-space as relations were to functions in two-space.
\begin{annotation}
\endnote{By now we have been foreshadowing relations and surfaces for some time. This is a central theme behind the way the course is run.  Rather than introducing a new concept, such as surfaces, just in time to introduce planes, we have foreshadowed the idea for some time.  When the concept becomes necessary, it is already understood at a conceptual, intuitive level. We first graph $y = x^2$ and $x = y^2$.  Both are relations.  One is a function, the other is not.  We recall that every function is a relation, but not every relation is a function.  Then we move to three-space.  We rewrite $f(x,y) = x^2 + y^2$ as $z = x^2 + y^2$ and remind them that this short hand really means $\{ (x,y,z): x^2 + y^2 = z \}$. We ask, ``What equation would result in the same shape, but rotated so  that it is symmetric about the x-axis?''  The class (sometimes prompted by a game of hangman if they are not forthcoming) suggests $x = y^2 + z^2$.  Then we graph a few simple surfaces in three-space and ask which are functions:  $z=2x$,  $y=x^2$, $z=\sin(x)$, $y=\sin(z)$.}
\end{annotation}

\begin{prb}
Sketch these \underline{functions} in ${\re}^3$.
\begin{enumerate}
\item $f(x,y)=\mid x \mid - \mid y \mid$
\item $h(x,y)=\sqrt{xy}$
\item $g(x,y)=\sin(x)$
\item $i(x,y)=2x^2-y^2$
\end{enumerate}
\end{prb}

\begin{prb}
Sketch these \underline{surfaces} in ${\re}^3$.
\begin{enumerate}
\item $x^2+y^2+z^2=1$
\item $y^2+z^2=4$
\item $x^2-y+z^2=0$
\item $\mid y \mid =1$
\end{enumerate}
\end{prb}

Surfaces may be expressed as $F(x,y,z)=k$, where $F: \re^3 \to \re$ is a function and $k$ is a real number.  The first example above represents the set of all points in $\re^3$ that satisfy $F(x,y,z)=1$ where $F$ is the function defined by $F(x,y,z)= x^2+y^2+z^2.$  The mathematical convention is to rewrite these as follows:
\begin{enumerate}
\item $F(x,y,z)=1$ where $F(x,y,z)= x^2+y^2+z^2.$
\item $F(x,y,z)=4$ where $F(x,y,z)= y^2+z^2$.
\item $F(x,y,z)=0$ where $F(x,y,z)= x^2-y+z^2$.
\item $F(x,y,z)=1$ where $F(x,y,z)= \mid y \mid $.
\end{enumerate}

\vskip .1in
\textbf{Tangent Planes to Functions}
\begin{annotation}
\endnote{This mini-lecture gives away too much in my opinion, but there is simply too much material in the syllabus for them to discover everything so we show them how to compute tangent planes to functions and surfaces.

Suppose we want to find the tangent plane to a function such as $f(x,y) = x^2+y^2$ at the point, $p=(2,3,13).$ We know that to find the equation of a plane it is enough to have a vector that is orthogonal to the plane and a point on the plane.  We already have the point $p$, so all we need is an orthogonal vector.  Our strategy will be to find two vectors that are tangent to the surface at the point $p$ and thus are in the plane.  Their cross product will yield the orthogonal vector.  We know that $f_x(x,y)=2x$ tells us the rate of change of $x$ with respect to $z$ and thus a line tangent to the function and parallel to the x-axis at $(2,3,13)$ would be $L_x(t) = (2,3,13) + (1,0,4)t.$ Hence, $(1,0,4)$ is a vector parallel to our plane and pointing in the x-direction.  Similarly, $(0,1,6)$ is parallel to our plane and points in the y-direction.  Taking the cross product will yield our vector that is perpendicular to the function and the plane at $p.$ This vector is $(-4,-6,1)$ so our plane is $(x-2,y-3,z-13) \cdot (-4,-6,1) = 0.$

To demonstrate how to find tangent planes to surfaces, we'll first note that every function may be written as a surface, where surfaces will always be written as $F(x,y,z)=k$ where $F: \re^3 \to \re$ and $k \in \re$. Thus our function,  $f(x,y) = x^2+y^2$ may be written as  $F(x,y,z) = 0$ where $F(x,y,z) = x^2+y^2-z$.   Note that  our point  $p=(2,3,13)$ is on this surface since $F(2,3,13)=0$.   Now suppose that  $c(t) = (x(t),y(t),z(t))$ is a curve on this surface.  Then $F(c(t))=0$ so $(x(t))^2 + (y(t))^2 - z(t) = 0$.  Taking the derivative of both sides  we have $2x(t)x'(t) + 2y(t)y'(t) - z'(t) = 0$.  Rewriting this as a dot product  we have $(2x(t),2y(t),-1) \cdot (x'(t),y'(t),z'(t)) = 0$ or  $\nabla F(c(t))\cdot c'(t) = 0$.  Thus for any curve $c$ on the surface, the  tangent to the curve is in the tangent plane and is orthogonal to the gradient of  $F$ evaluated at $c(t)$.   The tangent plane may now be computed by simply using  the fact that  $\nabla F(2,3,13)$ is an orthogonal vector to the plane and $(2,3,13)$ is  a point on the plane.  Of course it is the same plane we found when we worked it the other  way, treating $f$ as a function.    Once this is complete, we point out that there is nothing special about our surface or curve.  For any surface, $F(x,y,z)=k$ and any curve $c$ on the  surface, we have that $F(c(t))=k$ so $\nabla F(c(t)) \cdot c'(t) = 0$ so $\nabla F(c(t))$ must  always be a vector orthogonal to the tangent plane to $F$. \\}
\end{annotation}

%Larson
\begin{prb}
Find the equation of the plane tangent to the function $f(x,y) = 25 - x^2 - y^2$ at the point $(3,1,15).$ Sketch the graph of the function and the plane.
\end{prb}

%Larson
\begin{prb}
Find the equation of the plane tangent to the function $f(x,y) = \sqrt{x^2 + y^2}$ at the point $(3,4,5).$ Sketch the graph of the function and the plane.
\end{prb}

\textbf{Tangent Planes to Surfaces}
\begin{annotation}
\endnote{Here we spend the time to prove that the gradient of a surface is orthogonal to the surface. Let's look at an example of a tangent plane to a surface in three-space.    Suppose $F(x,y,z) = x^2 + y^2 + z^2$ and consider the surface (a sphere), $F(x,y,z)=1.$ Consider also the curve on the surface given by $\dsp \oa r(t) = (\frac{1}{2}\cos(t) , \frac{1}{2}\sin(t), \frac{ \sqrt{3}}{2}).$  First, let's verify that $\oa r$ is actually a curve on the surface by computing the composition of $F$ and $\oa r.$ We obtain, $$F(\oa r(t) ) = \left(\frac{1}{2}\cos(t)\right)^2 + \left(\frac{1}{2}\sin(t)\right)^2 + \left(\frac{ \sqrt{3}}{2}\right)^2 =  \frac{1}{4} + \frac{3}{4} = 1,$$ so $\oa r$ is a curve on the surface. Now, consider the point on the surface and on the curve at time $t = \frac{\pi}{4},$ or $\oa r(\frac{\pi}{4}).$  Can we find a vector that is orthogonal to this surface and this curve at this point?  For the sphere, the vector originating at the origin and passing through the point $\oa r(\frac{\pi}{4})$ is orthogonal to the surface, so $\dsp ( \frac{\sqrt{2}}{4}, \frac{\sqrt{2}}{4}, \frac{\sqrt{3}}{2})$ is both a point on the surface and a vector that is orthogonal to the surface.

Next, let's do something that does not immediately appear related.  Let's compose the gradient of the surface, $\nabla F,$ with the curve $\oa r.$ $$\nabla F(x,y,z)  = (2x, 2y, 2z)$$ thus $$\nabla F(\oa r (t) )  = ( \cos(t), \sin(t), \sqrt{3}).$$ Now let's compute $\oa r'.$ $$\oa r'(t) = (-\frac{1}{2}\sin(t) , \frac{1}{2}\cos(t), 0).$$ Imagine these two graphically.  Since $\oa r$ is a curve on the surface, $\oa r'$ represents a direction tangential to the curve, $r.$  And  $\nabla F$ represents the direction in which $F$ increases the most rapidly.  Does it seem natural that these two directions would be orthogonal?  Let's check. $$\nabla F(\oa r (t) ) \cdot \oa r'(t) = ( \cos(t), \sin(t), \sqrt{2}) \cdot (-\frac{1}{2}\sin(t) , \frac{1}{2}\cos(t), 0) = 0.$$

As you might expect, this is not a unique phenomena.  Any time you have a surface $F(x,y,z) = k$ and a curve $\oa r$ on that surface, you will have that $\nabla F(\oa r(t)) \cdot \oa r'(t) = 0.$ This says that $\nabla F(\oa r(t))$ is orthogonal to  $\oa r'(t)$ for every curve $\oa r$ on the surface, $F.$    In other words, $\nabla F$ is an orthogonal vector to the surface.  This is how we will define the tangent plane.  But first, let's show that it works in general. Suppose $\oa r(t)=\big(x(t),y(t), z(t)\big)$ is a parametric equation.  Suppose $F(x,y,z)=k$ is a surface.  \\
Composing: \hspace{0.4in} $$F(\oa r(t))=k$$
Differentiating: \hspace{0.2in}
$$\dsp{\frac{d}{dt} F(\oa r(t))=\frac{d}{dt} k}$$
$$\dsp{\frac{d}{dt} F(\oa r(t))=0}$$
$$\nabla F(\oa r(t)) \cdot \oa r'(t)=0$$

We conclude that: $F(\oa r(t)) \perp$ to $\oa r'(t)$ which is the tangent to $\oa r$.\\

Summarizing, if we are sitting on the point $p$ on the surface $F(x,y,z)=k$, then every curve $\oa r$ that passes through $p$ is $\perp$ to $\nabla F(p)$. The next definition uses this observation to \underline{define} the tangent plane.
As an example, we compute the tangent plane to $f(x,y) = x^2 + y^2$ at the point $(1,2,13).$ To use our definition of the tangent plane to a surface, we first rewrite $f$ as a surface.  Rewriting $f$ as $$z = x^2 + y^2$$ and yields $$x^2 + y^2 - z = 0$$ so if we define $F(x,y,z) = x^2 + y^2 - z$ then we have rewritten $f$ as the surface, $F(x,y,z) = 0.$  The gradient of the surface is orthogonal to the surface, so $(2x, 2y, -1)$ is orthogonal to the surface.  Evaluating at $(1,2,5)$ gives us our orthogonal vector, $\oa {(1,4,-1)}.$  Now we know a vector orthogonal to the plane and a point on the plane and we are done! \\}
\end{annotation}

\begin{dfn}
If $F(x,y,z)=k$ is a surface, then the \textbf{tangent plane} to F at $u=(x,y,z)$ is the plane passing through $u$ with normal vector, $\oa{\nabla F(u)}$.
\end{dfn}

\begin{prb}
Find the equation of the tangent plane to the surface $x^2-2y^2-3z^2+xyz=4$ at the point $(3,-2,-1)$.
\end{prb}

\begin{prb}
Find the equations of two lines perpendicular to the surface in the previous problem at the point $(3,-2,-1)$ on the surface.
\end{prb}

\begin{prb}
Find the equations of two lines perpendicular to the surface $z+1=ye^{y}\cos(z)$ at the point $\oa{p}=(1,0,0).$
\end{prb}

\section{Practice} \label{chap10probs}

We will not present the problems from this section, although you are welcome to ask about them in class.

\vskip .1in
\noindent
\textbf{Domains of Functions}

\begin{enumerate}
\item Find the domain of $f(x,y)= \ln(x^2+y^2 - 1).$
\item Find the domain of $\dsp g(x, y) = \invtan \Big({x \over y} \Big).$
\item Find the domain of $\dsp h(x, y) = {4 \over {|x| - |y|}}.$
\item Find the domain of $\dsp k(x,y,z)= {{2xy} \over{z^2+z -1}}.$
\item Find the domain of $\dsp \ell(x, y, z)={{xz} \over {\sqrt{1-y^2}}}.$
\item Determine the domain over which
$f(x, y) = \ln(x^3y^4)$  is continuous.
\item Determine the domain over which
$\dsp g(x, y) = {{x} \over {|x||y|}}$  is continuous.
\item Determine the domain over which
$\dsp h(x, y, z) = {{x^2 - 1} \over {z \sqrt{y^2 - 1}}}$  is continuous.
\end{enumerate}

\noindent
\textbf{Graphing Functions}

\begin{enumerate}
\item Graph $f(x, y)= 1-x^2 +y^2.$
\item Graph $g(x,y)= x+ y.$
\item Graph $\dsp h(x,y)= {{x^2} \over 9} + {{y^2} \over 4}.$
\item Graph $k(x, y)=|x|-|y|.$
\item Graph $r(x,y)= \sin(x).$
\end{enumerate}

\noindent
\textbf{Partial Derivatives and Gradients}

\begin{enumerate}
\item Let $g(x,y)=x^3 - 4 \log _7 x^2 + \invsin(xy)$ and compute $g_x.$
\item Let $g(x,y)= \dsp {{x^2} \over {\sin(xy)}}$ and compute $g_y.$
\item Let $g(x,y)= e^{xy^2}$ and compute $\nabla g.$
\item Let $g(x,y)=x \ln(y)-4xy+x$ and compute $g_x(1,1)$ and $g_y(1,1).$
\item Let $g(x,y)= 5^{x^2}y \sin(x - y)$ and compute $g_y(\pi,2\pi).$
\item Let $h(x,y,z)=\sqrt{2xz-5y}+\cos^{3}(z\sin(x))$ and compute
$\nabla h.$
\item Let $h(x, y,z) = z^{xy}$ and compute $h_x(2,3,4)$.
\item  Find $f_{xx}$, $f_{zy}$, and $f_{zxzy}$ for
$\dsp f(x,y,z)= \frac{y}{x^{3}}-\sin(zy)-3z^{3}$.
\end{enumerate}

%\item Prove that each of the following functions is differentiable at all points in its domain by using the theory discussed in this section.
%\begin{enumerate}
%\item $f(x, y) = 3^x \sin (2y) - 4^{-2y} \cos (5x)$
%\item $g(x, y) = \log(xy) + 5 \invtan(xy)$
%\item $\dsp h(x,y) = {{2x + 5y} \over {x^2 + 8xy - y^2}}$
%\end{enumerate}


\noindent
\textbf{Directional Derivatives}

\vskip .1 in
\noindent
Remember:  Directions vectors should be unit vectors.


\begin{enumerate}
\item Using the definition of directional derivative,
compute the derivative of $f(x,y) = x-y^2$ at $(1,2)$ in the
direction, $(1,1).$
\item Find $D_{\oa u}f(p)$ for
$f(x,y)=e^{xy}+2x^{2}y^{3}$; $p=(3,1)$; $\oa u$ is a unit
vector parallel to $\oa v= (-5, 12)$.
\item Find $D_{\oa u}f(p)$ for
$f(x,y)=e^{xy} + \ln(xy)$; $p=(1,2)$; $\oa u$ is a unit
vector which makes an angle $\dsp {-{\pi} \over 3}$ from the
positive $x$ axis.
\item Find $D_{\oa u}f(p)$ for
$f(x,y,z)=x^2y-4y^2z+xyz^2$; $p=(-2,1, -1)$; $\oa u$ is a
unit vector parallel to the vector $\overrightarrow{AB}$ where
$A=(2,1,-5)$ and $B=(-2,4,3)$.
\end{enumerate}

\noindent
\textbf{Derivatives}

\vskip .1 in
\noindent
For each of the following functions, compute the indicated
``derivative'' of the function. Because of the differing domains of
the functions, the derivative could be a function
(a partial derivative), a vector of functions (a gradient), or
a matrix of functions (the `total' derivative)!

\begin{enumerate}
\item $f(x,y) = x^2y^3$
\item $\dsp g(x,y) = \big( \frac{x^2}{y} - 3xy , \sin(\frac{x}{y}) \big)$
\item $h(x,y) = x^2- e^{xy\sqrt{z}} + \sinh(yz)$
\item $r(s,t,u) = \big( st, s^2tu, \sqrt{stu}, \ln(st^2u) \big)$
\end{enumerate}


\noindent
\textbf{Chain Rule}

\begin{enumerate}

\item Let $f(x,y)=2x^{2}-7y$ and $\oa g(t) = (\sin(t), \cos(t)).$
Compute the derivative of $f \circ g$ in two ways. First
compose  $f$ and $g$ and take the derivative.  Second,
apply the chain rule.  Verify that your solutions are the same.

\item Let $\dsp f(x,y)=4x^{3}y+e^{3y}+\frac{2}{x},$ $x(t)=t^{2}$,
$y(t)=4t-3,$ and $g(t)=(x(t),y(t)).$ Compute $(f \circ g)'(-1).$

\item For each of the following problems, find $\dsp \frac{dg}{dt}$
and evaluate at the given value of $t$.
\begin{enumerate}
\item $g(x,y)=3xy+e^{x}y^{2}$ where $x(t)=4t^{2}+t$, $y(t)=6+5t$,
and $t=0$
\item $\dsp g(x,y,z)=x^{3}y+xz+\frac{x}{y-z}$ where $x(t)=t^{2}+3$,
$y(t)=4t-t^2$, $z(t)= \cos(t-3 \pi)$, and $t=0$
\end{enumerate}


\item Let $f(x,y)=xy \ln(x)$ and $\oa g(s,t)=(2st, t-s^{3}).$
State the domain $f,$ $g,$ and $f \circ g.$
Compute the gradient of $(f \circ g).$

\item For each of the following problems, find $f_s$ and $f_t$
(i.e. $\dsp {{\partial f} \over {\partial s}}$ and
$\dsp {{\partial f} \over {\partial t}}$).
\begin{enumerate}
\item $f(x,y)=2x-y^2$, $x(s, t)=s \cos(t)$, and $y(s, t)=(s+t)e^t$
\item $f(x, y, z) = (x + 2y + 3z)^4$ and $x(s, t) = s + t$,
$y(s, t) = s-t$, $z(s, t)=st$
\end{enumerate}

\item For each of the following problems, find
$\dsp {{\partial g} \over {\partial u}},$
$\dsp {{\partial g} \over {\partial v}},$ and
$\dsp {{\partial g} \over {\partial w}}$
(i.e. $g_u$, $g_v$, and $g_w$).
\begin{enumerate}
\item $g(x, y) = (x+y) \ln(xy)$, $x(u, v, w) = u+v-3w$, and
$y(u, v, w) = uv+3w$
\item $g(x, y, z) = yz+xz+xy$, $x(u, v, w)=u+v-3vw$,
$y(u, v, w)=v+w+4uw$, and $z(u, v, w)=u+w-5uv$
\end{enumerate}

\item For each of the following three problems, find
$\dsp {{\partial z} \over {\partial x}}$ and
$\dsp {{\partial z} \over {\partial y}}$ at the indicated point.
\begin{enumerate}
\item $z^3 -xy+2yz+y^3-3=0$; $(1, 1,1)$
\item $\dsp {1 \over {x^2}}+{1 \over {y^2}} + {1 \over {z^2}} = {{49}
\over {36}}$; $(-1, 2, 3)$
\item $ye^{xyz} \cos(2xz) = 1$; $(\pi, 1, 4)$
\end{enumerate}

\end{enumerate}

\noindent
\textbf{Tangent Lines and Planes}

\begin{enumerate}

\item Find the (shortest) distance from the point $(0, 1, 0)$ to the plane $x + 2y+ 3z=4$.

\item Find the equation of the tangent plane to the function $z=x^{2}-4y^{2}$ at $(3,1,5).$

\item Find the equation of the tangent plane to $z+1=xe^y \cos(z)$ at the point $(1,0,0).$

\item Find the equation of the line normal (perpendicular) to $x^2+2y^2 + 3z^2 = 6$ and passing through $(1,-1,1).$

\item Find the tangent plane approximation of $h(x,y)=x+x \ln(xy)$ when $x=e$ and $y=1$.

\end{enumerate}


\vskip .5in
\noindent
\textbf{Chapter 10 Solutions}\\


\noindent
\textbf{Domains of Functions}

\begin{enumerate}
\item $\{ (x,y) \in \re^2 : x^2 + y^2 > 1\}$
\item  $\{ (x,y) \in \re^2 : y \neq 0 \}$
\item   $\{ (x,y) \in \re^2 : |x| \neq |y| \}$
\item all of $\re^3$ except where $z$ equals the Pisot numbers
\item   $\{ (x,y) \in \re^2 : |y| < 1 \}$
\end{enumerate}

\noindent
\textbf{Graphing Functions}

\begin{enumerate}
\item a saddle centered at $(0,0,1)$
\item a plane
\item squished paraboloid
\item planar saddle with a `point' at $(0,0,0)$
\item a three dimensional sine wave
\end{enumerate}

\noindent
\textbf{Partial Derivatives  and Gradients}

\begin{enumerate}
\item $\dsp g_x(x,y) = 3x^2 - \frac{8}{x\ln(7)} + \frac{y}{\sqrt{1-(xy)^2}}$
\item  $\dsp g_y(x,y) = \frac{-x^3\cos(xy)}{\sin^2(xy)}$
\item $\nabla g(x,y) = ( y^2e^{xy^2}, 2xye^{xy^2} )$
\item $g_x(1,1)=g_y(1,1) = -3$
\item $g_y(\pi,2\pi)= 2\pi 5^{\pi^2}$
\item $\dsp \nabla h(x,y,z) = \big( \frac{z}{\sqrt{2xz-5y}} - 3z\cos(x)\cos^2(z\sin(x))\sin(z\sin(x)), \frac{-5}{\sqrt{2xz-5y}},$\\
$\dsp \frac{x}{\sqrt{2xz-5y}} - 3\sin(x)\cos^2(z\sin(x))\sin(z\sin(x))  \big)$
\item $h_x(2,3,4) \approx 17,035$
\item $\dsp f_{xx}=\frac{12y}{x^5}, f_{zy} = -\cos(zy)+zy\sin(zy), f_{zxzy}=0$
\end{enumerate}

%\item Prove that each of the following functions is differentiable at all points in its domain by using the theory discussed in this section.
%\begin{enumerate}
%\item $f(x, y) = 3^x \sin (2y) - 4^{-2y} \cos (5x)$
%\item $g(x, y) = \log(xy) + 5 \invtan(xy)$
%\item $\dsp h(x,y) = {{2x + 5y} \over {x^2 + 8xy - y^2}}$
%\end{enumerate}

\newpage
\noindent
\textbf{Directional Derivatives}

\vskip .1 in
\noindent
Remember:  Directions vectors should be unit vectors.

\begin{enumerate}
\item $\frac{-3}{\sqrt{2}}$
\item $\frac{1}{13}(588 + 31e^3)$
\item $(1-\frac{\sqrt{3}}{2})(e^2 + \frac{1}{2}))$
\item $\frac{42}{\sqrt{89}}$
\end{enumerate}

\noindent
\textbf{Derivatives}

\begin{enumerate}
\item $\nabla f(x,y) = (2xy^3,3x^2y^2)$
\item $Dg(x,y) =
\begin{pmatrix} \frac{2x}{y} - 3y & - \frac{x^2}{y^2} - 3x \cr
\frac{1}{y}\cos(\frac{x}{y}) & - \frac{x}{y^2}\cos(\frac{x}{y}) \end{pmatrix} $
\item $\nabla h(x,y) =
(2x - y\sqrt{z}e^{xy\sqrt{z}}, -x\sqrt{z}e^{xy\sqrt{z}} + z \cosh(yz) )$
\item $\dsp Dr(s,t,u) =
\begin{pmatrix}
t & s & 0 \cr
2stu & s^2u & s^2t \cr
\frac{tu}{2\sqrt{stu}} & \frac{su}{2\sqrt{stu}} & \frac{st}{2\sqrt{stu}} \cr
\frac{t^2u}{st^2u} &
\frac{2stu}{st^2u} &
\frac{st^2}{st^2u}
\end{pmatrix}$
\end{enumerate}


\noindent
\textbf{Chain Rule}

\begin{enumerate}

\item $(f \circ g)'(t)  = \sin(t)(4\cos(t)+7)$

\item $(f \circ g)'(-1) \approx 188$

\item (a) $84$

\item $\nabla (f \circ g) =
( (2t^2-8s^3t)\ln(2st)+(2st^2-2s^4t)\frac{1}{s},
(4st-2s^4)\ln(2st) + (2st^2-2s^4t)\frac{1}{t}
)$

\item (a) $f_s(s,t)=2\cos(t)-2(s+t)e^{2t}$ and
$f_t(s,t) = -2s \sin(t) - 2(s+t)e^{2t} -2(s+t)^2e^{2t}$

\end{enumerate}

\noindent
\textbf{Tangent Lines and Planes}

\begin{enumerate}

\item $\dsp \frac{\sqrt{14}}{7}$

\item $6x-8y-z=5$

\item  $x + y - z = 1$

\item  $L(t) = (2,-4,6)t + (1,-1,1)$

\item $3x + ey - z = 2e$

\end{enumerate}


\chapter{Optimization and Lagrange Multipliers}

``The wise man doesn't give the right answers, he poses the right questions.''  - Claude Levi-Strauss\\ \\

Probably the most applied concept in all of calculus is finding the maxima and minima of functions.  In industry, these can be problems from engineering, such as trying to design an airplane wing that yields the maximum lift and stability while at the same time minimizing the drag coefficient. This way, we build a plane that flies easily while using less fuel.  Since planes measure fuel consumption in gallons per second, a small change in wing design can result in considerable profit for the company (and a big raise for you).  Have you noticed the addition of the upward turned tips at the ends of the airplane wings in recent years?

In the financial markets, mutual funds are sets of stocks.  People may buy shares of the fund instead of buying shares of individual stocks. Mutual fund managers (and their clients) want to choose groups of stocks that will increase in value and make them (and their clients) rich.  Thus, a fund manager wants to design a mutual fund that will maximize profits for the investors, but because investors fear volatility (large fluctuations in the value of their portfolios), the manager also wants to minimize the volatility of the mutual fund.  This is an example of an optimization problem with what is called a {\it constraint} because you want to maximize profit but are constrained by the customers' concerns about volatility.

Mathematicians have spent a considerable amount of time in industry working on both of these interesting problems that are representative of ``real-world'' applications. In mathematics, the difference between being able to understand or apply a formula to such a problem and the ability to derive or create your own formulas for the problems is the difference between working as an engineer, mutual fund manager, or biologist on a team (a wonderful job in it's own right) and working in a think tank such as Bell Labs or Los Alamos or MSRI (the Mathematical Sciences Research Institute) where you are tackling the problems that no one can solve and creating the mathematics that will be implemented by the teams in industry.

In Calculus I you solved \emph{optimization} or \emph{max-min} problems by setting the derivative of a function to zero to tell you where the rate of change (or slope) of the function was zero.  In Calculus III we do exactly the same thing.   Except that instead of solving $f'=0$ we are solving $\nabla f=0$ and we are seeking a horizontal tangent \emph{plane} instead of a horizontal tangent \emph{line}. The procedures are very similar and both find the point in the domain of the functions where potential maxima and minima are attained.

\begin{dfn}
A \textbf{critical point} of $f:{\re}^2 \to \re $ is any point $x \in \re^2$ where $\nabla f(x)$ is zero or undefined.
\end{dfn}

Since $\nabla f = 0$ translates to $\nabla f(x,y) = (f_x (x,y), f_y (x,y)) = (0,0)$ we are seeking points $(x,y)$ in the plane where both $f_x(x,y) =0$ and $f_y(x,y) = 0$, or where at least one is undefined. Review the definitions for ``local minimum, local maximum, and inflection point'' for functions from $\re$ to $\re$ (i.e. from Calculus I).  You may look these up in a book or on the web.

Recall from Definition \ref{nhood} that an $\epsilon$ - \textbf{neighborhood } of the point $(s,t)$ in $\re^2$ is the set of all points in $\re^2$ that are a distance of less than $\epsilon$ away from $(s,t).$

\begin{dfn}
If $f:\re^2 \to \re$ and $x$ is in the domain of $f$, then we say $(x,f(x))$ is a \textbf{local minimum} if $f(x)<f(y)$ for every $x$ in some $\epsilon$-neighborhood of $x$.
\end{dfn}

Local maximum is defined similarly. A function may have infinitely many local minima and maxima.

%I think I stole the upcoming problem from Harvard
\begin{prb}
\label{minimum}
Find the minimum of $f(x,y)=x^2-2x+y^2-4y+5$ in two different ways.
\begin{enumerate}
\item First, set $\nabla f(x,y)=0$ and solve for $(x,y)$.
\item Second, complete the square to write $f$ as $f(x,y) = (x-a)^2 + (y-b)^2$ and graph.
\end{enumerate}
\end{prb}

\begin{prb}
Let $f(x,y)=8y^3+12x^2-24xy$.
\begin{enumerate}
\item Find all critical points of $f.$
\item Sketch $f$ using any software to verify your answers.
\item Use any software to solve the whole problem.  In other words, use software to compute your partial derivatives and solve for the roots of these partial derivatives.
\end{enumerate}
\end{prb}

Recall in Calculus I that if $x$ were a critical point for $f$ then $(x,f(x))$ could have been a minimum, maximum, or an inflection point for $f.$  Inflection points were critical points where the function $f$ switched concavity (i.e. where the first derivative is zero and the second derivative switches signs).  In Calculus III the analogous points are critical points that are neither maxima nor minima and we call these {\it saddle points}.

\begin{dfn}
If $f:\re \to {\re}^2$ is differentiable at $x$, then we say $(x,f(x))$ is a \textbf{saddle point} if $\nabla f(x)=0$ and no matter how small an $\epsilon >0$ we choose, there are points $y$ and $z$ in the $\epsilon$-neighborhood of $x$ so that $f(x)<f(y)$ and $f(x)>f(z)$.
\end{dfn}

This definition of a {\it saddle point} says that if we are at a saddle point and we decide to walk away from it, then
there are paths away from the critical point along which $f$ increases and paths away from the point along which $f$ decreases. Given a critical point, how do we determine if it was a maximum, minimum, or saddle point?  That is, how do we {\it classify} the critical points of $f$?    How did we do it in Calculus I? We restate the {\it Second Derivative Test} which is a slick way to classify the the critical points of the single-variable, real-valued functions from Calculus I.  Notice how nicely it parallels the next theorem for classifying the multi-variable, real-valued functions of Calculus III.

\begin{thm}
\textbf{Second Derivative Test I.} If $f: \re \to \re$ is differentiable and $f'(x)=0$ then,
\begin{enumerate}
\item If $f''(x) > 0$, then $(x,f(x))$ is a minimum.
\item If $f''(x) < 0$, then $(x,f(x))$ is a maximum.
\item If $f''$ switches signs at $x$, then $(x,f(x))$ is an inflection point.
\end{enumerate}
\end{thm}

Here is a sweet theorem for classifying critical points of functions of two variables.

\begin{thm}
\textbf{Second Derivative Test II.}
If $f:{\re}^2 \to \re$ and $\nabla f(u)=0$ and $D=f_{xx}(u)f_{yy}(u)-(f_{xy}(u))^2$, then
\begin{enumerate}
\item $(u,f(u))$ is a local min if $D>0$ and $f_{xx}(u)>0$.
\item $(u,f(u))$ is a local max if $D>0$ and $f_{xx}(u)<0$.
\item $(u,f(u))$ is a saddle point if $D<0$.
\item No information if D=0.
\end{enumerate}
\end{thm}

\begin{expl}
Classify all critical points of a function $f$ with gradient $\nabla f(x,y) = (xy-2x-3y+6, yx-y+4x-4).$
\begin{annotation}
\endnote{As a class exercise, we attempt to compute and classify all critical points of a function $f$ with gradient $\nabla f(x,y) = (xy-2x-3y+6, yx-y+4x-4).$  The problem is intentionally misleading. We quickly solve for the critical points, but when we attempt to classify them by computing the determinant of the Hessian, $\begin{pmatrix} f_{xx} & f_{xy} \cr f_{yx} & f_{yy} \end{pmatrix}$ we find that $f_{xy} \neq f_{yx}$.  We ask, ``How could this happen?  Don't we have a theorem that says that $f_{xy} = f_{yx}$?'' After some thought, someone will suggest that perhaps there is not a function with that gradient and we integrate the first component of our gradient with respect to $x$ and the second component of the gradient with respect to $y$ to demonstrate that in fact, there is no function having this gradient.  This provides an example for solving two equations in two variables and foreshadows the proof technique needed later to show that a vector field $f = (P,Q)$ is conservative if $P_y = Q_x$.}
\end{annotation}
\end{expl}

%Mandy- VP problem
\begin{prb}
Compute the critical points of $f(x,y)=xy^2-6x^2-3y^2$ and classify these critical points as local maxima, local minima, or
saddle points.
\end{prb}

%Mandy VP problem
\begin{prb}
Compute and classify the critical points of $f(x,y)=xy+\dsp{\frac{2}{x}+\frac{4}{y}}$.
\end{prb}

%Mandy-VP problem
\begin{prb}
Compute and classify the critical points of $\dsp f(x,y)=e^{-(x^2+y^2-4y)}$.
\end{prb}

The next problem reminds you of an important aspect of max/min problems from Calculus I.  If you wanted the local maxima and minima of a function like $f(x) = x^2$ on $[-1,3]$, then you checked not only the places where $f'=0$ but also the values of $f$ at the endpoints (boundary) of the interval.  The same must be done for functions of several variables.  When you applied this technique, you were applying the Extreme Value Theorem.

\begin{thm}
\textbf{Extreme Value Theorem for Functions of One Variable.} If $f: [a,b] \to \re$ is a differentiable function, then $f$ has a maximum and a minimum on this interval.
\end{thm}

The corresponding theorem for real-valued functions of two variables  requires at least an intuitive idea of the notions of what it means for a set to be {\it closed} and {\it bounded.}  A set $S$ is \textbf{bounded} if there is a number $M$ so that $| x | \le M$ for all $x \in S.$ A set $S$ is \textbf{closed} if it contains its boundary points. Think of the open and closed intervals.  An open interval does not contain its end points, the points on the boundary of the set.  A closed interval does contain its boundary points.  The set of all points $(x,y)$ satisfying $x^2 + y^2 \leq 9$ is closed, while the set of all points $(x,y)$ satisfying $x^2 + y^2 < 9$ is not closed.

\begin{dfn}
A set $S$ in $\re^2$ is \textbf{closed} if it contains all its boundary points.
\end{dfn}

\begin{thm} \textbf{Extreme Value Theorem for Functions of Two Variables.}
If $M$ is a closed and bounded set and $f: M \to \re$ is a differentiable function, then $f$ has a maximum and a minimum on $M$.
\end{thm}

\begin{expl}
Optimize $f(x,y) = 2x^2 + y^2 + 1$ subject to $g=0$ where $g(x,y) = x^2 + y^2 - 1$ in four ways!
\begin{annotation}
\endnote{Somewhere before or after this problem is attempted, I deliver this seminal lecture which reinforces then notions of composition of functions, graphing and differentiation while introducing Lagrange multipliers.  Our goal is to optimize $f(x,y) = 2x^2 + y^2 + 1$ subject to $g=0$ where $g(x,y) = x^2 + y^2 - 1$ and we solve the problem in four distinct ways:
\begin{enumerate}
\item We solve $g=0$ for $y$, substitute the result into $f$ and optimize the resulting function of one variable.
\item We parameterize $g=0$ as $c(t) = (\cos(t), \sin(t))$ and optimize the composition $f \circ c$.
\item We graph $f$ and locate the extrema visually.
\item We apply Lagrange Multipliers, saving the theory of why it works for later.
\end{enumerate}
Because we solve for $y$, Method 1 finds only two of the critical points unless we go back and solve $g=0$ for $x$ and repeat the process.  I don't tell them this.  We simply move on to solve it another way, confident that our two extrema are correct.  Method 2 then surprises us when we find four critical points. At this point, we may revisit Method 1, asking how we can find the other two.  Whether we go back or not depends on the students' questions. Method 3, graphing, then verifies that there are exactly four solutions since the graph of $f$ constrained by $g$ looks like a Pringle potato chip with two maxima and two minima.  At this point, someone usually realizes that we would have found all critical points using Method 1 if we had solved for $x$ as well but all these methods seem onerous even for such a simple problem.

Method 2 foreshadows the proof that the method of Lagrange Multipliers works. Consider functions $f:\re^2 \to \re$ and $g: \re^2 \to \re$, both differentiable everywhere.  If we parameterize $g=0$ as $r(t) = (x(t),y(t))$, then our critical points will occur at any point $t$ in the domain of $r$ where $(f \circ r)'=0$ or where $\nabla f (r(t)) \cdot r'(t)=0$.   Since $r$ is a parametrization of $g=0$, we know that $g(r(t))=0$ and thus $\nabla g (r(t)) \cdot r'(t) =0$.  Thus, both $\nabla f (r(t))$and $\nabla g (r(t))$ are orthogonal to the same vector $r'(t)$ and therefore they must be scalar multiples of one another.  Hence we have Lagrange's Theorem, that solving the system, $\nabla f = \lambda \nabla g$ and $g=0$ is sufficient to find all critical points.}
\end{annotation}
\end{expl}

\begin{prb}
\label{temp}
Let $T(x,y)=2x^2+y^2-y$ be the temperature at the point $(x,y)$ on the circular disk of radius 1 centered at $(0,0).$
\begin{enumerate}
\item Find the critical points of T.
\item Find the minimum and maximum of T over the circle (perimeter), $x^2+y^2=1$ by parameterizing the circle.
\item Find the minimum and maximum of T over the disk, $x^2+y^2 \leq 1.$
\end{enumerate}
\end{prb}

For the next problem, you will need to parameterize each of the four line segments that form the line and check the maximum and minimum of the function over not only the interior of the square, but also over each of the four lines.

\begin{prb}
Find the maximum and minimum of $f(x,y) = 2x^2 - 3y^2 + 10$ over the square disk,  $S = \{ (x,y) | 0 \leq x \leq 3, 2 \le y \le 4 \}.$
\end{prb}

\begin{prb}
Find all maxima and minima of $f(x,y)=x^2-y^2+4y$ over the rectangle  $R = \{ (x,y) | -1 \leq x \leq 1, -3 \le y \le 3 \}.$
\end{prb}

\begin{prb}
Find all maxima and minima of $f(x,y)=2x^2 + 3y^2 +2$ over all points on or inside the triangle with vertices, $(-2,0)$, $(0,2)$, and $(2,0)$.
\end{prb}

In Calculus I an $n^{th}$ degree polynomial will have at most $n-1$ critical points. What about in Calculus III?  Here is a $6^{th}$ degree polynomial which has far more than 5 critical points.  All of these may be found by hand and easily seen when graphed.

%mclaughlin
\begin{prb}
Find all thirteen critical points of $f(x,y) = x^3y^3 - x^3y -3xy^3 + 3xy +1.$
\end{prb}

\begin{prb}
Let $f(x,y) = x^3 - y^3$ and $p = (2,4).$  Find the direction in which $f$ increases the most rapidly.
\end{prb}

%vp
\begin{prb}
Ted is riding his mountain bike and is at altitude (in feet) of $A(x,y) = 5000e^{-(3x^2+y^2)/100}.$  What is my slope of descent or ascent if I am riding in the direction $\oa {(-1,1)}$ starting at the point, $(10,10, 5000e^{-4})?$  In what direction should I travel to ascend the most rapidly?  To descend the most rapidly?
\end{prb}

%vp
\begin{prb}
Let $\dsp T(x,y,z)  = \frac{10}{x^2 + y^2 + z^2}$ and $\dsp \oa r(t) = (t \cos(\pi t) , t \sin(\pi t) , t ).$ If $T(x,y,z)$ represents the temperature in space at the point $(x,y,z)$ and $\oa r(t)$ represents Ted's position at time t, then compute $(T \circ \oa r)'(3)$ and explain what this number represents in a complete sentence.  To check your answer, compute it in two ways. First compose the functions and take the derivative.  Second, use the chain rule.
\end{prb}

We have tackled optimization problems before when we found maxima and minima of functions like $f(x,y)=x^2+x^2y+y^2+4$
from Problem \ref{minimum}. We have also sought maxima and minima of curves (or paths) on surfaces when we found the maxima and minima of $T(x,y)=2x^2+y^2-y$ over the circle $x^2+y^2+1$ as in Problem \ref{temp}.  Problem \ref{temp} is called a \textbf{constrained} optimization problem because we want a maxima or a minima of $T$ subject to the constraint that it is above the unit circle. The method of Lagrange multipliers is a slick way to tackle constrained optimization problems.

\begin{thm} \textbf{Lagrange Multipliers.}
Suppose that $f, g: \re^2 \to \re.$ To maximize (or minimize) the function $f$ subject to the constraint $g=0$  we solve the two equations,
\begin{enumerate}
\item $\nabla f(x)=\lambda \nabla g(x)$
\item $g(x)=0$
\end{enumerate}
for $x$ and for $\lambda.$ The variable $\lambda$ is called the Lagrange multiplier, and $x$ is the point at which $f$ is maximized (or minimized).
\end{thm}

\begin{expl}
Minimize the function $f(x,y) = y^2-x^2$ over the region $g(x,y)\leq 0$ where $g(x,y) = \frac{x^2}{4}+y^2 -1.$
\begin{annotation}
\endnote{Experience says that we have not seen enough examples of Lagrange Multipliers, so this is a stock example, I show.}
\end{annotation}
\end{expl}


\begin{prb}
Find any possible maxima or minima of $f(x,y)=x^2+y^2$ subject to $xy=3$ in three ways.  First use Lagrange multipliers by putting $g(x,y) = xy-3$ so that the constraint is $g(x,y) = 0.$  Second, substitute $y=3/x$ into the equation and solve.  Third, sketch $f$ and identify the portion of $f$ that is above the equation, $xy=3.$
\end{prb}

\begin{prb}
Find any minima and maxima of $f(x,y)=4x^2+y^2-4xy$ subject to $x^2+y^2=1$.  It may be helpful to eliminate $\lambda$ first.
\end{prb}

\begin{prb}
Find the minimum of $f(x,y,z)=3x+2y+z$ subject to $9x^2+4y^2-z=0$ via Lagrange multipliers.
\end{prb}

\section{Practice} \label{chap11probs}
%mostly from shing

We will not present the problems from this section, although you are welcome to ask about them in class.

\vskip .1in
\noindent
\textbf{Critical Points}

\begin{enumerate}

\item For each of the following functions, find the critical points
and determine if they are maxima, minima, or saddle points.
\begin{enumerate}
\item $f(x,y)=1-x^{2}-y^{2}$
\item $g(x,y)=e^{-x} \sin y$
\item $F(x,y)=2x^{2}+2xy+y^{2}-2x-2y+5$
\item $g(x,y)=x^{2}+xy+y^{2}$
\item $z=8x^{3}-24xy+y^{3}$
\end{enumerate}

\item For each of the following functions, find the absolute extrema
of the function on the given closed and bounded set $R$ in $\re ^2$.
\begin{enumerate}
\item $f(x,y)=2x^{2}-y^{2}$; $R = \{(x, y): x^{2}+y^{2}\leq 1 \}$
\item $g(x,y)=x^{2}+3y^{2}-4x+2y-3$; $R = \{(x, y): 0 \le x \le 3$,
$-3 \le y \le 0 \}$
\end{enumerate}

\item Find the direction at which the maximum rate of change of
$g(x,y)= \ln(xy)-3x+2y$ at $p=(3,2)$ will occur and find the
maximum rate of change.

\item Find the direction in which the function $f(x,y)=x^3-y^5$
increases the fastest at the point $(2, 4)$.

\item For each of the following four problems, find all critical
points of $f$ and classify these critical points as relative maxima,
relative minima, or saddle points using the second derivative test
whenever possible.
\begin{enumerate}
\item $f(x,y)=xy^2-6x^2-3y^2$
\item $\dsp f(x,y)=\frac{9x}{x^{2}+y^{2}+1}$
\item $\dsp g(x,y)=x^{2}+y^{3}+\frac{768}{x+y}$
\end{enumerate}

\end{enumerate}

\noindent
\textbf{Optimization and Lagrange Multipliers}

\begin{enumerate}

\item Find all extrema of $f(x,y)=1-x^{2}-y^{2}$ subject to
$x+y=1$, $x\geq 0$, and $y\geq 0$.

\item Find all extrema of $f(x,y)=1-x^{2}-y^{2}$ subject to
$x+y \leq 1$, $x\geq 0$, and $y\geq 0$.

\item Find the absolute extrema of the function $g(x,y)=2 \sin(x)+5 \cos(y)$
on the rectangular region $R = \{(x, y): 0 \le x \le 2, 0 \le y \le 5 \}$.
\item Find the minimum value of $z=x^{2}+y^{2}$ subject to $x+y=24$.
\item Find the extreme values of $f(x, y) = 2x^2 +y^2-y$ subject
to $x^2+y^2=4$ using Lagrange multipliers.

\item Find three positive numbers whose sum is 123 such that their product is as large as possible.

\item A container in $\re^{3}$ has the shape of a cube with each
edge length 1. A (triangular) plate is placed in the container so
that it intersects the cube in the plane $x+y+z=1$. If the
container is heated so that the temperature at each point is given
by $T(x,y,z)=4-2x^{2}-y^{2}-z^{2}$ in hundreds of degrees, what are
the hottest and coldest points on the plate?

\item A company has three production plants, each manufacturing the
same product. If plant A produces $x$ units at the cost of
$\$(x^2+2,000)$, plant B produces $y$ units at the cost of
$\$(2y^2+3,000)$, and plant C produces $z$ units at the cost of
$\$(z^2+4,000)$. If there is an order for 11,000 units to be filled,
determine how the production should be arranged among these three
plants so that the total production cost can be minimized.

\end{enumerate}


\vskip .5in
\noindent
\textbf{Chapter 11 Solutions}\\

\noindent
\textbf{Critical Points}

\begin{enumerate}

\item
\begin{enumerate}
\item $(0,0,1)$ is a max
\item none, why?
\item $(0,1, 4)$ is a min
\item $(0,0,0)$ is a min
\end{enumerate}

\item
\begin{enumerate}
\item $(\pm 1,0,2)$ are local minima and $(0,\pm 1, -1)$ are local
maxima
\item all points to consider should be:  $(2,-1/3)$, $(0,0$), $(3,0)$, $(3,-3)$, $(0,-3)$, $(2,0)$, $(0,-1/3)$, $(3,-1/3)$, and $(2,-3)$; $(2,-1/3,-22/3)$ is a min; $(0,-3,18)$ is a max
\end{enumerate}

\item $(-8/3, 5/2)$ and $\sqrt{481}/6$
\item  $(3,-320)$

\begin{enumerate}
\item $(3,\pm6,-54)$ are saddles and $(0,0,0)$ are all maxima
\item $(0,1,9/2)$  (D is long -- use software!)
\item icky algebra -- use software!
\end{enumerate}

\end{enumerate}

\noindent
\textbf{Optimization and Lagrange Multipliers}

\begin{enumerate}

\item $(1/2,1/2,1/2)$ is the max, each of $(1,0,0)$ and $(0,1,0)$ is a min
\item $(0,0,1)$ is the max, each of $(1,0,0)$ and $(0,1,0)$ is a min
\item there is one saddle on the interior, but no extrema on the interior; the
extrema occur on the boundaries at
$(0,0),$ $(\pi/2,5),$ $(2,0),$ and $(\pi/2,0)$
\item $(12,12,288)$
\item $(0, \pm 2)$ and $\dsp (\pm\frac{\sqrt{15}}{2},1/2)$
\item $x=41, y = 41, z= 41$

\end{enumerate}


\chapter{Integration}

``You cannot teach a man anything; you can only help him find it within himself.'' - Galileo Galilei\\ \\

Calculus III  continues to parallel Calculus I and here we are at integration.  Let's review integration in one variable
before we tackle integration in several variables. By now you have been using anti-derivatives (and hence the Fundamental Theorem of Calculus) to compute integrals for so long that it is worth remembering the original definition for the definite integral.   If $f:\re \to \re$ is a function, then we define $\dsp \int_a^b f(x) \  dx$ as the limit of Riemann sums.  If $f$ is positive, then this is the limit of sums of areas of rectangles.  Now we will define our integrals once again as limits of sums, but this time we will take limits of sums of volumes.

First, let's formally restate the definition of the definite integral from Calculus I.
\begin{annotation}
\endnote{To introduce multi-variable integration, we launch from the roots of Calculus I by reviewing the definition of a the definite integral as the limit of sums of rectangles, and computing $\int_0^2 x^2 \; dx$ using summation notation and proving that the limit of the sums of the areas of the rectangles is $\frac{8}{3}$.  Now that we truly \emph{know} what the area is, we recompute it in several ways using the Fundamental Theorem by showing that $$\int_0^2 x^2 \; dx = \int_0^2 \int_0^{x^2} 1 \; dy \; dx =  \int_0^4 \int_{\sqrt{y}}^4 1 \; dx \; dy.$$ To explain the integrand of ``1'', we sketch in three-space the solid of height one above the region bounded by $y=x^2$, $y=0$, and $x=4$.  The volume of this solid equals the area we first computed. We intentionally leave out integrating with respect to $y$ because this is a forthcoming problem. Finally, we define the double integral as it appears in the notes below as a limit of sums of volumes of right cylinders with square bases.}
\end{annotation}

\begin{dfn}
If $[a,b]$ is a closed interval, then a partition P of $[a,b]$ is an ordered sequence $a=x_{0}<x_{1}<x_{2}< \ldots <x_{n-1}<x_{n}=b$.  The \textbf{norm} or \textbf{mesh} of P is $max\{x_{i}-x_{i-1}:i=1,2,\ldots , n\}$ and is denoted $\| P \|$.
\end{dfn}

\begin{dfn}
The \textbf{integral of f over $[a,b]$}, denoted by $\dsp{\int_{a}^b f}$ is defined (assuming the limit exists) as $$\dsp{\int_{a}^b f}=\dsp{\lim_{\| P \| \to 0} \ \sum_{i=1}^{n} f(\hat{x_i})\cdot (x_{i}-x_{i-1})}$$ where $\hat{x_i}\in [x_{i-1}, x_{i}]$ and $n$ is the number of divisions of the partition $P$.  If the limit does not exist, we say that $f$ is \textbf{not} integrable.
\end{dfn}

Of course, this is the limit of the sum of the areas of a collection of rectangles where the width of the rectangles tends
toward zero as the mesh of the partition tends toward zero. In Calculus III, we do the same except that we must partition both the x and y-axes and we are summing volumes over rectangles rather than summing areas over intervals.

\begin{dfn}
Given any two sets, $A$ and $B$, the \textbf{Cartesian Product} of $A$ and $B$ is the set $$A \times B = \{ (a,b) : a \in A \mbox{ and } b \in B \}.$$ In the case where $A$ and $B$ are closed intervals these are simply rectangles in the plane, $$[a,b] \times [c,d] = \{(x,y) \mid x\in [a,b], y\in [c,d]\  \}$$
\end{dfn}

\begin{dfn}
If R is the rectangle $R=[a,b]\times [c,d]$, then a \textbf{partition, U, of R} is a partition of $[a,b]$, $P=\{a=x_{0}<x_{1}< \ldots <x_{n}=b\}$ along with a partition of $[c,d], Q =\{c=y_{0}<y_{1}<y_{2}< \ldots <y_{m}=d\}$.  The \textbf{norm} or \textbf{mesh} of U is the largest width of any of the rectangles, $[x_{i-1},x_i] \times [y_{j-1},y_j]$ where $i = 1,2,\dots n$ and $j=1,2,\dots m$.  The norm of $U$ is denoted by $\|U\|$.
\end{dfn}

\begin{dfn}
If $f \in C_{R}$  (i.e. f is continuous on the rectangle R), then $$\int _{R} f \  dA = \dsp{\lim_{\| U \|  \to 0} \
\sum_{i=1}^n \sum_{j=1}^m f(\hat{x_i}, \hat{y_j}) (x_{i}-x_{i-1})(y_{j}-y_{j-1})}$$ where $$(\hat{x_i}, \hat{y_j}) \in [x_{i-1}, x_{i}] \times [y_{j-1}, y_{j}].$$
\end{dfn}

Thankfully, in Calculus III we won't have to compute any such limits because there is a theorem that states:
$$\dsp{\int_{R} f \  dA = \int _{c}^d \int_{a}^b f(x,y)\ dx\ dy}$$
which gives us a straightforward way to compute integrals using anti-derivatives by using the Fundamental Theorem of Calculus from Calculus I:

\begin{thm}
\textbf{Fundamental Theorem of Calculus I.} If $f$ is continuous on $[a,b]$ and $F$ is any anti-derivative of $f$, then $\dsp{\int_{a}^b f=F(b)-F(a)}$.
\end{thm}

The next problem reminds us that in Calculus I we can integrate with respect to either $x$ or $y$ and obtain the same result. These techniques are especially handy in Calculus III.

\begin{prb}
Compute the area bounded by $y=x^2$, the x-axis, and $x=2$ two ways. First, use an integral with respect to x, $\dsp{\int_{\_}^{\_} \_\_\_ \  dx}$, then use an integral with respect to y, $\dsp{\int_{\_}^{\_} \_\_\_ \  dy}$. Sketch a picture to help explain your endpoints of integration.
\end{prb}

Most of the theorems that you proved in Calculus I have an analog in  Calculus III.  The following theorem provides a list of properties for double integrals that won't surprise you.  We won't state them again, but they also hold for any integrable function $f: \re^n \to \re$ where $n > 2.$

When we are integrating over a two dimensional region R, we will write $\int_R f \ dA$, where the $A$ reminds us that we are integrating over a region with \emph{area}.  This will be a double integral. When we are integrating over a three dimensional region, we will write $\int_R f \ dV$, where the $V$ reminds us that we are integrating over a region with \emph{volume}. This will be a triple integral.
\begin{annotation}
\endnote{In order to expedite multiple integrals, we discuss the notations $dA$ and $dV$.  We also work through several examples of integrals although they are unlikely to all occur in one lecture.
\begin{enumerate}
\item We use the fundamental theorem to demonstrate Fubini's theorem by integrating $\dsp \int_0^2 \int_1^3 x^2 + y^2 \; dx \; dy$  and  $\dsp \int_1^3 \int_0^2 x^2 + y^2 \; dy \; dx$ and illustrate graphically the volume that these integrals represent.
\item We graph and find the volume under $4x+3y+6z=12$ and above $xy$-plane.
\item We compute $\dsp \int_R x^2 + y^2 \; dA$ over $R= [0,1] \times [1,2]$
\item We compute $\dsp \int_R x^2 y \; dA$ over the region $R$ bounded by $y=x^2, x=0$ and $y=4.$
\end{enumerate}}
\end{annotation}

\begin{thm}
\textbf{Properties of the Integral.} Suppose $f$ is integrable on a closed and bounded rectangular region $R$. Then
\begin{enumerate}
\item $\dsp \int_{R}[f(x,y)+g(x,y)] \ dA=\int_{R} f(x,y) \ dA+\int_{R} g(x,y) \ dA$
\item $\dsp \int_{R}c[f(x,y)] \ dA=c\int_{R}f(x,y) \ dA$ where $c$ is a real number
\item $\dsp \int_{R}f(x,y) \ dA \geq \int_{R}g(x,y) \ dA$ if $f(x,y)\geq g(x,y)$ for all $(x,y)\in R$
\item $\dsp \int_{R}f(x,y) \ dA=\int_{R_{1}}f(x,y) \ dA+\int_{R_{2}}f(x,y) \ dA$,    where $R_{1}$ and $R_{2}$ are rectangular regions such that $R_1$ and $R_2$ have no points in common except for points on parts of their boundaries and $R=R_{1}\cup R_{2}$
\end{enumerate}
\end{thm}

\begin{prb}
Compute $\dsp{\int_{1}^2 \int_{0}^3 2x+3y \  dy\ dx}$ and $\dsp{\int_{0}^3 \int_{1}^2 2x+3y \  dx\ dy}$.  This is called ``reversing the order of integration.'' Sketch the solid that you found the volume of.
\end{prb}

\begin{prb}
Compute and compare these three integrals:
\begin{enumerate}
\item $\dsp{\int_{1}^2 \int_{x-2}^{0} 2x+3y \  dy\ dx}$
\item $\dsp{\int_{x-2}^{0} \int_{1}^2 2x+3y \  dx\ dy}$
\item $\dsp{\int_{-1}^{0} \int_{1}^{y+2} 2x+3y \  dx\ dy}$
\end{enumerate}
\end{prb}

As you can see from the previous two problems, when the limits of integration contain variables care must be taken in reversing the order of integration.  When the limits of integration are numbers, we can reverse the order of the integrals and obtain the same result.  This is known as Fubini's Theorem.

\begin{thm}
\textbf{Fubini's Theorem.} Suppose $f$ is a continuous function of two variables, $x$ and $y$, defined on the rectangle
$R=[a,b] \times [c,d]$. Then $$\int_{R}f(x,y)dA=\int_{a}^{b} \Bigg[\int_{c}^{d} f(x,y) dy \Bigg]dx =\int_{c}^{d} \Bigg[\int_{a}^{b} f(x,y) dx \Bigg]dy. $$
\end{thm}

\begin{prb}
Evaluate $\dsp \int_{B} 4x \ln(y)z \; dV$ over the box $B= [0, 2] \times [1, 4] \times [-2, 5]$. Now change the order of integration and verify the result. How many choices are there for orders of integration?
\end{prb}

%VP
\begin{prb}
Let $f(x,y)=4-x^2-y^2$ over the region $R=[0,1] \times [0,1].$ Compute $\dsp \int_0^1 \int_0^1 f(x,y) \ dy \ dx.$  Sketch
the solid that this integral represents the volume of.
\end{prb}

%VP
\begin{prb}
Let $f(x,y)=4-x^2-y^2.$ Write an integral for the volume of the solid that that is bounded by $f$ and by the planes
$z=0, x=0, x=2, y=0,$ and $y=2.$  Compute this integral.
\end{prb}

%vp
\begin{prb}
Compute $\dsp{\int_{0}^1 \int_{0}^1 \frac{y}{(xy+1)^2}\ \ dx\ dy}.$
\end{prb}

%vp
\begin{prb}
Compute $\dsp{\int_{0}^{\ln(3)} \int_{0}^1 xye^{xy^2}\ \ dy\ dx}.$
\end{prb}

%vp
\begin{prb}
Compute $\dsp{\int_{R}\sin(x+y)\ \ dA}$ where $R=\left[0,\dsp{\frac{\pi}{2}}\right]\times \left[0,\dsp{\frac{\pi}{2}}\right].$
\end{prb}

%vp
\begin{prb}
Compute $\dsp{\int_{R} xy\sqrt{1+x^2}\ \ dA}$ where $R=[0,\sqrt{3}]\times [1,2].$
\end{prb}

Thus far we have integrated primarily over domains that were rectangles, but we can also integrate over more general domains.

\begin{prb}
Let $f(x,y) = 4x + 2y.$
\begin{enumerate}
\item Sketch the region in the $xy$-plane bounded by by $x=2$, $x=4$, $y=-x$, $y=x^2$.
\item Compute the volume of the solid below $f$ and above this region.
\end{enumerate}
\end{prb}

%vp
\begin{prb}
Evaluate $\dsp{\int_{S} xy\ \ dA}$ where S is the region bounded by $y=x^2$ and $y=1$.
\end{prb}

%vp
\begin{prb}
Evaluate $\dsp{\int_{S} \frac{2}{1+x^2}\ \ dA}$ over the region S determined by the triangle with vertices $(0,0)$, $(2,2)$, and $(0,2)$ in the x-y plane.  Sketch the solid that you found the volume of.
\end{prb}

\begin{expl}
Find an integral expression for the volume of one-eighth of the sphere of radius 2 in three ways:
$$\int_{\_}^{\_} \int_{\_}^{\_} \_\_\_\_ \ dx dy  = \int_{\_}^{\_} \int_{\_}^{\_} \_\_\_\_ \ dy dz  =
\int_{\_}^{\_} \int_{\_}^{\_} \int_{\_}^{\_} 1 \ dx dy dz$$
\end{expl}

\begin{prb}
Fill in the blanks in order to change the order of integration.  $$\dsp{\int_0^1{\int_0^\frac{1-x}{2} \int_0^{1-x-2y}{f(x,y,z)\ dz\ dy\ dx}}}= \int_\_^\_{\int_\_^\_{\int_\_^\_   f(x,y,z)  {dx\ dz\ dy}}}$$
\end{prb}

%vp
\begin{prb}
Sketch and compute the volume of the solid bounded by $x^2=4y$, $z=0$, and $5y+9z-45=0$.  Write the integral both as $\dsp{\int_{\_}^{\_} \int_{\_}^{\_} \_\_\_\ dx\ dy}$ and $\dsp{\int_{\_}^{\_} \int_{\_}^{\_} \_\_\_\ dy\ dx}$.
\end{prb}

\begin{prb}
Fill in the blanks:  $$\dsp{\int_{0}^1 \int_{-y}^y f(x,y)\ dx\ dy}=\dsp{\int_{\_}^{\_} \int_{\_}^{\_} f(x,y)\ dy\ dx}$$
$$\dsp{\int_{0}^2 \int_{y^2}^{2y} f(x,y)\ dx\ dy}=\dsp{\int_{\_}^{\_} \int_{\_}^{\_} f(x,y)\ dy\ dx}$$
\end{prb}

\begin{prb}
Compute $\dsp{\int_{0}^1 \int _{-x}^x e^{x+y}\ dy\ dx}$ directly and by reversing the order of integration.
\end{prb}

\textbf{Coordinate Transformations}\\

Loosely speaking, a \emph{coordinate transformation} is a transformation from one coordinate system to another coordinate system and is also called \emph{change of variables}.  A cleverly chosen coordinate transformation can make a difficult integral easy.
There are infinitely many, but the three most common are: the conversion from rectangular coordinates to polar coordinates
(used in double integrals), from rectangular coordinates to spherical coordinates (used in triple integrals), and from rectangular coordinates to cylindrical coordinates (used in triple integrals).\\

In Calculus I when you did a trigonometric substitution, you did a change of variable. Written as a theorem, it would look like this.

\begin{thm} \label{cov9}
\textbf{Change of Variable Theorem.} If $f$ is an integrable function over $[a,b]$  and $u: [a,b] \to [c,d]$ is a differentiable function, then $$\dsp \int_a^b f(x) \  dx = \int_c^d f\big(u(t)\big) u'(t) \  dt$$ where we are making the substitution $x=u(t)$ and $u(c)=a$ and $u(d)=b.$
\end{thm}

Work the following problem, using the given substitution, but keeping all the independent variables in tact, i.e. don't toss out the ``t'' when you replace $x$ by $u(t).$

\begin{prb}
Apply Theorem \ref{cov9} to compute:
\begin{enumerate}
\item $\dsp \int_3^4 \frac{1}{x^2 + 9} \  dx$ using the substitution $u(t) = 3\tan(t)$
\item $\dsp \int_3^4 \frac{1}{\sqrt{x^2 + 9}} \  dx$ using the substitution $u(t) = 3\tan(t)$
\end{enumerate}
\end{prb}

For transformation of two variables the theorem is similar.
\begin{annotation}
\endnote{To demonstrate change of variables in two dimensions, we rewrite (but don't compute) $\dsp \int_0^1 \int_{x-3}^{x+3} \sqrt{x+2y}(y-x)^2  \ dy \ dx$ using $u=x+2y$ and $v=y-x$. We first solve the two equations for $x$ and $y$ before computing $J(u,v)$.  Then we note that $\dsp J(u,v) = \frac{1}{J(x,y)}$.}
\end{annotation}

\begin{thm} \label{cov}
\textbf{Change of Variable Theorem.} If $f$ is an integrable function over the domain $B$ and $x$ and $y$ are differentiable functions transforming the region $B$ to the region $B'$, then $$\dsp{\int_B \int f(x,y)\ dx\ dy=\int_{B'} \int f\big(x(u,v),y(u,v)\big) J(u,v)  \ du\ dv}$$
where we are making the substitution $x=x(u,v)$ and $y=y(u,v)$ and  $$J(u,v)= det \begin{pmatrix} x_{u}(u,v)&x_{v}(u,v)\\
y_{u}(u,v)&y_{v}(u,v) \end{pmatrix} = x_{u}(u,v)y_{v}(u,v)-y_{u}(u,v)x_{v}(u,v)$$
%$\dsp \int_{-3}^3 \int_{2v}^{3+2v} \frac{1}{3}\sqrt{u}v^2 du dv$
\end{thm}

\begin{prb}
Integrate $\dsp \int_0^4 \int_{\frac{y}{2}}^{\frac{y}{2}+1} \frac{2x-y}{2} \; dx \; dy$. Now make the change of variable, $\dsp u = \frac{2x-y}{2}$ and $\dsp v=\frac{y}{2}$ and integrate the result.
\end{prb}

\textbf{Polar Coordinates Refresher}
\begin{annotation}
\endnote{We spent little time on polar coordinates during the last semester, so we filled in gaps here. First we discussed a few graphs, $r = a$ (circles); $\theta = b$ (lines); $r = \pm \sin(n\theta)$  and $r = \pm \cos(n\theta)$ (roses); $r = a \pm b \sin(n\theta)$ and $r = a \pm b \cos(n\theta)$ (limacons and cardiods). We graph one, perhaps $r = 4\sin(2\theta)$, carefully labeling every point. Then we move on to calculus, finding four integral expressions for the area below $x^2 + y^2 = 9$ and above the $x-axis$.  For fun, I ask them how many ways they think I can compute the area of this half circle?  We know that the answer is $9\pi/2$ (that's one) so writing the integrals is an exercise to reinforce their understanding of the integral expressions.
\begin{enumerate}
\item Single Rectangular (two more):  $\dsp \int_{-3}^3 \sqrt{9-x^2} \  dx = \int_{0}^3 \sqrt{9-y^2} - - \sqrt{9-y^2} \  dy$
\item Double Rectangular (two more): $\dsp \int_{-3}^3 \int_0^{\sqrt{9-x^2}} 1 \ dx \ dy = \int_{0}^3 \int_{-\sqrt{9-y^2}}^{\sqrt{9-y^2}} 1 \ dy \ dx$
\item Double Polar (one more): $\dsp \int_{0}^\pi \int_0^3  1 \cdot J(r,\theta)  \ dr \ d\theta =\int_{0}^\pi \int_0^3  r  \ dr \ d\theta$
\end{enumerate}
What about single variable polar integration for this area? Since $r=3$ is the equation of the circle, we might try $\int_0^\pi 3 \ d\theta$ but this yields $3\pi$ which is incorrect. What's wrong? We return to Riemann Sums and illustrate the problem by dividing $r = f(\theta)$ into sectors of circles that $$\mbox{Area } = \lim_{n \to \infty} \sum_{i=1}^n \frac{1}{2} r_i^2 (\theta_{i+1} - \theta_i) =  \int_{\alpha}^{\beta} \frac{1}{2} r^2 \ d\theta = \int_{\alpha}^{\beta} \frac{1}{2} (f(\theta))^2 \ d\theta$$
Thus,
\begin{enumerate}
\item[4.] Single Polar (one more): $\dsp \frac{1}{2} \int_0^{\pi} (3)^2 \ d\theta$
\end{enumerate}
}
\end{annotation}
\\

% It would be nice to have a polar coordinate picture here.

In polar coordinates, the point in the plane $P = (x,y)$ may be denoted by $(r,\theta)$ where $r$ is the signed distance from $(0,0)$ to $P$, and $\theta $ is the angle between the vector $\oa{P}$ and the positive x-axis. We restrict $\theta$ to the interval $[0, 2\pi]$.  We say $r$ is the signed distance to allow equations such as $r = 4\sin(\theta)$ where if $\dsp \theta = \frac{11\pi}{6}$ then $r = -2$.  The corresponding point would be $(\dsp -2, \frac{11\pi}{6})$ which is the same point as $\dsp (2, \frac{5\pi}{6})$.  It follows that $x, y, r,$ and $\theta$ are related by the equations:
\begin{enumerate}
\item $x=rcos\theta,$
\item $y=rsin\theta,$
\item $r=\sqrt{x^2+y^2},$ and
\item $\theta= arctan\left(\dsp{\frac{y}{x}}\right)$.
\end{enumerate}


\begin{prb}
Sketch each of the following pairs of polar functions on the same graphs.
\begin{enumerate}
\item $r=5\cos(\theta)$ and $\theta = 2\pi/3$
\item $r=2\sin(\theta)$ and $r=2\cos(\theta)$
\item $r=2 + 2\cos(\theta)$ and $r=1$
\end{enumerate}
\end{prb}

\begin{prb}
\label{jacob1}
To make a change of variables from rectangular to polar coordinates, we let $x$ and $y$ be the functions,
$x(r,\theta)=rcos\theta$ and $y(r,\theta)=rsin\theta.$ Show that $J(r,\theta) = r$ using Theorem \ref{cov} by filling in the missing computation:
$$J(r,\theta)=
det
    \begin{pmatrix}
        x_r & x_{\theta} \\
        y_r & y_{\theta}
    \end{pmatrix} =  \dots = r.$$
\end{prb}

\begin{prb}
\label{jacob2}
To make a change of variables from polar to rectangular coordinates, we let $r$ and $\theta$ be the functions, $r(x,y)=\sqrt{x^2+y^2}$ and $\theta(x,y)=arctan(y/x).$ Compute and simplify $J(x,y)$.
\end{prb}


%VP
\begin{prb}
Compute $\dsp{\int_0^{\frac{\pi}{2}}\int_0^{\cos(\theta)}r^2\sin(\theta)\ dr\ d\theta}$.
\end{prb}

\begin{prb}
Convert the previous integral to rectangular coordinates and recompute to verify your answer.
\end{prb}

Using the previous problem and Theorem \ref{cov}, we now see that when making a change of variable from rectangular to polar coordinates we have, $$\dsp{\int_{B} f(x,y)\ dx\ dy=\int_{B'} f(r\cos(\theta),r\sin(\theta))\
r\ dr\ d\theta}$$ where $B'$ is the region $B$ represented in polar coordinates. The next problem demonstrates an intuitive geometric argument supporting this result.

\begin{prb}
Recall that the area of the sector of a circle with radius $r$ spanning $\theta$ radians is $A=\frac{1}{2} r^2\theta$. Let $0 < r_1 < r_2$ and $0 < \theta_1 < \theta_2 < \frac{\pi}{2}.$ Sketch the region in the first quadrant bounded by the two circles $r=r_1,$ $r=r_2,$ and the two lines $\theta = \theta_1,$ and $\theta = \theta_2.$ Show that the area of the bounded region is $\dsp{\left(\frac{r_{1}+r_{2}}{2}\right)} (r_{2}-r_{1})(\theta_{2}-\theta_{1})$.
\begin{annotation}
\endnote{This problem is a springboard to the next example/lecture.  Our goal is to demonstrate why if we are computing the volume under a function above a polar region then we must insert the Jacobian, $\int \int f \ r \ dr d\theta$.}
\end{annotation}
\end{prb}

\begin{expl}
Write an integral with respect to $\theta$ that represents the area inside the graph of $r=\sin(2\theta)$ using limits and demonstrate why $\dsp{\left(\frac{r_{i}+r_{i+1}}{2}\right)} (r_{i+1}-r_{i})(\theta_{i+1}-\theta_{i})$ appears in the sum.   Now write the same area as a double integral with respect to $r$ and $\theta$. Compute the volume of the cylinder of height 4 above $r=\cos(2\theta).$ Hopefully, the theme of looking at the areas or volumes of small pieces comes through.
\end{expl}

\begin{dfn}
Two circles are \textbf{concentric} if they share a common center, yet have distinct radii.
\end{dfn}

\begin{dfn}
An \textbf{annulus} is a region in the plane trapped between two concentric circles.
\end{dfn}

%VP
\begin{prb}
Let $r_1$ and $r_2$ be real numbers with $0 < r_1 < r_2.$ Let $R$ be the portion of the annulus centered at the origin, between the two circles of radii $r_1$ and $r_2$, and above the x-axis. Convert $\dsp{\int_R{e^{x^2+y^2}\ dA}}$ over the region $R$ to polar coordinates and compute.  Write down the endpoints of integration in rectangular coordinates.
\end{prb}

%VP
\begin{prb}
Convert to polar and compute $\dsp{\int_0^1\int_0^{\sqrt{1-y^2}}\sin(x^2+y^2)\ dx\ dy}$.
\end{prb}

%VP
\begin{prb}
Sketch the region bounded by $r=2$ and $r=2(1+\cos(\theta))$ from $\theta=0$ to $\theta = \pi$. Compute $\dsp{\int_R{y\ dA}}$ via polar coordinates.
\end{prb}

%VP
\begin{prb}
Sketch $\dsp{\theta=\frac{\pi}{6}}$ and $r=4\sin\theta$. Compute the area of the smaller of the two regions bounded by the curves in two ways:
\begin{enumerate}
\item First compute $\dsp{\int_R\int{r\ dr\ d\theta}}$.
\item Check your answer by using the formula for the area inside a polar graph, $\frac{1}{2}\int_\alpha^\beta (f(\theta))^2 \ d\theta$.
\end{enumerate}
\end{prb}

%VP
\begin{prb}
Convert to polar and compute $\dsp{\int_1^2\int_0^{\sqrt{2x-x^2}}(x^2+y^2)^{-\frac{1}{2}}\ dy\ dx}$.
\end{prb}
%solution
%$\dsp{y=\sqrt{2x-x^2}\Rightarrow x^2-2x+y^2=0\Rightarrow (x-1)^2+y^2=1\Rightarrow}$ circle $@\ (1,\ 0)$
%with radius $1$.
% {\Large $\cdot^{\cdot}\cdot\ $}$\theta\in[0,\ \dsp{\frac{\pi}{4}}]$ What is this curve in polar?
% $(x-1)^2+y^2=1$$\Rightarrow (r\ \cos\theta-1)^2+(r\ \sin\theta)^2=1$$\Rightarrow
%r^2\cos^2\theta-2r\ \cos\theta+1+r^2\sin^2\theta=1$$\Rightarrow r^2-2r\ \cos\theta=0$$\Rightarrow
%r(r-2\cos\theta)=0$$\Rightarrow r=0$ or
%\bm$r=2\cos\theta$\ubm$\dsp{\int_0^{\frac{\pi}{4}}\int_{???}^{2\cos\theta}\frac{1}{\sqrt{r^2}}\ r\ dr\ d\theta}$
%%FRANCESR picture goes here and i don't know what to do with the lower limit on the second integral
%$\dsp{\cos\theta=\frac{1}{r}\Rightarrow r=\frac{1}{\cos\theta}}$
%$\dsp{\int_0^{\frac{\pi}{4}}\int_{\frac{1}{\cos\theta}}^{2\cos\theta}dr\ d\theta=\int_0^\frac{\pi}{4}2\cos\theta-\frac{1}{\cos\theta}\ d\theta}$
%%FRANCESR * above next line
%$\dsp{=2\sin\theta-\ln\mid \sec(\theta)+\tan(\theta)\mid}${\Large $\mid$}$_0^{\frac{\pi}{4}}$
%%FRANCESR i know the line above doesn't look that great but i don't know how to make it look better
%$\dsp{=2\ \frac{\sqrt{2}}{2}-\ln\mid\frac{2\sqrt{2}}{\sqrt{2}\sqrt{2}}+1\mid-(0-\ln(i+0))}$
%$\dsp{=\sqrt{2}-\ln\mid\sqrt{2}+1\mid}$.
%$\dsp{\int\frac{1}{\cos(\theta)}\ d\theta=\int \sec(\theta)\ d\theta}$ $\dsp{\int
%sec(\theta)\ \frac{\sec\theta+\tan\theta}{\sec\theta+\tan\theta}\ d\theta}$
%%FRANCESR check and see if the denominator is correct above
%$\dsp{=\ln\mid sec\theta+\tan\theta\mid}$

%VP
\begin{prb}
Find the volume of the solid under $z=3xy,$ above $z=0,$ and within $x^2+y^2=2x$.
\end{prb}

%VP
\begin{prb}
Find the volume of the solid in the $1^{st}$ octant (i.e. where $x>0, y>0, z>0$) under $z=x^2+y^2$ and inside the surface $x^2+y^2=9$.
\end{prb}

\textbf{Cylindrical and Spherical Coordinates} \\

You already know how to represent points in the plane in two different ways, rectangular and polar coordinate systems. We wish to represent points in three-space in two new ways referred to as {\it cylindrical} and {\it spherical} coordinates.

\begin{dfn}
The \textbf{Cylindrical Coordinate Representation} for a point $P = (x,y,z)$ is denoted by $(r,\ \theta,\ z)$ where $(r,\ \theta)$ are the polar coordinates of the point $(x,y)$ in the x-y-plane and $z$ remains unchanged (i.e. $z$ is the height of the point above or below the x-y-plane).
\end{dfn}

\begin{dfn}
The \textbf{Spherical Coordinate Representation} for a point $P = (x,y,z)$ is denoted by $(\rho,\ \phi,\ \theta)$ where $\rho$ is the distance from the point to the origin, $\phi$ is the angle between $\oa P$ and positive z-axis, and $\theta$ is the angle between the x-axis and the projection of $\oa P$ onto the x-y-plane.  To avoid multiple representations of a single point in space we restrict $\rho \geq 0$, $0 \leq \phi \leq \pi$, and $0 \leq \rho \leq 2\pi$.
\begin{annotation}
\endnote{We briefly discuss cylindrical coordinates, as it is a sufficiently obvious extension of polar coordinates.  To introduce spherical coordinates, we discuss and sketch a few surfaces and solids such as $\rho \leq 6$, $\theta = \pi/4$ or the region above $\phi = 3\pi/4$ and below the $x-y$ plane.   Then we derive the relationship between rectangular and spherical, $x = \rho \sin(\phi) \cos(\theta)$, $y = \rho \sin(\phi) \sin(\theta)$ and $z = \rho \cos(\phi)$.  Finally we do the following example.}
\end{annotation}

\begin{expl}
Write an integral using spherical coordinates that represents the volume of an ice cream cone -- that is, the volume inside the cone $\phi = \pi/6$ and inside the sphere $\rho=6$.  Of course, we must insert the Jacobian $\rho^2 \sin(\phi)$.
\end{expl}

\end{dfn}

\begin{prb}
Sketch and write a sentence describing each of the following regions in spherical coordinates.
\begin{enumerate}
\item the region between $\rho = 2$ and $\rho = 3$
\item the region between $\phi = \pi/2$ and $\pi/4$
\item the region below $\rho = 10$ and above $\phi = \pi/3$
\end{enumerate}
\end{prb}

\begin{expl}
Convert $$\int_0^2 \int_0^{\sqrt{4-x^2}} \int_0^{\sqrt{4-x^2-y^2}} \sqrt{4-x^2-y^2} \ dz \ dy \ dx$$
to both spherical and cylindrical coordinates.
\begin{annotation}
\endnote{First we sketch the region defined by $x \in [0,2]$, $y \in [0, \sqrt{4-x^2}]$, and $z \in [0,\sqrt{4-x^2-y^2}]$. The obvious choice seems to be a conversion to spherical coordinates, so we do that first to obtain $$\int_0^{\frac{\pi}{2}} \int_0^{\frac{\pi}{2}} \int_0^2 \rho^3 \cos\phi \sin^2\phi\sqrt{4-\rho^2\sin^2\phi} \ d\rho \ d\phi \ d\theta$$  which we can't easily integrate! In cylindrical it becomes $$\int_0^{\frac{\pi}{2}}  \int_0^2 \int_0^{\sqrt{4-r^2}} \sqrt{4-r^2} \ r \ dz \ dr \ d\theta$$ which is easily integrated.}
\end{annotation}

\end{expl}

From Problem \ref{jacob2}, we know that the Jacobian for polar coordinates is $$J(r,\theta) = det \begin{pmatrix} \cos(\theta) & -r \ \sin(\theta) \cr \sin(\theta) & r \ \cos(\theta) \end{pmatrix}=r.$$  Two of the next three problems generate the Jacobian for cylindrical and spherical coordinate systems. \textbf{Warning.}  If you haven't had linear algebra yet, the next two problems require computing the determinant of $3 \times 3$ matrices.  You might skip them, ask me how to compute the determinant of a $3 \times 3$ matrix, or Google how to do it.

\begin{prb}
\label{jacob3}
To make a change of variables from rectangular to cylindrical coordinates, we let $x=x(r,\theta)=r\ \cos\theta$, $y=y(r,\theta)=r\ \sin\theta$, and $z=z$. Show that $J(r,\theta,z) = r$ using Theorem \ref{cov} by filling in the missing computation:
$$J(r,\theta,z)=
det \begin{pmatrix}
        x_r & x_{\theta} & x_z\cr
        y_r & y_{\theta} & y_z\cr
        z_r & z_{\theta} & z_z
    \end{pmatrix} =  \dots = r.$$
\end{prb}

\begin{prb}
Consider the solid trapped inside the cylinder $x^2+y^2=9$, above the function $f(x,y) = 4-x^2-y^2$ and below the plane $z=10$.
\begin{enumerate}
\item Write a triple integral in rectangular coordinates for the volume, but don't compute.
\item Write a triple integral in cylindrical coordinates for the volume and compute it.
\item Write the volume as the volume of the cylinder minus the volume under the upside down paraboloid.
\end{enumerate}
\end{prb}

\begin{prb}
\label{jacob4}
To make a change of variables from rectangular to spherical coordinates, we let $x(\rho,\phi,\theta)=\rho\ \sin\phi\ \cos\theta$,
$y(\rho,\phi,\theta)=\rho\ \sin\phi\ \sin\theta$, and $z(\rho,\phi,\theta)=\rho\ \cos\phi.$ Show that $J(\rho, \phi, \theta) = \rho^2\sin(\phi)$ using Theorem \ref{cov} by filling in the missing computation:
$$J(\rho, \phi, \theta) = det \begin{pmatrix}
x_{\rho} & x_{\phi} & x_{\theta} \cr
y_{\rho} & y_{\phi} & y_{\theta} \cr
z_{\rho} & z_{\phi} & z_{\theta}
\end{pmatrix} = \dots = \rho^2\sin(\phi).$$
\end{prb}

Summarizing, when integrating with respect to polar, cylindrical, or spherical coordinates, we will always use the appropriate Jacobian as illustrated below.

\begin{enumerate}
\item Polar: $$\dsp{\int\int\ldots\ r\ dr\ d\theta}, \ \ \mbox{where} \ \ x=r \ \cos(\theta), y = r \ \sin(\theta)$$
\item Cylindrical: $$\dsp{\int\int\int\ldots\ r\ dr\ d\theta\ dz},  \ \ \mbox{where}  \ \  x=r\cos(\theta), y = r\sin(\theta), \mbox{\&} \; z = z$$
\item Spherical: $$\dsp{\int\int\int\ldots\ \rho^2\sin(\phi)\ d\rho\ d\phi\ d\theta},  \ \  \mbox{where}  \ \  x=\rho\ \sin\phi\ \cos\theta, y=\rho\ \sin\phi\ \sin\theta, \mbox{\&} \; z=\rho\ \cos\phi$$
\end{enumerate}

The order of integration could change and you will find times when the choice of the order of integration transforms an apparently hard problem into an easy one.

\begin{prb}
Compute the volume of the region bounded by the two spheres $\rho=4$ and $\rho=6$ using spherical coordinates. Verify using
the formula for the volume of a sphere, $\dsp V = \frac{4\pi}{3}r^3.$
\end{prb}

\begin{prb}
Sketch and find the volume of the solid outside the cone $\dsp{\phi=\frac{\pi}{3}}$, inside the sphere $\rho=6$, and above the plane $z=0.$
\end{prb}

\begin{prb}
Convert to spherical coordinates then evaluate:
\begin{center}$\dsp {\int_{-3}^{3}
                {\int_{-\sqrt{9-x^{2}}}^{\sqrt{9-x^{2}}}
                {\int_{-\sqrt{9-x^{2}-z^{2}}}^{\sqrt{9-x^{2}-z^{2}}}
                \ \ {(x^2+y^2+z^2)^{\frac{3}{2}}\ dy\ dz\ dx}}}}$
\end{center}
\end{prb}

\begin{prb}
Convert to spherical coordinates and compute: $\dsp \int_R e^{ (x^2+y^2+z^2)^\frac{3}{2} } \ dV$ where $R$ is the region $x^2+y^2+z^2 \leq 1.$
\end{prb}

\begin{prb}
Find the volume of the given ellipsoids by first writing a triple integral for the volume in rectangular coordinates and then using the recommended change of variables to convert the triple integral into the new coordinate system.  If all goes well, the converted integral will be easier to compute because you are transforming the ellipsoid into a sphere.
\begin{enumerate}
\item $\dsp{\frac{x^2}{4}+\frac{y^2}{9}+\frac{z^2}{36}\leq1}$  via the change of variables  $x=2a$, $y=3b$, $z=6c$
\item $\dsp{\frac{x^2}{a^2}+\frac{y^2}{b^2}+\frac{z^2}{c^2}\leq1}$ via the change of variables  $x=ua$, $y=vb$, $z=wc$
\end{enumerate}
\end{prb}

\textbf{An Integration Application}
\begin{annotation}
\endnote{We imagine the x-axis as an infinite, massless beam and suppose that we have two masses: $m_1$ at $(x_1,0)$ and $m_2$ at $(x_2,0)$. At what point $(x,0)$ would we place a fulcrum in order to balance the beam (axis)? To determine this we need to solve $(x - x_1)*m_1 = (x_2 - x)*m_2$ for $x$ which yields $\dsp x = \frac{m_1 x_1 + m_2 x_2}{m_1 + m_2}$. We call $M_y = \sum m_i x_i$ the moment with respect to $y$. If we do the same thing with $n$ weights in the plane at $(x_1,y_1),...(x_n,y_n)$, then we have:
\begin{enumerate}
\item the moment with respect to $y$, $M_y = \sum_{i=1}^n m_i x_i$,
\item the moment with respect to $x$, $M_x = \sum_{i=1}^n m_i y_i$, and
\item the center of mass is $(M_y/m, M_x/m)$ where $m = \sum m_i$.
\end{enumerate}
We often skip the rather interesting case where we wish to find the center of mass (centroid) of an area bounded by a function $f$ on $[a,b]$ which turns out to be
\begin{eqnarray*}
M_y  & = &  \sum area*density*(distance from \ y-axis) \cr
& = & \sum (x_{i+1} - x_i)f(x_i)\rho(x_i) x_i \rightarrow \int_a^b x \rho(x) f(x) \ dx \cr
M_x &=& \sum area*density*(distance from  \ x-axis) \cr
&=& \sum (x_{i+1} - x_i)f(x_i)\rho(x_i) \frac{1}{2}f(x_i) \rightarrow \frac{1}{2} \int_a^b \rho(x) (f(x))^2 \ dx = \int \rho y \ dy.
\end{eqnarray*}
Instead of doing this, we move on to the formula's for center of mass in three dimensions, figuring that the simple finite case is enough of a transition.  The goal is, once again, to emphasize the transition from finite sums to integrals.}
\end{annotation}
\\

The \textbf{mass} of an object can be found by integrating the density function, $\delta,$ over the entire object. If $\delta$ is the density function, then the \textbf{center of mass} is given by $(\oa{x},\ \oa{y},\ \oa{z}) = \dsp{\left(\frac{M_{yz}}{m},\ \frac{M_{xz}}{m},\ \frac{M_{xy}}{m} \right)}$ where
\begin{enumerate}
\item $m = mass = \int \delta(x,y,z) \  dx \  dy \  dz,$
\item $M_{xy}=\int{\int{\int{z\ \delta(x,y,z)}}}\  dx \  dy \  dz,$
\item $M_{xz}=\int{\int{\int{y\ \delta(x,y,z)}}}\   dx \  dy \  dz,$ and
\item $M_{yz}=\int{\int{\int{x\ \delta(x,y,z)}}}\  dx \  dy \  dz.$
\end{enumerate}
The numbers, $M_{xy}, M_{xz},$ and $M_{yz}$ are called the \textbf{moments} of the object with respect to $z$, $y$, and $x$ respectively.

\begin{prb}
Find the center of the mass of the solid inside $x^2+y^2=4$, outside $x^2+y^2=1$, below $z=12-x^2-y^2$, and above $z=0$, assuming the constant density function $\rho(x,y,z) = k$.  Graph the solid and mark the center of mass to see if your answer makes sense!
\end{prb}

%VP
\begin{prb}
Find the mass of the solid bounded by $\dsp z=2-\frac{1}{2}x^{2},\ z=0,\ y=x$ and $y=0,$ assuming the density is $\delta(x,y,z)=kz$ where $k>0.$   Challenge:  Write this integral with the various orders of integration, $dz \ dy \ dx$, $dy \ dx \ dz$, $dz \ dx \ dy$, and $dx \ dz \ dy$.
\end{prb}

%VP
\begin{prb}
Find the center of the mass of the solid from the previous problem.
\end{prb}

\section{Practice} \label{chap12probs}

We will not present the problems from this section, although you are welcome to ask about them in class.

\vskip .1in
\noindent
\textbf{Basics}

\begin{enumerate}

\item Evaluate $\dsp \int x^3y-3x \  dx$ and $\dsp \int_1^3 x^3y-3x \   dy$
\item Evaluate $\dsp \int xye^{xy} \  dx$ and $\dsp \int_{-1}^1 xye^{xy} \  dy$
\item Compute the area of the region bounded by the parabola $y=x^{2}-2$ and the line $y=x$ by first computing a single integral with respect to $x$ and then computing a single integral with respect to $y.$
\item Compute the area bounded by $y=x^3$ and $y=5x$ in four ways.
(a) Single integral with respect to $x,$ (b) single integral w.r.t. $y,$
(c) double integral, $dx \ dy,$ and (d) double integral, $dy \ dx.$

\end{enumerate}

\noindent
\textbf{Regions}

\begin{enumerate}

\item Sketch the region $\phi= \pi/6$ in spherical coordinates.
\item Sketch the region bounded by $r=1$ and $r=2\sin(\theta)$ in
polar coordinates.
\item Sketch the region bounded by $x^{2}+y^{2}\leq 9$ and
write in polar coordinates.
\item Sketch the region between
$x^{2}+y^{2}=25$, $x^{2}+y^{2}=4$, and $x\geq 0$ and write in polar
coordinates.
\item
Find the area of the region $D$ bounded by $y= \cos(x)$ and
$y= \sin(x)$ on the interval $\dsp \Big[0,\frac{\pi }{4} \Big]$.

\end{enumerate}

\noindent
\textbf{Integration}

\begin{enumerate}

\item
Compute $\dsp \int_0^2 \int_0^3  x^2y^2-3xy^5\  dy \ dx$ and
$\dsp \int_0^3 \int_0^2 x^2y^2-3xy^5\  dx \ dy.$ Are they equal?
What theorem is this an example of?

\item
Evaluate $\dsp \int_{0}^{1}\int_{0}^{y} e^{y^{2}} \ dx \ dy$.

\item
Compute $\dsp \int_{R}x^{2}e^{xy}dA$ where
$R = \{(x, y): 0\leq x\leq 3, 0\leq y\leq 2 \}$.

\item
Compute $\dsp \int_{0}^{3} \int_{0}^{2} \sqrt{2x+y} \   dy  \  dx$.

\item
Compute $\dsp \int_{R}\frac{\ln\sqrt{y}}{xy}dA$ where
$R = \{(x, y): 1\leq x\leq 4, 1\leq y\leq e \}$.

\item
Evaluate $\dsp \int_{D} 160xy^{3} \  dA$ where $D$ is the region
bounded by $y=x^{2}$ and $y=\sqrt{x}$.


\item
Sketch and find the volume of the solid bounded above by the plane
$z=y$ and below in the $xy$-plane by the part of the disk
$x^{2}+y^{2}\leq 1$.

\item
Sketch the region, $D$, that is bounded by $x=y^{2}$ and $x=3-2y^{2}$
and evaluate $\int_{D} (y^{2}-x)dA$

\item Compute $\dsp \int_{0}^{2}\int_{0}^{\sin(x)} y \cos(x) \ dy \ dx$.

\item Determine the endpoints of integration for
$\dsp \int \int_S e^{xy} dA$ where $S$ is the region bounded
by $y= \sqrt x$ and $\dsp y={x \over 9}$.  Don't integrate.

\item Determine the endpoints of integration for
$\dsp \int\int_S dA$ where $S$ is bounded by a the
$x=y^{2}+4y$ and $x=3y+2$.

\item Determine the endpoints of integration for
$\dsp \int \int_{S} 2x \  dA$ where $S$ is the region bounded
by $yx^{2}=1$, $y=x$, $x=2$, and $y=0$.

\item
Evaluate $\dsp \int \int _D dA$ where $D$ is the region bounded by the
ellipse $\dsp \frac{x^{2}}{4}+\frac{y^{2}}{9}=1$

\end{enumerate}

\noindent
\textbf{Polar and Spherical Coordinate Integration}

\begin{enumerate}

\item Sketch the region $D$ between $r= \dsp \cos \Big(\frac{\theta }{2} \Big)$ and $x^{2}+y^{2}=1$ with $0\leq \theta \leq \pi$. Evaluate $\dsp \int_{D} 1 dA$.


\item Consider $\dsp \int_{R} x^{2}+y^{2}+1 \  dA$ where $R = \{(x, y): x \ge 0, 9 \le x^{2}+y^{2} \le 16 \}$. Write the integral in both rectangular and polar coordinates. Compute each to verify your answer.

\item
Find the volume of the solid bounded by the cone $\dsp \phi = {{\pi} \over 6}$ and the sphere $\rho=4.$
\end{enumerate}

\noindent
\textbf{Coordinate Transformations}

\begin{enumerate}

\item Find the Jacobian for the transformation: $x=u^{2}+v^{2}+w$, $y=uw-v$, and $\dsp z=\ln(w)-\frac{v}{u}$

\item Evaluate $\dsp \int_{0}^{4} \int_{y/2}^{(y/2)+1} \frac{2x-y}{2} \ dx \ dy$ using $u= \dsp \frac{2x-y}{2}$ and $v= \dsp \frac{y}{2}$.

\item Use the transformation $x= \dsp \frac{u}{v}$ and $y=v$ to rewrite (but not evaluate) the double integral $\int \int \sqrt{xy^{3}} \ dx \ dy$ over the region in the plane bounded by the $x$-axis, the $y$-axis, and the lines $y=-2x+2$ and $x+y=7$.


\item Compute $\dsp \int_0^1 \int_{0}^{y^2} (1-y) \sin \Big({x \over y} \Big) \ dx \ dy$ using $\dsp u = {x \over y}$ and $v = 1-y$.

\item Write an integral in rectangular coordinates that represents the area enclosed by the ellipse $\dsp \frac{x^{2}}{16}+\frac{y^{2}}{49}=1.$ Now, compute this integral by using the transformation $x=4u$ and $y=7v$.

\item Find the volume of the ellipsoid $\dsp {{x^2} \over {a^2}} + {{y^2} \over {b^2}} + {{z^2} \over {c^2}} = 1$ using the transformation $x=au$, $y=bv$, $z=cw$.

\item Evaluate the triple integral $\dsp \int_0^6 \int_0^8 \int_{y \over 2}^{{y \over 2}+4} {{2x-y} \over 2} + {y \over 6} + {z \over 3} dx\ dy\ dz$ using the transformation $\dsp u={{2x-y} \over 2}$, $\dsp v={y \over 2}$, and $\dsp w = {z \over 3}$.

\end{enumerate}

\newpage
\noindent
\textbf{Triple Integrals, Cylindrical, and Spherical Coordinates}

\begin{enumerate}

\item Sketch and find the volume of the solid formed by $f(x,y)=4x+2y$ above the region in the $xy$-plane bounded by $x=2$, $x=4$, $y=-x$, $y=x^2$.

\item Fill in the blanks:
$\dsp{\int_0^1{\int_0^\frac{1-x}{2}
\int_0^{1-x-2y}{f(x,y,z)\ dz\ dy\ dx}}}\\
=\int_\_^\_{\int_\_^\_{\int_\_^\_   f(x,y,z) \ {dy\ dz\ dx}}}\\
=\int_\_^\_{\int_\_^\_{\int_\_^\_   f(x,y,z) \ {dy\ dx\ dz}}}\\
=\int_\_^\_{\int_\_^\_{\int_\_^\_   f(x,y,z) \ {dx\ dy\ dz}}}$

 \item $\dsp \int_0^6 \int_0^{{12-2x} \over 3} \int_0^{{12-2x-3y} \over 4} f \ dz\ dy\ dx = \int_{\underline{\ }}^{\underline{\ }} \int_{\underline{\ }}^{\underline{\ }} \int_{\underline{\ }}^{\underline{\ }} f \ dy\ dx\ dz = \int_{\underline{\ }}^{\underline{\ }} \int_{\underline{\ }}^{\underline{\ }} \int_{\underline{\ }}^{\underline{\ }} f \ dx\ dz\ dy$

\item $\dsp \int_0^\frac{1}{6} \int_0^{{\sqrt{1-36x^2}} \over 3} \int_0^{{\sqrt{1-36x^2-9y^2}} \over 2} f \ dz\ dy\ dx = \int_{\underline{\ }}^{\underline{\ }} \int_{\underline{\ }}^{\underline{\ }} \int_{\underline{\ }}^{\underline{\ }} f \ dx\ dz\ dy$

\item Find the volume of the solid bounded by $x^2+y^2+z=8$ and $z=4$.

\item Evaluate $\dsp \int_0^2 \int_0^{\sqrt{4-x^2}} \int_0^{\sqrt{4-x^2-y^2}} z \sqrt{4-x^2 - y^2} \ dz\ dy\ dx$ by converting to (a) cylindrical coordinates; (b) spherical coordinates.

\end{enumerate}

\noindent
\textbf{Application - Center of Mass}

\begin{enumerate}
\item Suppose $\delta(x,y)=x+y$ is the density function of a thin sheet of material bounded by the curve $x^2=4y$ and $x+y=8$. Find its total mass.  Find its first moments.  Find its center of mass. Find its second moments.
\end{enumerate}


\vskip .5in
\noindent
\textbf{Chapter 12 Solutions}\\

\noindent
\textbf{Basics}

\begin{enumerate}

\item $\frac{1}{4}x^4y - \frac{3}{2}x^2$ and $4x^3-6x$
\item $e^{xy}(x-\frac{1}{y})$ and $e^x(1-\frac{1}{x})+e^{-x}(1+\frac{1}{x})$
\item $9/2$
\item $25/4$

\end{enumerate}

\noindent
\textbf{Regions}

\begin{enumerate}

\item it's a cone
\item it's the intersection of two circles
\item  $r \leq 3$
\item $2 \le r \le 5$ and $\theta \in [-\pi/2,\pi/2]$
\item $\sqrt{2}-1$

\end{enumerate}

\noindent
\textbf{Integration}

\begin{enumerate}

\item $-705$, Fubini's Theorem

\item $\frac{1}{2} (e-1)$

\item  $-\frac{1}{4}(17-5e^6)$

\item  $\approx 11.62$

\item $\approx .346$

\item $6$

\item $\frac{2}{3}$

\item $\frac{24}{5}$

\item  $\frac{1}{6}\sin^3(2)$

\item $\dsp \int_0^{81} \int_{\frac{x}{9}}^{\sqrt{x}} e^{xy} dA$

\item $4.5$

\item $\frac{2}{3} + 2\ln(2)$

\item $6\pi$

\end{enumerate}

\noindent
\textbf{Polar and Spherical Coordinate Integration}

\begin{enumerate}

\item $\dsp \frac{\pi}{4}$

\item $\dsp \frac{189\pi}{4}$

\item  $\dsp \frac{-64(\sqrt{3}-2)\pi}{3}$
\end{enumerate}

\noindent
\textbf{Coordinate Transformations}

\begin{enumerate}

\item $-2u(\frac{1}{w}-1)-2v+\frac{1}{u}(2v^2-w)+v/u^2$

\item $r$

\item $\rho^2\sin(\phi)$

\item  $2$

\item $\dsp \int_{0}^7 \int_{7v-v^2}^{v-v^2/2} \sqrt{u} du dv$

\item area of ellipse is $\pi a b$ in this case $28\pi$ (just like
circle)

\item volume of ellipsoid is $4/3 \pi abc$ (just like sphere!)

\end{enumerate}

\noindent
\textbf{Triple Integrals, Cylindrical, and Spherical Coordinates}

\begin{enumerate}

\item $2472/5$

\item $\dsp
\int_0^1{\int_0^{1-x}{\int_0^{\frac{1-x-z}{2}} \  f(x,y,z) \ {dy\ dz\ dx}}} \\
\int_0^1{\int_0^{1-z}{\int_0^{\frac{1-x-z}{2}} \  f(x,y,z) \ {dy\ dx\ dz}}} \\
\int_0^{1}{\int_0^{\frac{1-z}{2}}{\int_0^{1-z-2y} \  f(x,y,z)\  {dx\ dy\ dz}}}  $

\item $\dsp
\int_0^3{\int_0^{-2x+6}{\int_0^{\frac{-4x-2x+12}{3}} \  f(x,y,z) \ {dy\ dx\ dz}}} \\
\int_0^4{\int_0^{-\frac{3}{4}y+3}{\int_0^{\frac{-4z-3y+12}{2}} \  f(x,y,z) \ {dx\ dz\ dy}}}  $

\item $\dsp \int_0^\frac{1}{3}{\int_0^{\sqrt{\frac{1-9y^2}{2}}}{\int_0^{\frac{\sqrt{1-9y^2-4z^2}}{6}} \  f(x,y,z) \ {dy\ dx\ dz}}} $

\item $8\pi$

\end{enumerate}

\chapter{Line Integrals, Flux, Divergence, Gauss' and Green's Theorem}

``The important thing is not to stop questioning.''  - Albert Einstein\\ \\

The phrases {\it scalar field} and {\it vector field} are new to us, but the concept is not.  A scalar field is simply a function whose range consists of real numbers (\emph{a real-valued function}) and a vector field is a function whose range consists of vectors (\emph{a vector-valued function}).

\begin{expl}
$f(x,y) = x^2 -2xy$ is a {\it scalar field}. $f(x,\ y)=(2x,\ 3y)$ is a {\it vector field} on ${\re}^2$ and $f(x,\ y,\ z)=(2x-y,\ y\cos(z^2),\ x-yz)$ is a {\it vector field} on ${\re}^3$. For the point in the ocean at location $(x,y,z)$, the vector field $\oa f(x,y,z) = \big(p(x,y,z),q(x,y,z),r(x,y,z)\big)$ might represent the direction of the current at this point where $p,q,r: \re^n \to \re.$
\end{expl}

\begin{dfn}
Let $n$ and $m$ be integers greater than or equal to 2. A \textbf{scalar field} on ${\re}^n$ is a function $f:{\re^n}\to{\re}$. A \textbf{vector field} on ${\re}^n$ is a function $f:{\re}^n\to{\re}^n.$
\end{dfn}

We now modify our definition for $\nabla$.  If  $f: \re^2 \to \re$, then we previously defined $\dsp \nabla f = (f_x, f_y) = (\frac{ \partial f}{\partial x}, \frac{ \partial f}{\partial y}).$  We now redefine $\nabla$ as an {\it operator} that acts on the set of differentiable functions and write $$\nabla = \left(\frac{ \partial }{\partial x}, \frac{ \partial }{\partial y}\right).$$ This encompasses our previous notation as we may now write $$\dsp \nabla f = \big(\frac{ \partial }{\partial x}, \frac{ \partial }{\partial y}\big)(f)= \big(\frac{ \partial f}{\partial x}, \frac{ \partial f}{\partial y}\big)= (f_x, f_y).$$
We previously defined the dot product only between vectors (and points) in $\re^n$ and now redefine that notation as well.  If $\oa f: \re^2 \to  \re^2$ and $\oa f(x,y) = \left(p(x,y),q(x,y)\right)$, then we will write $$\nabla \cdot \oa f = \left(\frac{ \partial }{\partial x}, \frac{ \partial }{\partial y}\right) \cdot \big(p(x,y),q(x,y)\big) = \frac{ \partial p }{\partial x} + \frac{ \partial q }{\partial y} = p_x(x,y) + q_y(x,y).$$

The next definition formalizes what we just wrote, stating it for $\re^3$.

\begin{dfn}
If $\oa f(x,y,z)  = \Big(p(x,y,z) ,q(x,y,z) ,r(x,y,z) \Big): {\re}^3\to {\re}^3$ is a differentiable vector field, then the \textbf{divergence of $\oa{f}$} is the scalar field from ${\re}^3\to {\re}$ defined by: $\nabla \cdot \oa{f}=p_x + q_y + r_z$.
\end{dfn}

The divergence of a function at a point is the tendency for fluid to move away from or toward that point, where a positive divergence indicates fluid is moving away from the point and a negative divergence indicates that fluid is moving toward the point.

\begin{expl}
Sketch the vector fields, $F(x,y) = (x,y)$  and  $F(x,y) = (-x,y)$ and compute the divergence of each.
\begin{annotation}
\endnote{Discuss vector fields such as gravity and how the work done along any path will be equal.   Discuss another field such as river flow where if it is all flowing downstream it is conservative, but if there is an eddy in the middle, then two paths across the river may result in differing amounts of work.}
\end{annotation}
\end{expl}

The next definition extends our definition of the cross product, $\times.$

\begin{dfn}
If $\oa f(x,y,z)  = (p(x,y,z) ,q(x,y,z) ,r(x,y,z) ) : \re^3 \to \re^3$, then the \textbf{curl} of $\oa f$ denoted by $\nabla \times f$ is defined by the vector valued function $$\nabla \times \oa f = \left(\frac{ \partial }{\partial x}, \frac{ \partial }{\partial y}, \frac{ \partial }{\partial z}\right) \times  (p,q,r) = \left( \frac{ \partial r}{\partial y} - \frac{ \partial q}{\partial z}, \frac{ \partial p}{\partial z} - \frac{ \partial r}{\partial x}, \frac{ \partial q}{\partial x} - \frac{ \partial p}{\partial y} \right).$$
\end{dfn}

The divergence of $\oa f,$ denoted by $\nabla \cdot \oa f$, is a scalar field  while the curl of $\oa f,$ denoted by $ \nabla \times \oa f,$ is a vector field.  If our vector field represents a fluid flow, then the curl of a function at a point is a the direction from that point about which the fluid rotates most rapidly.  The points on the ray with base at the point and pointing in the direction of the curl tend not to move, but to spin as the fluid rotates (curls) around that ray.

\begin{prb}
Compute the divergence and curl of $\oa f:{\re}^3\to{\re}^3$ given by
\begin{enumerate}
\item $\oa f(x,\ y,\ z)=(x^2yz,\ x^2+y+\sqrt{z},\ \pi x^2/yz)$
\item $\oa f(x,\ y,\ z)=(e^x,\ \ln(xyz),\ \sin(x^2y^z))$
\end{enumerate}
\end{prb}

\begin{dfn}
Let $f:{\re}^3\to{\re},\ \oa g:{\re}^3\to{\re}^3,\ \oa h:{\re}^3\to{\re}^3$. Assume $f,\ \oa g,\ \oa h$ are differentiable. Then
\begin{enumerate}
\item $(f\oa g)(x,\ y,\ z)=f(x,\ y,\ z)\oa g(x,\ y,\ z)$
\item $(\oa g\cdot \oa h)(x,\ y,\ z)=\oa g(x,\ y,\ z)\cdot \oa h(x,\ y,\ z)$
\item $(\oa g\times \oa h)(x,\ y,\ z)=\oa g(x,\ y,\ z)\times \oa h(x,\ y,\ z)$
\end{enumerate}
\end{dfn}

\begin{prb}
Let $f:{\re}^3\to{\re},\ \oa g:{\re}^3\to{\re}^3,\ \oa h:{\re}^3\to{\re}^3$. Prove or give a counter example:
\begin{enumerate}
\item $\nabla\cdot(\oa{g}+\oa{h})=\nabla\cdot\oa{g}\ +\ \nabla\cdot\oa{h}$
\item $\nabla\cdot(f\oa{g})=f \ \nabla \cdot \oa{g}$  \hskip .2in (Note: $f \ \nabla\cdot\oa{g}$ means $f$ times $\nabla\cdot\oa{g}$.)
\item $\nabla\times(\oa{g}+\oa{h})=\nabla\times\oa{g}\ +\ \nabla\times\oa{h}$
\item $\nabla\cdot(f \oa{g})=f \ \nabla\cdot\oa{g} \ + \ \oa g\cdot\nabla f$
\item  $\nabla\times(f \oa{g})=f \nabla\times\oa{g} \ + \ \oa g\times \nabla f$
\item $\nabla\cdot(\nabla \times \oa{g})=0$
\end{enumerate}
\end{prb}

\textbf{Arc Length and Line Integrals over Scalar Fields}\\

\begin{prb}
Use Definition \ref{arclengthdfn} to write the integrals that would represent the arc length of $f(x)=x^2$ from $(-1,\ 1)$ to $(1,\ 1)$ and the arc length of $\oa f(t)=(t,\ t^2)$ for $-1 \leq t \leq 1$.  Demonstrate your powers of integration from Calculus II by computing this integral -- don't use a formula -- use a trig substitution and then integration by parts.
\end{prb}

\begin{expl}
Compute the area of a wall that has as its base the curve $c(t) = (t,t^2), \ \ 0 \leq t \leq 1$ and has a height  of $t^3$ at the point $(t,t^2).$
\begin{annotation}
\endnote{This is our introduction to the concept behind the line integral.  Our goal is to compute the area of a wall that has as its base the curve $c(t) = (t,t^2), \ \ 0 \leq t \leq 1$ and has a height  of $t^3$ at the point $(t,t^2)$. For entertainment's sake, we imagine the scenario where a mathematician lives in a house and likes the neighbor across the street, but does not care for the side neighbor.  Thus she hires our class to build a wall with a parabolic base that has its vertex in her front lawn and curves around her side lawn. The wall is to have height zero in front of her house at the vertex of the parabola and to get cubically taller as it wraps around the side of her house.  We'll need to estimate materials, so we'll need to know the area of the fence. By using a simple, physically possible example, we will see the importance of the arc length in the process.  When we ask what the area of this wall will be, students often say the area trapped between $f(x)=x^3$, the x-axis, and $x=1$.  This, however, does not compensate for the curvature of the base of the wall.  Thus, we return to Riemann Sums and compute from first principles. We  partition the base of the wall into $N$ pieces, $\{ c(t_0), c(t_1), \dots , c(t_N) \}.$ Then the area of the wall will simply be the sum of the areas of the sections of the wall, where the $i^{th}$ section of the wall has base the portion of the parabola between $c(t_{i-1})$ and $c(t_i)$ and has approximate height, $f(t_i)= t_i^3$.  If $d$ represents the formula for the distance between two points in the plane and $g(x) = x^2$, then an approximation of the area based on this partition would be:
\begin{eqnarray*}
area & \approx & \sum_{i=1}^N  d\big( c(t_{i-1}), c(t_i) \big) \ f(t_i) \cr
& = & \sum_{i=1}^N  d\big( (t_{i-1}, t_{i-1}^2), (t_i, t_i^2) \big) \ f(t_i) \cr
& = & \sum_{i=1}^N  \sqrt{ ( t_{i} - t_{i-1})^2 +  (t_i^2 - t_{i-1}^2 )^2 } \ f(t_i) \cr
& = & \sum_{i=1}^N  \sqrt{ 1 +  \frac{(t_i^2 - t_{i-1}^2)^2}{(t_1  - t_{i-1})^2 } } \ f(t_i)  ( t_{i} - t_{i-1}) \cr
& = & \sum_{i=1}^N  \sqrt{ 1 +  \big(\frac{ g(t_i) - g(t_{i-1}) }{ t_1  - t_{i-1} } \big)^2 } \ f(t_i)  ( t_{i} - t_{i-1}) \cr
& \rightarrow & \int_0^1  \sqrt{ 1 +  g'(t)^2 } \ f(t) \ dt \ \ \ \ \mbox{ as } n \to \infty
\end{eqnarray*}
Of course, after this derivation, we see that we have simply ended up integrating $f(x)=x^3$ with respect to the arc length of $g$.  While we have previously shown in this semester and in Calculus II that the arc length of an function $g$ from $a$ to $b$ is given by $\int_a^b \sqrt{ 1 + g'(x)^2 } \ dx$, I think the repetition in the different scenario is valuable.  After the derivation, or in a future class, we discuss other physically admissible examples of scalar integrals -- integrating the electrical charge of a piece of bent wire in three-space or integrating the density of a piece of wire to obtain the mass.}
\end{annotation}
\end{expl}

\begin{prb}
For the wall example worked in class, suppose we replace the base $c(t) = (t, t^2)$ with the curve $c(t) = (u(t),v(t))$ where $u$ and $v$ are some real-valued functions and we replace the height $f(t)=t^3$ with some real valued function $h$. What would the area of the wall be now?  Your answer will be an integral in terms of $u$, $v$, and $h$.
\end{prb}
% $ \int_0^1  \sqrt{ (u'(t)^2 +  (v'(t))^2 } \ h(c(t)) \ dt$

%TedProb
\begin{prb}
Two scalar line integrals:
\begin{enumerate}
\item A wall over $\oa{c}(t)=(3\ \sin(t),\ 3\ \cos(t))$ from $t=0$ to $\dsp{t=\frac{\pi}{2}}$ has height $h(x,\ y)=x^2y$. Graph the wall and determine its area.
\item A wall over $\oa{c}(t)=\dsp{(\sqrt{9-t^2},\ t)}$ from $t=0$ to $t=3$ has height $h(x,\ y)=x^2y$.  Determine its area.
\end{enumerate}
\end{prb}

What you have just been computing are called line integrals over scalar fields.

\begin{dfn}
The \textbf{line integral of a scalar field} $g:\ {\re}^2 \to{\re}$ over the curve $\oa{c}(t)=(x(t),\ y(t))$ is defined
by $L=\dsp{\int_c{g(\oa{c}(t)) |\oa{c}'(t) |\ dt}}.$
\end{dfn}

The line integral over a scalar field is computed with respect to {\it arc length} and we use the notation, $$\int_c{ g \ ds} \; \; \mbox{ to mean } \; \; \int_c{g(\oa{c}(t))\cdot |\oa{c}'(t) |\ dt}.$$

We may use line integrals to compute the mass of a curved piece of wire in a similar manner.  Suppose we have a piece of wire
in the plane that is bent into the shape of a curve, $\oa c: \re \to \re^2.$ If $\delta: \re^2 \to \re$ and $\delta(x,y)$ represents the density of the wire at $(x,y),$ then the line integral of the density with respect to the arc length is the mass of the wire.

\begin{prb}
Mass is the integral of density. Find the mass of a wire in the plane with density at $(x,\ y)$ of $\delta(x,\ y)=2xy$  and shape $\oa{c}(t)=(3\ \cos(t),\ 4\ \sin(t))$ from $t=0$ to $t=2\pi$. Could such a wire exist?
\end{prb}

\begin{prb}
Find the mass of the wire in three-space with density at $(x,\ y,\ z)$ of $\delta(x,\ y,\ z)=3z$ and in the shape $\oa{c}(t)=(2\ \cos(t),\ 2\ \sin(t),\ 5t),\  t\in[0,4\pi]$. Find its center of mass as well.  Center of mass was defined at the end of the last chapter.
\end{prb}

\begin{dfn}
A vector field is said to be \textbf{conservative} if it is the gradient of some function.  Conservative fields are also called \textbf{gradient} fields.  If $f$ is a conservative field, then any function $g$ satisfying $\nabla g = f$ is called a potential for $f.$
\end{dfn}

\begin{expl}
$\dsp{\oa f(x,\ y)=(-2xy+2e^y,\ -x^2+2xe^y)}$ is \textbf{conservative} since $\oa f=\nabla  g$ where $g$ is given by $\dsp{g(x,\ y)=-x^2y+2xe^y}$.
\end{expl}

\begin{prb}
Is $\oa f(x,\ y)=(xy,\ x-y)$ conservative? If $\oa f$ is conservative, then there must be a function $g$ so that $\nabla  g=\oa f.$ Is there a function $g$ so that $g_x(x,y) =xy$ and $g_y(x,y)=x-y$?  Why or why not?
\end{prb}

\begin{prb}
Is $\oa f(x,\ y)=(x^2+y^2,\ 2xy)$ conservative?
\end{prb}

\begin{prb}
\label{gradient}
Show that if $\oa f(x,\ y)=(p(x,\ y),\ q(x,\ y))$ and $p_y(x,\ y)=q_x(x,\ y)$ then $\oa f$ is a gradient field.
\end{prb}

This is part of a larger theory that we don't have time to cover, but is worth at least discussing.  Problem \ref{gradient} and its converse are valid in three dimensions as well. Suppose $f: \re^3 \to \re^3$ is a vector field.  If $f$ is conservative, then $\nabla \times f = 0.$  Conversely, if $f$ has continuous first partial derivatives and $\nabla \times f = 0$ then $f$ is conservative.  If $f:\re^2 \to \re^2$ and $f=(p,q)$, then Problem \ref{gradient} showed that if $p_y=q_x$ then $f$ was conservative.  If $p_y=q_x$, then $q_x-p_y=0$ which is the same statement as $\nabla \times f \cdot \oa{k} = 0.$  In summary, understanding the notation $\nabla \times f$ is enough to determine when a vector field is conservative.

\begin{prb}
Is $\dsp{\oa f(x,\ y)=\left(\frac{y}{x^2+y^2},\ \frac{-x}{x^2+y^2}\right)}$ a  gradient field?
\end{prb}

The next two examples remind us that work may be computed by integrating force.
\begin{annotation}
\endnote{We use the physics of work and force as the motivation for line integrals over vector fields.  To give an intuition for work, we work the two examples following this note: the work done in stretching a spring and the work done in pumping water out of a cone.  The recurring theme is first principles, the idea of adding up the force required to move one small slice over some distance. To introduce line integrals over vector fields, we sketch a vector field and a curve $c$ passing through the field, considering $c(t)$ as a boat's position at time $t$ as it moves through a turbulent bay with current $f(x,y)=(p(x,y),q(x,y))$ at $(x,y)$.  We point out that if the direction of the boat $c'(t)$ is perpendicular to the motion of the water $f(c(t))$, then no work is being done as we are not traveling against a current.  We neglect the wetted surface area, the coefficient of drag and other hydrodynamic factors affecting how much force is required to move the boat through the water. To find the total work associated with the current, we recall the fact that if $x$ and $y$ are vectors and $y$ is a unit vector then the length of the projection of $x$ onto $\{ \lambda y : \lambda \in \re \}$ is given by $x \cdot y$.  Therefore, if we make $c'(t)$ a unit vector by dividing by it's magnitude and integrate $\dsp f(c(t)) \cdot \frac{c'(t)}{|c'(t)|}$ with respect to arc length (to accommodate for the length of path), then we have $\dsp \int f(c(t)) \cdot \frac{c'(t)}{|c'(t)|} |c'(t)| \ dt$ and $|c'(t)|$ cancels leaving us with the total work as  $\int f(c(t)) \cdot c'(t) \ dt$.  We have thus illustrated that sometimes computing work requires writing out the sums and sometimes we may compute the work without resorting to Riemann sums.}
\end{annotation}

\begin{expl}
From physics we know that $Work=Force \cdot Distance$. So lifting a $5\ lb$ book $2\ feet$ requires $10\ ft \cdot lbs$ of work. Suppose we have a spring that has force $f$ proportional to the square of the distance it is stretched from equilibrium so that $f(x)=kx^2$ (where $k>0$ is the spring's coefficient). If $0=x_0 \leq x_1 \leq \dots \leq x_N=4$, then the work done to stretch it $4 \  units$ would be
\begin{eqnarray*}
W & = & \lim_{N\to\infty}\sum_{i=1}^Nk\hat x_i^2\ (x_{i+1}-x_i) \mbox{     where } \hat x_i \in [x_i, x_{i+1}]  \cr
& & =\int_0^4kx^2\ dx \mbox{     by the definition of the integral}.
\end{eqnarray*}
Thus in one dimension, work is the integral of force: $\dsp{W=\int{f(x)\ dx}}$.
\end{expl}

\begin{expl}
Suppose we have a cone full of water with radius $10\ ft$, height $15\ ft$, its point on the ground, and standing so that it holds the water. How much work is done in pumping all the water out of the cone?  Tools you will need are: Force = Mass $\cdot$ Acceleration (due to gravity) and Mass of one slice = volume of slice $\cdot$ Density of $H_2O$. Add (integrate) the amount or work done in moving each horizontal {\it slice} of water out of the tank.  The density of water is $62.5\ lbs/ft^3$ and acceleration due to gravity is $32\ ft/sec^2.$
\end{expl}

We now move to the study of vector (force) fields because we would like to be able to compute the total work done by an object as it moves through some field that acts on the object by either aiding or hindering its motion. As examples, think of a metal ball passing through a magnetic field, an electron passing through an electric field, a submarine passing through a fluid field, or a person passing through a gravitational field.  Suppose we have an object passing along a curve through a vector field. Let $\oa f: \re^2 \to \re^2$ be our (differentiable) vector field and $c: \re \to \re^2$ be our (differentiable) planar curve or path of the object. Therefore $\oa f$ is of the form, $\oa f(x,\ y)=\left(p(x,\ y),\ q(x,\ y)\right)$ for some $p,q: \re^2 \to \re$ and $c$ is of the form, $c(t)=(a(t),\ b(t))$ for some $a, b : \re \to \re.$  Thus we may compute the work done as the particle moves through the vector field by integrating over the field evaluated at the particle, dotted with the speed of the particle.  This gives the component of the force that is acting in the direction of travel of the particle.
\begin{eqnarray*}
W & = & \int \oa f \big(\oa c(t)\big) \cdot \oa c'(t)\ dt \cr
 & & =\int \big( \  p(a(t) \ ,\ b(t)),\ q(a(t),\ b(t)) \  \big) \  \cdot \  \big( a'(t) \ ,\ b'(t) \big) \ dt \cr
& & = \int p\big( \  a(t),b(t) \  \big) \  a'(t) + q\big( \  a(t),b(t) \  \big) \  b'(t) \  dt
\end{eqnarray*}

There are two additional ways in which vector line integrals are often written. First, the independent variable $t$ is often omitted:
$$W = \int_{\oa c} \oa f(\oa c(t)) \cdot \oa c'(t) \  dt = \int_{\oa c} \oa f(\oa c) \cdot d\oa c.$$
Second, if we $f(x,\ y)=(p(x,\ y),\ q(x,\ y))$ and $\oa c(t)=(x(t),\ y(t))$ then
\begin{eqnarray*}
W & = & \int f(\oa{c})\cdot d\oa{c} \cr
& & = \int \big(p(x,y),q(x,y)\big) \cdot \big(x'(t),y'(t)\big)\ dt \cr
& & = \int p\big(x(t),y(t)\big) \ x'(t)+q\big(x(t),y(t)\big) \ y'(t)\ dt \cr
& & = \int p \  dx + q \  dy
\end{eqnarray*}

We now have many different types of integrals to compute.  Here is a table of notations to help you determine which integral we are computing.  Some of these notations we have already seen, some are coming soon.
\begin{itemize}
\item $dx$ or $dt$ means the usual, for example: $$\dsp \int_1^2 x^2 \ dx = \int_1^2 t^2 \ dt$$
\item $ds$ means we integrate with respect to arc length, for example:  $$\dsp \int_{\oa c}  f \ ds = \int_{\oa c} f(\oa c(t))  |\oa c'(t) | \ dt$$
\item $d \oa c$ means a line integral over a vector field along some curve, for example: $$\dsp \int_{\oa c} \oa f \cdot  d \oa c =  \int_{\oa c} \oa f(\oa c(t)) \cdot \frac{\oa c'(t)}{|\oa c'(t)|} \ ds =  \int_{\oa c} \oa f(\oa c(t)) \cdot \frac{\oa c'(t)}{|\oa c'(t)|} |\oa c'(t)| \ dt   =  \int_{\oa c} \oa f(\oa c(t)) \cdot \oa c'(t) \ dt$$
\item $dA$ means we integrate over a two dimensional domain, for example the double integral: $$\dsp \int_D f \ dA =  \int  \int f(x,y) \ dx \ dy$$
\item $dV$ means we integrate over a three dimensional domain, for example the triple integral:  $$\dsp \int_D f \ dV =
\int \int \int f(x,y,z) \ dx \ dy \ dz$$

%\item $d \oa n$ means we integrate with respect to the outward unit normal $\oa n$ over the curve ${\oa c}$,
%$$\dsp \int_{\oa c} \oa f \cdot d \oa n
%= \int_{\oa c} \oa f \cdot \oa n \ ds
%= \int \oa f(\oa c(t)) \cdot \oa n(t) \ |\oa c'(t)| \ dt$$
\end{itemize}

%Gillette/Ted
\begin{prb}
Suppose we have a particle traveling through the force field, $\oa f(x,\ y)=(xy,\ 2x-y)$.
\begin{enumerate}
\item Compute the work done as the particle travels through the force field $f$ along the curve $\oa c(t)=(t,\ t^2)$ from $t=0$ to $t=1$.
\item Compute the work done as the particle travels through the force field $f$ along the curve $\oa c(t)=(2t,\ 4t^2)$ from $t=0$ to $t=\frac{1}{2}$.
\end{enumerate}
\end{prb}

What you have just computed is called a {\it line integral over a vector field}.

\begin{dfn}
The \textbf{line integral of the vector field} $\oa f: \re^2 \to \re^2$ given by $\oa f(x,y) = (p(x,y), q(x,y))$ over the curve $\oa c : \re \to \re^2$ given by $\oa c(t) = (x(t),y(t))$ is defined by  $$L = \int \oa f(\oa{c})\cdot d\oa{c}  = \int \Big(p\big(x(t),y(t)\big),q\big(x(t),y(t)\big)\Big) \cdot \big(x'(t),y'(t)\big)\ dt$$
\end{dfn}

%Gillette
\begin{prb}
\label{lineintegral} Compute the line integral of  $\oa f(x,\ y)=(4x+y,\ x+2y)$ over $\oa c$ where $\oa c$ is the curve bordering the rectangle with corners at $(0,\ 0),\ (1,\ 0),$ $(1,\ 2),\ (0,\ 2).$  Start your path at the origin and work counter-clockwise around the rectangle.  You will need to compute and sum four integrals along four parameterized lines.
\end{prb}

You have been using the Fundamental Theorem of Calculus for three semesters now.  This powerful result in one dimension is all that is needed to prove the version we need for three dimensions.

\begin{thm} \textbf{The Fundamental Theorem of Calculus.} If $F$ is any anti-derivative of $f$, then $\dsp{\int_a^b{f(x)\ dx=F(b)-F(a)}}$.
\end{thm}

\begin{thm}\textbf{The Fundamental Theorem of Calculus for Line Integrals.} \label{ftcIII}
If $\oa f$ is a \textbf{gradient field} and $g$ is any anti-derivative of $\oa f$ (i.e. $\nabla g=\oa f$), then $$\dsp{\int_{\oa c}{\oa f(\oa{c}) \cdot {d\oa{c}}= g\left(\oa c(b)\right)-g\left(\oa c(a)\right)}}.$$
\end{thm}

\begin{prb}
Prove Theorem \ref{ftcIII}.
\end{prb}

\begin{prb}
Compute the line integral from Problem \ref{lineintegral} using Theorem \ref{ftcIII}.
\end{prb}

The next two theorems are important results associated with line integrals. We don't assert all the necessary hypothesis -- curves are smooth, functions are assumed to be integrable, etc.

Theorem \ref{reverse} states that if we reverse the path along which we compute a line integral, then we change the sign of the result.  Thinking of the line integral as work, this makes sense, because the work done by traveling one direction along a path should be the opposite of the work done traveling the other way. When we have a curve $\oa{c}$ and we write $\oa{-c}$ we mean the same set of points in the plane, but we simply reverse the direction. If we have the curve $\oa{c}(t) = (x(t),y(t))$ from $t=a$ to $t=b$, then $\oa{-c}$ is simply the same curve where $t$ goes from $t=b$ to $t=a$.

\begin{thm}\label{reverse}
\textbf{Path Reversal Theorem for Line Integrals.} If $\oa c:{\re}\to{\re}^2$ is a parametric curve and $\oa f$ is a vector field, then $$ \int_c  \oa{f}(\oa{c})\cdot{d\oa{c}}=-\int_{-c}{\oa{f}(\oa{c})\cdot{d\oa{c}}}.$$
\end{thm}

%FRANCESR T
\begin{prb}
Show that $\dsp{\int_{\oa{c}}{\oa f(\oa{c})\cdot d{\oa{c}}}=-\int_{\oa{-c}}\oa f(\oa{c})\cdot{d{\oa{c}}}}$ where $\oa f(x,\ y)=(xy,\ y-x),\ \oa c(t)=(t^2,\ t),\ \ t\in[0,\ 2]$. Since $c$ is the path from $(0,\ 0)$ to $(4,\ 2)$, then $-c$ is the same path but from $(4,\ 2)$ to $(0,\ 0)$.
\end{prb}

Theorem \ref{independence} states that line integrals over conservative fields are independent of path.   Suppose you and I start at the same point at the bottom of a mountain and we walk  to the top following different paths.  Did we do the same amount of work? Yes.  Because gravity is a conservative (gradient) field it does not matter what path we follow as long as we both start at the same place and end at the same place. By the same logic, if we start at a point on the mountain, walk around, and return to the same spot, then the total work is zero.

\begin{thm}\label{independence}
\textbf{Independence of Path Theorem for Line Integrals.} If $\oa{c_1}$ and $\oa{c_2}: \re \to \re^2$ are two (distinct) paths beginning at the point $(x_1,\ y_1)$  and ending at the point $(x_2,\ y_2)$ in the plane and $\oa f$ is a  gradient field, then $$\int_{c_1}{\oa{f}(\oa{c})\cdot{d\oa{c}}= \int_{c_2}{\oa{f}(\oa{c})\cdot{d\oa{c}}}}$$
\end{thm}

% Ted
\begin{prb}
Let $f$ be the vector field $\oa f(x,\ y)=(3x^2y+x,\ x^3)$ and $c$ be the planar curve $\oa c(t)=(\cos(t),\sin(t))$ from $t=0$ to $t=\pi$.
\begin{enumerate}
\item Write out and simplify, but don't compute, the line integral of $f$ over $\oa c$.
\item Compute this line integral by choosing the simpler path $\oa p(t)= (-t,0)$ from $t=-1$ to $t=1$ and applying Theorem \ref{independence}.
\item Recompute this line integral by finding a function $g$ so that $\nabla  g = (3x^2y+x,\ x^3)$ and applying Theorem \ref{ftcIII}.
\end{enumerate}
\end{prb}

\begin{prb}
Let $\oa f(x,\ y)=(x-y,\ x+y).$
\begin{enumerate}
\item Compute $\dsp \int_{\oa c_1} \oa f(\oa{c})\ d\oa c$ where  $\oa c_1 (t)=(t,\ t^2),\ \ t\in[0,\ 1].$
\item Compute $\dsp \int_{\oa c_2} \oa f(\oa{c})\ d\oa c$ where $\dsp \oa c_2 (t)=\big(\sin(t),\ \sin^2(t)\big)\ \ t\in[2\pi,\ \frac{5\pi}{2}]$.
\item Explain your answer.
\end{enumerate}
\end{prb}

When we compute line integrals, we often integrate along a curve that is the boundary of some region. Here are a few of the buzz words about curves that we will use.

\begin{dfn}
Let $\oa{c}:\ [a,b] \to{\re}^2$.   We say that $\oa c$ is a  \textbf{simple closed curve} if $\oa c$ starts and ends at the same point (i.e. $\oa{c}(a)=\oa{c}(b)$) and never crosses itself.
\end{dfn}

When we are integrating around a simple closed curve $\oa{c}$ with respect to the arc length of the curve,  we will use the notation $\dsp \oint_{\oa c} \cdots \ ds.$   The circle indicates that the curve is closed, starting and ending at the same point in the plane.

\begin{dfn}
Let $\oa{c}:\ [a,b] \to{\re}^2$ be a simple closed curve on $[a,\ b]$. We say that $\oa c$ is \textbf{positively oriented} if as $t$ increases from $a$ to $b$ and we traverse the curve $\oa c$ then the enclosed region is on our left.
\end{dfn}

As we will show in a forthcoming lecture, Gauss' Divergence Theorem is equivalent to Green's Theorem.  They are simply the same theorem stated in two different ways.  Gauss' Divergence Theorem says that if we have a certain simple closed curve representing the boundary of a region over which we have a vector field (a flow), then the flow across boundary (the flux) must equal the integral of the divergence of the fluid (or the electricity or whatever) over the region enclosed by the boundary.
\begin{annotation}
\endnote{Rather than proving Gauss' Divergence Theorem, we spend considerable time trying to motivate the physical interpretation, explaining that if we are discussing the movement of particles inside a region via the function $f$ then Gauss' Theorem says that the integral of the divergence of the flow within the region equals the integral of the flow across the boundary of the region.  To put this in context, we give brief lectures on computing the flux (flow across a boundary), the divergence of elementary vector fields, and computing normals to curves.  Then we illustrate with a simple example.
\begin{enumerate}
\item FLUX: Suppose we cover the plane with a one-foot deep pool of water that is flowing to the right at 3 ft/sec. How much water flows across the line segment from $(2,1)$ to $(2,3)$?  We can compute this via common sense as $$flow \cdot width \cdot height = 3 ft/sec \cdot 2 \cdot 1  = 6 ft^3/sec.$$  We can also compute this by considering the flow as a vector,
$$flow \cdot (unit \ normal \ to \ flow) \cdot width = (3,0) \cdot (1,0) \cdot 2 .$$  What would be the flow across the line segment from $(0,0)$ to $(2,0)$. By logic, it is zero, which is also what we get when we consider the vector computation $$(3,0) \cdot (0,-1) \cdot 2.$$ What if we consider the flow across the line segment from $(1,1)$ to $(2,0)$?  Using our vector computation, $$(3,0) \cdot (\frac{\sqrt{2}}{2},\frac{\sqrt{2}}{2}) \cdot \sqrt{2} = 3.$$  This result makes sense because the flow across the line segment from $(0,0)$ to $(0,1)$ is also $3 ft^3/sec$ and whatever flows across this line segment must also flow across the line segment from $(1,1)$ to $(2,0)$.

\item DIVERGENCE: Suppose that we have the vector field, $\oa f: \re^2 \to \re^2$ given by $\oa f(x,y) = (p(x,y),q(x,y))$.  Recall that the divergence, $\nabla \cdot \oa f = p_x + q_y$, represents the sum of the velocity in the $x$ direction of the first component of the flow and the velocity in the $y$ direction of the second component of the flow.  Thus, divergence is a measure of the velocity of the flow.  We sketch vector fields for:
\begin{enumerate}
\item $\oa f(x,y) = (1,0)$ -- all flow to the right and $\nabla \cdot \oa f = 0$ since at every point, the flow in = the flow out
\item $\oa f(x,y) = (x,0)$ -- all flow outward and $\nabla \cdot \oa f = 1 > 0$ since at every point not on the y-axis, there is more flow out than in
\item $\oa f(x,y) = (x,y)$ -- all flow outward and $\nabla \cdot \oa f = 2$
\item $\oa f(x,y) = (-x,-y)$ -- all flow inward and $\nabla \cdot \oa f = -2$
\end{enumerate}
\item NORMALS TO CURVES: Suppose $\oa c(t)=(x(t),y(t))$ is a parametric curve.  We already know that the tangent to the curve is given by $\oa c'(t)= (x'(t),y'(t))$.  For any time $t$, the two vectors, $(-y'(t),x'(t))$ and $(y'(t),-x'(t))$ both perpendicular to $\oa c'(t)$.  By working a simple example, such as $\oa c(t) = (5\cos(t),2\sin(t))$ we can determine that the second one is the outward normal.
\item EXAMPLE: Verify Gauss' Divergence theorem, $\dsp{\oint_{\oa{c}}\oa{F}\cdot{n}\ ds=\int_D \nabla\cdot\oa{F}\ dA}$, for $\oa F(x,y) = (-2x,-2y)$ over the circle of radius 1 centered at (0,0).
\end{enumerate}}
\end{annotation}

For both Green's and Gauss' Theorems, when integrating along the boundary of the region the simple closed curve must be positively oriented. That is, you must integrate in a counter-clockwise direction so that the region is on your left as you travel along the parametric curve.

\begin{thm}
\textbf{Gauss' Divergence Theorem for the Plane.} If $F$ is a vector field and $\oa c$ is a simple closed curve and $\oa n$ is the unit normal to $\oa c$, then $\dsp{\oint_{\oa{c}}\oa{F}\cdot{n}\ ds=\int_D \nabla\cdot\oa{F}\ dA}$
\end{thm}

\begin{expl}
Verify Gauss' Divergence theorem for $\oa f(x,y) = (-2x,-2y)$ over the triangular domain with vertices (0,0), (0,1) and (1,1).
\end{expl}

%G/T
\begin{prb}
Verify Gauss' Divergence Theorem by computing both integrals (the flux integral and the divergence integral) with $\oa f(x,\ y)=-4(x,\ y)$ and $\oa c(t)=(\cos(t),\ \sin(t))$ assuming $t \in [0, 2\pi].$
\begin{annotation}
\endnote{As soon as this problem goes up, one can re-emphasize the understanding of Gauss' Theorem by computing the flow across the border at several points.  The flow across the border at (0,1) would be $\oa f(0,1) \cdot (0,-1) = (0,-4) \cdot (0,-1) = 4$ while the flow at ($\frac{\sqrt{2}}{2},\frac{\sqrt{2}}{2})$ would be $\oa f(\frac{\sqrt{2}}{2},\frac{\sqrt{2}}{2}) \cdot (-\frac{\sqrt{2}}{2},-\frac{\sqrt{2}}{2}) = (-2\sqrt{2},-2\sqrt{2}) \cdot (-\frac{\sqrt{2}}{2},-\frac{\sqrt{2}}{2}) = 4.$  Hence the flow across the border at every point on the circle is 4 and the circumference is $2\pi$ so it makes sense that the solution is $-8\pi$.}
\end{annotation}
\end{prb}


%G
\begin{prb}
Verify Gauss' Divergence Theorem for $\oa f(x,y) = (xy, 2x-y)$ over the region $D$ in the first quadrant bounded by $y=0$, $x=0$ and the line $y=1-x$.
\end{prb}

%G
\begin{prb}
Verify Gauss' Divergence Theorem for the flow $f(x,y) = (0,y)$ over the rectangular region, $\{(x,y) : 0 \le x \le 2, 0 \le y \le 3 \}.$
\end{prb}

\begin{thm}
\textbf{Green's Theorem}\begin{annotation}
\endnote{We sketch pictures for both Green's Theorem and Gauss' theorem to illustrate that they are close cousins. The line integral associated with Gauss' Theorem has us integrating the component of $f$ that is \emph{perpendicular} to the boundary, while the line integral associated with Green's Theorem has us integrating the component of $f$ that is \emph{parallel} to the boundary.  Then we verify Green's theorem for two problems.
\begin{enumerate}
\item $f(x,y) = (x-y,x+y)$ over the circle $c(t) = (2\sin(t),2\cos(t))$
\item $f(x,y) = (4y, 3x+y)$ over the region bounded by the x-axis and $y=\cos(x)$
\end{enumerate}
For the first example we get $8\pi$ one way and $-8\pi$ the other way.  Why?  Because our circle is not positively oriented. The second example reminds us how to integrate by parts!

While we don't prove either Gauss' or Green's Theorem, we do show that they are equivalent.  We may use the last few minutes of a class period to state the two theorems and reemphasize that they are very similar by reminding them that the line integral associated with Gauss' Theorem has us integrating the component of $f$ that is \emph{perpendicular} to the boundary, while the line integral associated with Green's Theorem has us integrating the component of $f$ that is \emph{parallel} to the boundary.  This and starting the theorem by writing the first two lines below may be enough that a student can give the proof the next day.  We conclude the class by saying that if a student has it, then s/he can present it and, if not, then after we present problems the next day, I will present it.  Even if I present it, then I assign as homework for the class to present the other direction, that Green's implies Gauss'. The proof we give follows.\\ \\ Suppose we have proven Gauss' Theorem and wish to prove Green's where
$f(x,y) = (u(x,y),v(x,y))$ and $c$ is the simple, closed, positively oriented curve $c(t) = (x(t),y(t))$.
\begin{eqnarray*}
\oint_{\oa c} f(\oa{c}) \ d\oa{c} & = & \oint_{\oa c} f(x(t),y(t)) \cdot (x'(t),y'(t)) \ dt \cr
& = &   \oint_{\oa c} (p(x(t),y(t)),q(x(t),y(t))) \cdot (x'(t),y'(t)) \ dt \cr
& = &   \oint_{\oa c} p(x(t),y(t)) x'(t) + q(x(t),y(t)) y'(t)) \ dt \cr
& = &   \oint_{\oa c} p  x' + q y' \ dt \cr
& = &   \oint_{\oa c} (q,-p) \cdot (y',-x') \ dt \cr
& = &   \oint_{\oa c} g \cdot n \ ds \ \ \mbox{ where } g(x,y) = (q(x,y),-p(x,y)) \cr
& = &   \oint_{\oa c} \nabla \cdot g \ dA \ \ \mbox{ by Gauss' Theorem } \cr
& = &   \oint_{\oa c} ( \frac{\partial}{\partial x}, \frac{\partial}{\partial y}) \cdot (q(x,y),-p(x,y)) \ dA \cr
& = &   \oint_{\oa c} q_x - p_y \ dA
\end{eqnarray*}}
\end{annotation}
If $\oa f:\re^2\to\re^2,\ \oa f(x,\ y)=\big(p(x,\ y),\ q(x,\ y)\big)$,  is a vector field and $\oa c: \re \to \re^2$ is a positively oriented  simple closed curve so that $\oa c(t) = (x(t),y(t))$ and  $D$ is the region enclosed by $\oa c$, then  $$\oint_{\oa c}{\oa{f}(\oa{c})\cdot{d\oa{c}}}=  \int_D q_x(x,y) - p_y(x,y) \ dA$$

\end{thm}

\begin{expl}
Verify Green's theorem for  $\oa f(x,y) = (4y, 3x+y)$ over the region bounded by the x-axis and $y=\cos(x)$.  Compute both sides of Green's theorem for $\oa f(x,y) = (x-y,x+y)$ over the circle $\oa c(t) = (2\sin(t),2\cos(t))$ and explain why they are not equal.
\end{expl}

\begin{prb}
Verify Green's Theorem for the flow $\oa f(x,y) = (-y^2, xy)$ over the region bounded by $x=0, y=0, x=3, \mbox{ and } y=3.$
\end{prb}

\begin{prb}
Verify Green's Theorem for the vector field $\oa f(x,y) = (-y,x)$ over the region bounded by the curve $\dsp \frac{x^2}{4} + \frac{y^2}{9} = 1.$
\end{prb}

\begin{prb}
Verify Green's Theorem for the vector field $\oa f(x,y) = (xy,3x)$ over the region formed by the three points, $(-1,0), (1,0),$ and
$(0,4).$
\end{prb}

\begin{thm}
\textbf{An Area Theorem.}
\begin{annotation}
\endnote{This is one I actually prove for them.  Proof.  We have intentionally written this formula as an engineer or physicist might.   To prove it, the first step we will take is to rewrite it as a mathematician.
\begin{eqnarray*}
\frac{1}{2} \int x \ dy - y \ dx  & = & \frac{1}{2} \int x(t)y'(t) -  y(t)x'(t) \ dt \cr
& = & \frac{1}{2}  \int (-y(t),x(t)) \cdot (x'(t),y'(t)) \ dt \cr
& = & \frac{1}{2}  \int f(x(t),y(t)) \cdot (x'(t),y'(t)) \ dt \mbox{ where } f(x,y) = (-y,x)  \cr
& = & \frac{1}{2}  \int f(c(t)) \cdot c'(t) \ dt  \mbox{ where } c(t)=(x(t),y(t))   \cr
& = & \frac{1}{2}  \int f(c) \ dc \cr
& = &  \frac{1}{2} \int v_x - u_y \ dA \ \mbox{ by Green's Theorem} \cr
& = &  \frac{1}{2} \int 1 - (-1) \ dA \cr
& = &  \int 1 \ dA  = \mbox{ area! }
\end{eqnarray*}\emph{q.e.d.}}
\end{annotation}
If we have a simple, closed, positively oriented parametric curve $\oa c(t) = (x(t),y(t))$, then the area enclosed by the curve may be computed by: $\dsp A = \frac{1}{2} \int_a^b x \ dy - y \ dx$.
\end{thm}

\begin{prb}
Find the area of the ellipse $\dsp \frac{x^2}{a^2}+\frac{y^2}{b^2}=1$ via the formula $\dsp{A=\frac{1}{2}\int{x\ dy-y\ dx}}$ and the parametrization $\oa c(t)=(a\ \cos(t),\  b\ \sin(t)),\ t\in[0,\ 2\pi]$.
\end{prb}

Gauss' Divergence Theorem is valid in higher dimensions as well, although it is often the case that integrating over certain parts of the boundary is challenging.
\begin{annotation}
\endnote{Before introducing Gauss' Theorem in higher dimensions, there should probably be some problems on surface integrals. At the very least, one needs to define and explain that if we have a simple parameterized surface, $h(s,t) = (s,t,g(s,t))$ then $dS = \sqrt{1 + g_s^2 + g_t^2}$.  We typically mention this while working through two examples.\\ \\
Example 1. Verify Gauss' Theorem for $F(x,y,z) = ( x, y, z)$ over the sphere of radius 1.
\begin{eqnarray*}
\int_{\delta S} F \cdot n \ dS &  = & \int_{\delta S} (x,y,z) \cdot \frac{(x,y,z)}{\sqrt{x^2+y^2+z^2}} \ dS \cr
& = & \int_{\delta S} \frac{x^2+y^2+z^2}{\sqrt{x^2+y^2+z^2}} \ dS \cr
& = & \int_{\delta S} 1 \ dS \cr
& = & \mbox{ surface area of the sphere } \cr
& = & 4\pi 1^2 = 4 \pi
\end{eqnarray*}
and
\begin{eqnarray*}
\int_S \nabla \cdot F \ dV & = & \int_S 3 \ dV \cr
& = & 3 \frac{4}{3}\pi 1^3 \cr
& = & 3 \mbox{ volume of a sphere} \cr
& = & \frac{4}{3}\pi 1^3 \cr
& = & 4\pi
\end{eqnarray*}
Example 2. Let $S$ be the solid bounded by $x^2+y^2=4$, $z=0$, and $x+z=6$. Let $F(x,y,z) = (x^2, xy,e^y)$. We can compute one side easily,
$$\int_S \nabla F \cdot n \ dV = \int_S 3x \ dV = \int_0^{2\pi} \int_0^2 \int_0^{6-r\cos(\theta)} 3r\cos(\theta) r \ dz \ dr \ d \theta = \dots = -12\pi.$$
We can set up the other side, although the integrals are sufficiently difficult that both Maple and a TI Voyage 200 numerically approximate the solutions.}
\end{annotation}

\begin{thm}
\textbf{Gauss' Divergence Theorem for Three Dimensions.}   If $S$ is a solid in three dimensional space, we write $$\int_{\delta{S}}f\cdot{n}\ dS\ =\ \int_S \nabla\cdot{f}\ dV$$ where $\delta S$ denotes the boundary of $S$, $n$ denotes the unit outward normal and $dV$ indicates that we are integrating over the entire volume of the solid.
\end{thm}

\begin{prb}
Verify Gauss' Divergence Theorem for the vector field $\oa f(x,y,z) = (x^2y, 2xz, yz^3)$ over the three dimensional box, $0 \le x \le 1, 0 \le y \le 2, \mbox{ and } 0 \le z \le 3.$
\end{prb}
%works out nicely on all sides

\begin{prb}
Let $S$ be the cylinder of radius 2 with base centered at the origin and height 3.  Let $\oa f(x,y,z) = (x^3 + \tan(yz),  y^3 - e^{xz}, 3z + x^3)$.    Use the divergence theorem to compute the flux across the side of the cylinder.
\end{prb}
%remember, need to compute \int divergence - \int_top - \int_bottom because side integral is horrid

\begin{prb}
Let $S$ be the solid bounded by $2x+2y+z=6$ in the first octant.  Let $\oa f(x,y,z) = (x, y^2, z)$.  Sketch the solid and verify Gauss' Divergence Theorem.
\end{prb}

Congratulations, for some of you this note constitutes having independently worked through three semesters of Calculus, which is quite an accomplishment.

\section{Practice} \label{chap13probs}

We will not present the problems from this section, although you are welcome to ask about them in class.

\vskip .1in
\noindent
\textbf{Note.}  In this section, we write the functions and integrals using many different notations.  If you are unsure about the meaning of a notation, please ask!

\vskip .1 in
\noindent
\textbf{Vector Fields, Curl, and Divergence}

%Gillette
\begin{enumerate}

\item Sketch the vector field, $\oa{f}(x,y)=x^{2}{\vec{i}}+{\vec{j}}.$
\item Sketch the vector field, $\oa{g}(x,y)=(x,-y).$
\item Sketch the vector field, $\oa{h}(x,y,z)=y{\vec{j}}.$

\item
Compute the divergence and curl of the vector field,
$\oa f(x,y,z) = ( y^2 z, x^3 + z + y, \cos(xyz) ).$

\item
Find $g$ satisfying $\nabla  g=F$ if it exists.
$\dsp F(x,\ y)=(ye^{xy}+2x) \oa i +  (xe^{xy}-2y)\oa j$

\item
Find $g$ satisfying $\nabla  g=F$ if it exists.
$\dsp{F(x,\ y)= (e^x\sin(y),\ e^x\cos(y))}$

\item
Is ${\oa{f}}(x,y)=(2xy,x^2)$ a conservative vector field? If so, find a potential for it.

\item
Is ${\oa{h}}(x,y)=(y \cos(x), \sin(x))$ a conservative vector field? If so, find a potential for it.

\item
Is ${\oa s}(x, y, z)=(e^x \sin z + yz) \vec i + (xz + y) \vec j + (e^x \cos z + xy + z^2) \vec k$
a conservative vector field? If so, find a potential for it.

\item
Is ${\oa{g}}(x,y)=2x{\vec{i}}+y{\vec{j}}$
a conservative vector field? If so, find a potential for it.

\item
Show that ${\bf{F}}(x, y, z)= x{\vec{i}}+y{\vec{j}}+2z{\vec{k}}$ is
conservative and find a function $f$ such that ${\bf{F}}=\nabla f$.

\end{enumerate}

\noindent
\textbf{Line Integrals over Scalar Fields}

\begin{enumerate}

\item Let $f(x,y) = x+y$ and ${\oa c}$  be the unit circle in $\re^{2}.$ Evaluate $\int_{\oa c} f \ ds.$   Recall that $ds$ means to evaluate with respect to the arc length.

\item Evaluate $\int_{\oa c} \sqrt{xy+2y+2}\ ds$ with ${\oa c}$ the line segment from $(0,1)$ to $(0,-1).$

\item Evaluate $\int_{\oa c} (x-y+z-2)\ ds$ where ${\oa c}$ is the line segment from $(0,1,1)$ to $(1,0,1).$

\end{enumerate}

\noindent
\textbf{Line Integrals over Vector Fields}
\begin{enumerate}

\item Compute $\dsp \int_{\oa r} f \cdot \; dr$ where $f(x,y) = (y,x^2)$ and $\oa r(t) = (4-t, 4t-t^2)$ for $0 \leq t \leq 3.$

\item Compute $\dsp \int_{\oa r} f \cdot \; dr$ where $f(x,y) = (y,x^2)$ and $\oa r(t) = (t, 4t-t^2)$ for $1 \leq t \leq 4.$

\item Compute $\dsp \int_{\oa r} F \cdot \; dr$ where $F(x,y) = (-\frac{1}{2}x, -\frac{1}{2}y, \frac{1}{4})$ and $\oa r(t) = (\cos(t),\sin(t),t),$ $1 \leq t \leq 4.$

\end{enumerate}

\noindent
\textbf{Divergence Theorem and Green's Theorem}

\begin{enumerate}

%G
\item
Verify the divergence theorem for the flow $f(x,y) = (0,y)$ over the circle, $x^2 + y^2 = 5.$

\item
Let $f(x,y) = (u(x,y),v(x,y))=(-x^2y,xy^2)$ and ${\oa c} = \{(x,y): x^2 + y^2 = 9 \}$ and $D$ be the region bounded by $\oa c.$ Verify Green's Theorem by evaluating both $\dsp \oint_{\oa c} f(\oa x) d\oa x$ and $\dsp \int \int_D v_x(x,y) - u_y(x,y) dA.$

%\item
% Ted is this problem ok, if so, keep it, if not fix it.
%Use Green's Theorem to evaluate:
%$\dsp \oint_{\oa c} (1- \cos(y))\ dx - (y- \sin(y))\ dy$ where
%${\oa c} = {\oa c}_1 \cup {\oa c}_2$ and ${\oa c}_1$ is represented
%by $y= \sin(x)$ from $(0,\pi)$ to $(0,0)$ (i.e. in the positive
%direction) and ${\oa c}_2$ is the line segment from $(0,0)$ to
%$(0, \pi)$.

\item
Verify Green's Theorem where $f(x,y) = (4xy,y^2)$ and ${\oa c}$ is the curve $y=x^3$ from the $(0,0)$ to $(2,8)$ and the line segment from $(2,8)$ to $(0,0)$.

\item If you want more practice on verifying Green's and Gauss' theorems, then note that each problem that asks you to verify Gauss' theorem could have asked you to verify Green's theorem and vice-versa.  You won't need solutions because you are computing both sides of the equation and they must be equal if all your integration is correct.

\end{enumerate}

\noindent
\textbf{Divergence Theorem in Three Dimensions}

\begin{enumerate}

%larson p1079
\item  Verify the divergence theorem for $f(x,y,z) = (xy,z,x+y)$ over the region in the
first octant bounded by $y=4$, $z=4-x$, $z=0$, $y=0$, and $x=0$.

\item Verify the divergence theorem for $f(x,y,z) = (2x,-2y,z^2)$ over the region
$[0,3] \times [0,3] \times [0,3]$.

\end{enumerate}


%TED
%\item
%Prove Green's Theorem.  Hint:
%With $D$, $u$ and $v$ as in Green's Theorem, prove the
%following results.
%\begin{enumerate}
%\item $\dsp \oint_{\partial D} u(x,y) \ dx = -\int \int_{D}
%\frac{\partial u}{\partial y} \ dx \ dy$
%\item $\dsp \oint_{\partial D} v(x,y) \ dy =\int \int_{D}
%\frac{\partial v}{\partial x} \ dx \ dy$
%\end{enumerate}


\vskip .5in
\noindent
\textbf{Chapter 13 Solutions}\\
\vskip .1 in
\noindent
\textbf{Vector Fields, Curl, and Divergence}

%Gillette
\begin{enumerate}

\item no sketch
\item no sketch
\item no sketch

\item no solution

\item  $g(x,y) = e^{xy}+x^2-y^2$

\item  $g(x,y) = e^{x}\sin(y)$

\item   yes, $g(x,y) = x^2y$

\item  yes, $g(x,y) = y\sin(x)$

\item yes, $g(x,y,z) = e^x\sin(z) + xyz + \frac{1}{2}y^2 + \frac{1}{3}
z^3$

\item no solution
\item no solution

\end{enumerate}

\noindent
\textbf{Line Integrals over Scalar Fields}

\begin{enumerate}

\item $0$

\item  $8/3$

\item $-\sqrt{2}$

\end{enumerate}

\noindent
\textbf{Line Integrals over Vector Fields}
\begin{enumerate}

\item $69/2$

\item $-69/2$

\item $3\pi/4$

\end{enumerate}

\noindent
\textbf{Divergence Theorem and Green's Theorem}
\begin{enumerate}

\item If both sides are equal, you probably have the right answer!

\item  $81\pi/2$

%\item  $\dsp \int_0^{\pi} \int_0^{\sin(x)} 1 - \cos(y) \ dy \ dx
%\approx .213513$

\item $-256/15$

\item no solution

\end{enumerate}

\noindent
\textbf{Divergence Theorem in Three Dimensions}
\begin{enumerate}

\item Hopefully you got the same answer both ways!

\item Hopefully you got the same answer both ways!

\end{enumerate}



\chapter{Appendices}

\section{Additional Materials on Limits} \label{applim}

\setcounter{dfn}{0}
\setcounter{prb}{0}
\setcounter{axm}{0}
\setcounter{expl}{0}
\setcounter{lem}{0}
\setcounter{thm}{0}

Our definition of limits in this book is an intuitive one.  The statement that ``$f(x)$ gets close to $f(a)$ as  $x$ gets close to $a$'' is not mathematically precise because of the word `close.' What does `close' mean?  Is $.5$ close to zero?  Is $.001$ close to zero? Mathematicians spent hundreds of years defining what `close' meant and the following definition was the result of the work of Newton and Leibniz, among others.  This definition gave rise to precise definitions for continuity, derivatives, and integrals, which had theretofore been defined only intuitively. This definition made precise the entire field of analysis which is one of the most applied fields in all of mathematics.  If the notion seems non-trivial or the definition challenging, then you are in the company of great men. If you find success in understanding the next few problems, then you are in the company of a very few and should exploit your understanding of such deep concepts with a major or minor in the subject.

\begin{dfn}
If $L$ and $a$ are numbers, then we say that {\bf the limit of $f$ at $a$ is $L$} if for any two horizontal lines $y=H$ and y=K with the line $y=L$ between them, there are two vertical lines, $x=h$ and $x=k$ with $x=a$ between them, so that if $t$ is any number between $h$ and $k$ (other than $a$), then $f(t)$ is between $H$ and $K$.
\end{dfn}

Here is a sketch of how we might use this definition to {\it prove} the statement: $\lim_{x \rightarrow 2} x^2 + 3 = 7.$  First, we do some preliminary analysis.  That is, even though to prove this statement, we must show that the definition holds for {\it any} choice of $H$ and $K$, we will start out with a specific choice of $H$ and $K$.

\begin{prb}
Let $f(x) = x^2 + 3,$ $H=6$, and $K=8$.  Find values $h$ and $k$ so that for all numbers $t$ between $h$ and $k$ we have $f(t)$ between $H$ and $K$. Be sure to {\it prove} that if $h < t < k$ then $H < f(t) < K$.
\end{prb}

Does this {\it prove} the theorem?  No.  We must show that we can solve the previous problem for {\it any} choice of $H$ and $K$.

\begin{prb}
Let $f(x) = x^2 + 3.$ Let $y=H$ and $y=K$ be two horizontal lines with $y=7$ between them. Find vertical lines $x=h$ and $x=k$ satisfying the definition of the limit.
\end{prb}

\begin{prb}
Prove: $\lim_{x \rightarrow 2} x^2 + 3 = 7$ by showing that if $t$ satisfies $h < t < k$ then $H < f(t) < K.$
\end{prb}

This definition can be used to prove some of the theorems about limits that we used. Here are a few that you might enjoy working on.

\begin{prb}
Prove that if $a \in \re$ then $\lim_{x \rightarrow a} x = a.$
\end{prb}

\begin{prb}
Prove that if $a \in \re$ then $\lim_{x \rightarrow a} x^2 + 3 = a^2 + 3.$
\end{prb}

\begin{prb}
Prove that if f and g are functions and $\lim_{x \rightarrow a} f(x) = f(a)$ and $\lim_{x \rightarrow a} g(x) = g(a)$ then and $\lim_{x \rightarrow a} \left( f(x) + g(x) \right)  = f(a) + g(a).$
\end{prb}

\section{The Extended Reals} \label{appreal}

\setcounter{dfn}{0}
\setcounter{prb}{0}
\setcounter{axm}{0}
\setcounter{expl}{0}
\setcounter{lem}{0}
\setcounter{thm}{0}

Have you ever wondered exactly what {\it infinity} really {\it is}? Is it a number?  Can you multiply by it?  Can you divide by it?  Does $\dsp \frac{\infty}{\infty} = 1$?  Well if you have wondered, then this appendix is for you!

In this text, we try to define precisely most of the words and symbols that we use.  Yet we did not define the real numbers, and to develop them from scratch is a non-trivial task.  Thus, the real numbers are an ``undefined'' term in this text.  One philosophy that separates mathematics from other less scientific fields is our constant attempt to make clear the distinction between that which we are assuming (axioms or undefined terms) and that which we have proved (theorems, lemmas, or corollaries). We will see examples in this course of each.  We will not define what a ``real number'' is, but we will now define the ``extended reals'' assuming that the reals exist.  We will give only an intuitive notion of the definition of a ``limit'' but we will define ``derivatives'' precisely based on our intuitive understanding of limits.

Where does this leave us, if we build the subject on undefined terms and intuitive definitions?    It leaves us in better shape than most subjects!  While we have not defined everything, we know what we have defined and what we have proved based on those definitions.  Should you find mathematics worthy of further exploration, you can take courses where the real numbers are developed (based on even more elementary axioms and assumptions) and where limits are studied in rigorous detail. Can we go all the way back to the beginning? I will if you will define the word ``beginning''.  Enough mathematical philosophy, let's get back to the extended reals.

\begin{dfn}
Assuming the existence of the real numbers, we define the {\bf extended reals} to be the set of all real numbers along with two new numbers which we will call ${\bf +\infty}$ and ${\bf -\infty}.$
\end{dfn}

We already know many properties of the real numbers.  We know how to add, subtract, multiply, and divide.  But which of these properties follow for the real numbers?  Here are a few:

\begin{axm}
$\infty + a = \infty$ for all real numbers, a.
\end{axm}

\begin{axm}
$\infty + \infty = \infty.$
\end{axm}

\begin{axm}
\label{ax}
$\infty \cdot \pm a = \pm \infty$ for all positive real numbers,
a.
\end{axm}

\begin{axm}
$\infty \cdot  \infty = \infty.$
\end{axm}

\begin{axm}
$\infty / \pm a = \pm \infty$ for all positive real numbers, a.
\end{axm}

\begin{axm}
$a / \infty  = 0$ for all real numbers, a.
\end{axm}

Our interest with the numbers $\pm \infty$ stems from limits.  For example, what is $$\dsp \lim_{x \rightarrow \infty} 3x?$$   Some would say that the limit does not exist since there is no {\it real} number that $f(x) = 3x$ approaches as $x$ approaches $\infty.$  Others would say that the function $f$ increases without bound as $x$ increases without bound.  The latter is our personal favorite, but in the interest of notational efficiency, and having defined the extended reals, we will write, $$\lim_{x \rightarrow \infty} 3x = \infty.$$  Note that this follows from Axiom \ref{ax} above.

Knowing what is {\bf not} true is at least as important as knowing what {\bf is} true.  Here are some cases called {\bf indeterminate} forms where the rules you might expect to be true do not necessarily hold.    Consider $f(x) = x-5$ and $g(x) = x.$ Clearly,  $$\dsp \lim_{x \rightarrow \infty} f(x) = \infty \;\;\; \mbox{     and     } \;\;\; \lim_{x \rightarrow \infty} g(x) = \infty.$$ But what about $\dsp \lim_{x \rightarrow \infty} \left( f(x) - g(x) \right) ?$
$$\lim_{x \rightarrow \infty} \left( f(x) - g(x) \right)  = \lim_{x \rightarrow \infty} \left( x-5 - x \right) = \lim_{x \rightarrow \infty} -5 = -5.$$ Yet, if we consider $h(x) = x-5$ and $k(x) = x^2$ then $$\dsp \lim_{x \rightarrow \infty} h(x) = \infty \;\;\; \mbox{     and     } \;\;\; \lim_{x \rightarrow \infty} k(x) = \infty.$$  But what about $\dsp \lim_{x \rightarrow \infty} \left( h(x) - k(x) \right)?$ $$\lim_{x \rightarrow \infty} \left( h(x) - k(x) \right) = \lim_{x \rightarrow \infty} \left( x-5 - x^2 \right) = \lim_{x \rightarrow \infty} \left( -x^2 + x - 5 \right) = -\infty.$$ Thus we say that $\infty - \infty$ is an {\bf indeterminate} form because when we consider two functions both of whose limits tend to $\infty$, the limit of their differences is not determined.

\begin{prb}
Show that $\dsp \frac{\infty}{\infty}$ is an indeterminate form by finding functions $f, g, h$ and $k$ so that
$$\dsp \lim_{x \rightarrow \infty} f(x) = \lim_{x \rightarrow \infty} g(x) =\lim_{x \rightarrow \infty} h(x) =\lim_{x \rightarrow \infty} k(x)=\infty,$$
but
$$\dsp \lim_{x \rightarrow \infty} \frac{f(x)}{g(x)} \neq  \lim_{x \rightarrow \infty} \frac{h(x)}{k(x)}.$$
\end{prb}

\begin{prb}
Show that each of  $\dsp \infty^0$, $\dsp 0^\infty$, and $\dsp 0 \cdot \infty$ is an indeterminate form.
\end{prb}

\section{Exponential and Logarithmic Functions} \label{appexp}

\setcounter{dfn}{0}
\setcounter{prb}{0}
\setcounter{axm}{0}
\setcounter{expl}{0}
\setcounter{lem}{0}
\setcounter{thm}{0}

The goal of this appendix is to review exponential and logarithmic functions. Even though you should definitely have seen these functions before you made it to Calculus, it is my experience that many students benefit from a review of the basics.  Use this as a review and ask if you have any troubles.

\begin{prb}
Use a limit table to convince yourself that the $\lim_{n \rightarrow \infty} (1 + \frac{1}{n})^n$ exists by
substituting in $n=10, n=100,$ and $n=1000.$
\end{prb}

\begin{dfn}
We define the real number {\bf $e$} by $e = \lim_{n \rightarrow \infty} (1 + \frac{1}{n})^n.$
\end{dfn}

\begin{dfn}
Any function of the form $f(x) = b^x$ where $b$ is any positive real number other than $1$ is called an {\bf exponential} function.
\end{dfn}

\begin{prb}
Graph $r(x) = 2^x, s(x) = 3^x, u(t) = 2^{-t},$ and $v(t) = 3^{-t}$ all on the same pair of coordinate axes by either plotting points (preferable) or by using your favorite technological weapon.  Pay special attention to the domain and range, x- and y-intercepts, and any vertical or horizontal asymptotes.
\end{prb}

\begin{dfn}
The function $f(x) = e^x$ is called the {\bf natural exponential} function.
\end{dfn}

\begin{prb}
Graph each of the following variants of the exponential function, listing domain, range, intercepts, and asymptotes.
\begin{enumerate}
\item $f(x) = e^x$
\item $g(x) = e^{-x}$
\item $h(x) = e^{x-2}$
\item $i(x) = e^x-3$
\end{enumerate}
\end{prb}

\begin{dfn}
We define the inverse of $f(x) = e^x$ by $g(x) = \ln(x)$ and refer to this as the {\bf natural logarithm} function.
\end{dfn}

\begin{dfn}
We define the inverse of $f(x) = b^x$ by $g(x) = \log_b(x)$ and refer to this as the {\bf logarithm to the base b}.
\end{dfn}

By definition of inverses, we have these Inverse Properties.
\begin{enumerate}
\item $\ln(e^x) = x$ for all $x \in \re$ and
\item $e^{\ln(x)}=x$ for all $x > 0.$
\end{enumerate}

The following theorem relates the natural logarithm function to all other base logarithm functions which means that if you really understand the natural logarithm function then you can always convert the other logarithms to the natural logarithm function.

\begin{thm}
\textbf{Change of Base Theorem.} For any $b \in {\nat}$,  $\dsp \log_b(x) = \frac{\ln(x)}{\ln(b)}.$
\end{thm}

\begin{thm}
\textbf{Algebra of Exponential Theorem.}
\begin{enumerate}
\item $e^xe^y = e^{x+y}$
\item $e^x/e^y = e^{x-y}$
\item $(e^x)^y = e^{xy}$
\end{enumerate}
\end{thm}

Each of these laws leads to a corresponding statement about logarithms.

\begin{thm}
\textbf{Algebra of Natural Log Theorem.}
\begin{enumerate}
\item $\ln(xy) = \ln(x) + \ln(y)$
\item $\ln(x/y) = \ln(x) - \ln(y)$
\item $\ln(x^y) = y\ln(x)$
\end{enumerate}
\end{thm}

\begin{prb}
Prove the Logarithmic Laws using the Exponential Laws and the Inverse Properties.
\end{prb}

\begin{prb} \label{e1}
Evaluate each of the following without using a calculator.

    \begin{enumerate}
    \item If $e^4 = 54.59815 . . .$,  then   $\ln(54.59815 . . .) = $ \hrulefill\ .
    \item If $\ln(24) = 3 .1780538 . . .$,   then  $e^{3 .1780538}  =$ \hrulefill\ .
    \item $e^{\ln(4)} = $ \hrulefill\ .
    \item $\ln(e^{x}) = $ \hrulefill\ .
    \end{enumerate}
\end{prb}

\begin{prb} \label{e2}
Evaluate each of the following using a calculator.
    \begin{enumerate}
    \item $\ln(3.41) = $ \hrulefill\ .
    \item $e^{4\ln(2)} = $ \hrulefill\ .
    \end{enumerate}
\end{prb}


\begin{prb} \label{e3}
Write each of the following as a single logarithm.
    \begin{enumerate}
    \item $3\ln(2x) - 2\ln(x) + \ln(y) - \ln(z)$
    \item $2\ln(x+y) + \ln(\frac{1}{x-y})$
    \item $\ln(x) + 3 \hspace{.25in}$. Remember that $3 = \ln(what)?$
    \item Which of the following is correct?  Why?
        \begin{enumerate}
           \item $\dsp \ln(x) - \ln(y) + \ln(z) = \ln(\frac{x}{yz})$ or \\
           \item $\dsp \ln(x) - \ln(y) + \ln(z) = \ln(\frac{xz}{y})$
        \end{enumerate}
    \end{enumerate}
\end{prb}

\begin{prb} \label{e4}
Expand each of the following into a sum, difference, or multiple of the logarithms.

    \begin{enumerate}
    \item $\ln(9x^2y)$
    \item $\ln(4x^{-1}y^2)$
    \item $\dsp \ln(\frac{xy^2}{z^3})$
    \item $\dsp \ln\big( \frac{ (x+y)^2 }{ (x-y)^2 } \big)$
    \end{enumerate}

\end{prb}

\begin{prb} \label{e5}
Solve each of the following for x; give both exact and approximate answers.
    \begin{enumerate}
    \item $\ln(3x) = 2$
    \item $\ln(6) + \ln(2x) = 3$
    \item $\ln(x^2-x-5)=0$
    \item $\ln(x+3) + \ln(x-2)=2$
    \item $\ln ^2x-\ln x^5 + 4 = 0$ \hspace{.25in} Notation: $\ln ^2x$  means $\ln(x) )^2$ and $\ln x^5$ means $\ln(x^5)$
    \item $\ln ^2x+\ln x^2-3=0$
    \end{enumerate}
\end{prb}

\begin{prb} \label{e6}
Solve for x; give both exact and approximate answers.
    \begin{enumerate}
    \item $e^x = 19$
    \item $4^x = \frac{2}{3}$
    \item $4^{x+1} = e$
    \item $(\frac{1}{2})^x = 3$
    \item $x^2 e^x - 9 e^x = 0$ \hspace{.25in} Factor, factor, factor!
    \item $e^x - 8e^{-x} = 2$
    \item $5^x = 4^{x+1}$
    \item $3^{2x+1} = 2^{x-1}$
    \end{enumerate}
\end{prb}

\noindent
\textbf{Solutions to Exponential and Logarithmic Exercises}

\begin{sol}
These are the solutions to Problem \ref{e1}.
    \begin{enumerate}
    \item $4$
    \item $24$
    \item $4$
    \item $x$
    \end{enumerate}
\end{sol}

\begin{sol}
These are the solutions to Problem \ref{e2}.
    \begin{enumerate}
    \item $1.2267123$
    \item $16$
    \end{enumerate}
\end{sol}

\begin{sol}
These are the solutions to Problem \ref{e3}.
    \begin{enumerate}
    \item $\dsp \ln(\frac{8xy}{z})$
    \item $\dsp \ln(\frac{(x+y)^2}{(x-y)})$
    \item $\ln(xe^3)$
    \item Only the second statement is true.
    \end{enumerate}
\end{sol}

\begin{sol}
These are the solutions to Problem \ref{e4}.
    \begin{enumerate}
    \item $\ln(9) + 2\ln(x) + \ln(y)$
    \item $\ln(4) + 2\ln(y) - \ln(x)$
    \item $\ln(x) + 2\ln(y) - 3\ln(z)$
    \item $2\ln(x+y) - 2\ln(x-y)$
    \end{enumerate}
\end{sol}

\begin{sol}
These are the solutions to Problem \ref{e5}.
    \begin{enumerate}
    \item $e^2/3$
    \item $e^3/12$
    \item $-2,3$
    \item $\dsp \frac{-1+\sqrt{25+4e^2}}{2}$
    \item $e, e^4$
    \item $\frac{1}{e^3},e$
    \end{enumerate}
\end{sol}

\begin{sol}
These are the solutions to Problem \ref{e6}.
    \begin{enumerate}
    \item $ln19$
    \item $\dsp \frac{\ln(\frac{2}{3})}{\ln4}$
    \item $\dsp \frac{1}{\ln4}-1$
    \item $\dsp \frac{\ln3}{\ln\frac{1}{2}}$
    \item $-3,3$
    \item $ln4$
    \item $\dsp \frac{\ln4}{\ln(\frac{5}{4})}$
    \item $\dsp \frac{\ln\frac{1}{6}}{\ln\frac{9}{2}}$
    \end{enumerate}
\end{sol}


\section{Trigonometry} \label{apptrig}

\setcounter{dfn}{0}
\setcounter{prb}{0}
\setcounter{axm}{0}
\setcounter{expl}{0}
\setcounter{lem}{0}
\setcounter{thm}{0}

We state here only the very minimal definitions and identities from trigonometry that we expect you to know.  By ``know'' I mean memorize and be prepared to use on a test.

\begin{dfn}
The \textbf{unit circle} is the circle of radius one, centered at the origin.
\end{dfn}

\begin{dfn} \label{sincos}
If $P=(x,y)$ is a point on the unit circle and $L$ is the line through the origin and $P$, and $\theta$ is the angle between the $x-$axis and $L$, then we define \textbf{cos}$(\theta)$ to be $x$ and \textbf{sin}$(\theta)$ to be $y$.
\end{dfn}

\begin{dfn} These are the definitions of the remaining four trigonometric functions:
$$\tan(\theta )=\frac{\sin(\theta )}{\cos(\theta )}, \ \ \ \cot(\theta )=\frac{\cos(\theta )}{\sin(\theta )}, \ \ \ \csc(\theta )=\frac{1}{\sin(\theta )}, \ \ \ \mbox{ and }\ \ \ \sec(\theta )=\frac{1}{\cos(\theta)}$$
\end{dfn}

\begin{prb}
Graph each of $\dsp \sin(-\theta )$, $\dsp -\sin(\theta )$, $\dsp \cos(-\theta )$ and $\dsp \cos(\theta )$ to convince yourself of the following theorem.
\end{prb}

\begin{thm} \textbf{Even/Odd Identities.}
\begin{enumerate}
\item $\dsp \sin(-\theta )=-\sin(\theta )$
\item  $\dsp \cos(-\theta )=\cos(\theta )$
\end{enumerate}
\end{thm}

The Pythagorean Identity follows immediately from Definition \ref{sincos}.

\begin{thm} \textbf{The Pythagorean Identity.}
$\dsp \sin^{2}(\theta )+\cos^{2}(\theta )=1$
\end{thm}

\begin{axm}
It's easier to memorize one identity than three.
\end{axm}

\begin{prb}
 Divide both sides of the Pythagorean Identity by $\sin^2(\theta)$ to show that $\dsp 1+\tan^{2}(\theta )=\sec^{2}(\theta )$.  Divide by $\cos^2(\theta)$ to show that $\dsp 1+\cot^{2}(\theta )=\csc^{2}(\theta )$.
\end{prb}

These next ones are a bit tricky to derive, but easy to remember if you memorize the Double Angle Identities.

\begin{thm} \textbf{Sum/Difference Identities.}
\begin{enumerate}
\item $\sin(x \pm y) = \sin(x)\cos(y) \pm \cos(x)\sin(y)$
\item $\cos(x \pm y) = \cos(x)\cos(y) \mp \sin(x)\sin(y)$
\end{enumerate}
\end{thm}

\begin{thm} \textbf{Double Angle Identities.}
\begin{enumerate}
\item $\sin(2x) = 2\sin(x)\cos(x)$
\item $\cos(2x) = \cos^2(x) - \sin^2(x)$
\end{enumerate}
\end{thm}

\begin{prb}
Use the first Sum/Difference Identity to prove the first Double Angle Identity.
\end{prb}

\begin{prb}
Use the second Sum/Difference Identity to prove the second Double Angle Identity.
\end{prb}

\begin{thm} \textbf{Half Angle Identities.}
\begin{enumerate}
\item $\dsp \sin^2(x/2) = \frac{1-\cos(x)}{2}$
\item $\dsp \cos^2(x/2) = \frac{1+\cos(x)}{2}$
\end{enumerate}
\end{thm}

\begin{thm} \textbf{Product Identities.}
\begin{enumerate}
\item $\sin(mx) \sin(nx)={1 \over 2}[ \cos(m-n)x- \cos(m+n)x]$
\item $\sin(mx) \cos(nx)={1 \over 2}[\sin(m-n)x+ \sin(m+n)x]$
\item $\cos(mx) \cos(nx)={1 \over 2}[\cos(m-n)x+ \cos(m+n)x]$
\end{enumerate}
\end{thm}

\section{Summation Formulas} \label{appsum}

\setcounter{dfn}{0}
\setcounter{prb}{0}
\setcounter{axm}{0}
\setcounter{expl}{0}
\setcounter{lem}{0}
\setcounter{thm}{0}

These formulas are useful for computing Riemann Sums.
\begin{enumerate}
\item $\dsp \Sigma_{i=1}^{n} c = nc$ where $c$ is any real number
\item $\dsp \Sigma_{i=1}^{n} i = \frac{n(n+1)}{2}$
\item $\dsp \Sigma_{i=1}^{n} i^{3} = [\frac{n(n+1)}{2}]^{2}$
\item $\dsp \Sigma_{i=1}^{n} i^{2} = \frac{n(n+1)(2n+1)}{6}$
\item $\dsp \Sigma_{i=1}^{n} i^{4} = \frac{n(n+1)(2n+1)(3n^{2}+3n+1)}{30}$
\end{enumerate}



\backmatter

\begin{annotation}
\chapter{Endnotes to Instructor}

\renewcommand\notesname{}
\vspace{-2cm}
\begingroup
\setlength{\parindent}{0pt}
\setlength{\parskip}{2ex}
\renewcommand{\enotesize}{\normalsize}
\theendnotes
\endgroup
\end{annotation}




\end{document}

